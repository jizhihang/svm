\section{Kernel principle component analysis}
Linear component analysis works by finding the main axes which describe a given data set best. Linear combinations of these principal component vectors can then be used to recreate each point in the original data set. In practice linear PCA operates on the mean:
\begin{equation}
\bar{\mathbf{x}} = \frac{1}{N} \sum\limits_{k = 1}^{N} \mathbf{x}_k . 
\end{equation}
Next from each data point's deviation from the mean the covariance matrix is computed:\footnote{Support Vector Machines: Methods and Applications, Suykens et al., page 20}
\begin{equation}
\Sigma = \frac{1}{N - 1} \sum\limits_{k = 1}^{N} (\mathbf{x}_k - \bar{\mathbf{x}})(\mathbf{x}_k - \bar{\mathbf{x}})^T
\end{equation}
The linear principle components are defined to be the eigenvectors $\mathbf{u}$ of the covariance matrix $\Sigma$;
\begin{equation}
\Sigma \mathbf{u} = \lambda \mathbf{u}. 
\end{equation}
If the eigenvectors are stored in a matrix $U$,
\begin{equation}
\mathbf{x} = \bar{\mathbf{x}} + \mathbf{U}\mathbf{b}
\end{equation}
can be used to recombine the prinipal components to match points close to the original data set. By enforcing limits on the weight vector $\mathbf{b}$ the closeness of remaps to the original data can be controlled.  \footnote{Active Shape Models - Their training and Application, Cootes et al, pages 43,44 and 49}
Another important application of linear PCA is dimensionality reduction. An reduction of input dimension is achieved by considering the remapped data,
\begin{equation}
\mathbf{z}_i = \mathbf{u}_i^T (\mathbf{x} - \bar{\mathbf{x}}_i) \;\; i = 1,\dots,m.
\end{equation}
With $m < N$, the dimension is reduced by neglecting the smallest eigenvalues and their corresponding eigenvectors. This method allows to reduce the input data complexity, while keeping the information loss as small as possible. 
\begin{figure}
\centering
% This file was created by matlab2tikz.
% Minimal pgfplots version: 1.3
%
%The latest updates can be retrieved from
%  http://www.mathworks.com/matlabcentral/fileexchange/22022-matlab2tikz
%where you can also make suggestions and rate matlab2tikz.
%
\documentclass[tikz]{standalone}
\usepackage{pgfplots}
\usepackage{grffile}
\pgfplotsset{compat=newest}
\usetikzlibrary{plotmarks}
\usepackage{amsmath}

\begin{document}
\definecolor{mycolor1}{rgb}{0.00000,0.44700,0.74100}%
\definecolor{mycolor2}{rgb}{0.85000,0.32500,0.09800}%
\definecolor{mycolor3}{rgb}{0.92900,0.69400,0.12500}%
\definecolor{mycolor4}{rgb}{0.49400,0.18400,0.55600}%
%
\begin{tikzpicture}

\begin{axis}[%
width=1.5in,
height=1.5in,
scale only axis,
xmin=-3,
xmax=3,
ymin=-3,
ymax=3,
xlabel={$x$},
ylabel={$y$},
title={Linear PCA}
]
\addplot [color=mycolor1,only marks,mark=o,mark options={solid},forget plot]
  table[row sep=crcr]{%
0.0252096057237872	1.99997607141453\\
0.0542652775567787	1.99668444030418\\
0.0808900749886523	1.99825642146195\\
0.101366643507615	1.99783968345816\\
0.121749309882475	2.01870873559643\\
0.151476321234637	2.00000925448584\\
0.153941035925459	1.985784151057\\
0.195915107091867	2.00343949071584\\
0.229527599596238	1.99999208318555\\
0.253293740908129	1.98187925622405\\
0.261700405104983	2.00111764341199\\
0.271385473595184	1.97747751659113\\
0.315570555217003	1.99585617344955\\
0.381400297805844	1.97713737642657\\
0.373580454000242	1.92852123498518\\
0.414389140828103	1.98220102654256\\
0.455388433761919	1.95143123134093\\
0.482903433473298	1.93988715035403\\
0.426222561526825	2.01718931305916\\
0.504681921807889	1.9610493573584\\
0.464013948059054	1.95836762525045\\
0.550631587344094	1.95562773492631\\
0.542614985661499	1.92235544156257\\
0.509053726746412	1.97571241044708\\
0.632863127742311	1.96021514372426\\
0.643811813851122	1.94578570474198\\
0.739428426070039	1.99902298867195\\
0.681858203372199	1.94108818448797\\
0.821795284580571	1.99798669003619\\
0.692964071256407	1.95009943613109\\
0.684782986797712	1.96610747700205\\
0.804415925100892	1.88606660233353\\
0.77214381332821	1.98107256061489\\
0.8561862693169	1.98577548052889\\
0.793890669288902	1.81953949716239\\
0.94472744757324	1.90168474257471\\
0.885402662894457	1.98820300499133\\
0.948941440096197	1.8765455109341\\
0.864960845375703	1.9240470623056\\
0.971074023417243	1.85248013057106\\
0.884407395425057	1.97099773774496\\
0.964369614528843	1.95752739248484\\
0.900909743147055	1.76973306189219\\
1.18413274323466	1.90986540350391\\
1.1099687289197	1.91748318176565\\
0.992093616502852	1.82410709369686\\
1.09772201804167	1.89070138505298\\
1.08325985003493	2.00081574628242\\
1.00508592510932	1.89944106486253\\
1.16758136262516	1.88068451641741\\
1.16174862882086	1.72753294086775\\
1.45415473356334	1.84738354744748\\
1.22573396992357	1.94041187708243\\
1.34290885014218	1.87170475153035\\
1.26605694328578	1.90839209788116\\
1.36838081426711	1.8769289731854\\
1.3482644008382	1.83847354419544\\
1.36484147657509	1.64164427591067\\
1.29929253776045	1.69210700090283\\
1.41025993360966	1.57612903890478\\
1.32765804067646	1.59986884248708\\
1.48915722514583	1.87752066426332\\
1.37176437757486	1.63157176344676\\
1.30615767530574	1.7695810376244\\
1.36082962109504	1.79217341207432\\
1.26816043231762	1.5030211372027\\
1.54670532739515	1.63534655231132\\
1.70242507488103	1.64874370554204\\
1.47050763499058	1.83265445970779\\
1.40873622219343	1.79141155794274\\
1.76073410124384	1.5284977034257\\
1.6083774527115	1.78461211184732\\
1.37815632311487	1.56277486547138\\
1.88651112110521	1.89165827523777\\
1.41958740058221	1.59376504378723\\
1.77907241600095	1.74514506053466\\
1.62438926635298	1.74230143262092\\
1.62616358251699	1.53079009192941\\
1.61672449680776	1.42925840353382\\
1.77144535440835	1.64999571315109\\
1.57727343328513	1.80020463874339\\
1.93749484153559	1.79318586916688\\
2.03648333513386	1.43316974609015\\
1.84072051057321	1.63627048194093\\
1.54836974045472	1.5384218679536\\
1.7121394430095	1.65018955501604\\
1.73772940487125	1.79972345690608\\
1.73414950154265	1.46321336270143\\
1.71289224533864	1.7482573597082\\
1.80239552071758	1.52028802153443\\
1.9001598650258	1.17510507196256\\
1.89777584244137	1.3742599723717\\
1.70018599898497	1.553477459584\\
1.76670459271109	1.6302998207571\\
1.7689700974124	1.46588083067326\\
1.90930590009211	1.61261528163429\\
1.90706237339255	1.71229195383381\\
1.92988023314953	1.39891532312873\\
1.80588875599868	1.52556044006396\\
1.93938309862924	1.29806402739107\\
2.02163475040791	1.56955330006056\\
1.92695664510478	1.60994667216631\\
2.08758846909191	1.25487131167417\\
1.90511760234139	1.20009212826382\\
1.98097581960974	1.43346340433535\\
1.94673066113215	1.52038050801568\\
2.16410361576076	1.34936564890888\\
2.03641851034698	1.42970383952132\\
1.88099807150121	1.41439043339432\\
1.83503077161051	1.09414055214155\\
1.93249875747306	1.52240931968773\\
2.05030679491722	1.36123573228126\\
1.84652424071819	1.508505203139\\
1.7851378296155	1.28944583310626\\
2.11692587967433	1.0024349005619\\
2.01025120616327	1.04568541162729\\
1.86392190545371	1.33014443617879\\
2.04005058950193	1.31869983122255\\
1.92377489509653	1.12498559312132\\
1.80213374304114	1.07774825150461\\
1.95787308202436	1.19915932111511\\
1.88072644151616	1.07168977152398\\
2.09967342333427	1.41560260960808\\
2.17013282497243	1.02793352331685\\
1.88366201076541	0.941586330839192\\
1.72338061446466	0.987802777870307\\
1.88003407879788	1.24642881458309\\
1.92086095498696	1.15507757415611\\
2.0209406240519	0.976974071862971\\
2.18306336077639	1.21079939581542\\
2.22050931478028	1.04405182371875\\
1.99049886077393	1.45525544822546\\
1.76374196045846	1.07710077973711\\
2.18791002180786	1.00178609065663\\
2.2343891038206	0.915987373321067\\
2.03132528382159	0.776418345485256\\
1.95234652319889	1.00128433025528\\
2.27201241382157	0.963776210910149\\
1.73188484222463	0.866795603804765\\
1.94918890303658	1.13975291322449\\
1.89059216903087	1.05665241422918\\
2.06848643567122	0.769978820981561\\
1.90978191853865	0.752115982916175\\
2.14074279111578	0.961138895055768\\
1.58804370795702	0.670012148006513\\
1.64811388889222	0.454075454265649\\
1.86768493975299	0.587060032085191\\
1.93276798468813	0.616684880312694\\
1.84371533329868	0.515018996654667\\
1.62186306892625	0.647536964126465\\
2.29539795949656	0.959873291571083\\
1.81870879996465	0.752776768843789\\
1.750810621844	0.576752218657265\\
2.26961555688771	1.00978512313861\\
1.91028666052759	1.05392653190567\\
1.61454262374459	0.539269601451133\\
2.0862411429314	0.553696058294345\\
1.51485776055253	0.77781145100211\\
1.71051698523901	0.911868115887879\\
2.1246500991133	0.627586790641797\\
1.67737075522286	0.712156868271379\\
1.87596062247687	0.645743934174888\\
1.65425838595063	0.761151178577534\\
1.95431941589544	0.258004803260987\\
1.68846700630448	0.396720371390254\\
1.15942875612853	0.682059857084357\\
1.67510314653331	1.35794951852743\\
2.14787501562466	0.662265040221223\\
1.78829903050311	0.833288442422855\\
1.65889502054155	0.753967950396459\\
1.35354321534726	0.389833698812968\\
1.38248349849842	0.15799939111346\\
1.3564051515429	0.614193666322743\\
1.7386368293161	0.869202377743102\\
1.654183904986	-0.129071466971251\\
1.65669891543695	0.308059726877272\\
2.13920936132072	0.0630892746374619\\
1.93337174510856	0.562990491371363\\
1.81837693190798	0.831394335583366\\
1.56129991037271	0.18142051068844\\
1.20528265297514	0.388150061797963\\
1.62406405429853	0.185230839564162\\
1.47846860776594	0.148500231124781\\
1.59824571893907	0.30738385730661\\
1.41834822412495	0.292100840301446\\
1.50449450370496	0.0480260558360136\\
1.35903596425556	0.318977738196747\\
1.49873496680872	0.231336734804274\\
1.4538930327726	0.0267626473546722\\
1.13247801468771	0.692953012631509\\
2.16055376039454	0.241981542213923\\
0.73570352634472	0.738575152348107\\
1.22916989516643	0.317365370662156\\
1.13292888674327	0.121099815874434\\
1.30420528581571	0.582852015125134\\
1.42439953609805	0.338288647338872\\
0.814891710269256	0.491134495021951\\
1.45178643231601	0.0266099177314904\\
1.56859872134606	-0.198785050743016\\
1.64764284840506	0.167120282661557\\
1.10418132336813	-0.0765693713188222\\
1.11561313521497	-0.0370879943892937\\
1.03725012300773	-0.525418966245873\\
0.865155686903486	0.23984647704458\\
1.64759738562179	0.251854933990762\\
1.26453745729961	0.538567239131884\\
1.09637029196772	-0.331366853457615\\
0.975372185297645	0.0887052103649253\\
0.908134070444304	0.163207274576677\\
0.325747349608221	-0.483595123297063\\
0.63900320144928	-0.191133294545802\\
0.765572074921706	0.14842786003316\\
0.432012971737715	0.219965363969939\\
0.935841417848618	-0.292809766334486\\
0.931008424738499	0.141342301643102\\
0.82068918643219	0.0700097725538053\\
0.868705646471464	0.0194735516826833\\
0.664969184277619	-0.0719641616437988\\
1.05603525313678	-0.320543112823234\\
0.870716641568264	-0.0796803715903072\\
0.727546794304449	-0.305203334106111\\
0.817214381033212	-0.381777940173187\\
0.417584137862282	-0.595300009129268\\
0.537394122154339	-0.131177120108996\\
1.29171247374217	-0.258219922078456\\
0.735889062336743	-0.517619756084823\\
-0.360156211969547	-0.22645623516825\\
0.765847250483584	0.0194198504627727\\
0.419885739362212	0.453904700200194\\
0.982946233923353	-0.766130712624162\\
1.00040538885305	0.0963942898427855\\
0.109318122566416	-0.352811745680175\\
0.24850253433445	-0.448947947236817\\
0.320455608374698	-0.435132847211648\\
0.689377307721671	-0.759070790628597\\
-0.33578147945411	0.0675875240972314\\
0.525217224667007	0.0267955045132378\\
0.142660428690845	-0.129992148158271\\
0.696771682066403	-0.935898798951477\\
0.393828848107038	-0.410889343396687\\
0.127602038208516	-0.473043655424041\\
-0.0881445419301995	-0.343255304273698\\
0.041819420710484	-0.379292134325631\\
-0.156820260984138	-0.581516653451297\\
0.0644456006617879	-0.204122388305596\\
0.037139223722029	-0.147140231337332\\
0.106125047270562	-1.21019483440756\\
0.039458551265047	-0.403905927954739\\
0.710004611360853	-0.385952702115495\\
0.736726894333947	-0.830175870665623\\
};
\addplot [color=mycolor2,only marks,mark=o,mark options={solid},forget plot]
  table[row sep=crcr]{%
-0.0257785645204607	-1.55229203060528\\
-0.0498967820323308	-1.54958408652859\\
-0.0793085440393965	-1.5521856456645\\
-0.109906706182643	-1.54244264956815\\
-0.125700594951115	-1.55056677981119\\
-0.145396775966016	-1.52266720764417\\
-0.183363593113896	-1.5436718904051\\
-0.218521027885843	-1.53145796039904\\
-0.243424323336384	-1.55381635661645\\
-0.237615583519078	-1.55429285083151\\
-0.291344111490577	-1.52582655959974\\
-0.261390803264229	-1.55785130159408\\
-0.317866778776156	-1.47916361210781\\
-0.335164610444007	-1.50665538033238\\
-0.382132030695385	-1.56797159655555\\
-0.372565345366419	-1.56095262005124\\
-0.382860735728689	-1.52815551424323\\
-0.423673739146399	-1.54810694054304\\
-0.460509695214743	-1.53362834901732\\
-0.540951449544596	-1.59412232640275\\
-0.499099485620386	-1.5276700295708\\
-0.534897840636601	-1.49296894083153\\
-0.550033405271042	-1.54270372283686\\
-0.571116804237858	-1.48924630736613\\
-0.613098443120687	-1.43597200674607\\
-0.583986794029655	-1.46227041459537\\
-0.64150759475729	-1.44726068894559\\
-0.725376984725954	-1.42641639112938\\
-0.740180883636255	-1.55661227096743\\
-0.797991982697432	-1.5717958918167\\
-0.860741470190061	-1.46939771985393\\
-0.831182861976291	-1.51977015144589\\
-0.813388288783237	-1.53131216839832\\
-0.922634280315665	-1.41696222165957\\
-0.847366984321111	-1.41511759758694\\
-0.83450863223717	-1.48147345422914\\
-0.882635034000288	-1.41316424308918\\
-0.885664451610224	-1.47513095591159\\
-0.941870579515756	-1.4940000332642\\
-0.841741837226549	-1.55881980092359\\
-1.03579875681091	-1.49398888862321\\
-0.999273268086965	-1.41383991354962\\
-1.01383153967895	-1.57170925428048\\
-0.958369484820464	-1.56208647491563\\
-0.989204533484056	-1.45724407348306\\
-1.13152065867109	-1.45077249145617\\
-0.955633356091973	-1.48449500691868\\
-0.945068862087621	-1.47242270928704\\
-1.2103581826042	-1.41662823218126\\
-1.28024199072939	-1.38591887283525\\
-1.20739023415321	-1.48445504095233\\
-1.22406254581856	-1.41235220199581\\
-1.35330297397533	-1.45231398373445\\
-1.27206333806534	-1.26414862634002\\
-1.27512531313791	-1.2173438313114\\
-1.37079695011101	-1.40800026862167\\
-1.44327444538879	-1.336862666656\\
-1.43885959666231	-1.2199555601228\\
-1.38468249557781	-1.23107583840802\\
-1.42201041480349	-1.49492121071153\\
-1.32203682868692	-1.2349671750763\\
-1.28185574844467	-1.40916406310554\\
-1.62125883782101	-1.41556579773739\\
-1.47900851684798	-1.26945250906947\\
-1.43761795182563	-1.47737299155127\\
-1.54419734155947	-1.29061526442799\\
-1.53742296443518	-1.53541217210406\\
-1.46771181969171	-1.4588131829972\\
-1.65867378942448	-1.22164404756545\\
-1.48574453111986	-1.28275884247852\\
-1.49602791407209	-1.0587428453439\\
-1.59397425552048	-1.15148551655787\\
-1.71530294333393	-1.37872386909668\\
-1.67650144571949	-1.18446066989326\\
-1.65311099273438	-1.19182891591923\\
-1.63508502899037	-1.06901153934879\\
-1.39644499863987	-1.35575644266041\\
-1.85695837160397	-1.07946274714682\\
-1.80449892357766	-1.06978411520179\\
-1.89155606295175	-1.25218887657974\\
-1.78784018469446	-1.0663938840251\\
-1.63347382282517	-1.33693290630827\\
-1.81897811080574	-1.16084726432498\\
-1.84829975789657	-1.28320320595491\\
-1.79177330776733	-1.15936777948833\\
-1.78391537349941	-1.21164903240922\\
-1.7672180421973	-0.973753765746598\\
-1.59151719035543	-1.02579853468873\\
-1.83246977130268	-1.11179032945606\\
-1.76882434632801	-1.29866098710694\\
-1.82269780507015	-1.24645497234626\\
-1.70271234835196	-1.17635729588757\\
-1.89467430278034	-1.14127816821242\\
-1.92425132278915	-1.06989313512453\\
-1.97936618203155	-0.964315740587365\\
-1.82973940494891	-0.98225250693239\\
-1.89717084682634	-1.07796013628671\\
-2.09360236790814	-1.10506045861807\\
-1.80987679349089	-1.03419428379549\\
-1.91781546405301	-1.03949735243231\\
-1.99843741942447	-1.29184975373077\\
-1.73516344833119	-0.86682375741105\\
-2.13118726690554	-0.882661035237501\\
-1.81330883999235	-0.717737612588012\\
-1.91630863711476	-1.05526290861572\\
-2.13707575361358	-0.763563264474698\\
-1.96872466916745	-0.67944397630352\\
-1.98041341127616	-1.04392024391266\\
-2.11974587620345	-0.896919918170228\\
-1.6776937288212	-0.727110976755328\\
-1.95437459022583	-0.784798165767336\\
-1.97101692439671	-0.859061437356317\\
-2.03119319740258	-0.853007054499041\\
-1.70728194559278	-1.03666056250699\\
-1.88559577878238	-0.538939976418723\\
-1.66819935386158	-1.00495881720213\\
-1.91969127204619	-0.812496071932997\\
-2.02415061280072	-0.735951198636854\\
-2.0916032513645	-0.758994299043236\\
-2.01919022750961	-0.693201570009749\\
-2.11940270456961	-1.04417764723986\\
-1.7193796837629	-0.537385996294986\\
-1.98322618377247	-0.836920625320379\\
-2.15472997804464	-0.604909415709117\\
-2.15739788279318	-0.827476312929507\\
-1.89359900360034	-0.250976564116217\\
-1.63765949653619	-0.678007807274102\\
-1.95600754725664	-0.830549852700785\\
-2.2239373356186	-0.624845361308943\\
-1.99569014331284	-0.490294444669216\\
-2.06302528412929	-0.474400072943434\\
-2.26597321931492	-0.926012479714969\\
-2.00226465601093	-0.648231245229342\\
-1.87655641999738	-0.235550563623002\\
-1.50474718970262	-0.361377221380458\\
-1.6103737405893	-0.582398314384744\\
-1.92775824706411	-0.0597983925607425\\
-1.62242124240915	-0.609171427872175\\
-2.10413872500622	-0.545858597917913\\
-2.12249299464889	-0.280635988466376\\
-1.54490337527676	-0.460516130452826\\
-1.90182562967955	-0.656817163696101\\
-1.92836950634556	-0.665578330585617\\
-2.23881279314758	-0.901815370399758\\
-1.93974611615187	-0.347710676280764\\
-2.04716621633192	-0.822141711474344\\
-1.81946223289969	-0.559360270798851\\
-1.50620392060465	-0.374847089418671\\
-1.82863574562032	-0.642186077553999\\
-2.06176998048506	-0.516609055630509\\
-1.56652763404705	-1.00053803407384\\
-1.96449951956457	0.0617111874950914\\
-1.7996270238056	-0.629902241339237\\
-1.29437721682435	-0.174381078429632\\
-1.82600834129156	-0.301578808781892\\
-1.85940024963483	-0.117293680231677\\
-1.95918335755553	-0.490310334975374\\
-1.90593167852852	0.0838445821010763\\
-1.95038085663778	-0.124631149677822\\
-1.75391837021145	-0.899207590900018\\
-1.77039037523319	0.217526100808311\\
-1.65235475905915	-0.53123004499732\\
-1.8370213222327	-0.805942987950651\\
-2.09591811644858	-0.0553912269795177\\
-2.16963637145989	-0.31676661349549\\
-1.76037440920594	-0.484630417356484\\
-1.66819866477871	-0.0167015603972909\\
-1.7124438566096	-0.0889767583364116\\
-1.62064394314317	0.223232368711154\\
-1.83004903761568	0.153065120049793\\
-1.81352744143549	0.429332678636591\\
-1.5878331126562	-0.381700405108424\\
-1.5687584730913	0.336776403893821\\
-1.41007323782575	0.433812757396804\\
-1.79922604361375	-0.0751474496139841\\
-1.45367770012475	0.0791754261870778\\
-1.58586883070679	-0.0776816206272764\\
-1.34279320848638	0.0084954821648237\\
-1.75521795586068	-0.203208625392691\\
-1.18652679732711	-0.00877192916902404\\
-1.75243814534615	-0.338480277372437\\
-1.39838209715547	0.44129597178331\\
-1.39485325305253	0.373858497298805\\
-1.49567650322448	-0.168034990057268\\
-1.27951781090358	0.377371343930035\\
-1.8528717859086	0.18802778371839\\
-2.12781248944898	0.361707357274624\\
-1.52469444791007	0.370235275558373\\
-1.38375584237211	-0.0822270668949999\\
-1.93707167166853	-0.149379996121342\\
-1.2295766177135	0.651890819878131\\
-1.00631790511102	-0.0692967630940565\\
-1.58052638134334	0.0699006975943077\\
-1.49614019771953	0.468949852135243\\
-1.51035427391087	0.59995877990134\\
-1.28397878801989	0.287452913084822\\
-1.29565122526886	0.45339821213825\\
-1.16346820613333	0.206087060865922\\
-0.781428551577179	0.690874607824667\\
-0.712796351128313	0.35541859779063\\
-1.00890707118453	0.256533842544474\\
-1.20344313602678	-0.282210061581616\\
-1.00908030410666	0.660904401713759\\
-0.953123811043815	0.439969881790494\\
-1.08186265933063	0.681272625317561\\
-1.50842915595659	0.70687810650817\\
-1.57815222688715	0.1868433144969\\
-0.782986337572477	0.391632724820207\\
-0.707484202031367	0.914582586013556\\
-0.86736687024804	0.881112489375594\\
-1.09531822682492	0.520486711347663\\
-1.44498996205859	1.24413307488435\\
-0.502314404249651	0.517058762374403\\
-0.605418759766275	0.698859334664919\\
-0.880765931100263	1.01772606947867\\
-0.802244008718661	0.797686318023879\\
-0.951499345493502	1.05101611976091\\
-1.17891607390623	1.13178697924617\\
-1.07013331151424	0.533116461522449\\
-0.498604163947393	0.486595516730365\\
-0.930607974569119	0.991804810346011\\
-0.830962201311836	0.701143278080427\\
-0.690697476117149	0.585645361419952\\
-0.0131881677094311	0.449136260383922\\
-0.861550476248566	0.99047077648405\\
-0.548311552086026	0.253412284171596\\
-0.963474983199058	1.01748298234106\\
-0.310222568826283	1.08344497375488\\
-0.423563171845147	0.963514984992154\\
0.618223739200304	0.384640233936662\\
-0.719757006489903	0.118530837848968\\
-0.276022732683633	0.979443102179219\\
0.243272061869179	1.03733130847546\\
-0.0411148471233378	0.77328973720547\\
-0.642071581165032	0.606049490839948\\
0.11791128421646	0.66340116681769\\
-0.649560237408564	0.565939767806741\\
-0.19145575665924	0.89081274106155\\
-0.676571056927597	0.973789966063794\\
-0.613197104481082	0.824279235801786\\
-0.167022223949248	1.08087244514927\\
0.205725157179282	0.933904376990845\\
-0.497232199841278	1.70152550898244\\
-0.456496779881341	0.86489382603738\\
-0.286344024648262	1.04227034733798\\
-0.891846291964865	1.29176216117567\\
-0.0639639362571633	1.05503744986388\\
-0.0443174616984984	1.37260526939801\\
-0.0732852737748077	0.831055071467393\\
0.419602061842569	1.46961485521533\\
};
\addplot [color=mycolor3,only marks,mark=+,mark options={solid},forget plot]
  table[row sep=crcr]{%
0.849791066365513	0.603052959005208\\
0.867561585773218	0.615663781519159\\
0.886010862889738	0.628756283425784\\
0.899432529294121	0.638280948911574\\
0.922837912784848	0.654890544292592\\
0.933782939133606	0.662657665867826\\
0.928708382332309	0.659056514225045\\
0.964956799677737	0.684780149368573\\
0.985684402892952	0.699489461982887\\
0.992941913867935	0.704639743789443\\
1.00761279856099	0.715050915165028\\
1.00289671221418	0.71170415154402\\
1.0409569875095	0.738713568971247\\
1.0759037428708	0.763513481634954\\
1.04775783456673	0.743539779911717\\
1.10023350506767	0.780779060981971\\
1.11297861085555	0.78982360623835\\
1.12582959985944	0.798943291360586\\
1.12461683941074	0.798082657722386\\
1.15030166185907	0.816309853549752\\
1.12198892400698	0.796217761487326\\
1.17830258717578	0.836180668312898\\
1.15726759524027	0.821253217753067\\
1.16012968048541	0.823284292265756\\
1.23515742315193	0.876527617611642\\
1.23562886226445	0.876862173672591\\
1.32434679420333	0.939820721355834\\
1.25871530695027	0.893245434607825\\
1.37863742767028	0.978347988179831\\
1.27035449124217	0.901505164328984\\
1.27246874748414	0.90300554310829\\
1.31425646334786	0.932660133158836\\
1.33763282743259	0.949249134961798\\
1.39574656645686	0.990489462925388\\
1.27585790116123	0.905410650945249\\
1.41494468608808	1.00411337980234\\
1.41632322389134	1.00509165709221\\
1.40588242448399	0.997682359411651\\
1.37244856146274	0.973956068533857\\
1.40924410231557	1.00006796905594\\
1.4075410006561	0.998859365511047\\
1.45436393358398	1.03208718981877\\
1.32352614001542	0.939238345339041\\
1.57802697292151	1.11984448069148\\
1.53229803991403	1.08739301179071\\
1.40983249638604	1.00048552202785\\
1.51151302256868	1.07264295532484\\
1.55386489899335	1.10269790107351\\
1.45402836256285	1.03184905234561\\
1.55324675962385	1.10225923939464\\
1.47708525112558	1.04821133882231\\
1.72812120320715	1.22635862668091\\
1.62011159909377	1.14970977269832\\
1.66561367802576	1.18200025494378\\
1.63181695507583	1.15801646106041\\
1.68501997345968	1.1957719034677\\
1.65349149542178	1.17339776619299\\
1.57161969558106	1.11529756591177\\
1.55184174815542	1.10126217510789\\
1.57090518894005	1.11479051734284\\
1.52717352284993	1.08375640592217\\
1.76562400579465	1.25297243443498\\
1.57147002661808	1.11519135355608\\
1.59297265445059	1.13045066123067\\
1.63999616038955	1.16382082187565\\
1.44189462399702	1.02323836292378\\
1.68959958570525	1.19902181844692\\
1.79948712722113	1.27700334789789\\
1.73204540008479	1.22914342712474\\
1.67149783701482	1.18617593956799\\
1.78151469359675	1.26424923726213\\
1.8010641340962	1.27812246846768\\
1.54325132351042	1.09516599310938\\
2.03656465469427	1.44524506061416\\
1.58543220206717	1.12509959047688\\
1.89596113764419	1.34546598507477\\
1.79174391112248	1.27150838617692\\
1.69309780060734	1.20150432141905\\
1.63890064555699	1.16304339141356\\
1.84598135333987	1.30999790591023\\
1.78773692425172	1.26866483393832\\
2.02399677582401	1.43632628417474\\
1.91991570002611	1.36246530443438\\
1.88557634323161	1.33809643125496\\
1.64496138432617	1.16734438561457\\
1.80663025432049	1.28207245735823\\
1.89422428472095	1.34423342999603\\
1.73302208520665	1.2298365302604\\
1.85341558716961	1.3152735988263\\
1.80534776567993	1.28116234121299\\
1.70745329796646	1.21169167864491\\
1.79986209167576	1.2772694408622\\
1.75303555871374	1.24403906179564\\
1.8335326237372	1.30116368357032\\
1.75743917494155	1.24716408146414\\
1.92002598248569	1.36254356621684\\
1.96557787733514	1.3948694002536\\
1.83285024711906	1.30067943602405\\
1.81015941398734	1.28457692023638\\
1.79157193877722	1.27138634625937\\
1.97440874729368	1.40113621390918\\
1.93050547159421	1.36998031998596\\
1.86975340076235	1.32686770380191\\
1.7225435808613	1.22240047532699\\
1.88313791029652	1.33636600097034\\
1.90138439599223	1.34931459224855\\
1.96523944512869	1.39462923234457\\
1.918236797046	1.36127387344608\\
1.80764391058713	1.28279179701148\\
1.62592539006784	1.15383552076595\\
1.89287671310501	1.34327712781467\\
1.89515886259124	1.34489665178359\\
1.82913536662542	1.298043176639\\
1.68492053473484	1.19570133692537\\
1.77012343842437	1.25616545006975\\
1.71959007680188	1.2203045255895\\
1.75652582009669	1.24651592056481\\
1.86826226968603	1.32580952486425\\
1.69950425888065	1.20605065494899\\
1.5963099533829	1.13281897042557\\
1.7571894174292	1.24698684141912\\
1.64572020733602	1.16788288322835\\
1.95365044488114	1.38640511565541\\
1.81754426670559	1.28981756992054\\
1.58626816996289	1.12569283384352\\
1.50148240308646	1.06552474121454\\
1.72773061956997	1.22608144958707\\
1.71176867558715	1.21475407991786\\
1.69426987443153	1.20233608185503\\
1.91245044775427	1.35716759932727\\
1.85865545085913	1.31899206025502\\
1.89975650294352	1.34815936038177\\
1.57047114377063	1.11448249783767\\
1.81702664957369	1.28945024369723\\
1.807444420378	1.282650229083\\
1.60652092275176	1.14006516956312\\
1.66012360665478	1.17810423400821\\
1.85502126233245	1.31641306380361\\
1.45002704203577	1.02900952121933\\
1.72337602710687	1.22299121955986\\
1.64518448636028	1.16750270963933\\
1.62819645564499	1.15544717905539\\
1.51421618194017	1.07456124832903\\
1.7664730437439	1.25357495294555\\
1.26148751630507	0.895212728829387\\
1.19952362376215	0.851240145180902\\
1.40831794999461	0.999410726446845\\
1.46558465452748	1.04004995765073\\
1.35837556533755	0.963969187886045\\
1.27337223862079	0.903646704241841\\
1.86873224338	1.32614304099312\\
1.45395804639041	1.03179915257904\\
1.32572340373836	0.940797630253033\\
1.87514182544473	1.33069159131697\\
1.65699617921749	1.1758848598058\\
1.21740511889892	0.863929721454971\\
1.53792647591222	1.091387222977\\
1.26369133680398	0.896776666749644\\
1.45708861208862	1.03402075384361\\
1.59834503369294	1.13426316212322\\
1.34078713436237	0.951487584155841\\
1.44151866353293	1.02297156314286\\
1.34853943683429	0.956988994008021\\
1.3106330756086	0.930088801470056\\
1.19929162647673	0.851075508654361\\
0.98211465522174	0.696956195888159\\
1.64407117028965	1.16671264655527\\
1.63015818390412	1.15683931657984\\
1.47173218944106	1.04441254660578\\
1.34823288440298	0.956771449533726\\
0.973293567146992	0.690696323931781\\
0.883122978966472	0.626706900919756\\
1.08108722489489	0.767191931899144\\
1.45565350186142	1.03300233001773\\
0.928335107397781	0.658791620226156\\
1.13631881343768	0.806386946085089\\
1.34160434534176	0.952067516704866\\
1.44064427314625	1.02235105331299\\
1.49084214702486	1.05797389941766\\
1.01310239134652	0.718946596473144\\
0.873895287330681	0.62015848335454\\
1.05664328959187	0.749845329778149\\
0.942476520652678	0.668827052987711\\
1.09712435536349	0.778572657545514\\
0.970266854565604	0.688548421876196\\
0.91236517311324	0.647458580251283\\
0.943505027789142	0.669556931538513\\
0.995051079991546	0.706136510373866\\
0.868676044825535	0.616454655718465\\
0.969331722840602	0.687884807046454\\
1.44023092860071	1.02205772397459\\
0.726980989295389	0.51590097152957\\
0.856374011410177	0.607724536108427\\
0.699736335667558	0.496566843839096\\
1.03157846449336	0.732058114142605\\
0.996090474103417	0.706874114850478\\
0.6628631093301	0.470399813929054\\
0.867202924758291	0.615409258266453\\
0.838512525384257	0.595049159269942\\
1.06377727294114	0.754907950389912\\
0.587324062436544	0.416793642303336\\
0.613560884533015	0.435412563855242\\
0.330968390869711	0.234871223470033\\
0.577692781916591	0.409958818490221\\
1.10373890890825	0.783266665573545\\
0.984295672326509	0.698503951413876\\
0.461873438551599	0.327767794730674\\
0.57966094008847	0.411355518992638\\
0.570105322312798	0.404574389132815\\
-0.122491367491308	-0.0869258156998028\\
0.223877749945785	0.158874510340125\\
0.468316227323661	0.332339910144707\\
0.280239323502886	0.198871416701082\\
0.373308124360626	0.264917552003929\\
0.57499884723336	0.408047071772246\\
0.467962299014021	0.332088745449228\\
0.476045232254365	0.337824786931611\\
0.297390722983632	0.211042881685036\\
0.440156249286277	0.312356223856269\\
0.430585407303113	0.305564290160374\\
0.228928265293443	0.162458601001295\\
0.252422883946251	0.179131521981592\\
-0.114134374423702	-0.0809952880709822\\
0.184597867966288	0.130999600853936\\
0.626312536635352	0.444461754693889\\
0.134223135324014	0.0952512471921979\\
-0.457304482483235	-0.324525441891669\\
0.407611847981749	0.289261138155159\\
0.382585270127524	0.271501064619304\\
0.181244747487011	0.128620063921902\\
0.599938758019214	0.425745642152681\\
-0.204706661853154	-0.145269776354092\\
-0.157512293014221	-0.111778363107744\\
-0.103138200988369	-0.073191870042274\\
-0.0106671116111196	-0.00756989979741397\\
-0.302314929728088	-0.214537337634895\\
0.251057081084312	0.178162282023784\\
-0.0773684250016563	-0.0549043870634821\\
-0.0892061371178754	-0.0633050017581854\\
-0.0428977475772432	-0.0304423223955004\\
-0.249291946964137	-0.176909657235947\\
-0.331522851943693	-0.23526469594174\\
-0.2620958695517	-0.185995941746241\\
-0.489648264989737	-0.347478158762873\\
-0.164373715710506	-0.116647561459849\\
-0.155640730352938	-0.110450211464921\\
-0.611485921706578	-0.433940069589491\\
-0.275282886109006	-0.195354088319075\\
0.179150530914901	0.127133906264338\\
-0.0127354974848403	-0.00903772673850744\\
-0.860669299136302	-0.610772680617756\\
-0.875431545941815	-0.621248686979772\\
-0.896220275512014	-0.636001377820458\\
-0.911971834302799	-0.647179447952866\\
-0.926310189792137	-0.657354640476478\\
-0.926241881353106	-0.657306165494905\\
-0.96140597001206	-0.682260308408173\\
-0.979023567783496	-0.694762610311707\\
-1.00613840602721	-0.714004614709037\\
-1.00250007616914	-0.711422679368001\\
-1.02479819049491	-0.727246502842524\\
-1.01999173054931	-0.723835605732332\\
-1.02041429902378	-0.724135480818099\\
-1.0448937479462	-0.741507285125975\\
-1.10506954020103	-0.784210946079909\\
-1.0953943062048	-0.777344930748113\\
-1.08676234418982	-0.771219271817126\\
-1.12332224490183	-0.797163950665783\\
-1.14098733375422	-0.809699954543872\\
-1.22303783448055	-0.867926969641237\\
-1.16384012378268	-0.82591740280104\\
-1.17127080473359	-0.831190574421964\\
-1.20481014017381	-0.854991713644101\\
-1.19360199655035	-0.847037871288483\\
-1.19637907048497	-0.849008617651837\\
-1.18942973797817	-0.844077034234202\\
-1.22060102038328	-0.86619768816245\\
-1.26654224098446	-0.898799806636522\\
-1.33783589556494	-0.949393241958139\\
-1.38345047876527	-0.981763562689329\\
-1.37685483134069	-0.977082971361205\\
-1.38097035967162	-0.980003549884639\\
-1.37458316111734	-0.975470883985494\\
-1.39327011523848	-0.988732053022901\\
-1.34234149072737	-0.952590630825034\\
-1.36510749203572	-0.968746489596806\\
-1.3648752976333	-0.968581713186482\\
-1.39613625229296	-0.990766002896023\\
-1.4424228272911	-1.02361320160127\\
-1.40642292361884	-0.998065923814139\\
-1.50488638548282	-1.06794037222995\\
-1.44276678791041	-1.02385729274022\\
-1.5269580149288	-1.08360347104833\\
-1.48553023863156	-1.05420431157268\\
-1.45655566047246	-1.03364254552647\\
-1.54815148725514	-1.09864338698044\\
-1.44708998889338	-1.0269252458505\\
-1.43436614661876	-1.01789579021449\\
-1.58446903522285	-1.12441608056665\\
-1.61645289079822	-1.14711337582961\\
-1.61450708717214	-1.14573253919718\\
-1.59156525609187	-1.1294519030908\\
-1.69637979589305	-1.20383338446387\\
-1.55354210131926	-1.10246882818698\\
-1.53348825876664	-1.08823764881897\\
-1.68709995003658	-1.19724795573397\\
-1.70172803452905	-1.20762875401123\\
-1.64361567491277	-1.16638940494246\\
-1.61283250021678	-1.1445441710694\\
-1.7621842185823	-1.25053139458547\\
-1.57300532154833	-1.11628087330662\\
-1.628497091504	-1.155660524843\\
-1.85724532387398	-1.31799136574948\\
-1.69367839500649	-1.20191633936587\\
-1.76428216181837	-1.25202019686453\\
-1.74702163063237	-1.23977128672921\\
-1.85805200523452	-1.31856382597024\\
-1.7755370941428	-1.2600072427518\\
-1.79060442089781	-1.2706997486436\\
-1.70443845294575	-1.209552198386\\
-1.60554969564432	-1.13937593969923\\
-1.71446225915878	-1.21666557747001\\
-1.90240310575751	-1.35003751810955\\
-1.78491177234962	-1.26665996911727\\
-1.77283304386803	-1.25808831751943\\
-1.70287889290137	-1.20844545893353\\
-1.67950039110047	-1.19185494010348\\
-1.85537282360422	-1.31666254873421\\
-1.8159156306822	-1.2886618108029\\
-1.9599035673211	-1.3908426346406\\
-1.80323632854359	-1.27966396300778\\
-1.82825709575489	-1.29741991302958\\
-1.86852401203757	-1.32599526993247\\
-1.9457728021649	-1.38081475828635\\
-1.84973264567606	-1.31266000490489\\
-1.86918154114166	-1.32646188447756\\
-1.74579817580975	-1.23890306384455\\
-1.65350816038532	-1.17340959245943\\
-1.85434370758438	-1.31593223809023\\
-1.9002116324599	-1.34848234236223\\
-1.9114017942597	-1.35642342394358\\
-1.79851934204462	-1.27631656059514\\
-1.9096313352045	-1.35516702032357\\
-1.89561081916093	-1.34521738208718\\
-1.88243712423655	-1.33586868919122\\
-1.79139037862955	-1.27125750237218\\
-1.88140775553853	-1.33513819923455\\
-2.02483890807373	-1.43692390206597\\
-1.80269510162365	-1.27927988213369\\
-1.87698474717365	-1.33199941796499\\
-2.04970574135177	-1.45457061310231\\
-1.67401230018867	-1.18796032459781\\
-1.94487053936245	-1.38017446883833\\
-1.65562087320737	-1.1749088759531\\
-1.88342340771415	-1.33656860378567\\
-1.89257670064708	-1.3430642244227\\
-1.74090999414445	-1.23543417303823\\
-1.92070422167259	-1.36302487764122\\
-1.9439909205274	-1.3795502486479\\
-1.56985123261036	-1.11404257881013\\
-1.78108977856539	-1.26394769694578\\
-1.82720783368252	-1.29667530576955\\
-1.86437180823027	-1.32304866471209\\
-1.7356263400663	-1.23168463582572\\
-1.61931035181827	-1.14914116876789\\
-1.69467153364418	-1.20262111871442\\
-1.77109541136913	-1.25685520921598\\
-1.80444161554829	-1.28051929312762\\
-1.86017790485282	-1.32007246745413\\
-1.78096624188766	-1.26386002932723\\
-2.01326332428272	-1.42870930634513\\
-1.5080315785488	-1.07017235378418\\
-1.82487813666468	-1.29502203976594\\
-1.82943877904973	-1.29825849281203\\
-1.93625712413403	-1.37406197160668\\
-1.48872412579456	-1.05647084882003\\
-1.52005063851245	-1.07870166170752\\
-1.80376904868879	-1.2800420070619\\
-1.88487560809163	-1.33759915561114\\
-1.66957178136662	-1.18480911705851\\
-1.70685276463558	-1.21126551106502\\
-2.05497304674821	-1.45830854849729\\
-1.74848508815841	-1.24080982717331\\
-1.47010906648733	-1.04326069982977\\
-1.28221578164831	-0.909922511321861\\
-1.45677927224367	-1.0338012312854\\
-1.42121282456821	-1.00856155489797\\
-1.47742770342282	-1.04845435958275\\
-1.76792207509052	-1.254603255873\\
-1.65495290409744	-1.17443485267348\\
-1.35571264959525	-0.962079453713104\\
-1.68573836068283	-1.19628170588594\\
-1.70752687083814	-1.21174388952323\\
-2.02548924844283	-1.43738541513603\\
-1.56507039757349	-1.11064986637809\\
-1.86042759325702	-1.32024965845665\\
-1.58496459483402	-1.12476775370339\\
-1.28954192908386	-0.915121500889982\\
-1.63015658566992	-1.15683818239533\\
-1.72593910322577	-1.22481010269342\\
-1.62496625811125	-1.15315487420777\\
-1.38829995789861	-0.985204988301747\\
-1.60506620076996	-1.13903282828492\\
-1.05404919613431	-0.748004435236614\\
-1.46765413774947	-1.04151856332339\\
-1.40288578663985	-0.995555799848361\\
-1.64529971102207	-1.16758447864819\\
-1.33890203155906	-0.950149823771486\\
-1.46685729986929	-1.04095308851374\\
-1.70176970450703	-1.20765832504877\\
-1.18566444181825	-0.841404997446996\\
-1.46054990863365	-1.03647705775888\\
-1.71302127062384	-1.21564298211183\\
-1.53097088604776	-1.08645119903486\\
-1.70335897781448	-1.20878615047393\\
-1.51039710488886	-1.0718510460127\\
-1.2282472805885	-0.871623845278698\\
-1.29178486667196	-0.916713198194063\\
-1.08337940965752	-0.768818577386962\\
-1.25576502102526	-0.891151768614611\\
-1.11438811588347	-0.790823859372873\\
-1.34706559215569	-0.955943082336645\\
-0.995283016515568	-0.706301103781174\\
-0.843948361595316	-0.598906692305405\\
-1.34297413749091	-0.953039587654376\\
-1.04032513575947	-0.738265176322055\\
-1.20227268655022	-0.853191013476007\\
-0.999937709092217	-0.709604299405027\\
-1.3741461931675	-0.97516079033357\\
-0.904159228894543	-0.641635244212129\\
-1.43614104086768	-1.0191553412631\\
-0.832641111485067	-0.590882519179697\\
-0.862122375997823	-0.611803854438811\\
-1.18493218595062	-0.840885353165947\\
-0.783758283023623	-0.556192893088078\\
-1.25444256002785	-0.890213286146123\\
-1.35532654058605	-0.961805452032319\\
-0.950185904025893	-0.674298005365696\\
-1.06999868614571	-0.759322966963639\\
-1.46968637280049	-1.04296073588723\\
-0.620980040163108	-0.440677556549485\\
-0.812873352903296	-0.576854358873631\\
-1.12906575148895	-0.801239821523364\\
-0.88460540638579	-0.627758903320244\\
-0.832227007161552	-0.590588650665992\\
-0.829163584751956	-0.588414697535734\\
-0.75860603921422	-0.538343640895119\\
-0.787417518736284	-0.558789664237551\\
-0.304531159672673	-0.21611008189971\\
-0.417209759464771	-0.296072281681134\\
-0.660814371898056	-0.468945930475203\\
-1.04446327259695	-0.741201798938347\\
-0.470080440749927	-0.33359188156349\\
-0.537139027796087	-0.381179907544809\\
-0.508872697365994	-0.361120748439982\\
-0.780484385168615	-0.553869576372844\\
-1.07229383683827	-0.760951717219194\\
-0.446799182052607	-0.31707037115643\\
-0.149770434107914	-0.106284364516282\\
-0.271900355750037	-0.192953680709914\\
-0.593707072226662	-0.421323335652258\\
-0.484726814704755	-0.343985658930357\\
-0.200935862320133	-0.142593834106393\\
-0.183703798323403	-0.130365125669342\\
-0.216334709052557	-0.153521602654224\\
-0.26796340450209	-0.190159829146273\\
-0.247665614014887	-0.175755532491417\\
-0.360792573152466	-0.25603590981164\\
-0.570996526912752	-0.405206831143979\\
-0.212845907874728	-0.151045780117433\\
-0.261717111875524	-0.185727156927943\\
-0.33262804393153	-0.236048993782602\\
-0.29385326163088	-0.208532527527904\\
0.0923102414097086	0.0655078247253142\\
-0.216418656223092	-0.153581175638213\\
-0.355959353791408	-0.252606022922297\\
-0.2714567517757	-0.192638877813133\\
0.194133682938539	0.137766677683935\\
0.0621513683304142	0.0441056255606218\\
0.48180346367675	0.341911107246507\\
-0.533642044302123	-0.378698278439614\\
0.167793625663939	0.119074495442303\\
0.540482306910935	0.383552460567298\\
0.226726453283574	0.160896088402282\\
-0.251883580245563	-0.178748805917168\\
0.28062630556405	0.199146037941883\\
-0.27579448560454	-0.195717144135756\\
0.182206074746783	0.129302268373706\\
-0.101267158425716	-0.0718640874866644\\
-0.129683017183271	-0.0920293591453888\\
0.288157798289356	0.204490750487694\\
0.466696992889429	0.331190822850741\\
0.361472156583122	0.256518175176585\\
-0.00629761913635371	-0.00446909599921872\\
0.190581565145192	0.13524592260566\\
-0.0943682154484243	-0.0669682629232719\\
0.344505540791924	0.244477841661493\\
0.507453026687363	0.360113281266642\\
0.232594137860654	0.165060081984671\\
0.861777233202843	0.611558924370668\\
};
\addplot [color=mycolor3,only marks,mark=+,mark options={solid},forget plot]
  table[row sep=crcr]{%
-0.82620492043016	1.16424527876782\\
-0.814919768004873	1.14834282514352\\
-0.806744247689519	1.13682230439466\\
-0.79968934557494	1.12688090090509\\
-0.802712062690807	1.13114035766234\\
-0.783930077687403	1.10467375497651\\
-0.776390806195283	1.09404980319046\\
-0.770665152374303	1.08598150770577\\
-0.757780263085147	1.06782478756116\\
-0.74127163274824	1.04456167879311\\
-0.747535853244442	1.05338889460546\\
-0.733134698407429	1.0330955314056\\
-0.727009892080928	1.02446477083681\\
-0.696126904853387	0.98094606115012\\
-0.675800840354923	0.952303621431961\\
-0.687467824028005	0.968744131919087\\
-0.65921363688206	0.928929791461078\\
-0.644549626174577	0.908266025351939\\
-0.700017737672344	0.986428821695273\\
-0.647243199839613	0.912061670167149\\
-0.659598435736356	0.929472030121622\\
-0.629294459620117	0.886769232971909\\
-0.616276069367207	0.868424390168\\
-0.652699413527432	0.919750284539821\\
-0.603917755198054	0.85100969247112\\
-0.593440508201758	0.836245697427891\\
-0.586541827921725	0.826524433674614\\
-0.578480563366505	0.81516491623864\\
-0.558465602878145	0.786960868214859\\
-0.579013879774193	0.815916438160607\\
-0.589309220474863	0.83042410025226\\
-0.511463998035405	0.720728635533189\\
-0.567112473892814	0.799145592011591\\
-0.541183756928395	0.762608183962005\\
-0.483590691660765	0.681451012575644\\
-0.47184069830327	0.664893529130866\\
-0.532544020785318	0.750433514257618\\
-0.458564444176229	0.646185317880948\\
-0.509111175875474	0.717413160130241\\
-0.439793538686761	0.619734327873611\\
-0.524757065019475	0.739460538592409\\
-0.491617778843574	0.692762369024571\\
-0.424239856656803	0.597816882911644\\
-0.395517689475285	0.557343089170919\\
-0.423952770782757	0.597412336333435\\
-0.419362339671626	0.590943738027509\\
-0.415414464315444	0.585380596086639\\
-0.472228508746855	0.665440011567409\\
-0.450565897241962	0.634914178875416\\
-0.387288856787114	0.54574744338127\\
-0.316960082093146	0.446643768403944\\
-0.275589929432248	0.38834708712507\\
-0.396001088958626	0.558024270742612\\
-0.324328287672011	0.457026662945069\\
-0.36738347157848	0.517697803179247\\
-0.318262618981004	0.448479236076201\\
-0.306850554372015	0.432397944360955\\
-0.208401678794407	0.293668876357396\\
-0.254172670183402	0.358166992153433\\
-0.162268715118825	0.22866068792044\\
-0.201138941961896	0.283434602923413\\
-0.278090240437254	0.391870396186838\\
-0.201329108831648	0.283702576249173\\
-0.288438438933285	0.406452542752226\\
-0.280789999082942	0.395674756557167\\
-0.175357651467831	0.247104940637417\\
-0.14451771809853	0.203646900222904\\
-0.0986855121285289	0.139062524002648\\
-0.263161224882637	0.370833198941553\\
-0.264385074609825	0.372557784733246\\
-0.0224040521413407	0.0315706325220668\\
-0.194310141173139	0.273811809738138\\
-0.166718460183988	0.234931038720505\\
-0.151676993377486	0.213735380982113\\
-0.167468261273394	0.23598761966885\\
-0.118512181431675	0.167001241818394\\
-0.168978104557926	0.238115212802502\\
-0.0685576778787805	0.0966079368688565\\
-0.0237996085376677	0.0335371784787673\\
-0.0761594587199535	0.107319973599351\\
-0.212086950755022	0.29886197116357\\
-0.0881253940768524	0.124181751350635\\
0.11494417531931	-0.161973391985724\\
-0.046479292446835	0.0654962170444625\\
-0.0982151036598781	0.138399648697531\\
-0.0961142710994221	0.135439264016308\\
-0.158118339638138	0.222812193268547\\
-0.000496043452427429	0.000698998799537098\\
-0.142146801619406	0.200305927240394\\
-0.00457570475079006	0.00644784667993653\\
0.191083107270904	-0.269264440323847\\
0.0962902909771705	-0.135687302131998\\
-0.054473019517205	0.0767605641468633\\
-0.068451490814544	0.0964583035452747\\
0.00990746268241353	-0.0139610844323741\\
-0.0123435421820052	0.0173938817759508\\
-0.0601389637310231	0.0847447199387038\\
0.0954065262420418	-0.134441946536828\\
-0.00589411777709615	0.00830568618607708\\
0.146187700063586	-0.206000152509796\\
0.0456025433257971	-0.0642607474901288\\
-0.00517228627786775	0.00728851853884856\\
0.21621160854112	-0.304674225769241\\
0.180950561691654	-0.254986180704672\\
0.0962144495247871	-0.135580430276495\\
0.043722805351486	-0.0616119178743805\\
0.197240710843632	-0.277941417077193\\
0.116558253512548	-0.164247867566262\\
0.0717307011256446	-0.101079197258662\\
0.207481921754238	-0.292372802265907\\
0.0379985845796125	-0.0535456417684375\\
0.153524472537548	-0.216338753143834\\
0.0157654143043343	-0.0222158071415063\\
0.0985938350922311	-0.138933337460614\\
0.345178981461531	-0.486408383149352\\
0.289037669572959	-0.407296947603712\\
0.105772625568582	-0.149049318027522\\
0.170164860027468	-0.239787527283208\\
0.222647176427442	-0.313742895469173\\
0.204200329869808	-0.287748552562454\\
0.199060204806726	-0.280505353945513\\
0.233382774391705	-0.328870945345873\\
0.144399518664691	-0.203480339688831\\
0.350965098478413	-0.494561880245201\\
0.295770381014081	-0.416784336645828\\
0.220274751589765	-0.310399796985737\\
0.150679999439473	-0.212330468645489\\
0.207468819611377	-0.292354339403255\\
0.325047289831937	-0.458039843633561\\
0.268989453233687	-0.379046037153359\\
0.360230404132716	-0.507618070177775\\
0.0891188980419792	-0.125581745797816\\
0.191647356899398	-0.270059551742064\\
0.369259912445727	-0.520341986682102\\
0.425321223654168	-0.59934068940343\\
0.423180901281393	-0.59632465771937\\
0.29059945675568	-0.409497737394429\\
0.415367691700692	-0.585314686534959\\
0.280234340400421	-0.394891751056068\\
0.224189416141277	-0.315916139976874\\
0.243784222882163	-0.343528129051651\\
0.4386665202378	-0.618146191715327\\
0.393942276810052	-0.555123099054359\\
0.372646287583439	-0.525113891531284\\
0.324932731863514	-0.457878414464376\\
0.446966805341643	-0.629842524556755\\
0.457743529969949	-0.645028528003156\\
0.465559870372222	-0.656042910979538\\
0.483716308172697	-0.681628024872879\\
0.346867370517022	-0.488787573756878\\
0.42504225632813	-0.59894758306354\\
0.363127293785798	-0.511700217376748\\
0.423463758317206	-0.59672324523727\\
0.392850271654547	-0.553584301819861\\
0.251667021521664	-0.354636161541628\\
0.39551404505724	-0.557337953645339\\
0.54669120723074	-0.770368998324155\\
0.24954296396011	-0.351643049389035\\
0.25180491336195	-0.354830471597236\\
0.524681605631924	-0.739354205122923\\
0.334960161072059	-0.472008549525964\\
0.43281849915551	-0.609905462609471\\
0.304095489327902	-0.428515649071989\\
0.642062880498413	-0.90476183185057\\
0.487551920039319	-0.687032970905608\\
0.175690641118353	-0.247574172445304\\
0.0294085164552242	-0.0414409616693331\\
0.516093371932102	-0.727252110000122\\
0.314943381273613	-0.443801937824424\\
0.309038676350135	-0.435481332778769\\
0.378626188411836	-0.533540458760314\\
0.497737059743517	-0.701385343447798\\
0.273694466859579	-0.385676099217903\\
0.28135986766624	-0.396477785916127\\
0.724225337799786	-1.02054092083891\\
0.518756642210837	-0.731005052849319\\
0.795981556190523	-1.12165607570891\\
0.491104012173881	-0.692038395583132\\
0.325911325094688	-0.4592573974758\\
0.546574059237759	-0.770203919426206\\
0.329763905856028	-0.464686255198079\\
0.565797304918224	-0.797292323855489\\
0.534368627324833	-0.753004655504432\\
0.499497903787142	-0.703866633880405\\
0.446457909770911	-0.629125415216251\\
0.590505870803284	-0.832110358056771\\
0.413907476677981	-0.583257026983267\\
0.502060427028743	-0.707477609211094\\
0.583593528158635	-0.822369842005294\\
0.16152283205867	-0.227609628056447\\
0.718699372005392	-1.01275401540217\\
0.00709907726089813	-0.0100036528229649\\
0.37117242396782	-0.523036999087773\\
0.431569091287276	-0.608144861606164\\
0.27100336153391	-0.381883932658974\\
0.426685602206204	-0.601263301153108\\
0.150405141150723	-0.211943152548604\\
0.582960047769288	-0.821477174176464\\
0.728462736173367	-1.02651204365446\\
0.582242115675486	-0.820465501369857\\
0.515233801143155	-0.72604084726366\\
0.50042879089352	-0.705178391886037\\
0.704658272349589	-0.992968023357408\\
0.285839445198461	-0.402790175087142\\
0.542235016925106	-0.764089565224285\\
0.278618325184664	-0.392614545923493\\
0.632873393627688	-0.891812481829791\\
0.394087785420741	-0.555328142269214\\
0.336405288343072	-0.47404494819764\\
0.446615257311096	-0.629347141238762\\
0.413501991715061	-0.582685638527429\\
0.295632387809611	-0.416589883753049\\
0.150150188446396	-0.211583886372645\\
0.560909833699558	-0.790405151979917\\
0.354386117716705	-0.499382603770646\\
0.351103427629736	-0.494756806536925\\
0.391036954428666	-0.55102906889043\\
0.365955001505554	-0.515684876970336\\
0.614255544062073	-0.865577170321005\\
0.438507774476718	-0.617922495392183\\
0.496995069222573	-0.700339768748908\\
0.563168037298528	-0.793587295796281\\
0.530095052497551	-0.746982554699787\\
0.351172794399618	-0.494854554604434\\
0.663776477318383	-0.935359510413846\\
0.600042467224296	-0.845548836918523\\
0.0955248107252546	-0.134608626918083\\
0.356611942713401	-0.502519121333888\\
0.0356770094462548	-0.0502741980606111\\
0.800078026647908	-1.12742861018757\\
0.398843171045406	-0.562029185951397\\
0.312401324631136	-0.440219802967585\\
0.404391367560237	-0.569847417770575\\
0.421970349574634	-0.594618810810876\\
0.698420959544357	-0.984178724472685\\
-0.035090009514455	0.0494470280906244\\
0.272536683794263	-0.384044611152048\\
0.218405393904068	-0.307765594736291\\
0.784354359395845	-1.10527163083479\\
0.435103135895848	-0.613124854642688\\
0.37527052538422	-0.528811831829596\\
0.24175485022506	-0.34066844197346\\
0.30229183047375	-0.425974026220892\\
0.331204544217165	-0.466716328329926\\
0.227195856583861	-0.320152660487249\\
0.191156494286534	-0.269367853513913\\
0.715987509188706	-1.00893259845957\\
0.313117977585619	-0.441229673277166\\
0.529230620657518	-0.745764442021336\\
0.747838932030354	-1.05381597756862\\
0.833267274827408	-1.17419718362903\\
0.82391130412105	-1.16101323319032\\
0.815288271684184	-1.14886210148554\\
0.800441668331722	-1.12794103525679\\
0.798986135052589	-1.12588997297621\\
0.779221645598656	-1.09803887579076\\
0.77641891710973	-1.09408941563843\\
0.75887908010922	-1.06937318372883\\
0.761090622902389	-1.07248957554892\\
0.763261032861625	-1.07554800510501\\
0.7318306192159	-1.03125789039871\\
0.756977467496649	-1.06669352950325\\
0.70092406045919	-0.98770596493121\\
0.70810567771376	-0.997825928847907\\
0.721314049717209	-1.01643848411715\\
0.721205501049952	-1.01628552294463\\
0.702278148672702	-0.989614076067604\\
0.698025045966996	-0.983620823518763\\
0.678854178751042	-0.956606228114951\\
0.680462925147517	-0.958873190403019\\
0.663117178373857	-0.934430460411261\\
0.634749504308558	-0.894456200051067\\
0.653153275114337	-0.920389842834262\\
0.620861732524061	-0.874886269719149\\
0.581657167575855	-0.819641222735731\\
0.603819484160077	-0.850871214002673\\
0.577469965837552	-0.813740834424646\\
0.539541796470074	-0.76029441813436\\
0.596031552140249	-0.839896862650797\\
0.583835036279408	-0.822710162768875\\
0.5144899013622	-0.724992582134226\\
0.548164037906893	-0.772444435202748\\
0.559571412545665	-0.788519118054325\\
0.469012375134383	-0.660908002278167\\
0.493351046617824	-0.695204800403412\\
0.528975400010113	-0.745404798273837\\
0.48061680384458	-0.677260363544195\\
0.508848340894297	-0.717042786657071\\
0.49892878798691	-0.703064665304441\\
0.563057626603859	-0.793431710750956\\
0.467464168883484	-0.658726350034763\\
0.441870060035016	-0.62266045445091\\
0.511503015461416	-0.720783616873652\\
0.525537294022658	-0.74055999698445\\
0.465727667199967	-0.656279361598099\\
0.415007368795618	-0.584806938117234\\
0.489833173012972	-0.690247594709674\\
0.487673824742702	-0.687204752714048\\
0.372487392830213	-0.524889985256109\\
0.334587440280399	-0.471483330647138\\
0.405493393230502	-0.571400335396651\\
0.365879250484874	-0.515578132546516\\
0.341453362129291	-0.481158432912082\\
0.279855303465491	-0.394357631794541\\
0.256739485840296	-0.361784016133938\\
0.314679540137136	-0.443430146529198\\
0.256830129351822	-0.36191174628627\\
0.20313261846203	-0.286243988821845\\
0.226526544850544	-0.319209500980122\\
0.338550343990373	-0.477067649767564\\
0.249345033072971	-0.35136413541119\\
0.345017883270895	-0.486181371904045\\
0.234363026264538	-0.330252265629411\\
0.213046418370077	-0.300214003345098\\
0.325040750204299	-0.458030628328237\\
0.201200829284472	-0.283521811340287\\
0.319005581010913	-0.449526179775319\\
0.30620181466265	-0.431483773886899\\
0.130307171684893	-0.183622132563358\\
0.217070462037453	-0.305884477734017\\
0.107898321783794	-0.152044739286169\\
0.118864543849859	-0.16749777272936\\
0.185476702635146	-0.261364184628632\\
0.106786866841696	-0.150478534417495\\
0.118098591345217	-0.166418432041306\\
0.0661704041225709	-0.0932439140567601\\
0.281431932672161	-0.396579336198428\\
-0.00320900778817737	0.0045219679458814\\
0.009793247316107	-0.0138001380403871\\
0.0667240445809172	-0.0940240755806457\\
0.0137726840606916	-0.0194077546588232\\
0.193159813141291	-0.272190826920199\\
0.0479224414433961	-0.0675298280340098\\
0.0958495844798974	-0.135066281310063\\
0.0563358781202961	-0.0793856082249495\\
0.0836427078538156	-0.117864981573168\\
-0.0230433261759833	0.0324714644564543\\
0.0603675102414538	-0.0850667758707955\\
0.0202504764932675	-0.0285359250073342\\
0.129763826343462	-0.182856478386216\\
0.0870805294011104	-0.122709382044179\\
0.0941835339042283	-0.132718568933936\\
0.0133335726357239	-0.0187889815303534\\
-0.0302639634166458	0.0426464133211495\\
-0.0985525175834365	0.138875114962351\\
-0.0399724861077909	0.0563271617982841\\
-0.0173865510762377	0.02450022930634\\
-0.0703869196228495	0.099185609806399\\
-0.00880515165567269	0.0124077646966981\\
-0.0424541766677909	0.0598242318911834\\
0.0496448621388614	-0.0699569742699622\\
-0.0627746079309521	0.0884587335452575\\
-0.187940187331527	0.264835599959329\\
-0.159311426573416	0.224493429723583\\
-0.0345086891890377	0.0486278615284518\\
-0.24612251275493	0.3468231263065\\
-0.229438134811436	0.323312363093206\\
-0.0613326493920013	0.0864268000870564\\
-0.177378415464486	0.249952496836167\\
-0.109465955999277	0.154253768413303\\
-0.174908271448877	0.246471697536942\\
-0.145432550502633	0.204936034771735\\
-0.168444848960738	0.237363776571548\\
0.0267209346850814	-0.0376537603227687\\
-0.267908886752542	0.377523358707663\\
0.0248487199941677	-0.0350155321292097\\
-0.150219320465495	0.211681303641482\\
-0.221332457040863	0.311890260849263\\
-0.233048806300114	0.328400334769394\\
-0.239847445410382	0.337980625675982\\
-0.107762840075324	0.151853825463769\\
-0.212971565002535	0.300108523847693\\
-0.159971506896223	0.225423580804063\\
-0.326914658783338	0.460671243461408\\
-0.222764218447583	0.313907825035673\\
-0.406498337594207	0.572816451062312\\
-0.119232317812183	0.168016020791918\\
-0.153861958356279	0.216814320719616\\
-0.340685187315411	0.480075960660691\\
-0.327741821734655	0.461836838747793\\
-0.357795979282137	0.504187604480081\\
-0.212623632355146	0.299618235140817\\
-0.255403027640957	0.359900748302463\\
-0.408070813298482	0.575032302565262\\
-0.224154867842741	0.315867456299902\\
-0.155217928134065	0.218725083259153\\
-0.508168882284335	0.716085328695728\\
-0.146616998774763	0.206605098069075\\
-0.337840109704138	0.476066824313587\\
-0.469163550339885	0.661121030565606\\
-0.19081418546994	0.268885489618777\\
-0.217710728785151	0.306786708548342\\
-0.222466095295852	0.313487725296115\\
-0.21494700449318	0.302892211094771\\
-0.376299178366813	0.530261356455824\\
-0.188362082863334	0.265430113340807\\
-0.236121097854108	0.332729649263035\\
-0.218285451309223	0.307596577829809\\
-0.200102619738833	0.281974271199832\\
-0.337454337047718	0.475523213421412\\
0.0568151642757696	-0.0800609935075719\\
-0.577823021454393	0.814238342155337\\
-0.196184282824076	0.276452753304184\\
-0.241951480478479	0.34094552316548\\
-0.359977663330519	0.507261920899994\\
-0.458137922783414	0.645584285975182\\
-0.315507106321899	0.444596310031319\\
-0.568653106757893	0.80131657223106\\
-0.485147016556927	0.683644105194413\\
-0.0537721254928464	0.0757729005072529\\
-0.586349393203376	0.826253264613805\\
-0.193428310213928	0.272569179120061\\
-0.125623511397295	0.177022160519679\\
-0.566570690189258	0.798382138413844\\
-0.467900853433841	0.659341703336943\\
-0.251600764105514	0.354542795014719\\
-0.441574843978644	0.622244451239906\\
-0.422282449726077	0.59505860621615\\
-0.538887993274085	0.759373112456614\\
-0.575907476378847	0.811539055022902\\
-0.700762785340451	0.987478704367962\\
-0.242390980288942	0.34156484358672\\
-0.575098916364164	0.810399674033493\\
-0.567748336018867	0.800041616060708\\
-0.45787536591127	0.64521430439889\\
-0.414976024153712	0.584762768867631\\
-0.385219603945009	0.542831559207228\\
-0.344478959182599	0.485421947928349\\
-0.382695222481621	0.539274331299377\\
-0.283991028220998	0.400185481401604\\
-0.317920564266898	0.447997230249159\\
-0.567364445458836	0.799500657321504\\
-0.534354336843144	0.752984518096114\\
-0.312367777062291	0.440172529467178\\
-0.497382987668387	0.700886403376611\\
-0.600052685669185	0.845563236223011\\
-0.774109408651363	1.09083497566544\\
-0.576132003672613	0.811855447282568\\
-0.315380616014834	0.444418066427137\\
-0.469008758656474	0.660902906124383\\
-0.610220037338822	0.859890542786114\\
-0.195068011996154	0.274879762138073\\
-0.453084089642815	0.63846268547617\\
-0.613158251122169	0.864030921813985\\
-0.679750726537751	0.957869596925829\\
-0.456438663056367	0.643189776979055\\
-0.538668645843074	0.759064019391867\\
-0.377674147185478	0.532198891461971\\
-0.47852085169294	0.674306856082875\\
-0.297210051451975	0.418813045830262\\
-0.349716159074904	0.492801939378176\\
-0.160603323218258	0.226313903715229\\
-0.540623323145165	0.761818449635747\\
-0.417608243036162	0.588471955693801\\
-0.574613421753071	0.809715540116041\\
-0.729568230576407	1.02806984923951\\
-0.507481849837317	0.715117198074593\\
-0.337810615308303	0.476025262335134\\
-0.559337227711886	0.788189116888336\\
-0.597089974286437	0.841388336444006\\
-0.503234614386695	0.709132213358419\\
-0.961886607142268	1.3554409001732\\
-0.303002001717952	0.426974762839294\\
-0.423338421231305	0.596546626692759\\
-0.666054681836139	0.938569838491394\\
-0.535904064005005	0.75516831352865\\
-0.705457191267049	0.994093818610825\\
-0.819746960542196	1.15514505541631\\
-0.500760244389926	0.705645459024926\\
-0.287381715861099	0.404963463206297\\
-0.670514322482029	0.944854133632452\\
-0.499957617168739	0.704514438221527\\
-0.398467674274703	0.561500055306355\\
-0.107121868907573	0.150950602017106\\
-0.646755279813908	0.911374118480762\\
-0.193975658083052	0.273340473452391\\
-0.693641691211792	0.977444026512696\\
-0.505979711553255	0.71300046242944\\
-0.487337999963994	0.68673152579003\\
0.13479681573512	-0.189948706951347\\
-0.187738421976213	0.264551282647081\\
-0.445439818136005	0.627690773095414\\
-0.29883370483019	0.42110101426666\\
-0.269464760195345	0.379715815161686\\
-0.391811460707902	0.552120463115615\\
-0.164338481136024	0.231577295234305\\
-0.375389211592457	0.528979078300996\\
-0.375285291194457	0.528832639046342\\
-0.576927358290315	0.812976219908957\\
-0.485137547086245	0.683630761305673\\
-0.456803482027038	0.643703861020079\\
-0.262595295498581	0.370035720498602\\
-0.860327816212833	1.21232950016435\\
-0.45182262053342	0.636685088395097\\
-0.478549049581887	0.67434659109082\\
-0.799101536304874	1.12605259045744\\
-0.410092936837521	0.577881774560884\\
-0.553393948174295	0.779814154489862\\
-0.307502871423895	0.43331715584122\\
-0.443798631148708	0.625378097203165\\
};
\addplot [color=mycolor4,only marks,mark=square,mark options={solid},forget plot]
  table[row sep=crcr]{%
-9.32587340685132e-16	5.51558798633778e-16\\
};
\end{axis}
\end{tikzpicture}%
\end{document}
% This file was created by matlab2tikz.
% Minimal pgfplots version: 1.3
%
%The latest updates can be retrieved from
%  http://www.mathworks.com/matlabcentral/fileexchange/22022-matlab2tikz
%where you can also make suggestions and rate matlab2tikz.
%
\documentclass[tikz]{standalone}
\usepackage{pgfplots}
\usepackage{grffile}
\pgfplotsset{compat=newest}
\usetikzlibrary{plotmarks}
\usepackage{amsmath}

\begin{document}
\definecolor{mycolor1}{rgb}{0.00000,0.44700,0.74100}%
\definecolor{mycolor2}{rgb}{0.85000,0.32500,0.09800}%
\definecolor{mycolor3}{rgb}{0.92900,0.69400,0.12500}%
\definecolor{mycolor4}{rgb}{0.49400,0.18400,0.55600}%
\definecolor{mycolor5}{rgb}{0.46600,0.67400,0.18800}%
\definecolor{mycolor6}{rgb}{0.30100,0.74500,0.93300}%
%
\begin{tikzpicture}

\begin{axis}[%
width=1.5in,
height=1.5in,
scale only axis,
xmin=-3,
xmax=3,
xlabel={$x$},
ymin=-2,
ymax=2.5,
ylabel={$y$},
title={Kernel PCA},
axis x line*=bottom,
axis y line*=left
]
\addplot [color=mycolor1,only marks,mark=o,mark options={solid},forget plot]
  table[row sep=crcr]{%
0.0252096057237872	1.99997607141453\\
0.0542652775567787	1.99668444030418\\
0.0808900749886523	1.99825642146195\\
0.101366643507615	1.99783968345816\\
0.121749309882475	2.01870873559643\\
0.151476321234637	2.00000925448584\\
0.153941035925459	1.985784151057\\
0.195915107091867	2.00343949071584\\
0.229527599596238	1.99999208318555\\
0.253293740908129	1.98187925622405\\
0.261700405104983	2.00111764341199\\
0.271385473595184	1.97747751659113\\
0.315570555217003	1.99585617344955\\
0.381400297805844	1.97713737642657\\
0.373580454000242	1.92852123498518\\
0.414389140828103	1.98220102654256\\
0.455388433761919	1.95143123134093\\
0.482903433473298	1.93988715035403\\
0.426222561526825	2.01718931305916\\
0.504681921807889	1.9610493573584\\
0.464013948059054	1.95836762525045\\
0.550631587344094	1.95562773492631\\
0.542614985661499	1.92235544156257\\
0.509053726746412	1.97571241044708\\
0.632863127742311	1.96021514372426\\
0.643811813851122	1.94578570474198\\
0.739428426070039	1.99902298867195\\
0.681858203372199	1.94108818448797\\
0.821795284580571	1.99798669003619\\
0.692964071256407	1.95009943613109\\
0.684782986797712	1.96610747700205\\
0.804415925100892	1.88606660233353\\
0.77214381332821	1.98107256061489\\
0.8561862693169	1.98577548052889\\
0.793890669288902	1.81953949716239\\
0.94472744757324	1.90168474257471\\
0.885402662894457	1.98820300499133\\
0.948941440096197	1.8765455109341\\
0.864960845375703	1.9240470623056\\
0.971074023417243	1.85248013057106\\
0.884407395425057	1.97099773774496\\
0.964369614528843	1.95752739248484\\
0.900909743147055	1.76973306189219\\
1.18413274323466	1.90986540350391\\
1.1099687289197	1.91748318176565\\
0.992093616502852	1.82410709369686\\
1.09772201804167	1.89070138505298\\
1.08325985003493	2.00081574628242\\
1.00508592510932	1.89944106486253\\
1.16758136262516	1.88068451641741\\
1.16174862882086	1.72753294086775\\
1.45415473356334	1.84738354744748\\
1.22573396992357	1.94041187708243\\
1.34290885014218	1.87170475153035\\
1.26605694328578	1.90839209788116\\
1.36838081426711	1.8769289731854\\
1.3482644008382	1.83847354419544\\
1.36484147657509	1.64164427591067\\
1.29929253776045	1.69210700090283\\
1.41025993360966	1.57612903890478\\
1.32765804067646	1.59986884248708\\
1.48915722514583	1.87752066426332\\
1.37176437757486	1.63157176344676\\
1.30615767530574	1.7695810376244\\
1.36082962109504	1.79217341207432\\
1.26816043231762	1.5030211372027\\
1.54670532739515	1.63534655231132\\
1.70242507488103	1.64874370554204\\
1.47050763499058	1.83265445970779\\
1.40873622219343	1.79141155794274\\
1.76073410124384	1.5284977034257\\
1.6083774527115	1.78461211184732\\
1.37815632311487	1.56277486547138\\
1.88651112110521	1.89165827523777\\
1.41958740058221	1.59376504378723\\
1.77907241600095	1.74514506053466\\
1.62438926635298	1.74230143262092\\
1.62616358251699	1.53079009192941\\
1.61672449680776	1.42925840353382\\
1.77144535440835	1.64999571315109\\
1.57727343328513	1.80020463874339\\
1.93749484153559	1.79318586916688\\
2.03648333513386	1.43316974609015\\
1.84072051057321	1.63627048194093\\
1.54836974045472	1.5384218679536\\
1.7121394430095	1.65018955501604\\
1.73772940487125	1.79972345690608\\
1.73414950154265	1.46321336270143\\
1.71289224533864	1.7482573597082\\
1.80239552071758	1.52028802153443\\
1.9001598650258	1.17510507196256\\
1.89777584244137	1.3742599723717\\
1.70018599898497	1.553477459584\\
1.76670459271109	1.6302998207571\\
1.7689700974124	1.46588083067326\\
1.90930590009211	1.61261528163429\\
1.90706237339255	1.71229195383381\\
1.92988023314953	1.39891532312873\\
1.80588875599868	1.52556044006396\\
1.93938309862924	1.29806402739107\\
2.02163475040791	1.56955330006056\\
1.92695664510478	1.60994667216631\\
2.08758846909191	1.25487131167417\\
1.90511760234139	1.20009212826382\\
1.98097581960974	1.43346340433535\\
1.94673066113215	1.52038050801568\\
2.16410361576076	1.34936564890888\\
2.03641851034698	1.42970383952132\\
1.88099807150121	1.41439043339432\\
1.83503077161051	1.09414055214155\\
1.93249875747306	1.52240931968773\\
2.05030679491722	1.36123573228126\\
1.84652424071819	1.508505203139\\
1.7851378296155	1.28944583310626\\
2.11692587967433	1.0024349005619\\
2.01025120616327	1.04568541162729\\
1.86392190545371	1.33014443617879\\
2.04005058950193	1.31869983122255\\
1.92377489509653	1.12498559312132\\
1.80213374304114	1.07774825150461\\
1.95787308202436	1.19915932111511\\
1.88072644151616	1.07168977152398\\
2.09967342333427	1.41560260960808\\
2.17013282497243	1.02793352331685\\
1.88366201076541	0.941586330839192\\
1.72338061446466	0.987802777870307\\
1.88003407879788	1.24642881458309\\
1.92086095498696	1.15507757415611\\
2.0209406240519	0.976974071862971\\
2.18306336077639	1.21079939581542\\
2.22050931478028	1.04405182371875\\
1.99049886077393	1.45525544822546\\
1.76374196045846	1.07710077973711\\
2.18791002180786	1.00178609065663\\
2.2343891038206	0.915987373321067\\
2.03132528382159	0.776418345485256\\
1.95234652319889	1.00128433025528\\
2.27201241382157	0.963776210910149\\
1.73188484222463	0.866795603804765\\
1.94918890303658	1.13975291322449\\
1.89059216903087	1.05665241422918\\
2.06848643567122	0.769978820981561\\
1.90978191853865	0.752115982916175\\
2.14074279111578	0.961138895055768\\
1.58804370795702	0.670012148006513\\
1.64811388889222	0.454075454265649\\
1.86768493975299	0.587060032085191\\
1.93276798468813	0.616684880312694\\
1.84371533329868	0.515018996654667\\
1.62186306892625	0.647536964126465\\
2.29539795949656	0.959873291571083\\
1.81870879996465	0.752776768843789\\
1.750810621844	0.576752218657265\\
2.26961555688771	1.00978512313861\\
1.91028666052759	1.05392653190567\\
1.61454262374459	0.539269601451133\\
2.0862411429314	0.553696058294345\\
1.51485776055253	0.77781145100211\\
1.71051698523901	0.911868115887879\\
2.1246500991133	0.627586790641797\\
1.67737075522286	0.712156868271379\\
1.87596062247687	0.645743934174888\\
1.65425838595063	0.761151178577534\\
1.95431941589544	0.258004803260987\\
1.68846700630448	0.396720371390254\\
1.15942875612853	0.682059857084357\\
1.67510314653331	1.35794951852743\\
2.14787501562466	0.662265040221223\\
1.78829903050311	0.833288442422855\\
1.65889502054155	0.753967950396459\\
1.35354321534726	0.389833698812968\\
1.38248349849842	0.15799939111346\\
1.3564051515429	0.614193666322743\\
1.7386368293161	0.869202377743102\\
1.654183904986	-0.129071466971251\\
1.65669891543695	0.308059726877272\\
2.13920936132072	0.0630892746374619\\
1.93337174510856	0.562990491371363\\
1.81837693190798	0.831394335583366\\
1.56129991037271	0.18142051068844\\
1.20528265297514	0.388150061797963\\
1.62406405429853	0.185230839564162\\
1.47846860776594	0.148500231124781\\
1.59824571893907	0.30738385730661\\
1.41834822412495	0.292100840301446\\
1.50449450370496	0.0480260558360136\\
1.35903596425556	0.318977738196747\\
1.49873496680872	0.231336734804274\\
1.4538930327726	0.0267626473546722\\
1.13247801468771	0.692953012631509\\
2.16055376039454	0.241981542213923\\
0.73570352634472	0.738575152348107\\
1.22916989516643	0.317365370662156\\
1.13292888674327	0.121099815874434\\
1.30420528581571	0.582852015125134\\
1.42439953609805	0.338288647338872\\
0.814891710269256	0.491134495021951\\
1.45178643231601	0.0266099177314904\\
1.56859872134606	-0.198785050743016\\
1.64764284840506	0.167120282661557\\
1.10418132336813	-0.0765693713188222\\
1.11561313521497	-0.0370879943892937\\
1.03725012300773	-0.525418966245873\\
0.865155686903486	0.23984647704458\\
1.64759738562179	0.251854933990762\\
1.26453745729961	0.538567239131884\\
1.09637029196772	-0.331366853457615\\
0.975372185297645	0.0887052103649253\\
0.908134070444304	0.163207274576677\\
0.325747349608221	-0.483595123297063\\
0.63900320144928	-0.191133294545802\\
0.765572074921706	0.14842786003316\\
0.432012971737715	0.219965363969939\\
0.935841417848618	-0.292809766334486\\
0.931008424738499	0.141342301643102\\
0.82068918643219	0.0700097725538053\\
0.868705646471464	0.0194735516826833\\
0.664969184277619	-0.0719641616437988\\
1.05603525313678	-0.320543112823234\\
0.870716641568264	-0.0796803715903072\\
0.727546794304449	-0.305203334106111\\
0.817214381033212	-0.381777940173187\\
0.417584137862282	-0.595300009129268\\
0.537394122154339	-0.131177120108996\\
1.29171247374217	-0.258219922078456\\
0.735889062336743	-0.517619756084823\\
-0.360156211969547	-0.22645623516825\\
0.765847250483584	0.0194198504627727\\
0.419885739362212	0.453904700200194\\
0.982946233923353	-0.766130712624162\\
1.00040538885305	0.0963942898427855\\
0.109318122566416	-0.352811745680175\\
0.24850253433445	-0.448947947236817\\
0.320455608374698	-0.435132847211648\\
0.689377307721671	-0.759070790628597\\
-0.33578147945411	0.0675875240972314\\
0.525217224667007	0.0267955045132378\\
0.142660428690845	-0.129992148158271\\
0.696771682066403	-0.935898798951477\\
0.393828848107038	-0.410889343396687\\
0.127602038208516	-0.473043655424041\\
-0.0881445419301995	-0.343255304273698\\
0.041819420710484	-0.379292134325631\\
-0.156820260984138	-0.581516653451297\\
0.0644456006617879	-0.204122388305596\\
0.037139223722029	-0.147140231337332\\
0.106125047270562	-1.21019483440756\\
0.039458551265047	-0.403905927954739\\
0.710004611360853	-0.385952702115495\\
0.736726894333947	-0.830175870665623\\
};
\addplot [color=mycolor2,only marks,mark=o,mark options={solid},forget plot]
  table[row sep=crcr]{%
-0.0257785645204607	-1.55229203060528\\
-0.0498967820323308	-1.54958408652859\\
-0.0793085440393965	-1.5521856456645\\
-0.109906706182643	-1.54244264956815\\
-0.125700594951115	-1.55056677981119\\
-0.145396775966016	-1.52266720764417\\
-0.183363593113896	-1.5436718904051\\
-0.218521027885843	-1.53145796039904\\
-0.243424323336384	-1.55381635661645\\
-0.237615583519078	-1.55429285083151\\
-0.291344111490577	-1.52582655959974\\
-0.261390803264229	-1.55785130159408\\
-0.317866778776156	-1.47916361210781\\
-0.335164610444007	-1.50665538033238\\
-0.382132030695385	-1.56797159655555\\
-0.372565345366419	-1.56095262005124\\
-0.382860735728689	-1.52815551424323\\
-0.423673739146399	-1.54810694054304\\
-0.460509695214743	-1.53362834901732\\
-0.540951449544596	-1.59412232640275\\
-0.499099485620386	-1.5276700295708\\
-0.534897840636601	-1.49296894083153\\
-0.550033405271042	-1.54270372283686\\
-0.571116804237858	-1.48924630736613\\
-0.613098443120687	-1.43597200674607\\
-0.583986794029655	-1.46227041459537\\
-0.64150759475729	-1.44726068894559\\
-0.725376984725954	-1.42641639112938\\
-0.740180883636255	-1.55661227096743\\
-0.797991982697432	-1.5717958918167\\
-0.860741470190061	-1.46939771985393\\
-0.831182861976291	-1.51977015144589\\
-0.813388288783237	-1.53131216839832\\
-0.922634280315665	-1.41696222165957\\
-0.847366984321111	-1.41511759758694\\
-0.83450863223717	-1.48147345422914\\
-0.882635034000288	-1.41316424308918\\
-0.885664451610224	-1.47513095591159\\
-0.941870579515756	-1.4940000332642\\
-0.841741837226549	-1.55881980092359\\
-1.03579875681091	-1.49398888862321\\
-0.999273268086965	-1.41383991354962\\
-1.01383153967895	-1.57170925428048\\
-0.958369484820464	-1.56208647491563\\
-0.989204533484056	-1.45724407348306\\
-1.13152065867109	-1.45077249145617\\
-0.955633356091973	-1.48449500691868\\
-0.945068862087621	-1.47242270928704\\
-1.2103581826042	-1.41662823218126\\
-1.28024199072939	-1.38591887283525\\
-1.20739023415321	-1.48445504095233\\
-1.22406254581856	-1.41235220199581\\
-1.35330297397533	-1.45231398373445\\
-1.27206333806534	-1.26414862634002\\
-1.27512531313791	-1.2173438313114\\
-1.37079695011101	-1.40800026862167\\
-1.44327444538879	-1.336862666656\\
-1.43885959666231	-1.2199555601228\\
-1.38468249557781	-1.23107583840802\\
-1.42201041480349	-1.49492121071153\\
-1.32203682868692	-1.2349671750763\\
-1.28185574844467	-1.40916406310554\\
-1.62125883782101	-1.41556579773739\\
-1.47900851684798	-1.26945250906947\\
-1.43761795182563	-1.47737299155127\\
-1.54419734155947	-1.29061526442799\\
-1.53742296443518	-1.53541217210406\\
-1.46771181969171	-1.4588131829972\\
-1.65867378942448	-1.22164404756545\\
-1.48574453111986	-1.28275884247852\\
-1.49602791407209	-1.0587428453439\\
-1.59397425552048	-1.15148551655787\\
-1.71530294333393	-1.37872386909668\\
-1.67650144571949	-1.18446066989326\\
-1.65311099273438	-1.19182891591923\\
-1.63508502899037	-1.06901153934879\\
-1.39644499863987	-1.35575644266041\\
-1.85695837160397	-1.07946274714682\\
-1.80449892357766	-1.06978411520179\\
-1.89155606295175	-1.25218887657974\\
-1.78784018469446	-1.0663938840251\\
-1.63347382282517	-1.33693290630827\\
-1.81897811080574	-1.16084726432498\\
-1.84829975789657	-1.28320320595491\\
-1.79177330776733	-1.15936777948833\\
-1.78391537349941	-1.21164903240922\\
-1.7672180421973	-0.973753765746598\\
-1.59151719035543	-1.02579853468873\\
-1.83246977130268	-1.11179032945606\\
-1.76882434632801	-1.29866098710694\\
-1.82269780507015	-1.24645497234626\\
-1.70271234835196	-1.17635729588757\\
-1.89467430278034	-1.14127816821242\\
-1.92425132278915	-1.06989313512453\\
-1.97936618203155	-0.964315740587365\\
-1.82973940494891	-0.98225250693239\\
-1.89717084682634	-1.07796013628671\\
-2.09360236790814	-1.10506045861807\\
-1.80987679349089	-1.03419428379549\\
-1.91781546405301	-1.03949735243231\\
-1.99843741942447	-1.29184975373077\\
-1.73516344833119	-0.86682375741105\\
-2.13118726690554	-0.882661035237501\\
-1.81330883999235	-0.717737612588012\\
-1.91630863711476	-1.05526290861572\\
-2.13707575361358	-0.763563264474698\\
-1.96872466916745	-0.67944397630352\\
-1.98041341127616	-1.04392024391266\\
-2.11974587620345	-0.896919918170228\\
-1.6776937288212	-0.727110976755328\\
-1.95437459022583	-0.784798165767336\\
-1.97101692439671	-0.859061437356317\\
-2.03119319740258	-0.853007054499041\\
-1.70728194559278	-1.03666056250699\\
-1.88559577878238	-0.538939976418723\\
-1.66819935386158	-1.00495881720213\\
-1.91969127204619	-0.812496071932997\\
-2.02415061280072	-0.735951198636854\\
-2.0916032513645	-0.758994299043236\\
-2.01919022750961	-0.693201570009749\\
-2.11940270456961	-1.04417764723986\\
-1.7193796837629	-0.537385996294986\\
-1.98322618377247	-0.836920625320379\\
-2.15472997804464	-0.604909415709117\\
-2.15739788279318	-0.827476312929507\\
-1.89359900360034	-0.250976564116217\\
-1.63765949653619	-0.678007807274102\\
-1.95600754725664	-0.830549852700785\\
-2.2239373356186	-0.624845361308943\\
-1.99569014331284	-0.490294444669216\\
-2.06302528412929	-0.474400072943434\\
-2.26597321931492	-0.926012479714969\\
-2.00226465601093	-0.648231245229342\\
-1.87655641999738	-0.235550563623002\\
-1.50474718970262	-0.361377221380458\\
-1.6103737405893	-0.582398314384744\\
-1.92775824706411	-0.0597983925607425\\
-1.62242124240915	-0.609171427872175\\
-2.10413872500622	-0.545858597917913\\
-2.12249299464889	-0.280635988466376\\
-1.54490337527676	-0.460516130452826\\
-1.90182562967955	-0.656817163696101\\
-1.92836950634556	-0.665578330585617\\
-2.23881279314758	-0.901815370399758\\
-1.93974611615187	-0.347710676280764\\
-2.04716621633192	-0.822141711474344\\
-1.81946223289969	-0.559360270798851\\
-1.50620392060465	-0.374847089418671\\
-1.82863574562032	-0.642186077553999\\
-2.06176998048506	-0.516609055630509\\
-1.56652763404705	-1.00053803407384\\
-1.96449951956457	0.0617111874950914\\
-1.7996270238056	-0.629902241339237\\
-1.29437721682435	-0.174381078429632\\
-1.82600834129156	-0.301578808781892\\
-1.85940024963483	-0.117293680231677\\
-1.95918335755553	-0.490310334975374\\
-1.90593167852852	0.0838445821010763\\
-1.95038085663778	-0.124631149677822\\
-1.75391837021145	-0.899207590900018\\
-1.77039037523319	0.217526100808311\\
-1.65235475905915	-0.53123004499732\\
-1.8370213222327	-0.805942987950651\\
-2.09591811644858	-0.0553912269795177\\
-2.16963637145989	-0.31676661349549\\
-1.76037440920594	-0.484630417356484\\
-1.66819866477871	-0.0167015603972909\\
-1.7124438566096	-0.0889767583364116\\
-1.62064394314317	0.223232368711154\\
-1.83004903761568	0.153065120049793\\
-1.81352744143549	0.429332678636591\\
-1.5878331126562	-0.381700405108424\\
-1.5687584730913	0.336776403893821\\
-1.41007323782575	0.433812757396804\\
-1.79922604361375	-0.0751474496139841\\
-1.45367770012475	0.0791754261870778\\
-1.58586883070679	-0.0776816206272764\\
-1.34279320848638	0.0084954821648237\\
-1.75521795586068	-0.203208625392691\\
-1.18652679732711	-0.00877192916902404\\
-1.75243814534615	-0.338480277372437\\
-1.39838209715547	0.44129597178331\\
-1.39485325305253	0.373858497298805\\
-1.49567650322448	-0.168034990057268\\
-1.27951781090358	0.377371343930035\\
-1.8528717859086	0.18802778371839\\
-2.12781248944898	0.361707357274624\\
-1.52469444791007	0.370235275558373\\
-1.38375584237211	-0.0822270668949999\\
-1.93707167166853	-0.149379996121342\\
-1.2295766177135	0.651890819878131\\
-1.00631790511102	-0.0692967630940565\\
-1.58052638134334	0.0699006975943077\\
-1.49614019771953	0.468949852135243\\
-1.51035427391087	0.59995877990134\\
-1.28397878801989	0.287452913084822\\
-1.29565122526886	0.45339821213825\\
-1.16346820613333	0.206087060865922\\
-0.781428551577179	0.690874607824667\\
-0.712796351128313	0.35541859779063\\
-1.00890707118453	0.256533842544474\\
-1.20344313602678	-0.282210061581616\\
-1.00908030410666	0.660904401713759\\
-0.953123811043815	0.439969881790494\\
-1.08186265933063	0.681272625317561\\
-1.50842915595659	0.70687810650817\\
-1.57815222688715	0.1868433144969\\
-0.782986337572477	0.391632724820207\\
-0.707484202031367	0.914582586013556\\
-0.86736687024804	0.881112489375594\\
-1.09531822682492	0.520486711347663\\
-1.44498996205859	1.24413307488435\\
-0.502314404249651	0.517058762374403\\
-0.605418759766275	0.698859334664919\\
-0.880765931100263	1.01772606947867\\
-0.802244008718661	0.797686318023879\\
-0.951499345493502	1.05101611976091\\
-1.17891607390623	1.13178697924617\\
-1.07013331151424	0.533116461522449\\
-0.498604163947393	0.486595516730365\\
-0.930607974569119	0.991804810346011\\
-0.830962201311836	0.701143278080427\\
-0.690697476117149	0.585645361419952\\
-0.0131881677094311	0.449136260383922\\
-0.861550476248566	0.99047077648405\\
-0.548311552086026	0.253412284171596\\
-0.963474983199058	1.01748298234106\\
-0.310222568826283	1.08344497375488\\
-0.423563171845147	0.963514984992154\\
0.618223739200304	0.384640233936662\\
-0.719757006489903	0.118530837848968\\
-0.276022732683633	0.979443102179219\\
0.243272061869179	1.03733130847546\\
-0.0411148471233378	0.77328973720547\\
-0.642071581165032	0.606049490839948\\
0.11791128421646	0.66340116681769\\
-0.649560237408564	0.565939767806741\\
-0.19145575665924	0.89081274106155\\
-0.676571056927597	0.973789966063794\\
-0.613197104481082	0.824279235801786\\
-0.167022223949248	1.08087244514927\\
0.205725157179282	0.933904376990845\\
-0.497232199841278	1.70152550898244\\
-0.456496779881341	0.86489382603738\\
-0.286344024648262	1.04227034733798\\
-0.891846291964865	1.29176216117567\\
-0.0639639362571633	1.05503744986388\\
-0.0443174616984984	1.37260526939801\\
-0.0732852737748077	0.831055071467393\\
0.419602061842569	1.46961485521533\\
};
\addplot [color=mycolor1,only marks,mark=o,mark options={solid},forget plot]
  table[row sep=crcr]{%
0.0252096057237872	1.99997607141453\\
0.0542652775567787	1.99668444030418\\
0.0808900749886523	1.99825642146195\\
0.101366643507615	1.99783968345816\\
0.121749309882475	2.01870873559643\\
0.151476321234637	2.00000925448584\\
0.153941035925459	1.985784151057\\
0.195915107091867	2.00343949071584\\
0.229527599596238	1.99999208318555\\
0.253293740908129	1.98187925622405\\
0.261700405104983	2.00111764341199\\
0.271385473595184	1.97747751659113\\
0.315570555217003	1.99585617344955\\
0.381400297805844	1.97713737642657\\
0.373580454000242	1.92852123498518\\
0.414389140828103	1.98220102654256\\
0.455388433761919	1.95143123134093\\
0.482903433473298	1.93988715035403\\
0.426222561526825	2.01718931305916\\
0.504681921807889	1.9610493573584\\
0.464013948059054	1.95836762525045\\
0.550631587344094	1.95562773492631\\
0.542614985661499	1.92235544156257\\
0.509053726746412	1.97571241044708\\
0.632863127742311	1.96021514372426\\
0.643811813851122	1.94578570474198\\
0.739428426070039	1.99902298867195\\
0.681858203372199	1.94108818448797\\
0.821795284580571	1.99798669003619\\
0.692964071256407	1.95009943613109\\
0.684782986797712	1.96610747700205\\
0.804415925100892	1.88606660233353\\
0.77214381332821	1.98107256061489\\
0.8561862693169	1.98577548052889\\
0.793890669288902	1.81953949716239\\
0.94472744757324	1.90168474257471\\
0.885402662894457	1.98820300499133\\
0.948941440096197	1.8765455109341\\
0.864960845375703	1.9240470623056\\
0.971074023417243	1.85248013057106\\
0.884407395425057	1.97099773774496\\
0.964369614528843	1.95752739248484\\
0.900909743147055	1.76973306189219\\
1.18413274323466	1.90986540350391\\
1.1099687289197	1.91748318176565\\
0.992093616502852	1.82410709369686\\
1.09772201804167	1.89070138505298\\
1.08325985003493	2.00081574628242\\
1.00508592510932	1.89944106486253\\
1.16758136262516	1.88068451641741\\
1.16174862882086	1.72753294086775\\
1.45415473356334	1.84738354744748\\
1.22573396992357	1.94041187708243\\
1.34290885014218	1.87170475153035\\
1.26605694328578	1.90839209788116\\
1.36838081426711	1.8769289731854\\
1.3482644008382	1.83847354419544\\
1.36484147657509	1.64164427591067\\
1.29929253776045	1.69210700090283\\
1.41025993360966	1.57612903890478\\
1.32765804067646	1.59986884248708\\
1.48915722514583	1.87752066426332\\
1.37176437757486	1.63157176344676\\
1.30615767530574	1.7695810376244\\
1.36082962109504	1.79217341207432\\
1.26816043231762	1.5030211372027\\
1.54670532739515	1.63534655231132\\
1.70242507488103	1.64874370554204\\
1.47050763499058	1.83265445970779\\
1.40873622219343	1.79141155794274\\
1.76073410124384	1.5284977034257\\
1.6083774527115	1.78461211184732\\
1.37815632311487	1.56277486547138\\
1.88651112110521	1.89165827523777\\
1.41958740058221	1.59376504378723\\
1.77907241600095	1.74514506053466\\
1.62438926635298	1.74230143262092\\
1.62616358251699	1.53079009192941\\
1.61672449680776	1.42925840353382\\
1.77144535440835	1.64999571315109\\
1.57727343328513	1.80020463874339\\
1.93749484153559	1.79318586916688\\
2.03648333513386	1.43316974609015\\
1.84072051057321	1.63627048194093\\
1.54836974045472	1.5384218679536\\
1.7121394430095	1.65018955501604\\
1.73772940487125	1.79972345690608\\
1.73414950154265	1.46321336270143\\
1.71289224533864	1.7482573597082\\
1.80239552071758	1.52028802153443\\
1.9001598650258	1.17510507196256\\
1.89777584244137	1.3742599723717\\
1.70018599898497	1.553477459584\\
1.76670459271109	1.6302998207571\\
1.7689700974124	1.46588083067326\\
1.90930590009211	1.61261528163429\\
1.90706237339255	1.71229195383381\\
1.92988023314953	1.39891532312873\\
1.80588875599868	1.52556044006396\\
1.93938309862924	1.29806402739107\\
2.02163475040791	1.56955330006056\\
1.92695664510478	1.60994667216631\\
2.08758846909191	1.25487131167417\\
1.90511760234139	1.20009212826382\\
1.98097581960974	1.43346340433535\\
1.94673066113215	1.52038050801568\\
2.16410361576076	1.34936564890888\\
2.03641851034698	1.42970383952132\\
1.88099807150121	1.41439043339432\\
1.83503077161051	1.09414055214155\\
1.93249875747306	1.52240931968773\\
2.05030679491722	1.36123573228126\\
1.84652424071819	1.508505203139\\
1.7851378296155	1.28944583310626\\
2.11692587967433	1.0024349005619\\
2.01025120616327	1.04568541162729\\
1.86392190545371	1.33014443617879\\
2.04005058950193	1.31869983122255\\
1.92377489509653	1.12498559312132\\
1.80213374304114	1.07774825150461\\
1.95787308202436	1.19915932111511\\
1.88072644151616	1.07168977152398\\
2.09967342333427	1.41560260960808\\
2.17013282497243	1.02793352331685\\
1.88366201076541	0.941586330839192\\
1.72338061446466	0.987802777870307\\
1.88003407879788	1.24642881458309\\
1.92086095498696	1.15507757415611\\
2.0209406240519	0.976974071862971\\
2.18306336077639	1.21079939581542\\
2.22050931478028	1.04405182371875\\
1.99049886077393	1.45525544822546\\
1.76374196045846	1.07710077973711\\
2.18791002180786	1.00178609065663\\
2.2343891038206	0.915987373321067\\
2.03132528382159	0.776418345485256\\
1.95234652319889	1.00128433025528\\
2.27201241382157	0.963776210910149\\
1.73188484222463	0.866795603804765\\
1.94918890303658	1.13975291322449\\
1.89059216903087	1.05665241422918\\
2.06848643567122	0.769978820981561\\
1.90978191853865	0.752115982916175\\
2.14074279111578	0.961138895055768\\
1.58804370795702	0.670012148006513\\
1.64811388889222	0.454075454265649\\
1.86768493975299	0.587060032085191\\
1.93276798468813	0.616684880312694\\
1.84371533329868	0.515018996654667\\
1.62186306892625	0.647536964126465\\
2.29539795949656	0.959873291571083\\
1.81870879996465	0.752776768843789\\
1.750810621844	0.576752218657265\\
2.26961555688771	1.00978512313861\\
1.91028666052759	1.05392653190567\\
1.61454262374459	0.539269601451133\\
2.0862411429314	0.553696058294345\\
1.51485776055253	0.77781145100211\\
1.71051698523901	0.911868115887879\\
2.1246500991133	0.627586790641797\\
1.67737075522286	0.712156868271379\\
1.87596062247687	0.645743934174888\\
1.65425838595063	0.761151178577534\\
1.95431941589544	0.258004803260987\\
1.68846700630448	0.396720371390254\\
1.15942875612853	0.682059857084357\\
1.67510314653331	1.35794951852743\\
2.14787501562466	0.662265040221223\\
1.78829903050311	0.833288442422855\\
1.65889502054155	0.753967950396459\\
1.35354321534726	0.389833698812968\\
1.38248349849842	0.15799939111346\\
1.3564051515429	0.614193666322743\\
1.7386368293161	0.869202377743102\\
1.654183904986	-0.129071466971251\\
1.65669891543695	0.308059726877272\\
2.13920936132072	0.0630892746374619\\
1.93337174510856	0.562990491371363\\
1.81837693190798	0.831394335583366\\
1.56129991037271	0.18142051068844\\
1.20528265297514	0.388150061797963\\
1.62406405429853	0.185230839564162\\
1.47846860776594	0.148500231124781\\
1.59824571893907	0.30738385730661\\
1.41834822412495	0.292100840301446\\
1.50449450370496	0.0480260558360136\\
1.35903596425556	0.318977738196747\\
1.49873496680872	0.231336734804274\\
1.4538930327726	0.0267626473546722\\
1.13247801468771	0.692953012631509\\
2.16055376039454	0.241981542213923\\
0.73570352634472	0.738575152348107\\
1.22916989516643	0.317365370662156\\
1.13292888674327	0.121099815874434\\
1.30420528581571	0.582852015125134\\
1.42439953609805	0.338288647338872\\
0.814891710269256	0.491134495021951\\
1.45178643231601	0.0266099177314904\\
1.56859872134606	-0.198785050743016\\
1.64764284840506	0.167120282661557\\
1.10418132336813	-0.0765693713188222\\
1.11561313521497	-0.0370879943892937\\
1.03725012300773	-0.525418966245873\\
0.865155686903486	0.23984647704458\\
1.64759738562179	0.251854933990762\\
1.26453745729961	0.538567239131884\\
1.09637029196772	-0.331366853457615\\
0.975372185297645	0.0887052103649253\\
0.908134070444304	0.163207274576677\\
0.325747349608221	-0.483595123297063\\
0.63900320144928	-0.191133294545802\\
0.765572074921706	0.14842786003316\\
0.432012971737715	0.219965363969939\\
0.935841417848618	-0.292809766334486\\
0.931008424738499	0.141342301643102\\
0.82068918643219	0.0700097725538053\\
0.868705646471464	0.0194735516826833\\
0.664969184277619	-0.0719641616437988\\
1.05603525313678	-0.320543112823234\\
0.870716641568264	-0.0796803715903072\\
0.727546794304449	-0.305203334106111\\
0.817214381033212	-0.381777940173187\\
0.417584137862282	-0.595300009129268\\
0.537394122154339	-0.131177120108996\\
1.29171247374217	-0.258219922078456\\
0.735889062336743	-0.517619756084823\\
-0.360156211969547	-0.22645623516825\\
0.765847250483584	0.0194198504627727\\
0.419885739362212	0.453904700200194\\
0.982946233923353	-0.766130712624162\\
1.00040538885305	0.0963942898427855\\
0.109318122566416	-0.352811745680175\\
0.24850253433445	-0.448947947236817\\
0.320455608374698	-0.435132847211648\\
0.689377307721671	-0.759070790628597\\
-0.33578147945411	0.0675875240972314\\
0.525217224667007	0.0267955045132378\\
0.142660428690845	-0.129992148158271\\
0.696771682066403	-0.935898798951477\\
0.393828848107038	-0.410889343396687\\
0.127602038208516	-0.473043655424041\\
-0.0881445419301995	-0.343255304273698\\
0.041819420710484	-0.379292134325631\\
-0.156820260984138	-0.581516653451297\\
0.0644456006617879	-0.204122388305596\\
0.037139223722029	-0.147140231337332\\
0.106125047270562	-1.21019483440756\\
0.039458551265047	-0.403905927954739\\
0.710004611360853	-0.385952702115495\\
0.736726894333947	-0.830175870665623\\
};
\addplot [color=mycolor2,only marks,mark=o,mark options={solid},forget plot]
  table[row sep=crcr]{%
-0.0257785645204607	-1.55229203060528\\
-0.0498967820323308	-1.54958408652859\\
-0.0793085440393965	-1.5521856456645\\
-0.109906706182643	-1.54244264956815\\
-0.125700594951115	-1.55056677981119\\
-0.145396775966016	-1.52266720764417\\
-0.183363593113896	-1.5436718904051\\
-0.218521027885843	-1.53145796039904\\
-0.243424323336384	-1.55381635661645\\
-0.237615583519078	-1.55429285083151\\
-0.291344111490577	-1.52582655959974\\
-0.261390803264229	-1.55785130159408\\
-0.317866778776156	-1.47916361210781\\
-0.335164610444007	-1.50665538033238\\
-0.382132030695385	-1.56797159655555\\
-0.372565345366419	-1.56095262005124\\
-0.382860735728689	-1.52815551424323\\
-0.423673739146399	-1.54810694054304\\
-0.460509695214743	-1.53362834901732\\
-0.540951449544596	-1.59412232640275\\
-0.499099485620386	-1.5276700295708\\
-0.534897840636601	-1.49296894083153\\
-0.550033405271042	-1.54270372283686\\
-0.571116804237858	-1.48924630736613\\
-0.613098443120687	-1.43597200674607\\
-0.583986794029655	-1.46227041459537\\
-0.64150759475729	-1.44726068894559\\
-0.725376984725954	-1.42641639112938\\
-0.740180883636255	-1.55661227096743\\
-0.797991982697432	-1.5717958918167\\
-0.860741470190061	-1.46939771985393\\
-0.831182861976291	-1.51977015144589\\
-0.813388288783237	-1.53131216839832\\
-0.922634280315665	-1.41696222165957\\
-0.847366984321111	-1.41511759758694\\
-0.83450863223717	-1.48147345422914\\
-0.882635034000288	-1.41316424308918\\
-0.885664451610224	-1.47513095591159\\
-0.941870579515756	-1.4940000332642\\
-0.841741837226549	-1.55881980092359\\
-1.03579875681091	-1.49398888862321\\
-0.999273268086965	-1.41383991354962\\
-1.01383153967895	-1.57170925428048\\
-0.958369484820464	-1.56208647491563\\
-0.989204533484056	-1.45724407348306\\
-1.13152065867109	-1.45077249145617\\
-0.955633356091973	-1.48449500691868\\
-0.945068862087621	-1.47242270928704\\
-1.2103581826042	-1.41662823218126\\
-1.28024199072939	-1.38591887283525\\
-1.20739023415321	-1.48445504095233\\
-1.22406254581856	-1.41235220199581\\
-1.35330297397533	-1.45231398373445\\
-1.27206333806534	-1.26414862634002\\
-1.27512531313791	-1.2173438313114\\
-1.37079695011101	-1.40800026862167\\
-1.44327444538879	-1.336862666656\\
-1.43885959666231	-1.2199555601228\\
-1.38468249557781	-1.23107583840802\\
-1.42201041480349	-1.49492121071153\\
-1.32203682868692	-1.2349671750763\\
-1.28185574844467	-1.40916406310554\\
-1.62125883782101	-1.41556579773739\\
-1.47900851684798	-1.26945250906947\\
-1.43761795182563	-1.47737299155127\\
-1.54419734155947	-1.29061526442799\\
-1.53742296443518	-1.53541217210406\\
-1.46771181969171	-1.4588131829972\\
-1.65867378942448	-1.22164404756545\\
-1.48574453111986	-1.28275884247852\\
-1.49602791407209	-1.0587428453439\\
-1.59397425552048	-1.15148551655787\\
-1.71530294333393	-1.37872386909668\\
-1.67650144571949	-1.18446066989326\\
-1.65311099273438	-1.19182891591923\\
-1.63508502899037	-1.06901153934879\\
-1.39644499863987	-1.35575644266041\\
-1.85695837160397	-1.07946274714682\\
-1.80449892357766	-1.06978411520179\\
-1.89155606295175	-1.25218887657974\\
-1.78784018469446	-1.0663938840251\\
-1.63347382282517	-1.33693290630827\\
-1.81897811080574	-1.16084726432498\\
-1.84829975789657	-1.28320320595491\\
-1.79177330776733	-1.15936777948833\\
-1.78391537349941	-1.21164903240922\\
-1.7672180421973	-0.973753765746598\\
-1.59151719035543	-1.02579853468873\\
-1.83246977130268	-1.11179032945606\\
-1.76882434632801	-1.29866098710694\\
-1.82269780507015	-1.24645497234626\\
-1.70271234835196	-1.17635729588757\\
-1.89467430278034	-1.14127816821242\\
-1.92425132278915	-1.06989313512453\\
-1.97936618203155	-0.964315740587365\\
-1.82973940494891	-0.98225250693239\\
-1.89717084682634	-1.07796013628671\\
-2.09360236790814	-1.10506045861807\\
-1.80987679349089	-1.03419428379549\\
-1.91781546405301	-1.03949735243231\\
-1.99843741942447	-1.29184975373077\\
-1.73516344833119	-0.86682375741105\\
-2.13118726690554	-0.882661035237501\\
-1.81330883999235	-0.717737612588012\\
-1.91630863711476	-1.05526290861572\\
-2.13707575361358	-0.763563264474698\\
-1.96872466916745	-0.67944397630352\\
-1.98041341127616	-1.04392024391266\\
-2.11974587620345	-0.896919918170228\\
-1.6776937288212	-0.727110976755328\\
-1.95437459022583	-0.784798165767336\\
-1.97101692439671	-0.859061437356317\\
-2.03119319740258	-0.853007054499041\\
-1.70728194559278	-1.03666056250699\\
-1.88559577878238	-0.538939976418723\\
-1.66819935386158	-1.00495881720213\\
-1.91969127204619	-0.812496071932997\\
-2.02415061280072	-0.735951198636854\\
-2.0916032513645	-0.758994299043236\\
-2.01919022750961	-0.693201570009749\\
-2.11940270456961	-1.04417764723986\\
-1.7193796837629	-0.537385996294986\\
-1.98322618377247	-0.836920625320379\\
-2.15472997804464	-0.604909415709117\\
-2.15739788279318	-0.827476312929507\\
-1.89359900360034	-0.250976564116217\\
-1.63765949653619	-0.678007807274102\\
-1.95600754725664	-0.830549852700785\\
-2.2239373356186	-0.624845361308943\\
-1.99569014331284	-0.490294444669216\\
-2.06302528412929	-0.474400072943434\\
-2.26597321931492	-0.926012479714969\\
-2.00226465601093	-0.648231245229342\\
-1.87655641999738	-0.235550563623002\\
-1.50474718970262	-0.361377221380458\\
-1.6103737405893	-0.582398314384744\\
-1.92775824706411	-0.0597983925607425\\
-1.62242124240915	-0.609171427872175\\
-2.10413872500622	-0.545858597917913\\
-2.12249299464889	-0.280635988466376\\
-1.54490337527676	-0.460516130452826\\
-1.90182562967955	-0.656817163696101\\
-1.92836950634556	-0.665578330585617\\
-2.23881279314758	-0.901815370399758\\
-1.93974611615187	-0.347710676280764\\
-2.04716621633192	-0.822141711474344\\
-1.81946223289969	-0.559360270798851\\
-1.50620392060465	-0.374847089418671\\
-1.82863574562032	-0.642186077553999\\
-2.06176998048506	-0.516609055630509\\
-1.56652763404705	-1.00053803407384\\
-1.96449951956457	0.0617111874950914\\
-1.7996270238056	-0.629902241339237\\
-1.29437721682435	-0.174381078429632\\
-1.82600834129156	-0.301578808781892\\
-1.85940024963483	-0.117293680231677\\
-1.95918335755553	-0.490310334975374\\
-1.90593167852852	0.0838445821010763\\
-1.95038085663778	-0.124631149677822\\
-1.75391837021145	-0.899207590900018\\
-1.77039037523319	0.217526100808311\\
-1.65235475905915	-0.53123004499732\\
-1.8370213222327	-0.805942987950651\\
-2.09591811644858	-0.0553912269795177\\
-2.16963637145989	-0.31676661349549\\
-1.76037440920594	-0.484630417356484\\
-1.66819866477871	-0.0167015603972909\\
-1.7124438566096	-0.0889767583364116\\
-1.62064394314317	0.223232368711154\\
-1.83004903761568	0.153065120049793\\
-1.81352744143549	0.429332678636591\\
-1.5878331126562	-0.381700405108424\\
-1.5687584730913	0.336776403893821\\
-1.41007323782575	0.433812757396804\\
-1.79922604361375	-0.0751474496139841\\
-1.45367770012475	0.0791754261870778\\
-1.58586883070679	-0.0776816206272764\\
-1.34279320848638	0.0084954821648237\\
-1.75521795586068	-0.203208625392691\\
-1.18652679732711	-0.00877192916902404\\
-1.75243814534615	-0.338480277372437\\
-1.39838209715547	0.44129597178331\\
-1.39485325305253	0.373858497298805\\
-1.49567650322448	-0.168034990057268\\
-1.27951781090358	0.377371343930035\\
-1.8528717859086	0.18802778371839\\
-2.12781248944898	0.361707357274624\\
-1.52469444791007	0.370235275558373\\
-1.38375584237211	-0.0822270668949999\\
-1.93707167166853	-0.149379996121342\\
-1.2295766177135	0.651890819878131\\
-1.00631790511102	-0.0692967630940565\\
-1.58052638134334	0.0699006975943077\\
-1.49614019771953	0.468949852135243\\
-1.51035427391087	0.59995877990134\\
-1.28397878801989	0.287452913084822\\
-1.29565122526886	0.45339821213825\\
-1.16346820613333	0.206087060865922\\
-0.781428551577179	0.690874607824667\\
-0.712796351128313	0.35541859779063\\
-1.00890707118453	0.256533842544474\\
-1.20344313602678	-0.282210061581616\\
-1.00908030410666	0.660904401713759\\
-0.953123811043815	0.439969881790494\\
-1.08186265933063	0.681272625317561\\
-1.50842915595659	0.70687810650817\\
-1.57815222688715	0.1868433144969\\
-0.782986337572477	0.391632724820207\\
-0.707484202031367	0.914582586013556\\
-0.86736687024804	0.881112489375594\\
-1.09531822682492	0.520486711347663\\
-1.44498996205859	1.24413307488435\\
-0.502314404249651	0.517058762374403\\
-0.605418759766275	0.698859334664919\\
-0.880765931100263	1.01772606947867\\
-0.802244008718661	0.797686318023879\\
-0.951499345493502	1.05101611976091\\
-1.17891607390623	1.13178697924617\\
-1.07013331151424	0.533116461522449\\
-0.498604163947393	0.486595516730365\\
-0.930607974569119	0.991804810346011\\
-0.830962201311836	0.701143278080427\\
-0.690697476117149	0.585645361419952\\
-0.0131881677094311	0.449136260383922\\
-0.861550476248566	0.99047077648405\\
-0.548311552086026	0.253412284171596\\
-0.963474983199058	1.01748298234106\\
-0.310222568826283	1.08344497375488\\
-0.423563171845147	0.963514984992154\\
0.618223739200304	0.384640233936662\\
-0.719757006489903	0.118530837848968\\
-0.276022732683633	0.979443102179219\\
0.243272061869179	1.03733130847546\\
-0.0411148471233378	0.77328973720547\\
-0.642071581165032	0.606049490839948\\
0.11791128421646	0.66340116681769\\
-0.649560237408564	0.565939767806741\\
-0.19145575665924	0.89081274106155\\
-0.676571056927597	0.973789966063794\\
-0.613197104481082	0.824279235801786\\
-0.167022223949248	1.08087244514927\\
0.205725157179282	0.933904376990845\\
-0.497232199841278	1.70152550898244\\
-0.456496779881341	0.86489382603738\\
-0.286344024648262	1.04227034733798\\
-0.891846291964865	1.29176216117567\\
-0.0639639362571633	1.05503744986388\\
-0.0443174616984984	1.37260526939801\\
-0.0732852737748077	0.831055071467393\\
0.419602061842569	1.46961485521533\\
};
\addplot [color=mycolor3,only marks,mark=+,mark options={solid},forget plot]
  table[row sep=crcr]{%
0.432441396585878	1.89015325029924\\
0.442514664273458	1.89386898808656\\
0.452418984842874	1.89738891480102\\
0.459740708556899	1.8998516169006\\
0.469889666052156	1.90336892579835\\
0.478013136611092	1.9056045905603\\
0.477046079449893	1.90513242963222\\
0.494328969885951	1.91025543419918\\
0.506035218373242	1.91323256895741\\
0.512805396711629	1.91462834134357\\
0.517815268117428	1.9160587503619\\
0.519052504722246	1.91601442345717\\
0.537195703315967	1.9201293138949\\
0.560927832117353	1.92415395259352\\
0.554036551854553	1.92223332931295\\
0.574320359510971	1.9262996808305\\
0.58932065304363	1.92783702731829\\
0.600451205685032	1.92895029072051\\
0.580962721628258	1.92775569530985\\
0.610912924466679	1.9304067252091\\
0.593335715490807	1.92844349166739\\
0.631277193720616	1.93207529486357\\
0.626457527127582	1.93109102379049\\
0.613389452738105	1.93089342306144\\
0.670769557744862	1.93429290785339\\
0.676292501110995	1.93418033385874\\
0.72667951069592	1.93557592171859\\
0.696366521341149	1.93445566301568\\
0.7759933207	1.93393163639726\\
0.702282149346909	1.9346867752609\\
0.697613423645222	1.93498518567529\\
0.771969698425749	1.93130122737548\\
0.746455311727691	1.93476163981239\\
0.799521869313979	1.9322451189367\\
0.768044284264491	1.92947840662908\\
0.875665480832575	1.9226128829694\\
0.819462185173297	1.93082682103373\\
0.882915948052067	1.92082131741389\\
0.811255418411225	1.92974870017839\\
0.906310177733586	1.91676564453267\\
0.820561169784229	1.9302907030553\\
0.883523182066696	1.92350840415022\\
0.855186078501652	1.91983756238452\\
1.11190100995576	1.88051736026223\\
1.02814626596911	1.89899212611594\\
0.931161431586614	1.91184078691288\\
1.02285882586525	1.89879060134193\\
0.979723988294557	1.91108651024\\
0.928971174898103	1.91536761033766\\
1.10352130337089	1.88091061049929\\
1.15695601788969	1.85668224529944\\
1.4512044190748	1.76410741770866\\
1.14773727206535	1.87340439165372\\
1.31588593717111	1.81913977978945\\
1.20774058845535	1.85557693246137\\
1.34261554899036	1.8100863379765\\
1.33938233162525	1.80814813005359\\
1.47430865344578	1.72876557041581\\
1.36123351618589	1.78509721241335\\
1.56638733865265	1.66847522244466\\
1.45667977735365	1.73042788970168\\
1.46998284058881	1.75843000298134\\
1.48879844870355	1.72007474174392\\
1.3254285595287	1.80700545573507\\
1.37929810434021	1.78884823019551\\
1.4455673227687	1.71666013872107\\
1.64915855805101	1.62794465000868\\
1.73689199295177	1.56480902818324\\
1.47547674984785	1.75171839834178\\
1.43402561707891	1.76614619156245\\
1.80905292646196	1.47371567957936\\
1.61765629460808	1.67025574619826\\
1.54136616780931	1.67996954621922\\
1.72803130248532	1.60475618213211\\
1.56446117788491	1.67301384535727\\
1.7342948242437	1.58210148752863\\
1.64922481466898	1.64451968896649\\
1.75061205920195	1.52863560539528\\
1.79145976030138	1.46221592429331\\
1.76779433523629	1.53837250049061\\
1.58666543172494	1.69065825640526\\
1.78071723398855	1.54797594883383\\
1.90176084140615	1.33925522444112\\
1.7991955349248	1.50636797943964\\
1.7019586789959	1.56996884948586\\
1.74103528615719	1.56161493695308\\
1.69156521748855	1.62234731649404\\
1.82329438631695	1.44141650494016\\
1.69982665559057	1.60955023601983\\
1.82585146271903	1.45431879817097\\
1.92602144321645	1.19202218179829\\
1.89066590698042	1.33424473160809\\
1.77565531963179	1.51177971685656\\
1.77336792435672	1.52975150082546\\
1.83362461373669	1.43126276397148\\
1.82851596517969	1.47171495958022\\
1.79645058167468	1.52172163705747\\
1.89098040242809	1.34259494437171\\
1.82519088125033	1.45632495374837\\
1.91305288229699	1.27437759575379\\
1.86802799960483	1.41743141964129\\
1.83426875848029	1.46493364919946\\
1.93713368548312	1.22301678359159\\
1.92358909812647	1.21017277832583\\
1.89241925309816	1.35183044521564\\
1.86435651113062	1.4108121109279\\
1.93257779526577	1.26797715591217\\
1.90245452871976	1.33727157808189\\
1.87758128286318	1.36460161159014\\
1.92549054926266	1.1334579463671\\
1.86054755102847	1.41587395939336\\
1.91732420830966	1.29415725653512\\
1.8426791544065	1.43302702113547\\
1.8884906757359	1.30012684756937\\
1.9530397026418	1.05666561629086\\
1.94487910351921	1.08807472281827\\
1.89436621994207	1.31204784311943\\
1.92305471493774	1.26996261767245\\
1.93340405240327	1.15152678601145\\
1.9223696527511	1.12134502825597\\
1.92999220737516	1.20261395887181\\
1.93185545657882	1.11238161974633\\
1.91433248386303	1.31641204558266\\
1.95687211811552	1.0728895926775\\
1.93003857819316	1.00494575493783\\
1.91256150895384	1.03794607536762\\
1.91332438527846	1.24842714677093\\
1.93052713045292	1.17457932208399\\
1.94503114950776	1.03801964892295\\
1.94994215065674	1.18521303430564\\
1.96001894774043	1.08197992641525\\
1.88922932058644	1.36241331306068\\
1.9173367597267	1.12285259336995\\
1.95781680447819	1.05609690884836\\
1.95772687268781	1.0030644870679\\
1.92540313630581	0.891578715910934\\
1.93962988738052	1.05532842316846\\
1.96204613788753	1.03303784451232\\
1.89974056368262	0.918936709713714\\
1.9349238238462	1.16043437084817\\
1.93339305572771	1.09982665601542\\
1.92849024943515	0.892238543889832\\
1.9041608361607	0.851671501156429\\
1.95362259103874	1.029388278292\\
1.78830341015881	0.6664945577936\\
1.68738074286044	0.501963236700989\\
1.84384748076542	0.706874109805062\\
1.87003423645452	0.750223678824008\\
1.80594811310079	0.641418764352397\\
1.79035170950914	0.660660190706674\\
1.96325320797836	1.03097920267912\\
1.88856919914524	0.83157894770324\\
1.80348559476544	0.654140406340625\\
1.96284103038376	1.06050896785212\\
1.93548988587661	1.09702636490564\\
1.72375545212526	0.555257150323591\\
1.88050117233456	0.745498633409933\\
1.81381498109659	0.750635749948871\\
1.90364431699264	0.96053603802289\\
1.90648763740183	0.803658237549131\\
1.84061783761099	0.749148385339541\\
1.86735849722354	0.75668324681161\\
1.85316788731048	0.790357759944437\\
1.71863806030646	0.50671005049418\\
1.672628966686	0.479215340365358\\
1.49020950229351	0.366767934217262\\
1.84096976290895	1.38510246073056\\
1.91733611557947	0.8310000077252\\
1.90260826644658	0.897614671180299\\
1.85177596999971	0.784571460098693\\
1.437201178045	0.280677466079939\\
1.30668055194447	0.175257200186736\\
1.61022850615104	0.459139199406141\\
1.90134848720552	0.922536364240042\\
1.31452527942837	0.174401042782826\\
1.59540814197591	0.400061936757891\\
1.69347126346789	0.466004671307524\\
1.85135109738377	0.709590105994231\\
1.90704260252125	0.901505155756244\\
1.44549419252515	0.274151585471542\\
1.32149524880663	0.195697971642376\\
1.49084959186967	0.307949052183259\\
1.36652799881635	0.216474154076811\\
1.55725132049407	0.368253823469089\\
1.41815915765396	0.259861191399351\\
1.32241860916452	0.183136792751419\\
1.3921854161263	0.242423922204752\\
1.43517692700852	0.269051569376731\\
1.27760619786013	0.152196249462972\\
1.47247938400855	0.354255245596737\\
1.78418354802572	0.572111717127147\\
1.16424934332094	0.142575244468848\\
1.29564569172531	0.173618811372572\\
1.13593721973088	0.0575585995631585\\
1.54253540275161	0.390526763941684\\
1.45487559420214	0.290213070999872\\
1.12085162581234	0.0691904239575446\\
1.27619099765085	0.151233178894407\\
1.22669077815897	0.114708432958353\\
1.49440635668704	0.309664438379447\\
1.04531614229075	-0.00992404040335263\\
1.06448798772341	0.00447233370964263\\
0.888356432063437	-0.137752091533613\\
1.05194567044447	0.00551443667925386\\
1.55150909028377	0.359235851178306\\
1.47410734196274	0.325457322915245\\
0.961266555333265	-0.0765564026397118\\
1.04736555495398	-0.00432076282823296\\
1.04405951284982	-0.00366047457908104\\
0.670465823323431	-0.316477068763741\\
0.850297105988727	-0.156847418854191\\
0.982112061180654	-0.045344453274686\\
0.885201838271737	-0.0868411962556867\\
0.917419405963132	-0.109874532235512\\
1.04640771876345	-0.00302211904402673\\
0.979192281788462	-0.0519111801333741\\
0.981958814389771	-0.0523290943862709\\
0.887725250020317	-0.122748876735956\\
0.950232000181461	-0.0849271586344384\\
0.953797842722555	-0.0761966572202242\\
0.849339893437074	-0.163412895401956\\
0.857491979520456	-0.159741221652613\\
0.670146702366599	-0.325654360095144\\
0.833521516825306	-0.164148789065628\\
1.0618443308303	-0.000388329463332649\\
0.799090610071847	-0.212127331039898\\
0.114497415163346	-0.483889434243594\\
0.945022668993817	-0.0778823641196132\\
0.941424984222419	-0.00740688575639386\\
0.808493829687837	-0.20982985660848\\
1.06120729982845	0.00548146491953264\\
0.612945226592328	-0.341263371549375\\
0.649054036790826	-0.329831955907729\\
0.683421245966745	-0.301351918465682\\
0.716455869870308	-0.291999186823954\\
-0.889146705315018	0.585748663356458\\
0.8679346301715	-0.126540251399998\\
0.688640077936248	-0.245159311033001\\
0.655649848722695	-0.353083135363693\\
0.717709271659759	-0.272327240218075\\
0.578767162148951	-0.389196890798899\\
0.476806704160302	-0.432151194200843\\
0.563632062992785	-0.382064583901467\\
-0.0956335320974561	-1.18754801689226\\
0.62995848353455	-0.293868726041073\\
0.625960949005879	-0.276962209658603\\
-0.321141268750437	-1.41868867052886\\
0.552351085499616	-0.396254499711838\\
0.824269357453641	-0.186568874199231\\
0.709988752254665	-0.299859723368208\\
-0.435111492670318	-1.4751260063262\\
-0.441515858742574	-1.47710472090963\\
-0.450031906280678	-1.4796818858698\\
-0.457701936375356	-1.48173084683257\\
-0.463008948020021	-1.48326540941372\\
-0.466033206974403	-1.4837057534208\\
-0.479140502494352	-1.48721715719647\\
-0.488676326304259	-1.48922758085975\\
-0.497952910089979	-1.49147679632099\\
-0.496217611399825	-1.49112646199014\\
-0.511059719640534	-1.49364115898218\\
-0.503742308886225	-1.49268150524193\\
-0.516592262169662	-1.4939564109664\\
-0.524311946438354	-1.49565658716649\\
-0.543462825755627	-1.49936310377956\\
-0.539896186875108	-1.49877364536786\\
-0.541954619603096	-1.49861062177223\\
-0.557364958177438	-1.50086634113094\\
-0.570624971765945	-1.5021147695904\\
-0.603525045596546	-1.50575289917437\\
-0.585642308753342	-1.50341785639627\\
-0.599875933399868	-1.50392464796252\\
-0.607083307245019	-1.50521290054051\\
-0.615941184272026	-1.50480267487679\\
-0.635620879076816	-1.50452932965315\\
-0.621632378606659	-1.50456200131603\\
-0.650286400800483	-1.50510009430736\\
-0.698485145840129	-1.5043734626734\\
-0.701605374443176	-1.50723796346416\\
-0.735799416574136	-1.50649009043689\\
-0.789748652964396	-1.50058410162582\\
-0.762618276053556	-1.50390502800243\\
-0.749263400805111	-1.50495336580159\\
-0.849525901999144	-1.49281604617866\\
-0.784923852833605	-1.49928663009378\\
-0.768641390487698	-1.50251257350269\\
-0.814312820871128	-1.4965266166967\\
-0.808984906453868	-1.49908376753226\\
-0.854402827952875	-1.49497833693181\\
-0.766226048496672	-1.50461964169345\\
-0.946873358241786	-1.48232889505828\\
-0.927340235675622	-1.48196510309381\\
-0.906099115570564	-1.4910584532939\\
-0.857204261199727	-1.49675964257268\\
-0.906878346603976	-1.48687948948049\\
-1.07319682208701	-1.45660403953751\\
-0.868679856265977	-1.49296516238914\\
-0.861039783372317	-1.49347042561881\\
-1.19249982198847	-1.42530123285896\\
-1.30260200716703	-1.38964141966903\\
-1.15707082300435	-1.43902164835282\\
-1.21320656897906	-1.41923392327645\\
-1.35799864175156	-1.37634012972266\\
-1.36264406035168	-1.35528886353523\\
-1.39655465163501	-1.33549618463287\\
-1.40219807508425	-1.35586016189347\\
-1.51685305530584	-1.29789501980994\\
-1.5768921728239	-1.249269411616\\
-1.51885609677574	-1.28204027894299\\
-1.41271384890326	-1.35948214408007\\
-1.44585446923741	-1.31708920073781\\
-1.29210471369272	-1.39509754500126\\
-1.61810547305477	-1.25259696329115\\
-1.58408277377378	-1.25316726883054\\
-1.43781921095243	-1.34797532444138\\
-1.62265770307317	-1.23357615921031\\
-1.50272538999529	-1.32481969272222\\
-1.47683740681881	-1.32969861004947\\
-1.7184871328183	-1.15565842982232\\
-1.58265000130517	-1.25597024418921\\
-1.70511862426627	-1.12919483688901\\
-1.71578430405215	-1.14384484034146\\
-1.68560917559899	-1.20416074287809\\
-1.74081604773011	-1.13040575396941\\
-1.72766202430498	-1.14270979499441\\
-1.76799311489495	-1.07949171734668\\
-1.45938992026089	-1.32674889534433\\
-1.83123198140553	-1.0175990438061\\
-1.82133674310568	-1.02621078357501\\
-1.79265459291844	-1.09555034661006\\
-1.81803818142157	-1.02891375721462\\
-1.65854896795286	-1.21724334147637\\
-1.79857952817181	-1.07218110561438\\
-1.77074923517152	-1.12169303815008\\
-1.79100167393557	-1.07947865684539\\
-1.77192527253494	-1.10823416839941\\
-1.83941925723965	-0.975793320065125\\
-1.76751253701184	-1.06726605030282\\
-1.81648083467609	-1.04223574240329\\
-1.73762130035159	-1.15306940325576\\
-1.77372415209903	-1.11283379302143\\
-1.75450416467157	-1.11701654781248\\
-1.82358591330236	-1.04118260237455\\
-1.84766336400745	-0.995407492937869\\
-1.87915110009507	-0.921246321088837\\
-1.850503323008	-0.965441009204562\\
-1.84032344629747	-1.00649836425436\\
-1.86830419325153	-0.977083524226158\\
-1.83261958785574	-1.00339103846596\\
-1.85356895408887	-0.979589182886264\\
-1.80949456473581	-1.08463134159527\\
-1.86053668398226	-0.904972039008513\\
-1.90827485310713	-0.845014316501353\\
-1.89814475762076	-0.764451622962312\\
-1.84960588991906	-0.989029494613671\\
-1.92066845058859	-0.765500663318145\\
-1.91478849201885	-0.716706741355289\\
-1.86334972399438	-0.968085898526927\\
-1.90546236205229	-0.855551815144628\\
-1.87971647378369	-0.79180032101504\\
-1.90451563750119	-0.801725818077717\\
-1.896132971947	-0.853437651213841\\
-1.90312323825509	-0.839989654993196\\
-1.80457920588541	-1.03469268925039\\
-1.90970358648234	-0.595606187172249\\
-1.80311771533564	-1.02621274944299\\
-1.89741369921539	-0.827667352081065\\
-1.9150810499611	-0.756525573327172\\
-1.91802571685291	-0.766781072247891\\
-1.9175698650659	-0.723883394228014\\
-1.8828400834599	-0.940570609683461\\
-1.89544500930248	-0.590591877756199\\
-1.9006124176406	-0.835943363804789\\
-1.92717652405615	-0.648368623121216\\
-1.91645907019926	-0.806348821946609\\
-1.86481093530631	-0.332844697151086\\
-1.88097700635565	-0.74664891388458\\
-1.89870946184436	-0.835494064901433\\
-1.93042057421424	-0.660973294338449\\
-1.91392796145808	-0.554535821531009\\
-1.91660897429422	-0.543601494184056\\
-1.91376030913567	-0.852566746875054\\
-1.91832189462101	-0.688863471209518\\
-1.85861812118215	-0.316292171963244\\
-1.8333392252421	-0.348841077553766\\
-1.88410966419853	-0.640062451122212\\
-1.81364038674894	-0.191220176311921\\
-1.88490906501074	-0.671033806978046\\
-1.92330788471153	-0.602914746695952\\
-1.89353914654792	-0.392512760805153\\
-1.86553743946167	-0.483795498697332\\
-1.91088513173865	-0.702327812061596\\
-1.91251305210392	-0.708056649224653\\
-1.91448735191656	-0.842181504692221\\
-1.89092248104345	-0.424316877584032\\
-1.90843217978156	-0.816540358946879\\
-1.90538210319375	-0.614666243411379\\
-1.83789802406318	-0.365966911566761\\
-1.90547647830007	-0.693904065720471\\
-1.91961558436773	-0.578674776507938\\
-1.76748753046468	-1.0590523340919\\
-1.77914390286555	-0.119466471641108\\
-1.90338295208269	-0.683961760442436\\
-1.66272806765452	-0.0267907345954278\\
-1.86915525521836	-0.364677143589636\\
-1.82043924210739	-0.216217689686652\\
-1.9115891631019	-0.553143509501509\\
-1.75771675064436	-0.0887481269142645\\
-1.83753074331425	-0.243920205669491\\
-1.85621861179971	-0.925697646733118\\
-1.65577352510011	0.0410751754864592\\
-1.88791191360303	-0.581167034166438\\
-1.88873011213448	-0.83580072966983\\
-1.83817446928321	-0.226356259930454\\
-1.90313257953907	-0.425404754079141\\
-1.89527978480948	-0.536780031425992\\
-1.73419158333849	-0.0772237906986639\\
-1.77859156945341	-0.149817493611021\\
-1.58434292074839	0.110654294935027\\
-1.70776636029244	-0.0208002922074123\\
-1.55899640406314	0.146094554319714\\
-1.85525088268025	-0.396586179632633\\
-1.478309644254	0.210557940411883\\
-1.29014882835291	0.365153220372937\\
-1.79367998232646	-0.166917491370556\\
-1.58486822995262	0.0975114052612325\\
-1.73611655440691	-0.0921641046749929\\
-1.56555548953928	0.108654696444632\\
-1.83020608469022	-0.259268012483259\\
-1.46376298838188	0.201899279900118\\
-1.86880733671868	-0.385789295990209\\
-1.27563334653415	0.376735177084684\\
-1.32383427715771	0.337223368244324\\
-1.74955346164965	-0.133128139877703\\
-1.23170585535955	0.410258689351522\\
-1.69926938910601	-0.00742264292629534\\
-1.70653767293356	0.00153676890274784\\
-1.42395645938383	0.256723312002912\\
-1.65179847698148	0.00590707119980908\\
-1.84283289697276	-0.2596746610071\\
-1.03465101920685	0.571188375210371\\
-1.36101998273596	0.286223927153005\\
-1.6558762595785	0.0248353561844379\\
-1.32916612243557	0.335060207877323\\
-1.24456547141507	0.403358080070203\\
-1.30311063142982	0.35154623684854\\
-1.19021210857576	0.444321470206518\\
-1.2692777227711	0.376130307106637\\
-0.842360381585732	0.727112342084179\\
-0.921941378068808	0.648773853011431\\
-1.11913635592928	0.495828479739184\\
-1.68104576223272	-0.0852355032757928\\
-0.930328855731985	0.65628057491596\\
-0.99164659684548	0.60137406511327\\
-0.9526896913874	0.63848603200673\\
-1.17193068103614	0.461388282291668\\
-1.58450748145331	0.107644005121465\\
-0.936346915840671	0.641179736502912\\
-0.767971394010862	0.793386608651071\\
-0.820520265842122	0.750614751922969\\
-1.02631051588963	0.575960063276149\\
-0.903866864514661	0.684674272092515\\
-0.800666081416434	0.743552699734345\\
-0.787815751335572	0.768022632559758\\
-0.792523164613916	0.776613967766307\\
-0.821689032610292	0.747213954667822\\
-0.804451678568495	0.767408685991273\\
-0.850257319518033	0.729853253210323\\
-1.00761627122878	0.591204868310725\\
-0.807364621489104	0.735887137047249\\
-0.812356542605213	0.759771710413007\\
-0.855231267652953	0.717496290517966\\
-0.841848207876687	0.722146357923605\\
-0.582465270327566	0.782386986142206\\
-0.79340553862898	0.77528970228663\\
-0.897268583966557	0.651157953661213\\
-0.815615510917989	0.757548901090002\\
-0.598843847800416	0.929956648836995\\
-0.673833269351898	0.866337005911276\\
0.997830338156933	-0.0108155136403219\\
-1.03084993260291	0.551341497955641\\
-0.615552402161662	0.90920148251701\\
0.112153855718498	1.61588468685705\\
-0.554112056648502	0.914755677027864\\
-0.821342949909268	0.738130160510277\\
-0.440146980237786	0.918330064002035\\
-0.834143752272728	0.726315693648677\\
-0.603387867427638	0.907635741032812\\
-0.746414820150035	0.812328380501904\\
-0.761650670567732	0.794251856564254\\
-0.526621038021826	0.984637885360713\\
-0.1754063827064	1.25038192831993\\
-0.37263113303824	1.15050919790937\\
-0.707004670037002	0.835990616924554\\
-0.601396718826955	0.924889171181134\\
-0.734312312738539	0.829780659485467\\
-0.466011018615715	1.0284506117015\\
0.0813957087330467	1.61536202862116\\
-0.558508983343424	0.925357925940092\\
0.501161635929118	1.89916578354788\\
};
\addplot [color=mycolor4,solid,forget plot,mark=square]
  table[row sep=crcr]{%
-9.32587340685132e-16	5.51558798633778e-16\\
};

\end{axis}
\end{tikzpicture}%
\end{document}
\caption{Linear PCA showing two principal components(left) and kernel PCA (right) .
		 The mean sample value $\bar{\mathbf{x}}$ is indicated by a square at $(0,0)$ in both plots.}
\label{fig:PCA}
\end{figure}
Figure~\ref{fig:PCA} shows a classical coordinate system consisting of the principal axes centered at the mean. The data set under consideration is nonlinear, therefore the linear approach fails to capture essential information present in the data. 

On the right of figure~\ref{fig:PCA} the result of using a non-linear approach is shown. The kernel-PCA is able to track the nonlinear lines despite the presence of progressively increasing noise.
The linear PCA can also be formulated as an optimization problem \footnote{Support Vector Machines: Methods and Applications, Suykens et al., page 202}:
\begin{equation}
\max\limits_{\mathbf{w}} \text{Var}(\mathbf{w}^T\mathbf{x}) = \text{Cov}(\mathbf{w}^T\mathbf{x}, \mathbf{w}^T\mathbf{x}) \simeq \mathbf{w}^T C \mathbf{w}
\end{equation} 
which is valid, when considering zero mean data and using $\mathbf{C} = \frac{1}{N}\sum_{k=1}^{N} \mathbf{x}_k \mathbf{x}_k^T $. Taking into account the constraint $\mathbf{w}^T \mathbf{w} = 1$ leads to the Lagrangian:
\begin{equation}
\mathcal{L}(w; \lambda) = \frac{1}{2}\mathbf{w}^T\mathbf{C}\mathbf{w} - \lambda (\mathbf{w}^T \mathbf{w} - 1) 
\end{equation}
The solution of the problem will be found where the gradient of the Lagrangian is zero $\bigtriangledown \mathcal{L} = 0$. Using this equation yields the eigenvalue problem,
\begin{equation}
\mathbf{C}\mathbf{w} = \lambda \mathbf{w}.
\end{equation}
Which can be solved as discussed before. The kernel case is an extension of the linear case \footnote{Support Vector Machines: Methods and Applications, Suykens et al., page 211} the mean equivalent is defined by,
\begin{equation}
\hat{\mu} = \frac{1}{N} \sum\limits_{k=1}^{N} \varphi(\mathbf{x}_k).
\end{equation}
The ls-svm approach to the problem looks at:
\begin{align}
\max\limits_{\mathbf{w},\mathbf{e}} J_p(\mathbf{w},\mathbf{e}) &= \gamma \frac{1}{2}\sum\limits_{k=1}^{N} e_k^2 - \frac{1}{2}\mathbf{w}^T\mathbf{w} \\
\text{such that } e_k &= \mathbf{w}^T (\varphi(\mathbf{x_k} - \hat{\mu})), \;\; k = 1,\dots,N
\end{align}
Trough an analysis of the optimality conditions once more an equivalent eigenvalue problem can be found,
\begin{equation}
\Omega_c \alpha = \lambda \alpha.
\end{equation} 
With $\lambda = \frac{1}{\gamma}$ and $\alpha_k = \gamma e_k$. And the kernel matrix defined by,
\begin{equation}
\Omega_{c; k,l} = (\varphi(\mathbf{x}_k) - \hat{\mu})^T (\varphi(\mathbf{x}_l) - \hat{\mu}).
\end{equation}
Which can be found by using the Kernel trick, after that the analysis can proceed like it did in the linear case. The projected space is given by:
\begin{equation}
z(\mathbf{x}) = \mathbf{w}^T(\varphi(\mathbf{x}) - \hat{\mu}).
\end{equation}
This space is also called target space an is interpreted as en error where the target is zero. A plot of the kernel matrix eigenvalues associated with the toy-problem under consideration is shown in figure~\ref{fig:toyEig}. The plots in~\ref{fig:toyProj} show the  projections of the kernel matrix onto the first six principal components. It can be observed that the projection quality decreases alongside the relevance of the associated eigenvalue.
\begin{figure}
\centering
% This file was created by matlab2tikz.
% Minimal pgfplots version: 1.3
%
%The latest updates can be retrieved from
%  http://www.mathworks.com/matlabcentral/fileexchange/22022-matlab2tikz
%where you can also make suggestions and rate matlab2tikz.
%
\documentclass[tikz]{standalone}
\usepackage{pgfplots}
\usepackage{grffile}
\pgfplotsset{compat=newest}
\usetikzlibrary{plotmarks}
\usepackage{amsmath}

\begin{document}
\definecolor{mycolor1}{rgb}{0.00000,0.44700,0.74100}%
%
\begin{tikzpicture}

\begin{axis}[%
width=2.0in,
height=1.5in,
scale only axis,
xmin=1,
xmax=6,
ymin=20,
ymax=100
]
\addplot [color=mycolor1,only marks,mark=*,mark options={solid},forget plot]
  table[row sep=crcr]{%
1	96.2157379309088\\
2	53.49452394308\\
3	50.5770078222607\\
4	49.2099481009167\\
5	35.0250267496617\\
6	23.409325836359\\
};
\end{axis}
\end{tikzpicture}%
\end{document}
\caption{Plot of the first six eigenvalues of the centered kernel matrix.}
\label{fig:toyEig}
\end{figure}
\begin{figure}
\centering
% This file was created by matlab2tikz.
% Minimal pgfplots version: 1.3
%
%The latest updates can be retrieved from
%  http://www.mathworks.com/matlabcentral/fileexchange/22022-matlab2tikz
%where you can also make suggestions and rate matlab2tikz.
%
\documentclass[tikz]{standalone}
\usepackage{pgfplots}
\usepackage{grffile}
\pgfplotsset{compat=newest}
\usetikzlibrary{plotmarks}
\usepackage{amsmath}

\begin{document}
\definecolor{mycolor1}{rgb}{0.00000,0.44700,0.74100}%
\definecolor{mycolor2}{rgb}{0.85000,0.32500,0.09800}%
%
\begin{tikzpicture}

\begin{axis}[%
width=1.5in,
height=1.5in,
scale only axis,
xmin=-3,
xmax=3,
ymin=-3.2,
ymax=3.2,
title={first component}
]
\addplot [color=mycolor1,only marks,mark=o,mark options={solid},forget plot]
  table[row sep=crcr]{%
0.0259388858243524	1.99651291936462\\
0.0479365430139229	2.00136242712261\\
0.0808011428276144	2.00015957097738\\
0.0960018408277835	2.00397357604688\\
0.123924398192739	1.99393565592043\\
0.144716297235007	1.99609680237626\\
0.172432773795358	2.00782480916528\\
0.211096968463268	2.00289657674787\\
0.241323554557747	2.00054011148902\\
0.248450748340069	1.98323290473631\\
0.278037158020472	1.99775137204772\\
0.295123845636812	2.024317082564\\
0.278089869880903	1.98953669925669\\
0.349224330459094	2.01774494803247\\
0.389769425315202	1.96915513448386\\
0.443104169526806	2.00181328676435\\
0.445366816453369	2.00702160442149\\
0.447621894481189	1.9639008226548\\
0.459268204218018	1.96874477518371\\
0.525152271376063	1.93928073190811\\
0.479912634035923	1.9531725626356\\
0.541892480429807	1.98767075962513\\
0.596086730914084	1.95159718447859\\
0.642409661124227	1.88862872560156\\
0.59672705695044	2.06269182867749\\
0.577160609928358	1.93769298667604\\
0.61270638646484	1.9863001413635\\
0.738131551064678	1.99453476973758\\
0.731223036941714	1.9720505976831\\
0.728694499517447	2.00387465318476\\
0.792620470002546	2.01820212338234\\
0.76404643370489	1.92712620890764\\
0.783621400051233	1.97170029129066\\
0.844923525499164	2.0002383443526\\
0.933088804435648	1.86929978545337\\
0.826522508566799	1.96489496451136\\
0.857943301337058	1.93574991230876\\
0.862415931324452	1.91824418132958\\
0.85957751145223	1.94915043906307\\
1.10549276410824	1.89209624896601\\
0.992380983980335	1.9389778055659\\
1.05163144833557	1.9077966876626\\
0.971362952394099	1.94526987131734\\
1.10648177744783	1.98917932950755\\
1.12898410068108	1.88302810812235\\
1.14935856334692	1.80741423289017\\
1.2475966096538	1.80328803923413\\
1.02094574055669	1.89016584759305\\
1.18313200649952	1.80836196844854\\
1.11713521982497	1.97114350530034\\
1.21095637410431	1.6606506602979\\
1.09752975594461	1.89186114518647\\
1.1286092446374	1.8521928055118\\
1.17641657796315	1.85179775653125\\
1.2027986521108	1.79291729003239\\
1.40862693119303	1.68188184454732\\
1.29643649567147	1.6835431657974\\
1.46182858553516	1.76685142278392\\
1.44694035368565	1.75403257372017\\
1.49092139521946	1.61038316326389\\
1.3379059086772	1.66077993928176\\
1.46168112951844	1.88866965377674\\
1.41023026436323	1.73928409103319\\
1.4714192456311	1.78600061226619\\
1.35577541010529	1.79014999256957\\
1.4051660334094	1.67552928758717\\
1.61761103881626	1.92386710385478\\
1.24305770227106	1.88248740497196\\
1.49729193411573	1.81459528264337\\
1.63186681467879	1.63329561110855\\
1.53796361955124	1.77930541551158\\
1.46647340378635	1.89266145452761\\
1.59115626629009	1.92971274617182\\
1.55598375769603	1.68322562636936\\
1.60162857304111	1.74522783244834\\
1.55820638402395	1.58511515395533\\
1.6445086680066	1.69602995640505\\
1.43117976594691	1.39522816594688\\
1.66046656580406	1.5020485139169\\
1.89457571586182	1.8060245580165\\
1.7690851756095	1.803354010806\\
1.51808575625609	1.62902388762982\\
1.71974982125714	1.56603639813406\\
1.63851255324909	1.37209586707968\\
1.73596507613967	1.53438459796941\\
1.8856967379105	1.45170562832068\\
1.83646220685143	1.51078810063935\\
1.6807272288616	1.41341108157641\\
1.6794592820606	1.42428516993435\\
1.84697126074185	1.69896546443275\\
1.65822816966463	1.41270723759625\\
1.78371507953502	1.42947586399192\\
1.87899798012913	1.57760927136629\\
1.89142792776872	1.58934082453833\\
1.89301329213783	1.04572276609604\\
1.75219375143258	1.4487568304138\\
2.1047354401674	1.41876136406707\\
2.01123547254846	1.3110508611222\\
1.96781276802938	1.52540722041352\\
1.51166361010094	1.32992221483295\\
2.22158609914819	1.4805071317003\\
1.97788676892182	1.64185849242464\\
1.94851336550087	1.45074289017143\\
2.05150009067797	1.21139868557201\\
1.93322927547617	1.22986741742318\\
1.99640422827958	0.927539227571993\\
2.02784448795865	1.33792744894102\\
2.05296048996672	1.40894341042615\\
1.79681210188576	1.18409186751766\\
2.04186829813144	1.41141478615\\
2.06457591322178	1.27562881170451\\
1.9610792821858	1.32599888091024\\
2.02517402706913	1.29054836978655\\
1.90898477009428	1.23168003520599\\
1.95382687452845	1.07890142563229\\
1.98007488912549	0.865945916419211\\
1.91942744299385	1.23088568855929\\
1.90203680990463	1.46675740172822\\
1.79605423362615	0.99553318086307\\
1.93675059768613	1.26622062159566\\
1.80589580300649	1.38156657036724\\
1.78116659707435	1.22679608853667\\
1.4859041706582	1.42125093730957\\
1.80527660882125	1.44977387260663\\
2.02922661129011	1.0207490772509\\
1.81798786252392	0.647952952360472\\
2.32931027046338	1.2400230228602\\
1.62944341809323	0.943292490703794\\
1.98834525677686	1.34211488379452\\
1.91927966663161	1.25902905273399\\
2.0814706955723	1.07488996429983\\
1.99353342912541	0.988261144028966\\
2.10800463348708	0.766165616265449\\
1.8550225815185	1.13356072105016\\
1.89272673258568	1.36537195907968\\
1.8902074337978	0.757056328403824\\
1.53885687334523	1.07694569130808\\
1.76112331184547	0.998690472291168\\
2.09154477395659	0.694075837194668\\
1.90858542651557	0.757727713019561\\
1.9330049256032	0.976874558375833\\
1.93983302339883	0.816051509797898\\
2.24506862568672	0.672826581876916\\
1.82718437556353	1.10936506743491\\
1.96057264525256	0.893925986104898\\
1.93646566414943	0.64731870645308\\
2.39122724922164	1.24345442192263\\
2.24889413709987	0.793537657333463\\
2.39607565749392	0.91178625334131\\
1.80986080122559	0.707169946665853\\
1.97704296558096	0.591120152065339\\
2.00293771191178	1.36005407696541\\
2.24796362886873	0.686533510972829\\
2.11424994022843	0.877619102581024\\
2.34406728575871	0.795199528805402\\
1.56821878862338	0.60163344885657\\
1.82330236671711	0.291695714299127\\
1.45039697294321	0.590849030637304\\
1.88899283418446	1.04552268182906\\
1.52058053004542	0.435160238141619\\
1.66251832326001	0.804648110802819\\
1.64664565454418	0.505949448785172\\
1.91508532580206	0.847486502505432\\
1.34263224779645	0.88256267651136\\
2.26055865989652	0.538404924773607\\
1.69524767103065	0.741992169305531\\
1.94016209417356	0.741514759245998\\
1.82019093577253	0.328368525179733\\
1.4839520831286	0.681796346611881\\
1.13805529399511	0.728796623608036\\
1.73651102992287	0.402875684463721\\
2.40970926685092	0.766519696154169\\
2.12006326132181	0.724614576326541\\
1.33511311753437	0.0606578626326615\\
1.21694480442495	0.939960475214773\\
1.49919597789097	0.779844441507984\\
1.63203786050577	0.615522830684201\\
1.28009907725904	0.603133599618405\\
1.83363385147587	0.447426311107601\\
1.25294254329276	0.891238649047264\\
1.38589448744568	0.110193693911327\\
2.12557258718458	0.480767020439752\\
1.76853338890148	0.345961715962911\\
1.1838175823629	-0.0335760456264687\\
1.07748933668513	0.396363550377259\\
1.91951574719719	0.710847307761236\\
0.890323101344239	0.539672205647595\\
1.65348820931753	0.512238368108479\\
1.65111543031058	-0.265446989923681\\
1.6070900231769	0.488398313760616\\
0.901758997337281	0.183058802837383\\
1.71870592048393	-0.251816153684363\\
1.55442462019974	0.00466747014038243\\
0.795021559669278	0.0361205015350433\\
0.958568445751217	0.212363974452783\\
1.67442948239855	0.4882871131843\\
1.2206685044667	-0.324863168774303\\
1.50767341273933	-0.389108334817502\\
1.37836608274919	-0.0373397947956927\\
1.11156784444486	0.701945681672377\\
1.4984756552433	0.00584512978944459\\
1.24690289872505	0.226154641123381\\
1.09194532469786	-0.130683470031559\\
0.678183222535072	-0.0365335083510088\\
1.73612089885204	0.397031801733272\\
0.933320654619196	0.746473890326733\\
1.15525443706948	-0.179109971394189\\
0.458916732169228	-0.213954536366935\\
1.58359507914698	-0.231297472151952\\
0.931434909226943	0.114256234330456\\
1.1322927346265	0.17900469841877\\
0.780331897364257	0.102401717565955\\
0.501871266865262	-0.0628649708451471\\
0.503048632418745	-0.459092444286199\\
0.604765068083075	-0.011679852416595\\
0.666178137609782	0.219390527881344\\
0.413245356250045	-0.0695882331429949\\
0.467795007897315	-0.0999511872763669\\
1.15246888796731	-0.0589372144022903\\
0.792490565783635	-0.386299556418842\\
0.747652249238335	0.192053680178325\\
0.6730227064451	-0.457495772861639\\
0.657608820873446	-0.397844794344219\\
0.80283938775403	0.00586599616798039\\
1.29819955341968	-0.204261874156702\\
0.700969575790672	-0.360362213508646\\
0.577484279146232	0.667638314415951\\
1.53670668302431	-0.432957390684138\\
-0.157947037327338	-0.206989394214058\\
-0.250033960160869	0.369996686022994\\
0.35679132363435	-0.672812036115748\\
0.307418930825796	-0.491736063949304\\
0.612754451313721	-0.591725661644702\\
0.513152577927538	-0.242776993818146\\
0.457076244801449	-0.714876480294682\\
0.374036335051672	-0.347506625594372\\
0.579283772090358	0.220733907465001\\
0.187849487939868	0.169639639117464\\
-0.0361733332976721	0.281051835212802\\
0.772027729747197	-0.505718105015689\\
0.269625343575084	-0.756939350867156\\
0.60570524403307	-0.119581878977858\\
-0.192532712828272	-0.0344061540276999\\
-0.179793749480468	-0.900222299631903\\
0.0699543465880473	-0.840398840157916\\
0.168779646202848	-0.973477052749994\\
-0.00586739214808801	-0.369383158623758\\
0.465908963186411	-0.973925709442261\\
0.160490995593304	-0.295058807475094\\
0.168397029916368	-0.980178812503749\\
};
\addplot [color=mycolor2,only marks,mark=o,mark options={solid},forget plot]
  table[row sep=crcr]{%
-0.026947097397927	-1.54967503001056\\
-0.0499096052447985	-1.548838725865\\
-0.0878921441942618	-1.54358858745209\\
-0.102243240432917	-1.54118951004161\\
-0.115588964597783	-1.54384949729\\
-0.166126369580897	-1.54954111761082\\
-0.171775059288128	-1.53255169798828\\
-0.202468902552123	-1.53693920317449\\
-0.212030786452003	-1.57194441918315\\
-0.242543396618313	-1.55982750802416\\
-0.268638030522741	-1.55184551810696\\
-0.307999262514236	-1.54141240307698\\
-0.34184706998786	-1.55670987888773\\
-0.321654511497463	-1.58426559222047\\
-0.392925170560618	-1.51563730344866\\
-0.369402481562153	-1.52198515289873\\
-0.416941632627096	-1.51482183374776\\
-0.47214126557288	-1.57339572749443\\
-0.493265130467738	-1.49632259648208\\
-0.512861932358619	-1.50290927620725\\
-0.557340464169725	-1.50842598279773\\
-0.515456646472125	-1.56774163478792\\
-0.608332652640224	-1.42764267683647\\
-0.571584245650045	-1.49817383565417\\
-0.662130333502388	-1.56697697704636\\
-0.669954622356556	-1.57493033524355\\
-0.694374248931051	-1.46017650026605\\
-0.763136406470192	-1.44654664398095\\
-0.64832005525922	-1.5311300933111\\
-0.721079189047473	-1.57176135978772\\
-0.762850186528795	-1.48791858695594\\
-0.771668694696658	-1.4609916619696\\
-0.82794824307844	-1.46985243432682\\
-0.757762090173658	-1.51439340279633\\
-0.73200316350585	-1.54643662607006\\
-0.849650868675982	-1.4739825416715\\
-0.869633988303967	-1.36513677682954\\
-0.929617676304213	-1.40255715057468\\
-0.976307028854808	-1.46241069529678\\
-0.929398084649915	-1.49557897888553\\
-1.08884466383786	-1.60008247625742\\
-1.1061052512272	-1.4667516416831\\
-0.989287983622614	-1.44461809994125\\
-0.985959410143955	-1.5033836797293\\
-1.18771458192431	-1.33744464545877\\
-1.00331746689445	-1.40505396838939\\
-0.892432339761976	-1.52461527768079\\
-1.06850301949737	-1.42382215016215\\
-1.21578154069187	-1.35143372749743\\
-1.1054763672977	-1.37105316655968\\
-1.05324198188984	-1.36449836483918\\
-1.36583986729047	-1.47144251941379\\
-1.21498723375497	-1.58153140750668\\
-1.15864586873963	-1.35211138149604\\
-1.29274883019543	-1.24706936498748\\
-1.21165267830111	-1.36424729300889\\
-1.13339068136518	-1.38791313506574\\
-1.38647009921518	-1.36164505679436\\
-1.38447268580765	-1.24424678906453\\
-1.30846082709955	-1.31326104146817\\
-1.26981825327827	-1.22686897127221\\
-1.49956364455672	-1.20155729619685\\
-1.44709290956236	-1.3499491526295\\
-1.42148110929907	-1.2096773418177\\
-1.48960505179888	-1.41647829527412\\
-1.3264878697695	-1.47045835775913\\
-1.68833293650043	-1.35607977232846\\
-1.58968210104544	-1.26255149104674\\
-1.55268500555184	-1.2291738166282\\
-1.68292402606647	-1.11026634760032\\
-1.72154852078698	-1.24202688963492\\
-1.43590169300641	-1.37871676926254\\
-1.69132652673162	-1.07877471235978\\
-1.70551483678654	-1.42804960005592\\
-1.53520709567707	-1.17299419798963\\
-1.63554863873912	-1.28753234363309\\
-1.65650253897248	-1.09178635476079\\
-1.62198948544192	-1.19413197420841\\
-1.47519706775498	-1.28565950980904\\
-1.43433635492227	-1.32753123102665\\
-1.73478223846456	-1.20660071287845\\
-1.90432859972523	-1.29120302815414\\
-1.9071129230002	-0.972886141766062\\
-1.69183439158024	-1.10780628569212\\
-1.68284658996465	-1.31833231939511\\
-1.85992398322545	-1.01141698924493\\
-1.72772929472913	-0.98157402042344\\
-1.74885476713543	-1.4351574402123\\
-1.84150517090143	-0.95425928858046\\
-1.69315517542604	-0.958656958688715\\
-1.7286495891314	-0.954601665072524\\
-1.81811333022962	-1.19579885376329\\
-1.91064332117842	-1.29835844701107\\
-2.115454982706	-1.0182126578052\\
-1.81014864687136	-1.11941452772307\\
-1.88306383492019	-1.01716868637323\\
-1.86797031445803	-1.04760399233855\\
-1.99290332758122	-1.38532172129099\\
-1.57013566090976	-0.954717363457782\\
-1.6305853956127	-0.926490388852283\\
-1.82503267963498	-1.2054049019875\\
-2.04745234990293	-0.70712356859999\\
-1.91302909485597	-0.765494691747267\\
-1.79315031506749	-1.07048160263372\\
-1.67520961679342	-0.869000607592547\\
-1.98886073309524	-1.32715925014926\\
-2.01375717222897	-0.9315973741601\\
-1.93408159383817	-0.878274086050914\\
-2.26246434296342	-0.886063271151665\\
-1.72639050831228	-0.560229703702508\\
-2.06045827186263	-0.977280310897143\\
-2.21522054773345	-0.832044314048742\\
-2.12802823080291	-0.765383980210819\\
-2.47819250264965	-0.688993880398068\\
-1.68985107972107	-0.732789288101983\\
-1.81469301090601	-0.558603652168838\\
-2.02534666967788	-0.825878531557616\\
-1.90668680745126	-0.687396711733751\\
-2.06031052575321	-0.615192926362136\\
-1.99585448031789	-0.990208999863819\\
-2.14213953096814	-0.765898153601535\\
-1.99073350792875	-0.38534769649198\\
-1.96145912243946	-0.756790477559718\\
-1.88353079013264	-0.493700231213568\\
-2.3752794639133	-0.852864483514567\\
-1.71756604008283	-0.839522058331343\\
-1.96581549537573	-0.658952618483849\\
-2.46496189427916	-0.82131795587749\\
-2.10629520423653	-0.845571433601728\\
-2.11100602865062	-0.528455628514166\\
-1.70854667582737	-0.522179557510307\\
-2.00450257438808	-1.16055808599304\\
-2.21156096039358	-0.482354346628644\\
-2.04160631785436	-0.533352916239695\\
-1.84559801482221	-0.658966536067479\\
-1.92091051009354	-0.644336627288141\\
-1.92864622951692	-0.408174908184284\\
-2.04092091269914	-0.53236208450286\\
-1.86431188937102	-0.409957901472978\\
-1.97521350506084	-0.613741462440286\\
-2.12216417761872	-0.431925911153432\\
-1.80339734046248	-0.583776426535711\\
-1.72974769480169	-0.586333850372364\\
-1.97641639419026	-0.40354436882594\\
-1.66790009602428	-0.535102761781156\\
-1.60550901999747	-0.517641138116707\\
-1.98093745067308	-0.248643233716586\\
-1.77048803449917	-0.0215214325064952\\
-1.82669905771791	-0.344780626915585\\
-2.19905576091282	-0.408420695054419\\
-1.83101161586013	-0.16573497405526\\
-1.99022362231772	-0.240850680829\\
-1.37200304413117	-0.218020520951735\\
-1.83273900042908	0.0375840492151414\\
-1.56236387312051	-0.0622109695320063\\
-1.83847741136532	-0.629170879373212\\
-1.62842916741517	-0.261531442505625\\
-1.760757004596	-0.122954408755528\\
-1.92693522151942	0.0317672214706444\\
-0.795922426677282	-0.305147429881106\\
-1.66083624793154	-0.460640523838259\\
-1.79682003601176	0.166845603602293\\
-1.64022836211743	-0.361636009773446\\
-2.53886522482741	-0.236580562503111\\
-2.03581414704918	0.013541805129293\\
-1.84630209912121	0.230376960175914\\
-1.91209527248309	-0.360580206613225\\
-1.81233556853063	0.0860452228596581\\
-1.85601948484678	-0.446016224299677\\
-1.44425457774794	0.133932704188332\\
-1.82082024175442	-0.524871278676388\\
-1.90283579546103	-0.024980003152043\\
-0.918616001910668	0.598439720318925\\
-1.37397981159442	-0.228688766885865\\
-2.13124001952824	0.0394435337110551\\
-1.51184628119662	-0.339909320146771\\
-1.42078378674312	0.259637468967449\\
-1.46934828152567	-0.48658770179399\\
-1.45036714720006	0.561650634597791\\
-1.6235939585632	0.100495743328196\\
-1.56361953447043	0.429563917963797\\
-1.72011242640493	0.125548136165587\\
-1.70161637144955	-0.00552791997817297\\
-1.3830196422212	0.484897744780771\\
-1.94541153808096	-0.142579452253231\\
-1.17461760721602	0.106879495508337\\
-1.09674397776978	0.918979238595924\\
-1.37782133309489	0.510300838744834\\
-1.88845389859122	0.238194094828098\\
-1.05273228631062	0.534963752730167\\
-1.29045492870113	0.541672326993093\\
-1.51108660201011	0.115095697290448\\
-1.60478470830124	0.761938729066085\\
-1.16297289088485	0.712515972958128\\
-1.36626122649495	0.0459699369006284\\
-1.00569677090725	0.566053726817701\\
-1.15605146422804	0.32048685029173\\
-0.822878786956236	0.299885052175656\\
-1.40380642068769	0.703790875671732\\
-1.59757716568005	0.713604210308872\\
-1.32350395473092	-0.0181115754076439\\
-1.15090617990717	0.0297575564492658\\
-1.21924769269614	0.636116178851027\\
-1.66330401562556	0.274767468591772\\
-1.34530843504664	0.53762093307587\\
-1.39326431751636	1.12504793697234\\
-0.833618774635974	0.512669380246674\\
-0.573367167351866	0.652510525923379\\
-0.276320238657219	0.18763449775689\\
-1.17194492857265	0.572769304168273\\
-0.784840550207506	0.714006761323611\\
-0.924013272173041	0.0557708685451278\\
-1.15282586642296	0.334082023567727\\
-0.935348154153094	0.774875910909737\\
-1.09740398474138	0.634480504642842\\
-1.11946267734234	0.610532582269006\\
-0.838968288924051	1.38322547926388\\
-1.22448365542895	1.32213641543783\\
-0.476481839809044	1.40105426500117\\
-0.769114345508466	0.66169599251629\\
-1.40027835036432	1.28884494337088\\
-1.50132634028759	0.613610138665922\\
-0.949826176623956	0.594374431492637\\
-0.88164099127801	0.552395160007791\\
-0.994300613393627	0.256827952699087\\
-0.632450277038536	0.881951176938878\\
-0.331331900379547	0.85471608258404\\
-1.13074002827094	0.521881807762682\\
-0.86984371951327	0.417083028748866\\
-0.420618985877099	0.726297500339449\\
-0.110157996853952	0.742495519066295\\
0.644069697204106	0.64901805109848\\
-0.262085393832434	0.700643157487521\\
-0.276442795861882	0.467852427149516\\
-0.14443352788167	0.705856371233983\\
-0.727695229224663	1.06740698127406\\
-0.626210638028053	0.986934718277524\\
-0.658293890852093	1.48386542068786\\
-0.71066942064998	0.843047675153383\\
0.185985214616693	0.552601466396477\\
-0.615839609456226	0.931267504635312\\
-0.533903453015464	1.49417117664361\\
-0.278043832459567	1.40015122194036\\
-0.374673778674507	0.760845850670364\\
0.0347222292805265	1.27907372368006\\
-0.41631147842867	0.89614152469315\\
-0.148017321318051	0.860943748299246\\
-0.318167564411286	1.66905910965134\\
0.565869201509072	1.63328081306408\\
0.456182094678272	1.2472468033001\\
};
\addplot[contour prepared, contour prepared format=matlab, contour/labels=false] table[row sep=crcr] {%
%
-0.6	29\\
-1.45884643612088	-1.1\\
-1.5	-1.13754570215345\\
-1.6	-1.1785840340606\\
-1.7	-1.17689288308075\\
-1.8	-1.14060100269158\\
-1.85964761439557	-1.1\\
-1.9	-1.06320184157811\\
-1.94734021495595	-1\\
-1.99449210803193	-0.9\\
-2	-0.867892122258616\\
-2.00886951443491	-0.8\\
-2	-0.718943834841697\\
-1.99703429594087	-0.7\\
-1.94568089622016	-0.6\\
-1.9	-0.54844298565257\\
-1.81544965928523	-0.5\\
-1.8	-0.493492352850289\\
-1.7	-0.489444115349822\\
-1.667740076461	-0.5\\
-1.6	-0.527577563607002\\
-1.5126074649847	-0.6\\
-1.5	-0.614041192328958\\
-1.44669613964686	-0.7\\
-1.40583046049001	-0.8\\
-1.4	-0.835940491724053\\
-1.39160986387914	-0.9\\
-1.4	-0.983273951036223\\
-1.40245538558091	-1\\
-1.45884643612088	-1.1\\
-0.4	69\\
-1.37191990888635	-1.7\\
-1.4	-1.70522136383611\\
-1.5	-1.71232050714977\\
-1.6	-1.70812219790915\\
-1.65339075613851	-1.7\\
-1.7	-1.6928742969923\\
-1.8	-1.66642716145916\\
-1.9	-1.62832752836094\\
-1.95636203795995	-1.6\\
-2	-1.57676045423641\\
-2.1	-1.50970235273787\\
-2.1121318183965	-1.5\\
-2.2	-1.42169621444033\\
-2.22094490997383	-1.4\\
-2.3	-1.30461469687696\\
-2.30342336620455	-1.3\\
-2.36567683625301	-1.2\\
-2.4	-1.12899689532199\\
-2.41314863156781	-1.1\\
-2.44693072481371	-1\\
-2.46887718663509	-0.9\\
-2.47962537568764	-0.8\\
-2.47941109555133	-0.7\\
-2.46811682304335	-0.6\\
-2.44529635353493	-0.5\\
-2.41017801916485	-0.4\\
-2.4	-0.378130357662609\\
-2.35979179457196	-0.3\\
-2.3	-0.209448459171389\\
-2.29271878721408	-0.2\\
-2.20106434955966	-0.1\\
-2.2	-0.0989780988223092\\
-2.1	-0.0206590540358018\\
-2.06475127547755	0\\
-2	0.0356150689222266\\
-1.9	0.0733363769504435\\
-1.8	0.0944051018391568\\
-1.7	0.0992593100290098\\
-1.6	0.087426719642879\\
-1.5	0.057491403468208\\
-1.4	0.00699191978206579\\
-1.38993762413155	0\\
-1.3	-0.0680969978015095\\
-1.26729305496576	-0.1\\
-1.2	-0.173015825339939\\
-1.17902918344932	-0.2\\
-1.10952490507525	-0.3\\
-1.1	-0.315347983508182\\
-1.05216283953355	-0.4\\
-1.00281646677347	-0.5\\
-1	-0.506463077783774\\
-0.96063862741931	-0.6\\
-0.924660892369549	-0.7\\
-0.9	-0.782945644322776\\
-0.89487561619711	-0.8\\
-0.87190117425089	-0.9\\
-0.856980581904882	-1\\
-0.851665273876761	-1.1\\
-0.858166271559011	-1.2\\
-0.879535467451499	-1.3\\
-0.9	-1.35276257201851\\
-0.920765315991924	-1.4\\
-0.989511388389717	-1.5\\
-1	-1.51120664360769\\
-1.1	-1.59665100616558\\
-1.10518583148216	-1.6\\
-1.2	-1.65011141471583\\
-1.3	-1.68524961794824\\
-1.37191990888635	-1.7\\
-0.2	111\\
-0.896698633316912	-2.1\\
-0.9	-2.10127926902931\\
-1	-2.13285211965792\\
-1.1	-2.15563799636691\\
-1.2	-2.17073029876577\\
-1.3	-2.17890909796732\\
-1.4	-2.18067590070681\\
-1.5	-2.17627240268217\\
-1.6	-2.16568446078941\\
-1.7	-2.14863091738172\\
-1.8	-2.12453532807281\\
-1.87730643930365	-2.1\\
-1.9	-2.09319138372631\\
-2	-2.05594734953922\\
-2.1	-2.00897038043205\\
-2.11634775188062	-2\\
-2.2	-1.95431941418611\\
-2.28234822182673	-1.9\\
-2.3	-1.88803978914169\\
-2.4	-1.81003030648855\\
-2.41148860365435	-1.8\\
-2.5	-1.71684253352669\\
-2.51626963854783	-1.7\\
-2.6	-1.60330784254418\\
-2.60268758775305	-1.6\\
-2.67554115369156	-1.5\\
-2.7	-1.46014027031814\\
-2.73621566805895	-1.4\\
-2.78566833941317	-1.3\\
-2.8	-1.26443361809775\\
-2.82666704850635	-1.2\\
-2.85890585730211	-1.1\\
-2.88235474822552	-1\\
-2.89798250364692	-0.9\\
-2.9	-0.876264997262088\\
-2.90714749059272	-0.8\\
-2.9088371843978	-0.7\\
-2.90289299550626	-0.6\\
-2.9	-0.578579329994041\\
-2.89020801577466	-0.5\\
-2.87029054526197	-0.4\\
-2.84209604003743	-0.3\\
-2.80455496021426	-0.2\\
-2.8	-0.189889593691818\\
-2.75967834276833	-0.1\\
-2.70356148877963	0\\
-2.7	0.00560221815314918\\
-2.63740525487015	0.1\\
-2.6	0.148721598427784\\
-2.55746598661616	0.2\\
-2.5	0.262434064638358\\
-2.46132146825457	0.3\\
-2.4	0.35577717911519\\
-2.34371390861878	0.4\\
-2.3	0.433309853462394\\
-2.2	0.497149658410564\\
-2.19465719423782	0.5\\
-2.1	0.55134439146485\\
-2	0.59358108592508\\
-1.98020724924617	0.6\\
-1.9	0.627573919213708\\
-1.8	0.651698501629493\\
-1.7	0.665965273796182\\
-1.6	0.670389892426057\\
-1.5	0.664400990928863\\
-1.4	0.646790251297601\\
-1.3	0.615619198357988\\
-1.26528445659999	0.6\\
-1.2	0.569695179788646\\
-1.1	0.504748793515267\\
-1.09412422682792	0.5\\
-1	0.417303368053329\\
-0.983804610814049	0.4\\
-0.9	0.300078588784389\\
-0.899942487454691	0.3\\
-0.832118993157241	0.2\\
-0.8	0.146270176204885\\
-0.774246601871442	0.1\\
-0.72324110600043	0\\
-0.7	-0.0493202651212574\\
-0.676557524919356	-0.1\\
-0.632656517789898	-0.2\\
-0.6	-0.278948365285755\\
-0.591088488966574	-0.3\\
-0.549467908557751	-0.4\\
-0.509983207617662	-0.5\\
-0.5	-0.526212327390649\\
-0.470149148348397	-0.6\\
-0.432135445786629	-0.7\\
-0.4	-0.792412131880071\\
-0.397143458150142	-0.8\\
-0.363458328511833	-0.9\\
-0.335194173593354	-1\\
-0.31326814844692	-1.1\\
-0.3	-1.19122150168659\\
-0.298585786841564	-1.2\\
-0.292087988762327	-1.3\\
-0.296393514642805	-1.4\\
-0.3	-1.42226234188896\\
-0.312438603188831	-1.5\\
-0.342477076625948	-1.6\\
-0.389945378028242	-1.7\\
-0.4	-1.71585470240328\\
-0.457471246200616	-1.8\\
-0.5	-1.84784187538306\\
-0.5523827703069	-1.9\\
-0.6	-1.93947824860443\\
-0.686975126878027	-2\\
-0.7	-2.00805549096821\\
-0.8	-2.06083559354846\\
-0.896698633316912	-2.1\\
0	110\\
-1.90364811954664	3.2\\
-1.9	3.19713914008638\\
-1.83983822348261	3.1\\
-1.8	3.06044299964779\\
-1.76876138536017	3\\
-1.7	2.91598591004573\\
-1.69294002459417	2.9\\
-1.62175383594343	2.8\\
-1.6	2.78048345473768\\
-1.54909899599428	2.7\\
-1.5	2.64857700341886\\
-1.47243261896984	2.6\\
-1.4	2.51341969758158\\
-1.39302987037861	2.5\\
-1.31670520582355	2.4\\
-1.3	2.38462631930917\\
-1.24019996078132	2.3\\
-1.2	2.25865468615695\\
-1.1613767765157	2.2\\
-1.1	2.13043783496102\\
-1.08109983534838	2.1\\
-1.00007393049969	2\\
-1	1.99992899368202\\
-0.923673591901519	1.9\\
-0.9	1.8748485089346\\
-0.846471845397807	1.8\\
-0.8	1.74572287283339\\
-0.769265200165042	1.7\\
-0.7	1.6114321845496\\
-0.692764883026958	1.6\\
-0.620487526720783	1.5\\
-0.6	1.4741914927793\\
-0.550690868324283	1.4\\
-0.5	1.32835468394907\\
-0.482773367795891	1.3\\
-0.419288561868795	1.2\\
-0.4	1.1699604029592\\
-0.360247759703159	1.1\\
-0.304236997800243	1\\
-0.3	0.992200891595417\\
-0.254342654274605	0.9\\
-0.207320240132952	0.8\\
-0.2	0.783444772053375\\
-0.165351589325632	0.7\\
-0.126212634440571	0.6\\
-0.0999999999999996	0.528655480424524\\
-0.0898116057416276	0.5\\
-0.0555410772069033	0.4\\
-0.0218434550133227	0.3\\
0	0.235099996397796\\
0.0119535206848333	0.2\\
0.0473157408075989	0.1\\
0.0850532707836501	0\\
0.0999999999999996	-0.0372156161921269\\
0.126577715220277	-0.1\\
0.172719419642716	-0.2\\
0.2	-0.255007371874934\\
0.224062120767782	-0.3\\
0.281247958132496	-0.4\\
0.3	-0.431540497500056\\
0.344753403600087	-0.5\\
0.4	-0.582442676730154\\
0.413118941777693	-0.6\\
0.487239090379632	-0.7\\
0.5	-0.71792714695298\\
0.566470749902975	-0.8\\
0.6	-0.844284832548111\\
0.64894578219568	-0.9\\
0.7	-0.964171055611633\\
0.733840878194419	-1\\
0.8	-1.07984254794803\\
0.820350065588313	-1.1\\
0.9	-1.19281452235338\\
0.907731048365681	-1.2\\
0.99626206367183	-1.3\\
1	-1.30531547037998\\
1.08643498488786	-1.4\\
1.1	-1.4192226116864\\
1.17648151127868	-1.5\\
1.2	-1.53308832031778\\
1.26606930056825	-1.6\\
1.3	-1.6471092324561\\
1.35488241310251	-1.7\\
1.4	-1.76131644400107\\
1.44259218483286	-1.8\\
1.5	-1.87561085689847\\
1.52882362597496	-1.9\\
1.6	-1.98979911998895\\
1.6131127160417	-2\\
1.69671961413924	-2.1\\
1.7	-2.10623936569983\\
1.78363422358351	-2.2\\
1.8	-2.22954649489429\\
1.86907624413669	-2.3\\
1.9	-2.35210464348623\\
1.95257694015576	-2.4\\
2	-2.47328574753687\\
2.03347339746581	-2.5\\
2.1	-2.59260704089331\\
2.11079138564741	-2.6\\
2.19104016389737	-2.7\\
2.2	-2.72044487073916\\
2.27408212938639	-2.8\\
2.3	-2.85154088081055\\
2.35435572076791	-2.9\\
2.4	-2.97775090169011\\
2.43067432425733	-3\\
2.5	-3.09934649777283\\
2.50113214310253	-3.1\\
2.5833099033029	-3.2\\
0.2	109\\
1.30665332404214	-0.1\\
1.4	-0.131708136074781\\
1.5	-0.153889074894203\\
1.6	-0.165656280880189\\
1.7	-0.167979805761927\\
1.8	-0.161332419528974\\
1.9	-0.145716030149428\\
2	-0.120655415933129\\
2.05925942912073	-0.1\\
2.1	-0.0864749163427538\\
2.2	-0.0432313179765836\\
2.27821319897083	0\\
2.3	0.0120485834710175\\
2.4	0.0789240781804717\\
2.4267707434676	0.1\\
2.5	0.160889215716054\\
2.54086300477044	0.2\\
2.6	0.262267876211809\\
2.63238684206902	0.3\\
2.7	0.390603410906937\\
2.70659066280506	0.4\\
2.76799758292538	0.5\\
2.8	0.564614949543128\\
2.8173484766437	0.6\\
2.85729606488014	0.7\\
2.88714687438859	0.8\\
2.9	0.859741115469427\\
2.90917806969768	0.9\\
2.92403902656377	1\\
2.9310196944667	1.1\\
2.93047779808102	1.2\\
2.92245618260174	1.3\\
2.90667962323613	1.4\\
2.9	1.42845032718333\\
2.88442392090388	1.5\\
2.8546273088952	1.6\\
2.81531546742319	1.7\\
2.8	1.73199503973697\\
2.76801248422496	1.8\\
2.70993706757788	1.9\\
2.7	1.91490061519685\\
2.64166485811129	2\\
2.6	2.05224327544669\\
2.55953693901312	2.1\\
2.5	2.16278582769883\\
2.46125270157235	2.2\\
2.4	2.25449895208647\\
2.34196074807048	2.3\\
2.3	2.33152902074516\\
2.2	2.39597327619668\\
2.1928054542093	2.4\\
2.1	2.45203741994375\\
2	2.49721953175113\\
1.99264624764411	2.5\\
1.9	2.53656452374312\\
1.8	2.56741664598364\\
1.7	2.59071706117839\\
1.64434177898739	2.6\\
1.6	2.60809192983267\\
1.5	2.61947033815305\\
1.4	2.6241122708258\\
1.3	2.62205397946067\\
1.2	2.61305521887241\\
1.11995701926875	2.6\\
1.1	2.59684079202316\\
1	2.57371835839165\\
0.9	2.54141370595308\\
0.803734174826957	2.5\\
0.8	2.49829847851869\\
0.7	2.44319984009333\\
0.63701896096918	2.4\\
0.6	2.37095714431989\\
0.523136700073074	2.3\\
0.5	2.27384279460446\\
0.4415226317764	2.2\\
0.4	2.1299964011805\\
0.383310278933173	2.1\\
0.342730985610366	2\\
0.317508059457909	1.9\\
0.304772836853602	1.8\\
0.30234465934028	1.7\\
0.308542811545783	1.6\\
0.322052958368932	1.5\\
0.341833966191687	1.4\\
0.367055963981096	1.3\\
0.397063558772871	1.2\\
0.4	1.19129248857246\\
0.428673441129285	1.1\\
0.463640537230009	1\\
0.5	0.905512017698712\\
0.502027305495171	0.9\\
0.541189279833537	0.8\\
0.584484378520255	0.7\\
0.6	0.66652052347626\\
0.630798813995697	0.6\\
0.68229433996832	0.5\\
0.7	0.46841302677039\\
0.739981730248451	0.4\\
0.8	0.309673454627051\\
0.80694463681404	0.3\\
0.886130841465655	0.2\\
0.9	0.184116867134371\\
0.98405388914767	0.1\\
1	0.0853336639425692\\
1.1	0.00795364758804202\\
1.11263927846994	0\\
1.2	-0.0528342453201263\\
1.3	-0.0977634397182949\\
1.30665332404214	-0.1\\
0.4	71\\
1.45687228449594	0.4\\
1.5	0.38107737182315\\
1.6	0.352833867208359\\
1.7	0.339980879555162\\
1.8	0.341641956223711\\
1.9	0.357692058500177\\
2	0.38876006446242\\
2.02442268168505	0.4\\
2.1	0.436817835768245\\
2.19394925367139	0.5\\
2.2	0.50452981944448\\
2.3	0.59954863101394\\
2.30039751435383	0.6\\
2.37537691950056	0.7\\
2.4	0.741678698992032\\
2.43072465737828	0.8\\
2.46996752024414	0.9\\
2.4959930897213	1\\
2.5	1.02810572075782\\
2.5099052906679	1.1\\
2.51262548937804	1.2\\
2.50447224854776	1.3\\
2.5	1.32361187544473\\
2.48534553738824	1.4\\
2.45485078905802	1.5\\
2.4120906422505	1.6\\
2.4	1.62229521509319\\
2.3548134687409	1.7\\
2.3	1.77618347082936\\
2.28090403997869	1.8\\
2.2	1.88606147725782\\
2.18480144025293	1.9\\
2.1	1.96942811804594\\
2.05480783741651	2\\
2	2.03445060590293\\
1.9	2.0845387896463\\
1.85942639520284	2.1\\
1.8	2.12204361636459\\
1.7	2.14799186141708\\
1.6	2.16312900907214\\
1.5	2.16754767924296\\
1.4	2.16077580280442\\
1.3	2.14174121640765\\
1.2	2.10869139105169\\
1.18124486134727	2.1\\
1.1	2.05597505854085\\
1.02477504599929	2\\
1	1.97657116152248\\
0.934809405496878	1.9\\
0.9	1.84177306379358\\
0.878330431273508	1.8\\
0.844575384459663	1.7\\
0.827231183370582	1.6\\
0.822902931234785	1.5\\
0.829168181154605	1.4\\
0.844364567428952	1.3\\
0.867448930935518	1.2\\
0.897909394683823	1.1\\
0.9	1.09436715604105\\
0.935439888616946	1\\
0.98055141680318	0.9\\
1	0.862776458876044\\
1.03486875540705	0.8\\
1.09990473989918	0.7\\
1.1	0.699868757576866\\
1.18188867546208	0.6\\
1.2	0.580468850140737\\
1.28940115884632	0.5\\
1.3	0.491361129333794\\
1.4	0.426447706995948\\
1.45687228449594	0.4\\
0.6	31\\
1.55987137883556	0.9\\
1.6	0.876860973120876\\
1.7	0.848493892880847\\
1.8	0.850626385231816\\
1.9	0.885326108822391\\
1.92148864079131	0.9\\
2	0.972818778951781\\
2.01943606137231	1\\
2.06389694016051	1.1\\
2.08118578596548	1.2\\
2.0735915323462	1.3\\
2.04280208563256	1.4\\
2	1.48066742140537\\
1.98675763898175	1.5\\
1.9	1.59086228195596\\
1.8874692018282	1.6\\
1.8	1.65098835161274\\
1.7	1.68415886893146\\
1.6	1.69046606884745\\
1.5	1.66547089917018\\
1.4	1.60110192218234\\
1.3989302362974	1.6\\
1.33813101367016	1.5\\
1.31256348986029	1.4\\
1.31344465070162	1.3\\
1.33556342797339	1.2\\
1.37622319590118	1.1\\
1.4	1.05932275690652\\
1.44409630132461	1\\
1.5	0.94213348185591\\
1.55987137883556	0.9\\
};
\end{axis}
\end{tikzpicture}%
\end{document}
% This file was created by matlab2tikz.
% Minimal pgfplots version: 1.3
%
%The latest updates can be retrieved from
%  http://www.mathworks.com/matlabcentral/fileexchange/22022-matlab2tikz
%where you can also make suggestions and rate matlab2tikz.
%
\documentclass[tikz]{standalone}
\usepackage{pgfplots}
\usepackage{grffile}
\pgfplotsset{compat=newest}
\usetikzlibrary{plotmarks}
\usepackage{amsmath}

\begin{document}
\definecolor{mycolor1}{rgb}{0.00000,0.44700,0.74100}%
\definecolor{mycolor2}{rgb}{0.85000,0.32500,0.09800}%
%
\begin{tikzpicture}

\begin{axis}[%
width=1.5in,
height=1.5in,
scale only axis,
xmin=-3,
xmax=3,
ymin=-3.2,
ymax=3.2,
title={second component}
]
\addplot [color=mycolor1,only marks,mark=o,mark options={solid},forget plot]
  table[row sep=crcr]{%
0.0259388858243524	1.99651291936462\\
0.0479365430139229	2.00136242712261\\
0.0808011428276144	2.00015957097738\\
0.0960018408277835	2.00397357604688\\
0.123924398192739	1.99393565592043\\
0.144716297235007	1.99609680237626\\
0.172432773795358	2.00782480916528\\
0.211096968463268	2.00289657674787\\
0.241323554557747	2.00054011148902\\
0.248450748340069	1.98323290473631\\
0.278037158020472	1.99775137204772\\
0.295123845636812	2.024317082564\\
0.278089869880903	1.98953669925669\\
0.349224330459094	2.01774494803247\\
0.389769425315202	1.96915513448386\\
0.443104169526806	2.00181328676435\\
0.445366816453369	2.00702160442149\\
0.447621894481189	1.9639008226548\\
0.459268204218018	1.96874477518371\\
0.525152271376063	1.93928073190811\\
0.479912634035923	1.9531725626356\\
0.541892480429807	1.98767075962513\\
0.596086730914084	1.95159718447859\\
0.642409661124227	1.88862872560156\\
0.59672705695044	2.06269182867749\\
0.577160609928358	1.93769298667604\\
0.61270638646484	1.9863001413635\\
0.738131551064678	1.99453476973758\\
0.731223036941714	1.9720505976831\\
0.728694499517447	2.00387465318476\\
0.792620470002546	2.01820212338234\\
0.76404643370489	1.92712620890764\\
0.783621400051233	1.97170029129066\\
0.844923525499164	2.0002383443526\\
0.933088804435648	1.86929978545337\\
0.826522508566799	1.96489496451136\\
0.857943301337058	1.93574991230876\\
0.862415931324452	1.91824418132958\\
0.85957751145223	1.94915043906307\\
1.10549276410824	1.89209624896601\\
0.992380983980335	1.9389778055659\\
1.05163144833557	1.9077966876626\\
0.971362952394099	1.94526987131734\\
1.10648177744783	1.98917932950755\\
1.12898410068108	1.88302810812235\\
1.14935856334692	1.80741423289017\\
1.2475966096538	1.80328803923413\\
1.02094574055669	1.89016584759305\\
1.18313200649952	1.80836196844854\\
1.11713521982497	1.97114350530034\\
1.21095637410431	1.6606506602979\\
1.09752975594461	1.89186114518647\\
1.1286092446374	1.8521928055118\\
1.17641657796315	1.85179775653125\\
1.2027986521108	1.79291729003239\\
1.40862693119303	1.68188184454732\\
1.29643649567147	1.6835431657974\\
1.46182858553516	1.76685142278392\\
1.44694035368565	1.75403257372017\\
1.49092139521946	1.61038316326389\\
1.3379059086772	1.66077993928176\\
1.46168112951844	1.88866965377674\\
1.41023026436323	1.73928409103319\\
1.4714192456311	1.78600061226619\\
1.35577541010529	1.79014999256957\\
1.4051660334094	1.67552928758717\\
1.61761103881626	1.92386710385478\\
1.24305770227106	1.88248740497196\\
1.49729193411573	1.81459528264337\\
1.63186681467879	1.63329561110855\\
1.53796361955124	1.77930541551158\\
1.46647340378635	1.89266145452761\\
1.59115626629009	1.92971274617182\\
1.55598375769603	1.68322562636936\\
1.60162857304111	1.74522783244834\\
1.55820638402395	1.58511515395533\\
1.6445086680066	1.69602995640505\\
1.43117976594691	1.39522816594688\\
1.66046656580406	1.5020485139169\\
1.89457571586182	1.8060245580165\\
1.7690851756095	1.803354010806\\
1.51808575625609	1.62902388762982\\
1.71974982125714	1.56603639813406\\
1.63851255324909	1.37209586707968\\
1.73596507613967	1.53438459796941\\
1.8856967379105	1.45170562832068\\
1.83646220685143	1.51078810063935\\
1.6807272288616	1.41341108157641\\
1.6794592820606	1.42428516993435\\
1.84697126074185	1.69896546443275\\
1.65822816966463	1.41270723759625\\
1.78371507953502	1.42947586399192\\
1.87899798012913	1.57760927136629\\
1.89142792776872	1.58934082453833\\
1.89301329213783	1.04572276609604\\
1.75219375143258	1.4487568304138\\
2.1047354401674	1.41876136406707\\
2.01123547254846	1.3110508611222\\
1.96781276802938	1.52540722041352\\
1.51166361010094	1.32992221483295\\
2.22158609914819	1.4805071317003\\
1.97788676892182	1.64185849242464\\
1.94851336550087	1.45074289017143\\
2.05150009067797	1.21139868557201\\
1.93322927547617	1.22986741742318\\
1.99640422827958	0.927539227571993\\
2.02784448795865	1.33792744894102\\
2.05296048996672	1.40894341042615\\
1.79681210188576	1.18409186751766\\
2.04186829813144	1.41141478615\\
2.06457591322178	1.27562881170451\\
1.9610792821858	1.32599888091024\\
2.02517402706913	1.29054836978655\\
1.90898477009428	1.23168003520599\\
1.95382687452845	1.07890142563229\\
1.98007488912549	0.865945916419211\\
1.91942744299385	1.23088568855929\\
1.90203680990463	1.46675740172822\\
1.79605423362615	0.99553318086307\\
1.93675059768613	1.26622062159566\\
1.80589580300649	1.38156657036724\\
1.78116659707435	1.22679608853667\\
1.4859041706582	1.42125093730957\\
1.80527660882125	1.44977387260663\\
2.02922661129011	1.0207490772509\\
1.81798786252392	0.647952952360472\\
2.32931027046338	1.2400230228602\\
1.62944341809323	0.943292490703794\\
1.98834525677686	1.34211488379452\\
1.91927966663161	1.25902905273399\\
2.0814706955723	1.07488996429983\\
1.99353342912541	0.988261144028966\\
2.10800463348708	0.766165616265449\\
1.8550225815185	1.13356072105016\\
1.89272673258568	1.36537195907968\\
1.8902074337978	0.757056328403824\\
1.53885687334523	1.07694569130808\\
1.76112331184547	0.998690472291168\\
2.09154477395659	0.694075837194668\\
1.90858542651557	0.757727713019561\\
1.9330049256032	0.976874558375833\\
1.93983302339883	0.816051509797898\\
2.24506862568672	0.672826581876916\\
1.82718437556353	1.10936506743491\\
1.96057264525256	0.893925986104898\\
1.93646566414943	0.64731870645308\\
2.39122724922164	1.24345442192263\\
2.24889413709987	0.793537657333463\\
2.39607565749392	0.91178625334131\\
1.80986080122559	0.707169946665853\\
1.97704296558096	0.591120152065339\\
2.00293771191178	1.36005407696541\\
2.24796362886873	0.686533510972829\\
2.11424994022843	0.877619102581024\\
2.34406728575871	0.795199528805402\\
1.56821878862338	0.60163344885657\\
1.82330236671711	0.291695714299127\\
1.45039697294321	0.590849030637304\\
1.88899283418446	1.04552268182906\\
1.52058053004542	0.435160238141619\\
1.66251832326001	0.804648110802819\\
1.64664565454418	0.505949448785172\\
1.91508532580206	0.847486502505432\\
1.34263224779645	0.88256267651136\\
2.26055865989652	0.538404924773607\\
1.69524767103065	0.741992169305531\\
1.94016209417356	0.741514759245998\\
1.82019093577253	0.328368525179733\\
1.4839520831286	0.681796346611881\\
1.13805529399511	0.728796623608036\\
1.73651102992287	0.402875684463721\\
2.40970926685092	0.766519696154169\\
2.12006326132181	0.724614576326541\\
1.33511311753437	0.0606578626326615\\
1.21694480442495	0.939960475214773\\
1.49919597789097	0.779844441507984\\
1.63203786050577	0.615522830684201\\
1.28009907725904	0.603133599618405\\
1.83363385147587	0.447426311107601\\
1.25294254329276	0.891238649047264\\
1.38589448744568	0.110193693911327\\
2.12557258718458	0.480767020439752\\
1.76853338890148	0.345961715962911\\
1.1838175823629	-0.0335760456264687\\
1.07748933668513	0.396363550377259\\
1.91951574719719	0.710847307761236\\
0.890323101344239	0.539672205647595\\
1.65348820931753	0.512238368108479\\
1.65111543031058	-0.265446989923681\\
1.6070900231769	0.488398313760616\\
0.901758997337281	0.183058802837383\\
1.71870592048393	-0.251816153684363\\
1.55442462019974	0.00466747014038243\\
0.795021559669278	0.0361205015350433\\
0.958568445751217	0.212363974452783\\
1.67442948239855	0.4882871131843\\
1.2206685044667	-0.324863168774303\\
1.50767341273933	-0.389108334817502\\
1.37836608274919	-0.0373397947956927\\
1.11156784444486	0.701945681672377\\
1.4984756552433	0.00584512978944459\\
1.24690289872505	0.226154641123381\\
1.09194532469786	-0.130683470031559\\
0.678183222535072	-0.0365335083510088\\
1.73612089885204	0.397031801733272\\
0.933320654619196	0.746473890326733\\
1.15525443706948	-0.179109971394189\\
0.458916732169228	-0.213954536366935\\
1.58359507914698	-0.231297472151952\\
0.931434909226943	0.114256234330456\\
1.1322927346265	0.17900469841877\\
0.780331897364257	0.102401717565955\\
0.501871266865262	-0.0628649708451471\\
0.503048632418745	-0.459092444286199\\
0.604765068083075	-0.011679852416595\\
0.666178137609782	0.219390527881344\\
0.413245356250045	-0.0695882331429949\\
0.467795007897315	-0.0999511872763669\\
1.15246888796731	-0.0589372144022903\\
0.792490565783635	-0.386299556418842\\
0.747652249238335	0.192053680178325\\
0.6730227064451	-0.457495772861639\\
0.657608820873446	-0.397844794344219\\
0.80283938775403	0.00586599616798039\\
1.29819955341968	-0.204261874156702\\
0.700969575790672	-0.360362213508646\\
0.577484279146232	0.667638314415951\\
1.53670668302431	-0.432957390684138\\
-0.157947037327338	-0.206989394214058\\
-0.250033960160869	0.369996686022994\\
0.35679132363435	-0.672812036115748\\
0.307418930825796	-0.491736063949304\\
0.612754451313721	-0.591725661644702\\
0.513152577927538	-0.242776993818146\\
0.457076244801449	-0.714876480294682\\
0.374036335051672	-0.347506625594372\\
0.579283772090358	0.220733907465001\\
0.187849487939868	0.169639639117464\\
-0.0361733332976721	0.281051835212802\\
0.772027729747197	-0.505718105015689\\
0.269625343575084	-0.756939350867156\\
0.60570524403307	-0.119581878977858\\
-0.192532712828272	-0.0344061540276999\\
-0.179793749480468	-0.900222299631903\\
0.0699543465880473	-0.840398840157916\\
0.168779646202848	-0.973477052749994\\
-0.00586739214808801	-0.369383158623758\\
0.465908963186411	-0.973925709442261\\
0.160490995593304	-0.295058807475094\\
0.168397029916368	-0.980178812503749\\
};
\addplot [color=mycolor2,only marks,mark=o,mark options={solid},forget plot]
  table[row sep=crcr]{%
-0.026947097397927	-1.54967503001056\\
-0.0499096052447985	-1.548838725865\\
-0.0878921441942618	-1.54358858745209\\
-0.102243240432917	-1.54118951004161\\
-0.115588964597783	-1.54384949729\\
-0.166126369580897	-1.54954111761082\\
-0.171775059288128	-1.53255169798828\\
-0.202468902552123	-1.53693920317449\\
-0.212030786452003	-1.57194441918315\\
-0.242543396618313	-1.55982750802416\\
-0.268638030522741	-1.55184551810696\\
-0.307999262514236	-1.54141240307698\\
-0.34184706998786	-1.55670987888773\\
-0.321654511497463	-1.58426559222047\\
-0.392925170560618	-1.51563730344866\\
-0.369402481562153	-1.52198515289873\\
-0.416941632627096	-1.51482183374776\\
-0.47214126557288	-1.57339572749443\\
-0.493265130467738	-1.49632259648208\\
-0.512861932358619	-1.50290927620725\\
-0.557340464169725	-1.50842598279773\\
-0.515456646472125	-1.56774163478792\\
-0.608332652640224	-1.42764267683647\\
-0.571584245650045	-1.49817383565417\\
-0.662130333502388	-1.56697697704636\\
-0.669954622356556	-1.57493033524355\\
-0.694374248931051	-1.46017650026605\\
-0.763136406470192	-1.44654664398095\\
-0.64832005525922	-1.5311300933111\\
-0.721079189047473	-1.57176135978772\\
-0.762850186528795	-1.48791858695594\\
-0.771668694696658	-1.4609916619696\\
-0.82794824307844	-1.46985243432682\\
-0.757762090173658	-1.51439340279633\\
-0.73200316350585	-1.54643662607006\\
-0.849650868675982	-1.4739825416715\\
-0.869633988303967	-1.36513677682954\\
-0.929617676304213	-1.40255715057468\\
-0.976307028854808	-1.46241069529678\\
-0.929398084649915	-1.49557897888553\\
-1.08884466383786	-1.60008247625742\\
-1.1061052512272	-1.4667516416831\\
-0.989287983622614	-1.44461809994125\\
-0.985959410143955	-1.5033836797293\\
-1.18771458192431	-1.33744464545877\\
-1.00331746689445	-1.40505396838939\\
-0.892432339761976	-1.52461527768079\\
-1.06850301949737	-1.42382215016215\\
-1.21578154069187	-1.35143372749743\\
-1.1054763672977	-1.37105316655968\\
-1.05324198188984	-1.36449836483918\\
-1.36583986729047	-1.47144251941379\\
-1.21498723375497	-1.58153140750668\\
-1.15864586873963	-1.35211138149604\\
-1.29274883019543	-1.24706936498748\\
-1.21165267830111	-1.36424729300889\\
-1.13339068136518	-1.38791313506574\\
-1.38647009921518	-1.36164505679436\\
-1.38447268580765	-1.24424678906453\\
-1.30846082709955	-1.31326104146817\\
-1.26981825327827	-1.22686897127221\\
-1.49956364455672	-1.20155729619685\\
-1.44709290956236	-1.3499491526295\\
-1.42148110929907	-1.2096773418177\\
-1.48960505179888	-1.41647829527412\\
-1.3264878697695	-1.47045835775913\\
-1.68833293650043	-1.35607977232846\\
-1.58968210104544	-1.26255149104674\\
-1.55268500555184	-1.2291738166282\\
-1.68292402606647	-1.11026634760032\\
-1.72154852078698	-1.24202688963492\\
-1.43590169300641	-1.37871676926254\\
-1.69132652673162	-1.07877471235978\\
-1.70551483678654	-1.42804960005592\\
-1.53520709567707	-1.17299419798963\\
-1.63554863873912	-1.28753234363309\\
-1.65650253897248	-1.09178635476079\\
-1.62198948544192	-1.19413197420841\\
-1.47519706775498	-1.28565950980904\\
-1.43433635492227	-1.32753123102665\\
-1.73478223846456	-1.20660071287845\\
-1.90432859972523	-1.29120302815414\\
-1.9071129230002	-0.972886141766062\\
-1.69183439158024	-1.10780628569212\\
-1.68284658996465	-1.31833231939511\\
-1.85992398322545	-1.01141698924493\\
-1.72772929472913	-0.98157402042344\\
-1.74885476713543	-1.4351574402123\\
-1.84150517090143	-0.95425928858046\\
-1.69315517542604	-0.958656958688715\\
-1.7286495891314	-0.954601665072524\\
-1.81811333022962	-1.19579885376329\\
-1.91064332117842	-1.29835844701107\\
-2.115454982706	-1.0182126578052\\
-1.81014864687136	-1.11941452772307\\
-1.88306383492019	-1.01716868637323\\
-1.86797031445803	-1.04760399233855\\
-1.99290332758122	-1.38532172129099\\
-1.57013566090976	-0.954717363457782\\
-1.6305853956127	-0.926490388852283\\
-1.82503267963498	-1.2054049019875\\
-2.04745234990293	-0.70712356859999\\
-1.91302909485597	-0.765494691747267\\
-1.79315031506749	-1.07048160263372\\
-1.67520961679342	-0.869000607592547\\
-1.98886073309524	-1.32715925014926\\
-2.01375717222897	-0.9315973741601\\
-1.93408159383817	-0.878274086050914\\
-2.26246434296342	-0.886063271151665\\
-1.72639050831228	-0.560229703702508\\
-2.06045827186263	-0.977280310897143\\
-2.21522054773345	-0.832044314048742\\
-2.12802823080291	-0.765383980210819\\
-2.47819250264965	-0.688993880398068\\
-1.68985107972107	-0.732789288101983\\
-1.81469301090601	-0.558603652168838\\
-2.02534666967788	-0.825878531557616\\
-1.90668680745126	-0.687396711733751\\
-2.06031052575321	-0.615192926362136\\
-1.99585448031789	-0.990208999863819\\
-2.14213953096814	-0.765898153601535\\
-1.99073350792875	-0.38534769649198\\
-1.96145912243946	-0.756790477559718\\
-1.88353079013264	-0.493700231213568\\
-2.3752794639133	-0.852864483514567\\
-1.71756604008283	-0.839522058331343\\
-1.96581549537573	-0.658952618483849\\
-2.46496189427916	-0.82131795587749\\
-2.10629520423653	-0.845571433601728\\
-2.11100602865062	-0.528455628514166\\
-1.70854667582737	-0.522179557510307\\
-2.00450257438808	-1.16055808599304\\
-2.21156096039358	-0.482354346628644\\
-2.04160631785436	-0.533352916239695\\
-1.84559801482221	-0.658966536067479\\
-1.92091051009354	-0.644336627288141\\
-1.92864622951692	-0.408174908184284\\
-2.04092091269914	-0.53236208450286\\
-1.86431188937102	-0.409957901472978\\
-1.97521350506084	-0.613741462440286\\
-2.12216417761872	-0.431925911153432\\
-1.80339734046248	-0.583776426535711\\
-1.72974769480169	-0.586333850372364\\
-1.97641639419026	-0.40354436882594\\
-1.66790009602428	-0.535102761781156\\
-1.60550901999747	-0.517641138116707\\
-1.98093745067308	-0.248643233716586\\
-1.77048803449917	-0.0215214325064952\\
-1.82669905771791	-0.344780626915585\\
-2.19905576091282	-0.408420695054419\\
-1.83101161586013	-0.16573497405526\\
-1.99022362231772	-0.240850680829\\
-1.37200304413117	-0.218020520951735\\
-1.83273900042908	0.0375840492151414\\
-1.56236387312051	-0.0622109695320063\\
-1.83847741136532	-0.629170879373212\\
-1.62842916741517	-0.261531442505625\\
-1.760757004596	-0.122954408755528\\
-1.92693522151942	0.0317672214706444\\
-0.795922426677282	-0.305147429881106\\
-1.66083624793154	-0.460640523838259\\
-1.79682003601176	0.166845603602293\\
-1.64022836211743	-0.361636009773446\\
-2.53886522482741	-0.236580562503111\\
-2.03581414704918	0.013541805129293\\
-1.84630209912121	0.230376960175914\\
-1.91209527248309	-0.360580206613225\\
-1.81233556853063	0.0860452228596581\\
-1.85601948484678	-0.446016224299677\\
-1.44425457774794	0.133932704188332\\
-1.82082024175442	-0.524871278676388\\
-1.90283579546103	-0.024980003152043\\
-0.918616001910668	0.598439720318925\\
-1.37397981159442	-0.228688766885865\\
-2.13124001952824	0.0394435337110551\\
-1.51184628119662	-0.339909320146771\\
-1.42078378674312	0.259637468967449\\
-1.46934828152567	-0.48658770179399\\
-1.45036714720006	0.561650634597791\\
-1.6235939585632	0.100495743328196\\
-1.56361953447043	0.429563917963797\\
-1.72011242640493	0.125548136165587\\
-1.70161637144955	-0.00552791997817297\\
-1.3830196422212	0.484897744780771\\
-1.94541153808096	-0.142579452253231\\
-1.17461760721602	0.106879495508337\\
-1.09674397776978	0.918979238595924\\
-1.37782133309489	0.510300838744834\\
-1.88845389859122	0.238194094828098\\
-1.05273228631062	0.534963752730167\\
-1.29045492870113	0.541672326993093\\
-1.51108660201011	0.115095697290448\\
-1.60478470830124	0.761938729066085\\
-1.16297289088485	0.712515972958128\\
-1.36626122649495	0.0459699369006284\\
-1.00569677090725	0.566053726817701\\
-1.15605146422804	0.32048685029173\\
-0.822878786956236	0.299885052175656\\
-1.40380642068769	0.703790875671732\\
-1.59757716568005	0.713604210308872\\
-1.32350395473092	-0.0181115754076439\\
-1.15090617990717	0.0297575564492658\\
-1.21924769269614	0.636116178851027\\
-1.66330401562556	0.274767468591772\\
-1.34530843504664	0.53762093307587\\
-1.39326431751636	1.12504793697234\\
-0.833618774635974	0.512669380246674\\
-0.573367167351866	0.652510525923379\\
-0.276320238657219	0.18763449775689\\
-1.17194492857265	0.572769304168273\\
-0.784840550207506	0.714006761323611\\
-0.924013272173041	0.0557708685451278\\
-1.15282586642296	0.334082023567727\\
-0.935348154153094	0.774875910909737\\
-1.09740398474138	0.634480504642842\\
-1.11946267734234	0.610532582269006\\
-0.838968288924051	1.38322547926388\\
-1.22448365542895	1.32213641543783\\
-0.476481839809044	1.40105426500117\\
-0.769114345508466	0.66169599251629\\
-1.40027835036432	1.28884494337088\\
-1.50132634028759	0.613610138665922\\
-0.949826176623956	0.594374431492637\\
-0.88164099127801	0.552395160007791\\
-0.994300613393627	0.256827952699087\\
-0.632450277038536	0.881951176938878\\
-0.331331900379547	0.85471608258404\\
-1.13074002827094	0.521881807762682\\
-0.86984371951327	0.417083028748866\\
-0.420618985877099	0.726297500339449\\
-0.110157996853952	0.742495519066295\\
0.644069697204106	0.64901805109848\\
-0.262085393832434	0.700643157487521\\
-0.276442795861882	0.467852427149516\\
-0.14443352788167	0.705856371233983\\
-0.727695229224663	1.06740698127406\\
-0.626210638028053	0.986934718277524\\
-0.658293890852093	1.48386542068786\\
-0.71066942064998	0.843047675153383\\
0.185985214616693	0.552601466396477\\
-0.615839609456226	0.931267504635312\\
-0.533903453015464	1.49417117664361\\
-0.278043832459567	1.40015122194036\\
-0.374673778674507	0.760845850670364\\
0.0347222292805265	1.27907372368006\\
-0.41631147842867	0.89614152469315\\
-0.148017321318051	0.860943748299246\\
-0.318167564411286	1.66905910965134\\
0.565869201509072	1.63328081306408\\
0.456182094678272	1.2472468033001\\
};
\addplot[contour prepared, contour prepared format=matlab, contour/labels=false] table[row sep=crcr] {%
%
-0.5	39\\
-0.500568292583098	0.3\\
-0.6	0.270534962190567\\
-0.7	0.255210561217125\\
-0.8	0.252551954432356\\
-0.9	0.261720273700712\\
-1	0.282361390929562\\
-1.05445460902904	0.3\\
-1.1	0.319093332890317\\
-1.2	0.377064231141353\\
-1.2310294804785	0.4\\
-1.3	0.477511217377222\\
-1.3165023180166	0.5\\
-1.35204221677138	0.6\\
-1.35209863723038	0.7\\
-1.31940868280714	0.8\\
-1.3	0.831120295795939\\
-1.24781890582572	0.9\\
-1.2	0.944638715179678\\
-1.12413747836254	1\\
-1.1	1.01434886920175\\
-1	1.05799266056873\\
-0.9	1.08645865474191\\
-0.8	1.09977030002859\\
-0.7	1.09699285966019\\
-0.6	1.07606880673853\\
-0.5	1.03354657714055\\
-0.449177472256002	1\\
-0.4	0.955812465480715\\
-0.353328304528695	0.9\\
-0.3	0.802484359462388\\
-0.298851768470286	0.8\\
-0.273411794195118	0.7\\
-0.270162752997606	0.6\\
-0.292801381487081	0.5\\
-0.3	0.486384791782124\\
-0.356442770444033	0.4\\
-0.4	0.362301657397569\\
-0.5	0.30024216352541\\
-0.500568292583098	0.3\\
-0.4	61\\
-0.156635334683154	0.1\\
-0.2	0.0848521891978131\\
-0.3	0.059837671929797\\
-0.4	0.0442577638155116\\
-0.5	0.0360391905461676\\
-0.6	0.0341058338904147\\
-0.7	0.03794602308365\\
-0.8	0.0473692325112521\\
-0.9	0.062365156810127\\
-1	0.0830230096087495\\
-1.06350047533493	0.1\\
-1.1	0.111030929077424\\
-1.2	0.148872985154166\\
-1.3	0.194142582412746\\
-1.31111576640592	0.2\\
-1.4	0.258280455581407\\
-1.4554584431759	0.3\\
-1.5	0.345683715624345\\
-1.54747287584728	0.4\\
-1.6	0.494101540166176\\
-1.60306340967523	0.5\\
-1.62867413190646	0.6\\
-1.63018711971526	0.7\\
-1.60961210997733	0.8\\
-1.6	0.823663872863082\\
-1.56623813831466	0.9\\
-1.50043462085894	1\\
-1.5	1.00052054763824\\
-1.40554277426876	1.1\\
-1.4	1.10492603753859\\
-1.3	1.18149210032865\\
-1.27088931230368	1.2\\
-1.2	1.24112670056234\\
-1.1	1.2882191440732\\
-1.06831189729401	1.3\\
-1	1.32468979090745\\
-0.9	1.35209603227459\\
-0.8	1.37134728135967\\
-0.7	1.38249523691161\\
-0.6	1.38494893584815\\
-0.5	1.37724729170794\\
-0.4	1.35665112922412\\
-0.3	1.3184347480829\\
-0.268158900383014	1.3\\
-0.2	1.25271954978504\\
-0.146427948256347	1.2\\
-0.0999999999999996	1.14326020247654\\
-0.071920337550766	1.1\\
-0.019038742660019	1\\
0	0.955100417695114\\
0.0208559795620558	0.9\\
0.0515131483019774	0.8\\
0.0749677229844131	0.7\\
0.0906908581481487	0.6\\
0.0969051552979138	0.5\\
0.0898766396028256	0.4\\
0.0622020382702216	0.3\\
0	0.202047805208392\\
-0.00169594670644772	0.2\\
-0.0999999999999996	0.127850829598819\\
-0.156635334683154	0.1\\
-0.3	87\\
0.242281031869579	-0.2\\
0.2	-0.207112009155391\\
0.0999999999999996	-0.212697756303381\\
0	-0.21084630025258\\
-0.0999999999999996	-0.204821954392572\\
-0.154679776513274	-0.2\\
-0.2	-0.196219378779733\\
-0.3	-0.185885383158545\\
-0.4	-0.174513358563385\\
-0.5	-0.16204345195212\\
-0.6	-0.148219076365759\\
-0.7	-0.132695610322313\\
-0.8	-0.115106792518254\\
-0.874524328768393	-0.1\\
-0.9	-0.0947279524042192\\
-1	-0.070222498761438\\
-1.1	-0.0424043175251228\\
-1.2	-0.0109813085514448\\
-1.23095988578524	0\\
-1.3	0.0272283800147162\\
-1.4	0.0715170347551908\\
-1.45804630904744	0.1\\
-1.5	0.124538871438617\\
-1.6	0.189090680354012\\
-1.61572064534886	0.2\\
-1.7	0.275533771852164\\
-1.72566174753254	0.3\\
-1.8	0.399771242579414\\
-1.80016532096703	0.4\\
-1.84693211732667	0.5\\
-1.87030810600804	0.6\\
-1.873768478242	0.7\\
-1.85931789353759	0.8\\
-1.82778561343377	0.9\\
-1.8	0.958297407951783\\
-1.77918663099225	1\\
-1.71288824988671	1.1\\
-1.7	1.11604790806994\\
-1.62628908373646	1.2\\
-1.6	1.22591010662621\\
-1.51547604135611	1.3\\
-1.5	1.3123846396706\\
-1.4	1.38299643110306\\
-1.3720712085596	1.4\\
-1.3	1.44238888977352\\
-1.2	1.49227046052001\\
-1.18190248238646	1.5\\
-1.1	1.53549877054297\\
-1	1.5722090324008\\
-0.910739691693891	1.6\\
-0.9	1.60354163432339\\
-0.8	1.63129089555146\\
-0.7	1.65463353793702\\
-0.6	1.67390708382467\\
-0.5	1.68897720256656\\
-0.4	1.69902748586169\\
-0.366737462025682	1.7\\
-0.3	1.70221940224035\\
-0.268209174764127	1.7\\
-0.2	1.69471491128842\\
-0.0999999999999996	1.66978489925748\\
0	1.61439564346606\\
0.0155884404826957	1.6\\
0.0999999999999996	1.50097993846085\\
0.100562446772988	1.5\\
0.150847613925095	1.4\\
0.191376375488049	1.3\\
0.2	1.2767502435044\\
0.223579122964981	1.2\\
0.252928752315995	1.1\\
0.281557268626012	1\\
0.3	0.935661192193481\\
0.309485754477353	0.9\\
0.336808188920937	0.8\\
0.364930814054012	0.7\\
0.393671165791724	0.6\\
0.4	0.576730403113042\\
0.421245549709509	0.5\\
0.44757558263137	0.4\\
0.471290094122203	0.3\\
0.489474194196996	0.2\\
0.497083520441781	0.1\\
0.48491352532001	0\\
0.434680353166265	-0.1\\
0.4	-0.133300017701279\\
0.3	-0.186254314191649\\
0.242281031869579	-0.2\\
-0.2	119\\
0.72256608942021	-0.6\\
0.7	-0.609205564923012\\
0.6	-0.630498125273813\\
0.5	-0.633705530061871\\
0.4	-0.622785237755883\\
0.3	-0.601103297439621\\
0.296538403044926	-0.6\\
0.2	-0.571080579988968\\
0.0999999999999996	-0.536783216086507\\
0	-0.500676456511692\\
-0.0017079486905269	-0.5\\
-0.0999999999999996	-0.46428065843587\\
-0.2	-0.429725921807025\\
-0.291274541301534	-0.4\\
-0.3	-0.39738750170295\\
-0.4	-0.367088374566761\\
-0.5	-0.338917987939765\\
-0.6	-0.312248735694573\\
-0.645647659794204	-0.3\\
-0.7	-0.286142656732347\\
-0.8	-0.260051453224734\\
-0.9	-0.233644906452192\\
-1	-0.206410443224074\\
-1.0219615644611	-0.2\\
-1.1	-0.17688088790618\\
-1.2	-0.145337322460578\\
-1.3	-0.111767281455312\\
-1.33288706451906	-0.1\\
-1.4	-0.0739105650860377\\
-1.5	-0.0322455905614149\\
-1.57251295061347	0\\
-1.6	0.0141321944394464\\
-1.7	0.0685132516252183\\
-1.75474305555011	0.1\\
-1.8	0.132238723959866\\
-1.89044440424814	0.2\\
-1.9	0.209466227001449\\
-1.98928278657694	0.3\\
-2	0.315594737377723\\
-2.05860818988904	0.4\\
-2.1	0.495861586120196\\
-2.10187423201137	0.5\\
-2.12627893953793	0.6\\
-2.13247665936338	0.7\\
-2.12286656905129	0.8\\
-2.1	0.893817949255231\\
-2.09854452377969	0.9\\
-2.06217868430082	1\\
-2.01109617277606	1.1\\
-2	1.1177135862469\\
-1.94690452854058	1.2\\
-1.9	1.26017966773074\\
-1.86704018098389	1.3\\
-1.8	1.37100011252549\\
-1.77010275702879	1.4\\
-1.7	1.46244419324599\\
-1.65284891992407	1.5\\
-1.6	1.54022372742111\\
-1.51052242281391	1.6\\
-1.5	1.60695647605006\\
-1.4	1.66677983282744\\
-1.33612447032913	1.7\\
-1.3	1.71939186547626\\
-1.2	1.76748667675792\\
-1.1238178774828	1.8\\
-1.1	1.81087302964883\\
-1	1.85279432211237\\
-0.9	1.89125199503923\\
-0.875758705099526	1.9\\
-0.8	1.93020254839217\\
-0.7	1.96816672590707\\
-0.613441628585776	2\\
-0.6	2.00550015366442\\
-0.5	2.04438328649388\\
-0.4	2.08173274375661\\
-0.347026154956252	2.1\\
-0.3	2.11771429904655\\
-0.2	2.15001955762384\\
-0.0999999999999996	2.17546191379373\\
0	2.19075629586298\\
0.0999999999999996	2.19079590972042\\
0.2	2.16719833527734\\
0.3	2.10473899921257\\
0.304345342009483	2.1\\
0.371232722701201	2\\
0.4	1.93038006712474\\
0.408514912114291	1.9\\
0.427281813544451	1.8\\
0.439000565094233	1.7\\
0.447241785981442	1.6\\
0.454584481155503	1.5\\
0.462874066114033	1.4\\
0.473416779754041	1.3\\
0.487129719888512	1.2\\
0.5	1.12694888278807\\
0.504290613325195	1.1\\
0.524562843355145	1\\
0.549415644861153	0.9\\
0.579117718446011	0.8\\
0.6	0.738877829232378\\
0.613058545596848	0.7\\
0.650656115642886	0.6\\
0.693077369710446	0.5\\
0.7	0.484154103132084\\
0.737435689781057	0.4\\
0.784405964842942	0.3\\
0.8	0.265668502234371\\
0.830667185251051	0.2\\
0.874649477815993	0.1\\
0.9	0.0337704070579986\\
0.913346702718148	0\\
0.942198977510913	-0.1\\
0.958613445356983	-0.2\\
0.957359951262917	-0.3\\
0.931936232611812	-0.4\\
0.9	-0.458663317561347\\
0.869041230107365	-0.5\\
0.8	-0.559464147745143\\
0.72256608942021	-0.6\\
-0.1	157\\
1.10325551710148	-1\\
1.1	-1.00191241015909\\
1	-1.04691672900235\\
0.9	-1.07449868332402\\
0.8	-1.0868252229774\\
0.7	-1.08515400830724\\
0.6	-1.07020044725462\\
0.5	-1.04242941501318\\
0.4	-1.00230297820633\\
0.395732840923387	-1\\
0.3	-0.953906829501976\\
0.20588306552403	-0.9\\
0.2	-0.896966602186677\\
0.0999999999999996	-0.836800401575241\\
0.0436797330443289	-0.8\\
0	-0.774487806824109\\
-0.0999999999999996	-0.713520081491093\\
-0.121470242908821	-0.7\\
-0.2	-0.656091865103295\\
-0.3	-0.602573771142058\\
-0.304798193384111	-0.6\\
-0.4	-0.554486118835924\\
-0.5	-0.510491674987116\\
-0.524480061220402	-0.5\\
-0.6	-0.470619440594234\\
-0.7	-0.43387500140626\\
-0.797749698718707	-0.4\\
-0.8	-0.39926932995632\\
-0.9	-0.366266062128908\\
-1	-0.334114294674453\\
-1.1	-0.302274488296867\\
-1.10698572620119	-0.3\\
-1.2	-0.269754943812612\\
-1.3	-0.236729901477737\\
-1.4	-0.202971248021947\\
-1.40868093377626	-0.2\\
-1.5	-0.166693107620453\\
-1.6	-0.12904861999343\\
-1.67513438973865	-0.1\\
-1.7	-0.0891442904027817\\
-1.8	-0.0449969349518753\\
-1.89745603937559	0\\
-1.9	0.00141189488739411\\
-2	0.0562679195859394\\
-2.07668056771225	0.1\\
-2.1	0.117103837195231\\
-2.2	0.190512727178415\\
-2.21318368271274	0.2\\
-2.3	0.286500343542567\\
-2.31398294997507	0.3\\
-2.38494669256193	0.4\\
-2.4	0.433964754637153\\
-2.43234188414523	0.5\\
-2.45941408844021	0.6\\
-2.46958824320083	0.7\\
-2.46600341434535	0.8\\
-2.45006931877238	0.9\\
-2.42200422975688	1\\
-2.4	1.05579496579474\\
-2.38360817235476	1.1\\
-2.33566066739427	1.2\\
-2.3	1.2597016732211\\
-2.27622315808201	1.3\\
-2.20543434901698	1.4\\
-2.2	1.40689424367157\\
-2.1243042306334	1.5\\
-2.1	1.52674004921827\\
-2.02941024853981	1.6\\
-2	1.62836354311675\\
-1.91928457496758	1.7\\
-1.9	1.71645854239109\\
-1.8	1.79419619955512\\
-1.79183614868767	1.8\\
-1.7	1.86554292787936\\
-1.64516549805237	1.9\\
-1.6	1.92928644737966\\
-1.5	1.98742851430726\\
-1.47644199664573	2\\
-1.4	2.04352546705531\\
-1.3	2.09451157253892\\
-1.288679779074	2.1\\
-1.2	2.1470210790598\\
-1.1	2.19595101695964\\
-1.09155712362098	2.2\\
-1	2.24858729420653\\
-0.9	2.2991177940354\\
-0.898253655587776	2.3\\
-0.8	2.35467062043073\\
-0.715352718708421	2.4\\
-0.7	2.40894902513045\\
-0.6	2.46588698240248\\
-0.536633174664932	2.5\\
-0.5	2.52098190670866\\
-0.4	2.57414371076671\\
-0.345849500733761	2.6\\
-0.3	2.62291695905246\\
-0.2	2.66604138369041\\
-0.102586174671366	2.7\\
-0.0999999999999996	2.7009456952726\\
0	2.72906701821762\\
0.0999999999999996	2.74714694029828\\
0.2	2.75451610066399\\
0.3	2.74974722663395\\
0.4	2.73013822935637\\
0.477912040541221	2.7\\
0.5	2.69074074399573\\
0.6	2.62065831626089\\
0.619417003339108	2.6\\
0.690166068813734	2.5\\
0.7	2.47958434572911\\
0.727818915307121	2.4\\
0.746850520640847	2.3\\
0.753603081957575	2.2\\
0.751836737066392	2.1\\
0.744503553418747	2\\
0.733971801879541	1.9\\
0.722178003888233	1.8\\
0.710729891219314	1.7\\
0.700976312650192	1.6\\
0.7	1.58642392812891\\
0.693510866905697	1.5\\
0.690161459308162	1.4\\
0.691816583740698	1.3\\
0.699125054197285	1.2\\
0.7	1.19354114777726\\
0.711829496734932	1.1\\
0.730837071161094	1\\
0.756626364152432	0.9\\
0.789558131265448	0.8\\
0.8	0.773249684892695\\
0.828528758858558	0.7\\
0.874206082297059	0.6\\
0.9	0.54951819913737\\
0.925917318713273	0.5\\
0.98277332172454	0.4\\
1	0.370681887206189\\
1.04338228129328	0.3\\
1.1	0.209809955936104\\
1.10653935728861	0.2\\
1.1695357831062	0.1\\
1.2	0.0482527258202479\\
1.23072514292836	0\\
1.28714991305389	-0.1\\
1.3	-0.127921014570187\\
1.33665885150828	-0.2\\
1.37599588455315	-0.3\\
1.4	-0.391064305200162\\
1.40267420713867	-0.4\\
1.41434505683295	-0.5\\
1.40771546880674	-0.6\\
1.4	-0.630147413745954\\
1.38000241719854	-0.7\\
1.32706228098818	-0.8\\
1.3	-0.835628950349307\\
1.24102669766351	-0.9\\
1.2	-0.935669083452067\\
1.10325551710148	-1\\
0	82\\
1.77155027756242	-3.2\\
1.7	-3.1276166010376\\
1.68760530079975	-3.1\\
1.60379488972518	-3\\
1.6	-2.99760957955604\\
1.52681601165928	-2.9\\
1.5	-2.88008874485278\\
1.4467368147863	-2.8\\
1.4	-2.7596146066049\\
1.3644400743392	-2.7\\
1.3	-2.63596540411836\\
1.28055369085899	-2.6\\
1.2	-2.50900197100308\\
1.19554155734474	-2.5\\
1.1150985094161	-2.4\\
1.1	-2.3880655733604\\
1.03501642959205	-2.3\\
1	-2.26874311081469\\
0.953214768425952	-2.2\\
0.9	-2.14700363429771\\
0.870163364313912	-2.1\\
0.8	-2.02301691158688\\
0.786210978492042	-2\\
0.702223381731945	-1.9\\
0.7	-1.89807962437522\\
0.622064161524117	-1.8\\
0.6	-1.77914373387441\\
0.540225479217427	-1.7\\
0.5	-1.65906717340697\\
0.456921405622105	-1.6\\
0.4	-1.53878351942599\\
0.372192602681882	-1.5\\
0.3	-1.41962846199877\\
0.285879624858159	-1.4\\
0.2	-1.30330103958497\\
0.197561868025342	-1.3\\
0.107538839088458	-1.2\\
0.0999999999999996	-1.1929601956167\\
0.0127864602375759	-1.1\\
0	-1.08844693955489\\
-0.0892613644994377	-1\\
-0.0999999999999996	-0.990915937176685\\
-0.2	-0.90172599675866\\
-0.20186176824987	-0.9\\
-0.3	-0.822612148869353\\
-0.328798522658828	-0.8\\
-0.4	-0.751872845389596\\
-0.479934650628185	-0.7\\
-0.5	-0.688619155803831\\
-0.6	-0.633544648241048\\
-0.664888666129201	-0.6\\
-0.7	-0.583789383101105\\
-0.8	-0.539437919653281\\
-0.895270008654168	-0.5\\
-0.9	-0.498192395388073\\
-1	-0.460479713339449\\
-1.1	-0.424545432985414\\
-1.17042721073122	-0.4\\
-1.2	-0.389976549524738\\
-1.3	-0.35655270649572\\
-1.4	-0.323812427483707\\
-1.47417942448994	-0.3\\
-1.5	-0.29142660185158\\
-1.6	-0.259041171749457\\
-1.7	-0.2270636115422\\
-1.78639293694695	-0.2\\
-1.8	-0.195281869183079\\
-1.9	-0.162665852576186\\
-2	-0.130623252501434\\
-2.09776676343561	-0.1\\
-2.1	-0.0991677400833103\\
-2.2	-0.065973719601263\\
-2.3	-0.0337032583920374\\
-2.4	-0.0024317702183381\\
-2.4092319675485	0\\
-2.5	0.0310966828546874\\
-2.6	0.0635901718911817\\
-2.7	0.0947197460768831\\
-2.72075734517193	0.1\\
-2.8	0.128357300367505\\
-2.9	0.161101591835461\\
-3	0.192124227122032\\
0	73\\
1.41818069604664	3.2\\
1.4	3.1567213926634\\
1.38362259444142	3.1\\
1.34943021196719	3\\
1.31378130922119	2.9\\
1.3	2.86757445559712\\
1.27848129978411	2.8\\
1.24316427873544	2.7\\
1.20706443812109	2.6\\
1.2	2.58317969558379\\
1.1715665732681	2.5\\
1.13619416807411	2.4\\
1.10105151891381	2.3\\
1.1	2.29729726097256\\
1.066714769175	2.2\\
1.03356241373366	2.1\\
1.00211272298113	2\\
1	1.99330595656819\\
0.972258143989461	1.9\\
0.945383689531073	1.8\\
0.922176764076925	1.7\\
0.903281585360914	1.6\\
0.9	1.57712084890792\\
0.888868181569126	1.5\\
0.880202407033319	1.4\\
0.87803053194435	1.3\\
0.882913748899786	1.2\\
0.895381980773884	1.1\\
0.9	1.07727205426803\\
0.915397145070959	1\\
0.943537290813644	0.9\\
0.980328402056672	0.8\\
1	0.755880365867059\\
1.02556424942068	0.7\\
1.07913981198173	0.6\\
1.1	0.565058128590054\\
1.14099807331403	0.5\\
1.2	0.414046628605467\\
1.21041307714817	0.4\\
1.28685658495212	0.3\\
1.3	0.282784960106545\\
1.36990093763516	0.2\\
1.4	0.163164466513325\\
1.45846444575827	0.1\\
1.5	0.0517840324927044\\
1.55181839653266	0\\
1.6	-0.0537492383408141\\
1.64942852480696	-0.1\\
1.7	-0.154863079416263\\
1.75092986608399	-0.2\\
1.8	-0.252375237742649\\
1.85607681470318	-0.3\\
1.9	-0.34669276419556\\
1.96467450223273	-0.4\\
2	-0.437915380194556\\
2.07649952025704	-0.5\\
2.1	-0.52587195553123\\
2.19122157148709	-0.6\\
2.2	-0.610098702502387\\
2.3	-0.693215583283146\\
2.31156950048959	-0.7\\
2.4	-0.776886286943183\\
2.43848724623463	-0.8\\
2.5	-0.857311384662804\\
2.56753389737243	-0.9\\
2.6	-0.933408909781538\\
2.69710688329871	-1\\
2.7	-1.00340241260638\\
2.8	-1.07981641218854\\
2.8406807865304	-1.1\\
2.9	-1.15301215179708\\
2.98226594083429	-1.2\\
3	-1.21930788701325\\
0.1	99\\
-3	-1.42544206490208\\
-2.97723109971635	-1.5\\
-2.93702667369692	-1.6\\
-2.9	-1.6724752616829\\
-2.88652019014499	-1.7\\
-2.82768302228068	-1.8\\
-2.8	-1.83971304817614\\
-2.75762935260751	-1.9\\
-2.7	-1.97026685520746\\
-2.67454388955937	-2\\
-2.6	-2.07774560551856\\
-2.57693396371937	-2.1\\
-2.5	-2.16890423552509\\
-2.46112540288559	-2.2\\
-2.4	-2.24710227393293\\
-2.3203568962227	-2.3\\
-2.3	-2.31349803747115\\
-2.2	-2.37218031031542\\
-2.14315492318305	-2.4\\
-2.1	-2.42209588041932\\
-2	-2.46550014636868\\
-1.9000736558062	-2.5\\
-1.9	-2.50002783133597\\
-1.8	-2.53222363787806\\
-1.7	-2.55710481993607\\
-1.6	-2.57572618114819\\
-1.5	-2.58881234738441\\
-1.4	-2.59681666224675\\
-1.3	-2.59995772343556\\
-1.2	-2.5982374875949\\
-1.1	-2.59144267860685\\
-1	-2.57912924109955\\
-0.9	-2.56058746561537\\
-0.8	-2.53478280145508\\
-0.7	-2.50026374719969\\
-0.699358436688314	-2.5\\
-0.6	-2.46022155627781\\
-0.5	-2.40786489717176\\
-0.48709841374906	-2.4\\
-0.4	-2.34428246937933\\
-0.343129005438987	-2.3\\
-0.3	-2.26236308015176\\
-0.238336445432869	-2.2\\
-0.2	-2.15270217065329\\
-0.16138688991846	-2.1\\
-0.10737125724844	-2\\
-0.0999999999999996	-1.98036075419522\\
-0.0708821000753309	-1.9\\
-0.0515078392720902	-1.8\\
-0.0475297225340705	-1.7\\
-0.057596713578628	-1.6\\
-0.0812023114400904	-1.5\\
-0.0999999999999996	-1.45057953467572\\
-0.116805946740045	-1.4\\
-0.163844819524586	-1.3\\
-0.2	-1.24165955255497\\
-0.223665800844181	-1.2\\
-0.296550224178863	-1.1\\
-0.3	-1.09615396191133\\
-0.383149971466747	-1\\
-0.4	-0.983994511482996\\
-0.488139279046607	-0.9\\
-0.5	-0.890545535524204\\
-0.6	-0.812366246372276\\
-0.616332670914799	-0.8\\
-0.7	-0.74598350800141\\
-0.775869979435984	-0.7\\
-0.8	-0.687189596326935\\
-0.9	-0.636492645530605\\
-0.977254762445261	-0.6\\
-1	-0.590248462951641\\
-1.1	-0.549292416170437\\
-1.2	-0.511009622712328\\
-1.23045243848889	-0.5\\
-1.3	-0.475951498697338\\
-1.4	-0.443358379728986\\
-1.5	-0.412706168515762\\
-1.54499489077107	-0.4\\
-1.6	-0.384210215608199\\
-1.7	-0.357986885839505\\
-1.8	-0.33413294940172\\
-1.9	-0.313024163935677\\
-1.97458159556906	-0.3\\
-2	-0.295116287489451\\
-2.1	-0.281078340521604\\
-2.2	-0.27203683804026\\
-2.3	-0.269224547286687\\
-2.4	-0.274287178739893\\
-2.5	-0.289446622900879\\
-2.54131060792365	-0.3\\
-2.6	-0.317682604855507\\
-2.7	-0.36329796345245\\
-2.75811334192746	-0.4\\
-2.8	-0.433685548109681\\
-2.86778997302489	-0.5\\
-2.9	-0.542894492482842\\
-2.93969284636974	-0.6\\
-2.98832902478629	-0.7\\
-3	-0.73511459025451\\
0.1	78\\
3	1.85446037434794\\
2.97540289221416	1.9\\
2.90997837310627	2\\
2.9	2.01343012613882\\
2.83268251915898	2.1\\
2.8	2.13674323137026\\
2.73837315049261	2.2\\
2.7	2.23594050414747\\
2.62124941425639	2.3\\
2.6	2.31644654813143\\
2.5	2.3823201428135\\
2.46661197872874	2.4\\
2.4	2.43563751690911\\
2.3	2.4760777022589\\
2.21446709283419	2.5\\
2.2	2.5043543798122\\
2.1	2.5231897537315\\
2	2.52953760403466\\
1.9	2.52291760623581\\
1.8	2.50155270769672\\
1.79588750572041	2.5\\
1.7	2.46730159173341\\
1.6	2.4125171022988\\
1.58376773718071	2.4\\
1.5	2.33755340869957\\
1.46312422594894	2.3\\
1.4	2.23506376969981\\
1.37334939025094	2.2\\
1.30121651578824	2.1\\
1.3	2.0982463325771\\
1.24281551354191	2\\
1.2	1.91666800083033\\
1.19243738112286	1.9\\
1.15075046472164	1.8\\
1.11596222354203	1.7\\
1.1	1.64366336597911\\
1.08829861581932	1.6\\
1.06801784195354	1.5\\
1.05534468522853	1.4\\
1.05070378934962	1.3\\
1.05455906376387	1.2\\
1.06738741423228	1.1\\
1.08965399893108	1\\
1.1	0.967253333348838\\
1.12193145929741	0.9\\
1.16445985312519	0.8\\
1.2	0.731863844102969\\
1.21778646404891	0.7\\
1.28238901616411	0.6\\
1.3	0.575372397114663\\
1.35954146141996	0.5\\
1.4	0.452050878947417\\
1.4499434890498	0.4\\
1.5	0.348956047351019\\
1.55626759208478	0.3\\
1.6	0.261198926822864\\
1.6834141159005	0.2\\
1.7	0.187090587323865\\
1.8	0.124392051524134\\
1.85085694445731	0.1\\
1.9	0.0737292480731175\\
2	0.0349948211937748\\
2.1	0.00949977795589116\\
2.17331808946427	0\\
2.2	-0.00410519558592017\\
2.3	-0.00517248674241172\\
2.34126554260059	0\\
2.4	0.00711729334843647\\
2.5	0.0327988052710905\\
2.6	0.0747088673678463\\
2.64442460220111	0.1\\
2.7	0.13353953634515\\
2.78511901314083	0.2\\
2.8	0.212939170639175\\
2.88523340700342	0.3\\
2.9	0.317667882643434\\
2.9629403228343	0.4\\
3	0.459904828635754\\
0.2	85\\
-1.83787893190138	-2.2\\
-1.8	-2.21390625598031\\
-1.7	-2.24290730863447\\
-1.6	-2.26401540673969\\
-1.5	-2.27798153577242\\
-1.4	-2.28520878954\\
-1.3	-2.28578510187116\\
-1.2	-2.27949093097507\\
-1.1	-2.26578194008696\\
-1	-2.24374329815678\\
-0.9	-2.21200789768928\\
-0.870654450782591	-2.2\\
-0.8	-2.16984645460942\\
-0.7	-2.11377256576286\\
-0.679707105483279	-2.1\\
-0.6	-2.03820227572117\\
-0.559240723918386	-2\\
-0.5	-1.93045502148958\\
-0.477353973689002	-1.9\\
-0.423022309061781	-1.8\\
-0.4	-1.73046149791793\\
-0.390553432531435	-1.7\\
-0.376489487602857	-1.6\\
-0.379422302113984	-1.5\\
-0.398368007517729	-1.4\\
-0.4	-1.39543933368648\\
-0.431850381758791	-1.3\\
-0.481371966845838	-1.2\\
-0.5	-1.17229063700687\\
-0.547450292056455	-1.1\\
-0.6	-1.03835196861344\\
-0.632881537380279	-1\\
-0.7	-0.937508021185789\\
-0.741662579069615	-0.9\\
-0.8	-0.856611514564493\\
-0.88066333671932	-0.8\\
-0.9	-0.788405827712751\\
-1	-0.731855782088459\\
-1.06093608863143	-0.7\\
-1.1	-0.681853411302527\\
-1.2	-0.638835158238917\\
-1.29888793048491	-0.6\\
-1.3	-0.599594078560804\\
-1.4	-0.566410705925592\\
-1.5	-0.536697995557067\\
-1.6	-0.510600172061937\\
-1.64901190917685	-0.5\\
-1.7	-0.489003815904089\\
-1.8	-0.472292568580219\\
-1.9	-0.460752769689031\\
-2	-0.455434724689133\\
-2.1	-0.457759843518302\\
-2.2	-0.469638773774818\\
-2.3	-0.493643216587674\\
-2.31805910259983	-0.5\\
-2.4	-0.535648399768804\\
-2.4996145316555	-0.6\\
-2.5	-0.600330344466868\\
-2.59429422626591	-0.7\\
-2.6	-0.708757644627007\\
-2.6538398480017	-0.8\\
-2.69031566574602	-0.9\\
-2.7	-0.947792354101034\\
-2.71074527943248	-1\\
-2.71771866168889	-1.1\\
-2.7123205232726	-1.2\\
-2.7	-1.27213169461705\\
-2.69555039324804	-1.3\\
-2.6688227771109	-1.4\\
-2.63022828736614	-1.5\\
-2.6	-1.55966267368743\\
-2.57953241445767	-1.6\\
-2.51601643351814	-1.7\\
-2.5	-1.72131943470631\\
-2.43766981880058	-1.8\\
-2.4	-1.8409865865747\\
-2.34067729797615	-1.9\\
-2.3	-1.93647281806498\\
-2.21956853431377	-2\\
-2.2	-2.01451472399331\\
-2.1	-2.07944735402346\\
-2.06242129370794	-2.1\\
-2	-2.13370457542987\\
-1.9	-2.17796534659778\\
-1.83787893190138	-2.2\\
0.2	69\\
1.77519594045598	0.4\\
1.8	0.385883265950348\\
1.9	0.34482233225754\\
2	0.318871051064584\\
2.1	0.307671889323542\\
2.2	0.311542989353432\\
2.3	0.331567247086914\\
2.4	0.369748024822593\\
2.45357965055002	0.4\\
2.5	0.429355875104532\\
2.58389460573808	0.5\\
2.6	0.516082158027604\\
2.67041927514598	0.6\\
2.7	0.644577359797264\\
2.73333386941072	0.7\\
2.77910725764724	0.8\\
2.8	0.863858123787646\\
2.81165722134622	0.9\\
2.83325779476911	1\\
2.844383140133	1.1\\
2.84584258653208	1.2\\
2.83790920150687	1.3\\
2.82035614037239	1.4\\
2.8	1.47373605469597\\
2.79288268003517	1.5\\
2.75566063655346	1.6\\
2.70569692019267	1.7\\
2.7	1.70945750875935\\
2.64212556075836	1.8\\
2.6	1.85412570936905\\
2.56005132621391	1.9\\
2.5	1.95972187268426\\
2.45178499019344	2\\
2.4	2.03935170614472\\
2.3	2.098523673118\\
2.29667568231125	2.1\\
2.2	2.14167950009497\\
2.1	2.16879844701759\\
2	2.18088259132596\\
1.9	2.17792441178465\\
1.8	2.15878404945141\\
1.7	2.12094590919013\\
1.66426792508487	2.1\\
1.6	2.0617447947624\\
1.52711730394541	2\\
1.5	1.9754696513341\\
1.43598338942681	1.9\\
1.4	1.85216048009818\\
1.36773803056204	1.8\\
1.31529607114319	1.7\\
1.3	1.6640696811116\\
1.27594797479966	1.6\\
1.24788178372879	1.5\\
1.23002016162938	1.4\\
1.2225902807303	1.3\\
1.22591038583758	1.2\\
1.24033953256955	1.1\\
1.26623003442034	1\\
1.3	0.91010778840945\\
1.30414223958277	0.9\\
1.35690407666454	0.8\\
1.4	0.733042281211278\\
1.42417352018022	0.7\\
1.5	0.609520083532159\\
1.50943426902891	0.6\\
1.6	0.515726037096178\\
1.62087188424543	0.5\\
1.7	0.442292193697711\\
1.77519594045598	0.4\\
0.3	65\\
-1.71082343913866	-2\\
-1.7	-2.00354397315291\\
-1.6	-2.02731561179546\\
-1.5	-2.04185044751143\\
-1.4	-2.04744500982477\\
-1.3	-2.04396197932742\\
-1.2	-2.03081624894502\\
-1.1	-2.00692061889536\\
-1.07992445238317	-2\\
-1	-1.96985686463502\\
-0.9	-1.91663117858962\\
-0.875499660944723	-1.9\\
-0.8	-1.83726300915275\\
-0.763916945658892	-1.8\\
-0.7	-1.7077190633202\\
-0.695410911092135	-1.7\\
-0.657294422421047	-1.6\\
-0.642406318122801	-1.5\\
-0.648099103194521	-1.4\\
-0.673148234810259	-1.3\\
-0.7	-1.24008729692332\\
-0.717841952894497	-1.2\\
-0.783522874944573	-1.1\\
-0.8	-1.081263553804\\
-0.873977993781339	-1\\
-0.9	-0.97747098200078\\
-0.9950713192156	-0.9\\
-1	-0.896687161277149\\
-1.1	-0.834304197784665\\
-1.16063108250019	-0.8\\
-1.2	-0.78076717341629\\
-1.3	-0.737106213480548\\
-1.39690069081444	-0.7\\
-1.4	-0.69892343805572\\
-1.5	-0.669774433135302\\
-1.6	-0.646232569908757\\
-1.7	-0.628957297998317\\
-1.8	-0.618964906039322\\
-1.9	-0.617707648340479\\
-2	-0.627190529392616\\
-2.1	-0.650141169448849\\
-2.2	-0.690255112550674\\
-2.21751518576795	-0.7\\
-2.3	-0.76218066398268\\
-2.33825177856326	-0.8\\
-2.4	-0.892630865080555\\
-2.40424984912067	-0.9\\
-2.4400611942442	-1\\
-2.4552484299817	-1.1\\
-2.45354094965534	-1.2\\
-2.43666518523486	-1.3\\
-2.40494778821759	-1.4\\
-2.4	-1.41087782090099\\
-2.35820485604316	-1.5\\
-2.3	-1.59210980787794\\
-2.29465291968168	-1.6\\
-2.2111235401641	-1.7\\
-2.2	-1.71133296474553\\
-2.1014730939703	-1.8\\
-2.1	-1.80118156713615\\
-2	-1.87122092873824\\
-1.94992541019833	-1.9\\
-1.9	-1.92703603834824\\
-1.8	-1.97055342918878\\
-1.71082343913866	-2\\
0.3	49\\
1.88705044356399	0.6\\
1.9	0.595010720557418\\
2	0.576070035737513\\
2.1	0.576588623819681\\
2.2	0.597872955150262\\
2.20516477813525	0.6\\
2.3	0.645586593837847\\
2.37246626489238	0.7\\
2.4	0.725956526271838\\
2.46037501173163	0.8\\
2.5	0.867007844313672\\
2.51671611010501	0.9\\
2.55126432739884	1\\
2.57000023333681	1.1\\
2.57456370061535	1.2\\
2.56568976229236	1.3\\
2.54332313852841	1.4\\
2.50664158262627	1.5\\
2.5	1.51321491737685\\
2.45137798974328	1.6\\
2.4	1.66987413083215\\
2.37345452144045	1.7\\
2.3	1.76854115805889\\
2.25543802395783	1.8\\
2.2	1.83428131033102\\
2.1	1.87535878177924\\
2	1.89544247555691\\
1.9	1.89475052287138\\
1.8	1.87214895024651\\
1.7	1.82500490540372\\
1.66603273893832	1.8\\
1.6	1.74476402194843\\
1.56128590497594	1.7\\
1.5	1.61294044443612\\
1.49276482767845	1.6\\
1.44995175222251	1.5\\
1.42234675475055	1.4\\
1.40986657319446	1.3\\
1.41255937575134	1.2\\
1.43051664212708	1.1\\
1.46379360287324	1\\
1.5	0.925136977527595\\
1.5144030809065	0.9\\
1.58805946865228	0.8\\
1.6	0.786399499699579\\
1.6948973721031	0.7\\
1.7	0.695848478028796\\
1.8	0.635060486819138\\
1.88705044356399	0.6\\
0.4	49\\
-1.61490787062628	-1.8\\
-1.6	-1.80391538625915\\
-1.5	-1.81814329498551\\
-1.4	-1.82010877222819\\
-1.3	-1.80909982771075\\
-1.26301279927778	-1.8\\
-1.2	-1.78173741440185\\
-1.1	-1.73368433269603\\
-1.05133304578586	-1.7\\
-1	-1.65042953525652\\
-0.960186603725901	-1.6\\
-0.916086741641576	-1.5\\
-0.904513934436287	-1.4\\
-0.921131021498869	-1.3\\
-0.963758930510822	-1.2\\
-1	-1.14749665838137\\
-1.03500449334167	-1.1\\
-1.1	-1.03696140136947\\
-1.14201881630003	-1\\
-1.2	-0.960387527959088\\
-1.3	-0.900226672278478\\
-1.30045203982106	-0.9\\
-1.4	-0.8586539112062\\
-1.5	-0.825390700585599\\
-1.6	-0.801410408666986\\
-1.61141015193803	-0.8\\
-1.7	-0.790299971232611\\
-1.8	-0.790554416960737\\
-1.87004709844323	-0.8\\
-1.9	-0.805378775628237\\
-2	-0.84353693706006\\
-2.0871128900548	-0.9\\
-2.1	-0.913154139561207\\
-2.16324752334856	-1\\
-2.19789443575878	-1.1\\
-2.2	-1.13408059090116\\
-2.20360671269189	-1.2\\
-2.2	-1.22204772295246\\
-2.18652930787597	-1.3\\
-2.14804868793856	-1.4\\
-2.1	-1.48120444196827\\
-2.08751621433485	-1.5\\
-2	-1.59862618671013\\
-1.99856091838934	-1.6\\
-1.9	-1.67709454385771\\
-1.8633924491274	-1.7\\
-1.8	-1.73475200896236\\
-1.7	-1.77602369318399\\
-1.61490787062628	-1.8\\
0.4	27\\
1.89739289051417	0.9\\
1.9	0.898785061940268\\
2	0.885472241150608\\
2.07008470764074	0.9\\
2.1	0.908633405080997\\
2.2	0.987682999740872\\
2.20960576017271	1\\
2.25591279266881	1.1\\
2.27139212269839	1.2\\
2.25931502745723	1.3\\
2.22065620022939	1.4\\
2.2	1.43180788683713\\
2.1366962887458	1.5\\
2.1	1.52833222297757\\
2	1.56850888577786\\
1.9	1.56968975790948\\
1.8	1.529238618439\\
1.76566155422257	1.5\\
1.7	1.41776609651079\\
1.69077939618763	1.4\\
1.66445100294532	1.3\\
1.66442635418621	1.2\\
1.68986598526544	1.1\\
1.7	1.07975144580062\\
1.75619555136507	1\\
1.8	0.956803621571868\\
1.89739289051417	0.9\\
0.5	23\\
-1.67817620337855	-1.5\\
-1.6	-1.52781368101897\\
-1.5	-1.54068803190582\\
-1.4	-1.5279634835959\\
-1.33450642399306	-1.5\\
-1.3	-1.47256320591281\\
-1.24978412170238	-1.4\\
-1.24299049093677	-1.3\\
-1.28648423787677	-1.2\\
-1.3	-1.18501252836703\\
-1.39392568790722	-1.1\\
-1.4	-1.09630217006383\\
-1.5	-1.05383850343272\\
-1.6	-1.03077236123135\\
-1.7	-1.03340987722955\\
-1.8	-1.07136206261238\\
-1.83582079887812	-1.1\\
-1.88250226612436	-1.2\\
-1.86929842647757	-1.3\\
-1.81195461453724	-1.4\\
-1.8	-1.41239308499046\\
-1.7	-1.48940544905611\\
-1.67817620337855	-1.5\\
};
\end{axis}
\end{tikzpicture}%
\end{document}
% This file was created by matlab2tikz.
% Minimal pgfplots version: 1.3
%
%The latest updates can be retrieved from
%  http://www.mathworks.com/matlabcentral/fileexchange/22022-matlab2tikz
%where you can also make suggestions and rate matlab2tikz.
%
\documentclass[tikz]{standalone}
\usepackage{pgfplots}
\usepackage{grffile}
\pgfplotsset{compat=newest}
\usetikzlibrary{plotmarks}
\usepackage{amsmath}

\begin{document}
\definecolor{mycolor1}{rgb}{0.00000,0.44700,0.74100}%
\definecolor{mycolor2}{rgb}{0.85000,0.32500,0.09800}%
%
\begin{tikzpicture}

\begin{axis}[%
width=1.5in,
height=1.5in,
scale only axis,
xmin=-3,
xmax=3,
ymin=-3.2,
ymax=3.2,
title={third component}
]
\addplot [color=mycolor1,only marks,mark=o,mark options={solid},forget plot]
  table[row sep=crcr]{%
0.0259388858243524	1.99651291936462\\
0.0479365430139229	2.00136242712261\\
0.0808011428276144	2.00015957097738\\
0.0960018408277835	2.00397357604688\\
0.123924398192739	1.99393565592043\\
0.144716297235007	1.99609680237626\\
0.172432773795358	2.00782480916528\\
0.211096968463268	2.00289657674787\\
0.241323554557747	2.00054011148902\\
0.248450748340069	1.98323290473631\\
0.278037158020472	1.99775137204772\\
0.295123845636812	2.024317082564\\
0.278089869880903	1.98953669925669\\
0.349224330459094	2.01774494803247\\
0.389769425315202	1.96915513448386\\
0.443104169526806	2.00181328676435\\
0.445366816453369	2.00702160442149\\
0.447621894481189	1.9639008226548\\
0.459268204218018	1.96874477518371\\
0.525152271376063	1.93928073190811\\
0.479912634035923	1.9531725626356\\
0.541892480429807	1.98767075962513\\
0.596086730914084	1.95159718447859\\
0.642409661124227	1.88862872560156\\
0.59672705695044	2.06269182867749\\
0.577160609928358	1.93769298667604\\
0.61270638646484	1.9863001413635\\
0.738131551064678	1.99453476973758\\
0.731223036941714	1.9720505976831\\
0.728694499517447	2.00387465318476\\
0.792620470002546	2.01820212338234\\
0.76404643370489	1.92712620890764\\
0.783621400051233	1.97170029129066\\
0.844923525499164	2.0002383443526\\
0.933088804435648	1.86929978545337\\
0.826522508566799	1.96489496451136\\
0.857943301337058	1.93574991230876\\
0.862415931324452	1.91824418132958\\
0.85957751145223	1.94915043906307\\
1.10549276410824	1.89209624896601\\
0.992380983980335	1.9389778055659\\
1.05163144833557	1.9077966876626\\
0.971362952394099	1.94526987131734\\
1.10648177744783	1.98917932950755\\
1.12898410068108	1.88302810812235\\
1.14935856334692	1.80741423289017\\
1.2475966096538	1.80328803923413\\
1.02094574055669	1.89016584759305\\
1.18313200649952	1.80836196844854\\
1.11713521982497	1.97114350530034\\
1.21095637410431	1.6606506602979\\
1.09752975594461	1.89186114518647\\
1.1286092446374	1.8521928055118\\
1.17641657796315	1.85179775653125\\
1.2027986521108	1.79291729003239\\
1.40862693119303	1.68188184454732\\
1.29643649567147	1.6835431657974\\
1.46182858553516	1.76685142278392\\
1.44694035368565	1.75403257372017\\
1.49092139521946	1.61038316326389\\
1.3379059086772	1.66077993928176\\
1.46168112951844	1.88866965377674\\
1.41023026436323	1.73928409103319\\
1.4714192456311	1.78600061226619\\
1.35577541010529	1.79014999256957\\
1.4051660334094	1.67552928758717\\
1.61761103881626	1.92386710385478\\
1.24305770227106	1.88248740497196\\
1.49729193411573	1.81459528264337\\
1.63186681467879	1.63329561110855\\
1.53796361955124	1.77930541551158\\
1.46647340378635	1.89266145452761\\
1.59115626629009	1.92971274617182\\
1.55598375769603	1.68322562636936\\
1.60162857304111	1.74522783244834\\
1.55820638402395	1.58511515395533\\
1.6445086680066	1.69602995640505\\
1.43117976594691	1.39522816594688\\
1.66046656580406	1.5020485139169\\
1.89457571586182	1.8060245580165\\
1.7690851756095	1.803354010806\\
1.51808575625609	1.62902388762982\\
1.71974982125714	1.56603639813406\\
1.63851255324909	1.37209586707968\\
1.73596507613967	1.53438459796941\\
1.8856967379105	1.45170562832068\\
1.83646220685143	1.51078810063935\\
1.6807272288616	1.41341108157641\\
1.6794592820606	1.42428516993435\\
1.84697126074185	1.69896546443275\\
1.65822816966463	1.41270723759625\\
1.78371507953502	1.42947586399192\\
1.87899798012913	1.57760927136629\\
1.89142792776872	1.58934082453833\\
1.89301329213783	1.04572276609604\\
1.75219375143258	1.4487568304138\\
2.1047354401674	1.41876136406707\\
2.01123547254846	1.3110508611222\\
1.96781276802938	1.52540722041352\\
1.51166361010094	1.32992221483295\\
2.22158609914819	1.4805071317003\\
1.97788676892182	1.64185849242464\\
1.94851336550087	1.45074289017143\\
2.05150009067797	1.21139868557201\\
1.93322927547617	1.22986741742318\\
1.99640422827958	0.927539227571993\\
2.02784448795865	1.33792744894102\\
2.05296048996672	1.40894341042615\\
1.79681210188576	1.18409186751766\\
2.04186829813144	1.41141478615\\
2.06457591322178	1.27562881170451\\
1.9610792821858	1.32599888091024\\
2.02517402706913	1.29054836978655\\
1.90898477009428	1.23168003520599\\
1.95382687452845	1.07890142563229\\
1.98007488912549	0.865945916419211\\
1.91942744299385	1.23088568855929\\
1.90203680990463	1.46675740172822\\
1.79605423362615	0.99553318086307\\
1.93675059768613	1.26622062159566\\
1.80589580300649	1.38156657036724\\
1.78116659707435	1.22679608853667\\
1.4859041706582	1.42125093730957\\
1.80527660882125	1.44977387260663\\
2.02922661129011	1.0207490772509\\
1.81798786252392	0.647952952360472\\
2.32931027046338	1.2400230228602\\
1.62944341809323	0.943292490703794\\
1.98834525677686	1.34211488379452\\
1.91927966663161	1.25902905273399\\
2.0814706955723	1.07488996429983\\
1.99353342912541	0.988261144028966\\
2.10800463348708	0.766165616265449\\
1.8550225815185	1.13356072105016\\
1.89272673258568	1.36537195907968\\
1.8902074337978	0.757056328403824\\
1.53885687334523	1.07694569130808\\
1.76112331184547	0.998690472291168\\
2.09154477395659	0.694075837194668\\
1.90858542651557	0.757727713019561\\
1.9330049256032	0.976874558375833\\
1.93983302339883	0.816051509797898\\
2.24506862568672	0.672826581876916\\
1.82718437556353	1.10936506743491\\
1.96057264525256	0.893925986104898\\
1.93646566414943	0.64731870645308\\
2.39122724922164	1.24345442192263\\
2.24889413709987	0.793537657333463\\
2.39607565749392	0.91178625334131\\
1.80986080122559	0.707169946665853\\
1.97704296558096	0.591120152065339\\
2.00293771191178	1.36005407696541\\
2.24796362886873	0.686533510972829\\
2.11424994022843	0.877619102581024\\
2.34406728575871	0.795199528805402\\
1.56821878862338	0.60163344885657\\
1.82330236671711	0.291695714299127\\
1.45039697294321	0.590849030637304\\
1.88899283418446	1.04552268182906\\
1.52058053004542	0.435160238141619\\
1.66251832326001	0.804648110802819\\
1.64664565454418	0.505949448785172\\
1.91508532580206	0.847486502505432\\
1.34263224779645	0.88256267651136\\
2.26055865989652	0.538404924773607\\
1.69524767103065	0.741992169305531\\
1.94016209417356	0.741514759245998\\
1.82019093577253	0.328368525179733\\
1.4839520831286	0.681796346611881\\
1.13805529399511	0.728796623608036\\
1.73651102992287	0.402875684463721\\
2.40970926685092	0.766519696154169\\
2.12006326132181	0.724614576326541\\
1.33511311753437	0.0606578626326615\\
1.21694480442495	0.939960475214773\\
1.49919597789097	0.779844441507984\\
1.63203786050577	0.615522830684201\\
1.28009907725904	0.603133599618405\\
1.83363385147587	0.447426311107601\\
1.25294254329276	0.891238649047264\\
1.38589448744568	0.110193693911327\\
2.12557258718458	0.480767020439752\\
1.76853338890148	0.345961715962911\\
1.1838175823629	-0.0335760456264687\\
1.07748933668513	0.396363550377259\\
1.91951574719719	0.710847307761236\\
0.890323101344239	0.539672205647595\\
1.65348820931753	0.512238368108479\\
1.65111543031058	-0.265446989923681\\
1.6070900231769	0.488398313760616\\
0.901758997337281	0.183058802837383\\
1.71870592048393	-0.251816153684363\\
1.55442462019974	0.00466747014038243\\
0.795021559669278	0.0361205015350433\\
0.958568445751217	0.212363974452783\\
1.67442948239855	0.4882871131843\\
1.2206685044667	-0.324863168774303\\
1.50767341273933	-0.389108334817502\\
1.37836608274919	-0.0373397947956927\\
1.11156784444486	0.701945681672377\\
1.4984756552433	0.00584512978944459\\
1.24690289872505	0.226154641123381\\
1.09194532469786	-0.130683470031559\\
0.678183222535072	-0.0365335083510088\\
1.73612089885204	0.397031801733272\\
0.933320654619196	0.746473890326733\\
1.15525443706948	-0.179109971394189\\
0.458916732169228	-0.213954536366935\\
1.58359507914698	-0.231297472151952\\
0.931434909226943	0.114256234330456\\
1.1322927346265	0.17900469841877\\
0.780331897364257	0.102401717565955\\
0.501871266865262	-0.0628649708451471\\
0.503048632418745	-0.459092444286199\\
0.604765068083075	-0.011679852416595\\
0.666178137609782	0.219390527881344\\
0.413245356250045	-0.0695882331429949\\
0.467795007897315	-0.0999511872763669\\
1.15246888796731	-0.0589372144022903\\
0.792490565783635	-0.386299556418842\\
0.747652249238335	0.192053680178325\\
0.6730227064451	-0.457495772861639\\
0.657608820873446	-0.397844794344219\\
0.80283938775403	0.00586599616798039\\
1.29819955341968	-0.204261874156702\\
0.700969575790672	-0.360362213508646\\
0.577484279146232	0.667638314415951\\
1.53670668302431	-0.432957390684138\\
-0.157947037327338	-0.206989394214058\\
-0.250033960160869	0.369996686022994\\
0.35679132363435	-0.672812036115748\\
0.307418930825796	-0.491736063949304\\
0.612754451313721	-0.591725661644702\\
0.513152577927538	-0.242776993818146\\
0.457076244801449	-0.714876480294682\\
0.374036335051672	-0.347506625594372\\
0.579283772090358	0.220733907465001\\
0.187849487939868	0.169639639117464\\
-0.0361733332976721	0.281051835212802\\
0.772027729747197	-0.505718105015689\\
0.269625343575084	-0.756939350867156\\
0.60570524403307	-0.119581878977858\\
-0.192532712828272	-0.0344061540276999\\
-0.179793749480468	-0.900222299631903\\
0.0699543465880473	-0.840398840157916\\
0.168779646202848	-0.973477052749994\\
-0.00586739214808801	-0.369383158623758\\
0.465908963186411	-0.973925709442261\\
0.160490995593304	-0.295058807475094\\
0.168397029916368	-0.980178812503749\\
};
\addplot [color=mycolor2,only marks,mark=o,mark options={solid},forget plot]
  table[row sep=crcr]{%
-0.026947097397927	-1.54967503001056\\
-0.0499096052447985	-1.548838725865\\
-0.0878921441942618	-1.54358858745209\\
-0.102243240432917	-1.54118951004161\\
-0.115588964597783	-1.54384949729\\
-0.166126369580897	-1.54954111761082\\
-0.171775059288128	-1.53255169798828\\
-0.202468902552123	-1.53693920317449\\
-0.212030786452003	-1.57194441918315\\
-0.242543396618313	-1.55982750802416\\
-0.268638030522741	-1.55184551810696\\
-0.307999262514236	-1.54141240307698\\
-0.34184706998786	-1.55670987888773\\
-0.321654511497463	-1.58426559222047\\
-0.392925170560618	-1.51563730344866\\
-0.369402481562153	-1.52198515289873\\
-0.416941632627096	-1.51482183374776\\
-0.47214126557288	-1.57339572749443\\
-0.493265130467738	-1.49632259648208\\
-0.512861932358619	-1.50290927620725\\
-0.557340464169725	-1.50842598279773\\
-0.515456646472125	-1.56774163478792\\
-0.608332652640224	-1.42764267683647\\
-0.571584245650045	-1.49817383565417\\
-0.662130333502388	-1.56697697704636\\
-0.669954622356556	-1.57493033524355\\
-0.694374248931051	-1.46017650026605\\
-0.763136406470192	-1.44654664398095\\
-0.64832005525922	-1.5311300933111\\
-0.721079189047473	-1.57176135978772\\
-0.762850186528795	-1.48791858695594\\
-0.771668694696658	-1.4609916619696\\
-0.82794824307844	-1.46985243432682\\
-0.757762090173658	-1.51439340279633\\
-0.73200316350585	-1.54643662607006\\
-0.849650868675982	-1.4739825416715\\
-0.869633988303967	-1.36513677682954\\
-0.929617676304213	-1.40255715057468\\
-0.976307028854808	-1.46241069529678\\
-0.929398084649915	-1.49557897888553\\
-1.08884466383786	-1.60008247625742\\
-1.1061052512272	-1.4667516416831\\
-0.989287983622614	-1.44461809994125\\
-0.985959410143955	-1.5033836797293\\
-1.18771458192431	-1.33744464545877\\
-1.00331746689445	-1.40505396838939\\
-0.892432339761976	-1.52461527768079\\
-1.06850301949737	-1.42382215016215\\
-1.21578154069187	-1.35143372749743\\
-1.1054763672977	-1.37105316655968\\
-1.05324198188984	-1.36449836483918\\
-1.36583986729047	-1.47144251941379\\
-1.21498723375497	-1.58153140750668\\
-1.15864586873963	-1.35211138149604\\
-1.29274883019543	-1.24706936498748\\
-1.21165267830111	-1.36424729300889\\
-1.13339068136518	-1.38791313506574\\
-1.38647009921518	-1.36164505679436\\
-1.38447268580765	-1.24424678906453\\
-1.30846082709955	-1.31326104146817\\
-1.26981825327827	-1.22686897127221\\
-1.49956364455672	-1.20155729619685\\
-1.44709290956236	-1.3499491526295\\
-1.42148110929907	-1.2096773418177\\
-1.48960505179888	-1.41647829527412\\
-1.3264878697695	-1.47045835775913\\
-1.68833293650043	-1.35607977232846\\
-1.58968210104544	-1.26255149104674\\
-1.55268500555184	-1.2291738166282\\
-1.68292402606647	-1.11026634760032\\
-1.72154852078698	-1.24202688963492\\
-1.43590169300641	-1.37871676926254\\
-1.69132652673162	-1.07877471235978\\
-1.70551483678654	-1.42804960005592\\
-1.53520709567707	-1.17299419798963\\
-1.63554863873912	-1.28753234363309\\
-1.65650253897248	-1.09178635476079\\
-1.62198948544192	-1.19413197420841\\
-1.47519706775498	-1.28565950980904\\
-1.43433635492227	-1.32753123102665\\
-1.73478223846456	-1.20660071287845\\
-1.90432859972523	-1.29120302815414\\
-1.9071129230002	-0.972886141766062\\
-1.69183439158024	-1.10780628569212\\
-1.68284658996465	-1.31833231939511\\
-1.85992398322545	-1.01141698924493\\
-1.72772929472913	-0.98157402042344\\
-1.74885476713543	-1.4351574402123\\
-1.84150517090143	-0.95425928858046\\
-1.69315517542604	-0.958656958688715\\
-1.7286495891314	-0.954601665072524\\
-1.81811333022962	-1.19579885376329\\
-1.91064332117842	-1.29835844701107\\
-2.115454982706	-1.0182126578052\\
-1.81014864687136	-1.11941452772307\\
-1.88306383492019	-1.01716868637323\\
-1.86797031445803	-1.04760399233855\\
-1.99290332758122	-1.38532172129099\\
-1.57013566090976	-0.954717363457782\\
-1.6305853956127	-0.926490388852283\\
-1.82503267963498	-1.2054049019875\\
-2.04745234990293	-0.70712356859999\\
-1.91302909485597	-0.765494691747267\\
-1.79315031506749	-1.07048160263372\\
-1.67520961679342	-0.869000607592547\\
-1.98886073309524	-1.32715925014926\\
-2.01375717222897	-0.9315973741601\\
-1.93408159383817	-0.878274086050914\\
-2.26246434296342	-0.886063271151665\\
-1.72639050831228	-0.560229703702508\\
-2.06045827186263	-0.977280310897143\\
-2.21522054773345	-0.832044314048742\\
-2.12802823080291	-0.765383980210819\\
-2.47819250264965	-0.688993880398068\\
-1.68985107972107	-0.732789288101983\\
-1.81469301090601	-0.558603652168838\\
-2.02534666967788	-0.825878531557616\\
-1.90668680745126	-0.687396711733751\\
-2.06031052575321	-0.615192926362136\\
-1.99585448031789	-0.990208999863819\\
-2.14213953096814	-0.765898153601535\\
-1.99073350792875	-0.38534769649198\\
-1.96145912243946	-0.756790477559718\\
-1.88353079013264	-0.493700231213568\\
-2.3752794639133	-0.852864483514567\\
-1.71756604008283	-0.839522058331343\\
-1.96581549537573	-0.658952618483849\\
-2.46496189427916	-0.82131795587749\\
-2.10629520423653	-0.845571433601728\\
-2.11100602865062	-0.528455628514166\\
-1.70854667582737	-0.522179557510307\\
-2.00450257438808	-1.16055808599304\\
-2.21156096039358	-0.482354346628644\\
-2.04160631785436	-0.533352916239695\\
-1.84559801482221	-0.658966536067479\\
-1.92091051009354	-0.644336627288141\\
-1.92864622951692	-0.408174908184284\\
-2.04092091269914	-0.53236208450286\\
-1.86431188937102	-0.409957901472978\\
-1.97521350506084	-0.613741462440286\\
-2.12216417761872	-0.431925911153432\\
-1.80339734046248	-0.583776426535711\\
-1.72974769480169	-0.586333850372364\\
-1.97641639419026	-0.40354436882594\\
-1.66790009602428	-0.535102761781156\\
-1.60550901999747	-0.517641138116707\\
-1.98093745067308	-0.248643233716586\\
-1.77048803449917	-0.0215214325064952\\
-1.82669905771791	-0.344780626915585\\
-2.19905576091282	-0.408420695054419\\
-1.83101161586013	-0.16573497405526\\
-1.99022362231772	-0.240850680829\\
-1.37200304413117	-0.218020520951735\\
-1.83273900042908	0.0375840492151414\\
-1.56236387312051	-0.0622109695320063\\
-1.83847741136532	-0.629170879373212\\
-1.62842916741517	-0.261531442505625\\
-1.760757004596	-0.122954408755528\\
-1.92693522151942	0.0317672214706444\\
-0.795922426677282	-0.305147429881106\\
-1.66083624793154	-0.460640523838259\\
-1.79682003601176	0.166845603602293\\
-1.64022836211743	-0.361636009773446\\
-2.53886522482741	-0.236580562503111\\
-2.03581414704918	0.013541805129293\\
-1.84630209912121	0.230376960175914\\
-1.91209527248309	-0.360580206613225\\
-1.81233556853063	0.0860452228596581\\
-1.85601948484678	-0.446016224299677\\
-1.44425457774794	0.133932704188332\\
-1.82082024175442	-0.524871278676388\\
-1.90283579546103	-0.024980003152043\\
-0.918616001910668	0.598439720318925\\
-1.37397981159442	-0.228688766885865\\
-2.13124001952824	0.0394435337110551\\
-1.51184628119662	-0.339909320146771\\
-1.42078378674312	0.259637468967449\\
-1.46934828152567	-0.48658770179399\\
-1.45036714720006	0.561650634597791\\
-1.6235939585632	0.100495743328196\\
-1.56361953447043	0.429563917963797\\
-1.72011242640493	0.125548136165587\\
-1.70161637144955	-0.00552791997817297\\
-1.3830196422212	0.484897744780771\\
-1.94541153808096	-0.142579452253231\\
-1.17461760721602	0.106879495508337\\
-1.09674397776978	0.918979238595924\\
-1.37782133309489	0.510300838744834\\
-1.88845389859122	0.238194094828098\\
-1.05273228631062	0.534963752730167\\
-1.29045492870113	0.541672326993093\\
-1.51108660201011	0.115095697290448\\
-1.60478470830124	0.761938729066085\\
-1.16297289088485	0.712515972958128\\
-1.36626122649495	0.0459699369006284\\
-1.00569677090725	0.566053726817701\\
-1.15605146422804	0.32048685029173\\
-0.822878786956236	0.299885052175656\\
-1.40380642068769	0.703790875671732\\
-1.59757716568005	0.713604210308872\\
-1.32350395473092	-0.0181115754076439\\
-1.15090617990717	0.0297575564492658\\
-1.21924769269614	0.636116178851027\\
-1.66330401562556	0.274767468591772\\
-1.34530843504664	0.53762093307587\\
-1.39326431751636	1.12504793697234\\
-0.833618774635974	0.512669380246674\\
-0.573367167351866	0.652510525923379\\
-0.276320238657219	0.18763449775689\\
-1.17194492857265	0.572769304168273\\
-0.784840550207506	0.714006761323611\\
-0.924013272173041	0.0557708685451278\\
-1.15282586642296	0.334082023567727\\
-0.935348154153094	0.774875910909737\\
-1.09740398474138	0.634480504642842\\
-1.11946267734234	0.610532582269006\\
-0.838968288924051	1.38322547926388\\
-1.22448365542895	1.32213641543783\\
-0.476481839809044	1.40105426500117\\
-0.769114345508466	0.66169599251629\\
-1.40027835036432	1.28884494337088\\
-1.50132634028759	0.613610138665922\\
-0.949826176623956	0.594374431492637\\
-0.88164099127801	0.552395160007791\\
-0.994300613393627	0.256827952699087\\
-0.632450277038536	0.881951176938878\\
-0.331331900379547	0.85471608258404\\
-1.13074002827094	0.521881807762682\\
-0.86984371951327	0.417083028748866\\
-0.420618985877099	0.726297500339449\\
-0.110157996853952	0.742495519066295\\
0.644069697204106	0.64901805109848\\
-0.262085393832434	0.700643157487521\\
-0.276442795861882	0.467852427149516\\
-0.14443352788167	0.705856371233983\\
-0.727695229224663	1.06740698127406\\
-0.626210638028053	0.986934718277524\\
-0.658293890852093	1.48386542068786\\
-0.71066942064998	0.843047675153383\\
0.185985214616693	0.552601466396477\\
-0.615839609456226	0.931267504635312\\
-0.533903453015464	1.49417117664361\\
-0.278043832459567	1.40015122194036\\
-0.374673778674507	0.760845850670364\\
0.0347222292805265	1.27907372368006\\
-0.41631147842867	0.89614152469315\\
-0.148017321318051	0.860943748299246\\
-0.318167564411286	1.66905910965134\\
0.565869201509072	1.63328081306408\\
0.456182094678272	1.2472468033001\\
};
\addplot[contour prepared, contour prepared format=matlab, contour/labels=false] table[row sep=crcr] {%
%
-0.5	23\\
-1.99905814885931	-0.5\\
-2	-0.500133723625852\\
-2.0004363842562	-0.5\\
-2.1	-0.44036160358306\\
-2.12480555292848	-0.4\\
-2.13949728290364	-0.3\\
-2.11824865708978	-0.2\\
-2.1	-0.163906808958695\\
-2.05502763932388	-0.1\\
-2	-0.0466151525353341\\
-1.91822023027297	0\\
-1.9	0.00815742441956426\\
-1.8	0.0141707513122246\\
-1.76510949477104	0\\
-1.7	-0.0525535751322851\\
-1.6767922083847	-0.1\\
-1.67267901163111	-0.2\\
-1.7	-0.2925469661608\\
-1.70298528190328	-0.3\\
-1.77927594689506	-0.4\\
-1.8	-0.420167060755567\\
-1.9	-0.482116124671566\\
-1.99905814885931	-0.5\\
-0.4	47\\
-1.98489138295682	-0.8\\
-2	-0.805060091385689\\
-2.1	-0.814985356055461\\
-2.19344952403964	-0.8\\
-2.2	-0.798639389607547\\
-2.3	-0.741583299410887\\
-2.34254753944296	-0.7\\
-2.4	-0.614823294602659\\
-2.40759670943997	-0.6\\
-2.43831581041222	-0.5\\
-2.44949550258463	-0.4\\
-2.4441028771929	-0.3\\
-2.4237266621161	-0.2\\
-2.4	-0.131559602566179\\
-2.38799813334859	-0.1\\
-2.33501857709701	0\\
-2.3	0.0518426656230525\\
-2.26232736508623	0.1\\
-2.2	0.166739061157658\\
-2.16196989508252	0.2\\
-2.1	0.247975004630187\\
-2.01186610644646	0.3\\
-2	0.306492994238831\\
-1.9	0.344080061505899\\
-1.8	0.363747606558536\\
-1.7	0.363815690907172\\
-1.6	0.340367132577877\\
-1.52376818313032	0.3\\
-1.5	0.283494431181304\\
-1.42770192881825	0.2\\
-1.4	0.146204865877756\\
-1.38294284139194	0.1\\
-1.36788752018958	0\\
-1.37286326220627	-0.1\\
-1.39469059391835	-0.2\\
-1.4	-0.21378910719565\\
-1.4352036397849	-0.3\\
-1.4892620319271	-0.4\\
-1.5	-0.415791294512327\\
-1.56261442473052	-0.5\\
-1.6	-0.543614259486498\\
-1.65512702754579	-0.6\\
-1.7	-0.641375663721785\\
-1.77636696981402	-0.7\\
-1.8	-0.716968019885662\\
-1.9	-0.770611637846997\\
-1.98489138295682	-0.8\\
-0.3	65\\
-1.9640536003253	-1\\
-2	-1.01596815052357\\
-2.1	-1.04203286257822\\
-2.2	-1.04928191701934\\
-2.3	-1.03496099322811\\
-2.38936744295805	-1\\
-2.4	-0.995011816343002\\
-2.5	-0.91937213544083\\
-2.51864983532103	-0.9\\
-2.59194779495298	-0.8\\
-2.6	-0.784587276812\\
-2.63842292110516	-0.7\\
-2.66734064569315	-0.6\\
-2.68224644977903	-0.5\\
-2.68506711337492	-0.4\\
-2.67688788602771	-0.3\\
-2.65813513970509	-0.2\\
-2.62867313032671	-0.1\\
-2.6	-0.0289302371035772\\
-2.58796163911656	0\\
-2.5354724577869	0.1\\
-2.5	0.155372682615215\\
-2.4691165746496	0.2\\
-2.4	0.285257035542166\\
-2.38657895823226	0.3\\
-2.3	0.384946104159482\\
-2.28204677271534	0.4\\
-2.2	0.463946245249311\\
-2.14335649105381	0.5\\
-2.1	0.526584514306788\\
-2	0.574860426602621\\
-1.92846519583615	0.6\\
-1.9	0.610080155547817\\
-1.8	0.632689185633718\\
-1.7	0.641962706593195\\
-1.6	0.636812120428373\\
-1.5	0.61483731997157\\
-1.46361756716474	0.6\\
-1.4	0.570885380193629\\
-1.30522218190779	0.5\\
-1.3	0.495146636605362\\
-1.22706521514412	0.4\\
-1.2	0.348039660919931\\
-1.18036817302073	0.3\\
-1.15610995359754	0.2\\
-1.14815965028584	0.1\\
-1.15453179795759	0\\
-1.17386743706017	-0.1\\
-1.2	-0.182528988156613\\
-1.2055369569812	-0.2\\
-1.25024373825037	-0.3\\
-1.3	-0.390281212669749\\
-1.30548056132959	-0.4\\
-1.37347666325642	-0.5\\
-1.4	-0.534367557446948\\
-1.45308616355651	-0.6\\
-1.5	-0.652990763658741\\
-1.54499246443704	-0.7\\
-1.6	-0.754200944201568\\
-1.65209497904126	-0.8\\
-1.7	-0.840977675068836\\
-1.78063501203372	-0.9\\
-1.8	-0.914223948164177\\
-1.9	-0.972544844742308\\
-1.9640536003253	-1\\
-0.3	45\\
1.71915367607952	1.5\\
1.7	1.491366393692\\
1.6	1.46209997452352\\
1.5	1.44834431986758\\
1.4	1.44729835366639\\
1.3	1.45689678427305\\
1.2	1.47566822887228\\
1.10990019687903	1.5\\
1.1	1.50336383847252\\
1	1.54703670729356\\
0.9	1.59931900078326\\
0.898867832102558	1.6\\
0.8	1.68193778302415\\
0.78044373847102	1.7\\
0.711871735105705	1.8\\
0.7	1.8384893337898\\
0.682167669259685	1.9\\
0.687241686623202	2\\
0.7	2.03284337504202\\
0.731177982972174	2.1\\
0.8	2.17660741740752\\
0.827774138993975	2.2\\
0.9	2.24268938741539\\
1	2.28239585224489\\
1.0785826642393	2.3\\
1.1	2.30399015049166\\
1.2	2.31012527968845\\
1.3	2.30409628573232\\
1.32297917505402	2.3\\
1.4	2.28484124875689\\
1.5	2.25243586927695\\
1.6	2.20696682839803\\
1.61198137341675	2.2\\
1.7	2.13899516446605\\
1.7461001146574	2.1\\
1.8	2.04041466118117\\
1.83155918052621	2\\
1.88306195855241	1.9\\
1.9	1.82556313806319\\
1.90534412039401	1.8\\
1.9	1.73724952165104\\
1.89627947071669	1.7\\
1.84525478466286	1.6\\
1.8	1.55582983370438\\
1.71915367607952	1.5\\
-0.2	85\\
-1.97439537831403	-1.2\\
-2	-1.2149320772669\\
-2.1	-1.25571179214429\\
-2.2	-1.28029239872555\\
-2.3	-1.28786284476777\\
-2.4	-1.27692995055961\\
-2.5	-1.24507201705556\\
-2.58184112250253	-1.2\\
-2.6	-1.18856846367845\\
-2.7	-1.10128540149259\\
-2.70120078897917	-1.1\\
-2.77913407316979	-1\\
-2.8	-0.964973209974712\\
-2.83543237231558	-0.9\\
-2.87567667259948	-0.8\\
-2.9	-0.711536820388351\\
-2.9031943199455	-0.7\\
-2.92116727426243	-0.6\\
-2.92937611706343	-0.5\\
-2.92866768671396	-0.4\\
-2.91940461724011	-0.3\\
-2.90152068323772	-0.2\\
-2.9	-0.194153314068729\\
-2.87686640782561	-0.1\\
-2.84346114674939	0\\
-2.8000043814739	0.1\\
-2.8	0.100008598574892\\
-2.74910656726109	0.2\\
-2.7	0.279050005092447\\
-2.68656192753509	0.3\\
-2.61293403828098	0.4\\
-2.6	0.415749510694025\\
-2.52555506524719	0.5\\
-2.5	0.526391005613894\\
-2.42078647629576	0.6\\
-2.4	0.618231611883596\\
-2.3	0.695065818312758\\
-2.29255271507173	0.7\\
-2.2	0.760515135913084\\
-2.12568235737718	0.8\\
-2.1	0.813858497260071\\
-2	0.857907619975404\\
-1.9	0.891027175697234\\
-1.86161404104414	0.9\\
-1.8	0.915408080612244\\
-1.7	0.930015189331237\\
-1.6	0.933931372444417\\
-1.5	0.926270843601872\\
-1.4	0.905146909943458\\
-1.38542818953517	0.9\\
-1.3	0.868659300114465\\
-1.2	0.809935039562434\\
-1.18767444694261	0.8\\
-1.1	0.71742121821157\\
-1.08617275992935	0.7\\
-1.02230321620636	0.6\\
-1	0.551575920243745\\
-0.980506207890227	0.5\\
-0.955220614897106	0.4\\
-0.942552122017628	0.3\\
-0.941049453474766	0.2\\
-0.949904912924098	0.1\\
-0.968779324611778	0\\
-0.997676016236761	-0.1\\
-1	-0.105724364451025\\
-1.03696986176234	-0.2\\
-1.08607248582628	-0.3\\
-1.1	-0.323130362774498\\
-1.14534755640997	-0.4\\
-1.2	-0.479890238113755\\
-1.21371440021735	-0.5\\
-1.29072828346961	-0.6\\
-1.3	-0.6109953310957\\
-1.37628106597256	-0.7\\
-1.4	-0.726202374011889\\
-1.46965301436706	-0.8\\
-1.5	-0.831369083385768\\
-1.57131527600986	-0.9\\
-1.6	-0.927736330526302\\
-1.68302799281739	-1\\
-1.7	-1.01527635029145\\
-1.8	-1.09304826985736\\
-1.81091229752734	-1.1\\
-1.9	-1.16043044525074\\
-1.97439537831403	-1.2\\
-0.2	69\\
2.01321512678451	1.3\\
2	1.29371515216809\\
1.9	1.25710118536281\\
1.8	1.23324336324839\\
1.7	1.21968142489198\\
1.6	1.2145413338266\\
1.5	1.21639303910559\\
1.4	1.22415011979727\\
1.3	1.23699976989858\\
1.2	1.25435519154214\\
1.1	1.27582500014404\\
1.00467043447567	1.3\\
1	1.30134555728165\\
0.9	1.33409781320293\\
0.8	1.3710874091365\\
0.729602681286473	1.4\\
0.7	1.41473513358682\\
0.6	1.46884393962413\\
0.547196387204895	1.5\\
0.5	1.53605093025039\\
0.422065530229951	1.6\\
0.4	1.62529581482257\\
0.337413867970155	1.7\\
0.3	1.76999602522771\\
0.28409661167469	1.8\\
0.255911900009964	1.9\\
0.250121648207229	2\\
0.265340447887314	2.1\\
0.3	2.19417348171605\\
0.302263377860216	2.2\\
0.363781850273408	2.3\\
0.4	2.34215747229138\\
0.457126081921146	2.4\\
0.5	2.43474736646296\\
0.598648624665921	2.5\\
0.6	2.50077723151391\\
0.7	2.54837713183254\\
0.8	2.58337116861859\\
0.867457107709788	2.6\\
0.9	2.60770248010475\\
1	2.62285300464\\
1.1	2.62957246009309\\
1.2	2.62838596655393\\
1.3	2.61950450889744\\
1.4	2.60282997940094\\
1.41162633110365	2.6\\
1.5	2.57881262511994\\
1.6	2.54650777726195\\
1.7	2.50476826131564\\
1.70955685599987	2.5\\
1.8	2.45294522687816\\
1.88359891067816	2.4\\
1.9	2.38867649925518\\
2	2.30870464468856\\
2.00966670491614	2.3\\
2.1	2.20581513735131\\
2.10508134375649	2.2\\
2.17763735816218	2.1\\
2.2	2.05934321477979\\
2.2313515279099	2\\
2.26767294483768	1.9\\
2.28723477475033	1.8\\
2.28955186132578	1.7\\
2.27239648008394	1.6\\
2.23096252637221	1.5\\
2.2	1.45604120531152\\
2.1544088895082	1.4\\
2.1	1.35521622833057\\
2.01321512678451	1.3\\
-0.1	88\\
-3	0.383144949928337\\
-2.99050231251812	0.4\\
-2.92670117594257	0.5\\
-2.9	0.536762439565806\\
-2.85330119418039	0.6\\
-2.8	0.66435627367933\\
-2.76909464005947	0.7\\
-2.7	0.773416744896755\\
-2.67304159669187	0.8\\
-2.6	0.86842716729353\\
-2.56262781072857	0.9\\
-2.5	0.951740896331269\\
-2.43332951365827	1\\
-2.4	1.02424575384081\\
-2.3	1.08757244892769\\
-2.27724201445364	1.1\\
-2.2	1.1440581908535\\
-2.1	1.19058971562075\\
-2.07568612299227	1.2\\
-2	1.23175282724955\\
-1.9	1.26476363674027\\
-1.8	1.28946027037286\\
-1.73924193419481	1.3\\
-1.7	1.30775919724512\\
-1.6	1.3195183989728\\
-1.5	1.32319709242344\\
-1.4	1.31846924064942\\
-1.3	1.30435925642632\\
-1.28155465418412	1.3\\
-1.2	1.28135063405936\\
-1.1	1.24521755927432\\
-1.01778323076445	1.2\\
-1	1.1894339575415\\
-0.9	1.10392082541534\\
-0.896597696969557	1.1\\
-0.824787057849913	1\\
-0.8	0.952414774837168\\
-0.778339967542093	0.9\\
-0.748052204169371	0.8\\
-0.728650270521467	0.7\\
-0.717563285253912	0.6\\
-0.713432359987947	0.5\\
-0.715611148996508	0.4\\
-0.723927029271193	0.3\\
-0.738565000383961	0.2\\
-0.760012632922898	0.1\\
-0.78903566897196	0\\
-0.8	-0.0284947303332114\\
-0.826014180363051	-0.1\\
-0.871668911022412	-0.2\\
-0.9	-0.250571878158638\\
-0.926470483504461	-0.3\\
-0.990158485621222	-0.4\\
-1	-0.413282336505866\\
-1.06194408396605	-0.5\\
-1.1	-0.547454064526061\\
-1.14114116203211	-0.6\\
-1.2	-0.669021528662932\\
-1.22623376282692	-0.7\\
-1.3	-0.782473963727782\\
-1.31590574917684	-0.8\\
-1.4	-0.890438885916022\\
-1.40925763294852	-0.9\\
-1.5	-0.994226898917507\\
-1.50596519451184	-1\\
-1.6	-1.09418352121971\\
-1.60644398834981	-1.1\\
-1.7	-1.18988523442082\\
-1.71212319564557	-1.2\\
-1.8	-1.28021523756214\\
-1.82598927294914	-1.3\\
-1.9	-1.36333166108873\\
-1.95369539810375	-1.4\\
-2	-1.43651202712987\\
-2.1	-1.4963729130246\\
-2.10871189788428	-1.5\\
-2.2	-1.54528091897305\\
-2.3	-1.57670309713365\\
-2.4	-1.59149013369314\\
-2.5	-1.5899338357462\\
-2.6	-1.57121444532659\\
-2.7	-1.53326936208292\\
-2.75843596258057	-1.5\\
-2.8	-1.4746705438843\\
-2.8917806576295	-1.4\\
-2.9	-1.39242570718543\\
-2.98498323448383	-1.3\\
-3	-1.28042510420619\\
-0.1	99\\
2.31896923983921	1.1\\
2.3	1.09208373215515\\
2.2	1.05865696290908\\
2.1	1.03603141309355\\
2	1.02194329414892\\
1.9	1.01469657124102\\
1.8	1.01300511440054\\
1.7	1.01588151785379\\
1.6	1.02255849794419\\
1.5	1.0324333866923\\
1.4	1.04502926514152\\
1.3	1.05996826494558\\
1.2	1.07695385989825\\
1.1	1.0957598126169\\
1.07905189475021	1.1\\
1	1.11699097845451\\
0.9	1.14000920080013\\
0.8	1.16457845431946\\
0.7	1.19073922047228\\
0.665442936480691	1.2\\
0.6	1.2199806012997\\
0.5	1.25189285802246\\
0.4	1.28613466329005\\
0.360273338738243	1.3\\
0.3	1.32562532282622\\
0.2	1.37047610992334\\
0.137939089063721	1.4\\
0.0999999999999996	1.42354178110108\\
0	1.4894187500242\\
-0.0156545644903221	1.5\\
-0.0999999999999996	1.58215006057455\\
-0.117619851904679	1.6\\
-0.181734569813954	1.7\\
-0.2	1.74967755203042\\
-0.219275722868466	1.8\\
-0.237799972176544	1.9\\
-0.240356973480497	2\\
-0.229306226451457	2.1\\
-0.204956080323635	2.2\\
-0.2	2.21337526172519\\
-0.169386824713184	2.3\\
-0.11875923409801	2.4\\
-0.0999999999999996	2.42894748435022\\
-0.052568796627106	2.5\\
0	2.56277802623196\\
0.0336963618160049	2.6\\
0.0999999999999996	2.66223530060263\\
0.145674495881218	2.7\\
0.2	2.74042069435819\\
0.295642965721203	2.8\\
0.3	2.80257389642069\\
0.4	2.85420488131268\\
0.5	2.89441356218787\\
0.517023917954894	2.9\\
0.6	2.92792332924944\\
0.7	2.95353272790378\\
0.8	2.97190017316712\\
0.9	2.98392592412794\\
1	2.99023758063731\\
1.1	2.99122977372914\\
1.2	2.98708720585811\\
1.3	2.97779320050813\\
1.4	2.963124539864\\
1.5	2.94263203467129\\
1.6	2.91560482422916\\
1.64666230450731	2.9\\
1.7	2.88344955694939\\
1.8	2.84564076321214\\
1.89728262532493	2.8\\
1.9	2.79876451094306\\
2	2.74697540567837\\
2.07536013530645	2.7\\
2.1	2.68441655291207\\
2.2	2.61214772672774\\
2.21513837866471	2.6\\
2.3	2.52743649721738\\
2.32906113032895	2.5\\
2.4	2.42564819669784\\
2.42292739637388	2.4\\
2.49990403941635	2.3\\
2.5	2.29985441740256\\
2.56486005297933	2.2\\
2.6	2.1305733659702\\
2.61579473407136	2.1\\
2.65619024710663	2\\
2.68406821493518	1.9\\
2.7	1.80357874836577\\
2.70064968317304	1.8\\
2.70671422685892	1.7\\
2.7	1.60656012668534\\
2.69956988341727	1.6\\
2.67936092170612	1.5\\
2.64139278559524	1.4\\
2.6	1.33184344524445\\
2.57974321693553	1.3\\
2.5	1.21513774214751\\
2.48368078804344	1.2\\
2.4	1.14250413817906\\
2.31896923983921	1.1\\
0	107\\
3	0.670895364091001\\
2.9	0.677040960085266\\
2.8	0.684325056102779\\
2.7	0.692741214064507\\
2.62766782067653	0.7\\
2.6	0.702097180152401\\
2.5	0.711895841889273\\
2.4	0.7227200125555\\
2.3	0.734555830945959\\
2.2	0.747383726117829\\
2.1	0.761176290368746\\
2	0.775895912675981\\
1.9	0.791492307259451\\
1.85016106301908	0.8\\
1.8	0.807553112336373\\
1.7	0.823944976935563\\
1.6	0.840954933670556\\
1.5	0.858478493934987\\
1.4	0.876394894942564\\
1.3	0.894567746621794\\
1.27045903130663	0.9\\
1.2	0.912699064263798\\
1.1	0.930727996453405\\
1	0.948554642257712\\
0.9	0.965998477884254\\
0.8	0.982855311022979\\
0.7	0.998883916774768\\
0.692090837782924	1\\
0.6	1.01404045924791\\
0.5	1.02769044955155\\
0.4	1.0392583185042\\
0.3	1.04793147635779\\
0.2	1.05247633259698\\
0.0999999999999996	1.0509417910292\\
0	1.04007325676464\\
-0.0999999999999996	1.01401885092281\\
-0.131324746488689	1\\
-0.2	0.962464954776398\\
-0.267649046475316	0.9\\
-0.3	0.86116611942428\\
-0.336662406265561	0.8\\
-0.382720131883132	0.7\\
-0.4	0.655363956300719\\
-0.417672211569305	0.6\\
-0.447099150057091	0.5\\
-0.474785291389262	0.4\\
-0.5	0.311123255041197\\
-0.502825246232811	0.3\\
-0.532152914161151	0.2\\
-0.565271280587037	0.1\\
-0.6	0.010601612701646\\
-0.603835406511441	0\\
-0.647943446227753	-0.1\\
-0.7	-0.198556066710695\\
-0.700720005157546	-0.2\\
-0.760757038285438	-0.3\\
-0.8	-0.355315343797258\\
-0.830091892887411	-0.4\\
-0.9	-0.489744192786042\\
-0.907620174145114	-0.5\\
-0.991214994946121	-0.6\\
-1	-0.609592795074304\\
-1.07935812775537	-0.7\\
-1.1	-0.722168925783512\\
-1.17039021716301	-0.8\\
-1.2	-0.831886473711267\\
-1.26259423378851	-0.9\\
-1.3	-0.940932338455696\\
-1.35470006671335	-1\\
-1.4	-1.05073840691264\\
-1.44583728731596	-1.1\\
-1.5	-1.16219410860777\\
-1.53541864420223	-1.2\\
-1.6	-1.27576248134763\\
-1.62301231013362	-1.3\\
-1.7	-1.39153907779385\\
-1.70822771993638	-1.4\\
-1.79204589411599	-1.5\\
-1.8	-1.51135487051403\\
-1.87512641660387	-1.6\\
-1.9	-1.6358521821169\\
-1.95613250229173	-1.7\\
-2	-1.76256765677169\\
-2.03456949737199	-1.8\\
-2.1	-1.89029538242761\\
-2.10969224648766	-1.9\\
-2.185496484472	-2\\
-2.2	-2.02564649901383\\
-2.2610707665652	-2.1\\
-2.3	-2.16412499602944\\
-2.33318962942112	-2.2\\
-2.4	-2.29968038042841\\
-2.4003434577777	-2.3\\
-2.47372476495315	-2.4\\
-2.5	-2.45074317836333\\
-2.54319268593725	-2.5\\
-2.6	-2.5939189037158\\
-2.60657722724523	-2.6\\
-2.67579464149818	-2.7\\
-2.7	-2.75202954070264\\
-2.74220233372028	-2.8\\
-2.8	-2.89956623479554\\
-2.80050531132027	-2.9\\
-2.86928953851458	-3\\
-2.9	-3.06762563816423\\
-2.93123002426935	-3.1\\
-2.98971251492905	-3.2\\
0.1	157\\
-0.832287549338605	-2.8\\
-0.8	-2.80541421432706\\
-0.7	-2.8155185814163\\
-0.6	-2.81905343575408\\
-0.5	-2.81623309656419\\
-0.4	-2.8069507299769\\
-0.356085481260723	-2.8\\
-0.3	-2.79231680510456\\
-0.2	-2.77246748122528\\
-0.0999999999999996	-2.74527703244311\\
0	-2.7093705729757\\
0.0217778680493097	-2.7\\
0.0999999999999996	-2.66835282431535\\
0.2	-2.617602058778\\
0.229783904620448	-2.6\\
0.3	-2.55897943243726\\
0.384324492198335	-2.5\\
0.4	-2.48878960704439\\
0.5	-2.40813331742625\\
0.509237007723348	-2.4\\
0.6	-2.31518350387666\\
0.615151409423915	-2.3\\
0.7	-2.20798078528971\\
0.707125260026812	-2.2\\
0.790255927310179	-2.1\\
0.8	-2.08729397724886\\
0.868120009987515	-2\\
0.9	-1.95581958253565\\
0.942125378146426	-1.9\\
1	-1.81920490932853\\
1.01468806885946	-1.8\\
1.09051502903304	-1.7\\
1.1	-1.68767228742272\\
1.17315808427793	-1.6\\
1.2	-1.56872625084434\\
1.26371098616581	-1.5\\
1.3	-1.46252126119488\\
1.36437917820707	-1.4\\
1.4	-1.36694183768329\\
1.47487119929634	-1.3\\
1.5	-1.27832534284763\\
1.59167800814651	-1.2\\
1.6	-1.19301779482019\\
1.7	-1.10861298191798\\
1.71001930459449	-1.1\\
1.8	-1.02150865468891\\
1.82371825220881	-1\\
1.9	-0.927769232248656\\
1.92815254879271	-0.9\\
2	-0.823597639386622\\
2.02144924251072	-0.8\\
2.1	-0.703453418770733\\
2.10275406523595	-0.7\\
2.17409852102884	-0.6\\
2.2	-0.556300516290622\\
2.23374376504565	-0.5\\
2.28183325830883	-0.4\\
2.3	-0.35025081466811\\
2.31931723129838	-0.3\\
2.34575054592769	-0.2\\
2.35918392599594	-0.1\\
2.35863477557352	0\\
2.34153797640692	0.1\\
2.30276936112217	0.2\\
2.3	0.204327082375234\\
2.23945452985649	0.3\\
2.2	0.339825223747914\\
2.13802910836696	0.4\\
2.1	0.426024908400365\\
2	0.489770193981069\\
1.98308336730898	0.5\\
1.9	0.538720563751779\\
1.8	0.580995069254703\\
1.75196835350277	0.6\\
1.7	0.617074248422577\\
1.6	0.648146890858733\\
1.5	0.676722152598849\\
1.41191088070672	0.7\\
1.4	0.70282618251179\\
1.3	0.72533225863569\\
1.2	0.745935840495318\\
1.1	0.764544998262238\\
1	0.780981422498865\\
0.9	0.794978754419798\\
0.854334059752209	0.8\\
0.8	0.805977224267777\\
0.7	0.813661301446775\\
0.6	0.817533023860807\\
0.5	0.816667666552153\\
0.4	0.809754182611612\\
0.332296284341345	0.8\\
0.3	0.794806856671223\\
0.2	0.768872534308333\\
0.0999999999999996	0.72781171079188\\
0.0524334584787649	0.7\\
0	0.664282961505891\\
-0.0697410317020116	0.6\\
-0.0999999999999996	0.567215618293462\\
-0.149875200140302	0.5\\
-0.2	0.421288386937451\\
-0.211494370621519	0.4\\
-0.262360035413389	0.3\\
-0.3	0.222061475925546\\
-0.309548875510636	0.2\\
-0.354843711979808	0.1\\
-0.4	0.0069667543213807\\
-0.403133341684418	0\\
-0.454435283147218	-0.1\\
-0.5	-0.176274166061664\\
-0.513370100222342	-0.2\\
-0.57986230805529	-0.3\\
-0.6	-0.325694351404936\\
-0.655172719481208	-0.4\\
-0.7	-0.451796963734665\\
-0.739423144226292	-0.5\\
-0.8	-0.565182658322858\\
-0.83054405602707	-0.6\\
-0.9	-0.671862903518825\\
-0.925743908645458	-0.7\\
-1	-0.776167137181608\\
-1.02218133635384	-0.8\\
-1.1	-0.881195511281029\\
-1.11745747470957	-0.9\\
-1.2	-0.989315701232298\\
-1.20976172008621	-1\\
-1.29773943433493	-1.1\\
-1.3	-1.10276534770355\\
-1.38077145147897	-1.2\\
-1.4	-1.22570904107967\\
-1.4582777684601	-1.3\\
-1.5	-1.36091070343472\\
-1.52907222833528	-1.4\\
-1.59191868200472	-1.5\\
-1.6	-1.51611154530967\\
-1.64780136805132	-1.6\\
-1.69247034992527	-1.7\\
-1.7	-1.72431152878461\\
-1.72800353106994	-1.8\\
-1.7507476441775	-1.9\\
-1.75933930274654	-2\\
-1.75252348463268	-2.1\\
-1.7280784299783	-2.2\\
-1.7	-2.26531806145466\\
-1.68388289814442	-2.3\\
-1.61741485459673	-2.4\\
-1.6	-2.42079437758582\\
-1.52312119610314	-2.5\\
-1.5	-2.52057871444111\\
-1.4	-2.59471439449184\\
-1.39142233170155	-2.6\\
-1.3	-2.65475921126385\\
-1.20025352138914	-2.7\\
-1.2	-2.70011877668489\\
-1.1	-2.73955320840436\\
-1	-2.76890238633588\\
-0.9	-2.79011998871965\\
-0.832287549338605	-2.8\\
0.2	135\\
-0.811549004370996	-2.5\\
-0.8	-2.50249401250061\\
-0.7	-2.51565470275251\\
-0.6	-2.52039179121663\\
-0.5	-2.51708197576582\\
-0.4	-2.50567750168653\\
-0.370799613186468	-2.5\\
-0.3	-2.48711661874582\\
-0.2	-2.46065724649945\\
-0.0999999999999996	-2.4245837714513\\
-0.0460916925454747	-2.4\\
0	-2.3789729075096\\
0.0999999999999996	-2.3229404498097\\
0.134708364568663	-2.3\\
0.2	-2.25474948104453\\
0.26806697065728	-2.2\\
0.3	-2.17222630568382\\
0.375059901630763	-2.1\\
0.4	-2.07344026329256\\
0.465117377884432	-2\\
0.5	-1.95594897279409\\
0.543528367992521	-1.9\\
0.6	-1.81905972696062\\
0.61353376041391	-1.8\\
0.679683564162582	-1.7\\
0.7	-1.66737697693623\\
0.744600098326379	-1.6\\
0.8	-1.51412593931162\\
0.809874549955101	-1.5\\
0.881096435052903	-1.4\\
0.9	-1.37451446006357\\
0.960597723406271	-1.3\\
1	-1.25425419293221\\
1.05075069180658	-1.2\\
1.1	-1.15026386736098\\
1.15290417173448	-1.1\\
1.2	-1.05710396434962\\
1.26472250573009	-1\\
1.3	-0.969428618005571\\
1.3802544231065	-0.9\\
1.4	-0.882698179395259\\
1.49227219748725	-0.8\\
1.5	-0.792728678220709\\
1.59497759957364	-0.7\\
1.6	-0.694634632944201\\
1.68507883395807	-0.6\\
1.7	-0.580926338067012\\
1.7611832025426	-0.5\\
1.8	-0.43722473654418\\
1.82256088041235	-0.4\\
1.86926481933508	-0.3\\
1.9	-0.20119110501243\\
1.90037498762739	-0.2\\
1.91584619617529	-0.1\\
1.91290757943909	0\\
1.9	0.0533366045497275\\
1.88870365159513	0.1\\
1.83842744486735	0.2\\
1.8	0.246690115549617\\
1.75321693835206	0.3\\
1.7	0.342166605570091\\
1.61910768655855	0.4\\
1.6	0.410417884676984\\
1.5	0.460146879105056\\
1.40755790303454	0.5\\
1.4	0.502703055197247\\
1.3	0.53464401919933\\
1.2	0.561751244507775\\
1.1	0.584060294133704\\
1.00823863628493	0.6\\
1	0.601309974478742\\
0.9	0.612444127839403\\
0.8	0.618445123226849\\
0.7	0.618614095713047\\
0.6	0.611974906862368\\
0.518220035280341	0.6\\
0.5	0.596978554519203\\
0.4	0.570649299162082\\
0.3	0.531248420511867\\
0.24228105525981	0.5\\
0.2	0.473583471411276\\
0.109187075653846	0.4\\
0.0999999999999996	0.391440305318224\\
0.0187195955188328	0.3\\
0	0.276365608400133\\
-0.0525711255354456	0.2\\
-0.0999999999999996	0.126457130184648\\
-0.11534862618838	0.1\\
-0.173792106717834	0\\
-0.2	-0.0424711905175911\\
-0.233237862552171	-0.1\\
-0.298275878891854	-0.2\\
-0.3	-0.202326651432696\\
-0.369339668706806	-0.3\\
-0.4	-0.336449266274878\\
-0.451214308409738	-0.4\\
-0.5	-0.451335257495559\\
-0.544021305837869	-0.5\\
-0.6	-0.553603018647519\\
-0.645676012537027	-0.6\\
-0.7	-0.649283746976093\\
-0.752244850415296	-0.7\\
-0.8	-0.742899095997276\\
-0.859186854479199	-0.8\\
-0.9	-0.837839542291609\\
-0.962663140481915	-0.9\\
-1	-0.936993477819239\\
-1.0601457495717	-1\\
-1.1	-1.04333814524207\\
-1.15024660703862	-1.1\\
-1.2	-1.16048628260663\\
-1.23215144796213	-1.2\\
-1.3	-1.29324870655208\\
-1.30500842152475	-1.3\\
-1.36796699996734	-1.4\\
-1.4	-1.46228439718025\\
-1.42059798026049	-1.5\\
-1.46164025217387	-1.6\\
-1.48966509193888	-1.7\\
-1.5	-1.77603837186488\\
-1.50367009468762	-1.8\\
-1.5017550886293	-1.9\\
-1.5	-1.90952594442117\\
-1.48186702197113	-2\\
-1.44078342232138	-2.1\\
-1.4	-2.16433210484731\\
-1.37361559507981	-2.2\\
-1.3	-2.27649568142299\\
-1.27217438613206	-2.3\\
-1.2	-2.35303088412451\\
-1.11673991252757	-2.4\\
-1.1	-2.40892989350063\\
-1	-2.45146939934069\\
-0.9	-2.48183228465625\\
-0.811549004370996	-2.5\\
0.3	111\\
-0.994002507000206	-2.2\\
-0.9	-2.24148374438844\\
-0.8	-2.27107835930396\\
-0.7	-2.28844707461311\\
-0.6	-2.29507838679235\\
-0.5	-2.29172185588938\\
-0.4	-2.27854588577209\\
-0.3	-2.2552170219706\\
-0.2	-2.22092403643791\\
-0.153574598525486	-2.2\\
-0.0999999999999996	-2.17472532837494\\
0	-2.11526334733207\\
0.0216897994278071	-2.1\\
0.0999999999999996	-2.03963818819368\\
0.144677029696215	-2\\
0.2	-1.94464560964147\\
0.240829715692824	-1.9\\
0.3	-1.82572802363644\\
0.319645366541472	-1.8\\
0.386664533414034	-1.7\\
0.4	-1.67736914811363\\
0.446390904775572	-1.6\\
0.5	-1.5012976597277\\
0.500742067884398	-1.5\\
0.554738589440407	-1.4\\
0.6	-1.31617761136787\\
0.609568531400957	-1.3\\
0.670407535617583	-1.2\\
0.7	-1.15511921084315\\
0.740512388591236	-1.1\\
0.8	-1.02657923959921\\
0.823755203064771	-1\\
0.9	-0.921264307762674\\
0.922130444727939	-0.9\\
1	-0.828703269117485\\
1.03248617868056	-0.8\\
1.1	-0.74081205110103\\
1.14645270742654	-0.7\\
1.2	-0.651081764698706\\
1.25414419714938	-0.6\\
1.3	-0.552494730822967\\
1.34823218808079	-0.5\\
1.4	-0.433813678961674\\
1.42508095246501	-0.4\\
1.48270109387789	-0.3\\
1.5	-0.254848837664009\\
1.52012554921011	-0.2\\
1.53554242516108	-0.1\\
1.52645768733162	0\\
1.5	0.069758659951685\\
1.48726324816925	0.1\\
1.40729278353551	0.2\\
1.4	0.206013975929727\\
1.3	0.276603281751039\\
1.25892100807281	0.3\\
1.2	0.32544783710441\\
1.1	0.358924172346864\\
1	0.382368815734012\\
0.9	0.395711167025139\\
0.8	0.398639365783745\\
0.7	0.390582620041942\\
0.6	0.370662136116142\\
0.5	0.337604901847068\\
0.420953984343764	0.3\\
0.4	0.288251384329882\\
0.3	0.215922515581422\\
0.282131028234488	0.2\\
0.2	0.116374376665623\\
0.186180347299068	0.1\\
0.10876389138867	0\\
0.0999999999999996	-0.0118903861473933\\
0.0402647386512449	-0.1\\
0	-0.156942664085941\\
-0.0288738255465496	-0.2\\
-0.0999999999999996	-0.29402410041416\\
-0.104374278690726	-0.3\\
-0.189775924101258	-0.4\\
-0.2	-0.409993851029548\\
-0.291044038203221	-0.5\\
-0.3	-0.507401067295397\\
-0.4	-0.592796366698508\\
-0.40802862343722	-0.6\\
-0.5	-0.671986497237429\\
-0.53370250859581	-0.7\\
-0.6	-0.749898933388041\\
-0.661474099478469	-0.8\\
-0.7	-0.829642891989302\\
-0.783389990149374	-0.9\\
-0.8	-0.913827679320785\\
-0.89422964764255	-1\\
-0.9	-1.00544987758857\\
-0.992018338361929	-1.1\\
-1	-1.10887553159674\\
-1.07680977889951	-1.2\\
-1.1	-1.23123199494385\\
-1.14917198214166	-1.3\\
-1.2	-1.38473738383554\\
-1.20910565353539	-1.4\\
-1.2543897057648	-1.5\\
-1.28557918862231	-1.6\\
-1.3	-1.69561224067994\\
-1.30069388925671	-1.7\\
-1.3	-1.72344196859626\\
-1.29744378451276	-1.8\\
-1.27276011834818	-1.9\\
-1.22264928695413	-2\\
-1.2	-2.03074789922273\\
-1.13691314467113	-2.1\\
-1.1	-2.13170743028366\\
-1	-2.19708044837151\\
-0.994002507000206	-2.2\\
0.4	85\\
-0.899580376952201	-2\\
-0.8	-2.0464879750442\\
-0.7	-2.07396322648952\\
-0.6	-2.08537423410227\\
-0.5	-2.0824566658886\\
-0.4	-2.06595686313441\\
-0.3	-2.03587174698102\\
-0.218460041315565	-2\\
-0.2	-1.9909630061721\\
-0.0999999999999996	-1.92752487330081\\
-0.064528916047016	-1.9\\
0	-1.84175716252331\\
0.0403151876415718	-1.8\\
0.0999999999999996	-1.72600071297447\\
0.119383043483722	-1.7\\
0.181151952898485	-1.6\\
0.2	-1.56247408181959\\
0.231174450421229	-1.5\\
0.272545621703598	-1.4\\
0.3	-1.32194432472884\\
0.308184180861568	-1.3\\
0.339816645338024	-1.2\\
0.370420734764569	-1.1\\
0.4	-1.00944740510586\\
0.403536473603522	-1\\
0.441847173918891	-0.9\\
0.495833713748708	-0.8\\
0.5	-0.793429997660324\\
0.571377817132969	-0.7\\
0.6	-0.670851852547642\\
0.678840248997094	-0.6\\
0.7	-0.582716660407395\\
0.8	-0.502398743059263\\
0.802837014425606	-0.5\\
0.9	-0.411128513704315\\
0.91126966749811	-0.4\\
0.992003192372248	-0.3\\
1	-0.283546090321021\\
1.03452568095772	-0.2\\
1.0346751082852	-0.1\\
1	-0.039831275987709\\
0.959977326197339	0\\
0.9	0.027789172881296\\
0.8	0.0427159200959204\\
0.7	0.0298273257880723\\
0.619172361774309	0\\
0.6	-0.0095109211182664\\
0.5	-0.0843172802904879\\
0.483643046782235	-0.1\\
0.4	-0.195113593392465\\
0.396244072150001	-0.2\\
0.327716581281196	-0.3\\
0.3	-0.338361499840579\\
0.256094352459323	-0.4\\
0.2	-0.462540377929804\\
0.164458411211829	-0.5\\
0.0999999999999996	-0.550863633041237\\
0.0303290369768436	-0.6\\
0	-0.616280717421329\\
-0.0999999999999996	-0.671027271995772\\
-0.15279768544273	-0.7\\
-0.2	-0.721584013302707\\
-0.3	-0.7716126684397\\
-0.351525263799792	-0.8\\
-0.4	-0.824071511005497\\
-0.5	-0.880658006199477\\
-0.529638669826678	-0.9\\
-0.6	-0.944366866298763\\
-0.677836629427079	-1\\
-0.7	-1.01615502847053\\
-0.8	-1.09952142092953\\
-0.800508427918479	-1.1\\
-0.898576583543783	-1.2\\
-0.9	-1.20167110390038\\
-0.975613097298311	-1.3\\
-1	-1.33921500189246\\
-1.03521744987203	-1.4\\
-1.07671303025389	-1.5\\
-1.09956363291109	-1.6\\
-1.09995693187682	-1.7\\
-1.07344421559907	-1.8\\
-1.01455629247701	-1.9\\
-1	-1.91645192055764\\
-0.9	-1.99975963266704\\
-0.899580376952201	-2\\
0.5	33\\
-0.728840927540519	-1.8\\
-0.7	-1.81564167720382\\
-0.6	-1.84193940306994\\
-0.5	-1.84315419242148\\
-0.4	-1.82252128655524\\
-0.344869966388001	-1.8\\
-0.3	-1.77730048490814\\
-0.2	-1.70439445535515\\
-0.19526237525781	-1.7\\
-0.112691966455118	-1.6\\
-0.0999999999999996	-1.5778878478335\\
-0.0605404135120126	-1.5\\
-0.029559802615027	-1.4\\
-0.0191978071559133	-1.3\\
-0.0363548666609262	-1.2\\
-0.0999999999999996	-1.1037899623679\\
-0.105401840802578	-1.1\\
-0.2	-1.06363936264226\\
-0.3	-1.0613629311786\\
-0.4	-1.08101264185243\\
-0.450423806928842	-1.1\\
-0.5	-1.11978008302863\\
-0.6	-1.17671279057755\\
-0.631783517131451	-1.2\\
-0.7	-1.25805724490082\\
-0.740509943476718	-1.3\\
-0.8	-1.37946202660435\\
-0.813085686363857	-1.4\\
-0.851116704874705	-1.5\\
-0.859662100834873	-1.6\\
-0.829746610029662	-1.7\\
-0.8	-1.74050840458762\\
-0.728840927540519	-1.8\\
};
\end{axis}
\end{tikzpicture}%
\end{document}
% This file was created by matlab2tikz.
% Minimal pgfplots version: 1.3
%
%The latest updates can be retrieved from
%  http://www.mathworks.com/matlabcentral/fileexchange/22022-matlab2tikz
%where you can also make suggestions and rate matlab2tikz.
%
\documentclass[tikz]{standalone}
\usepackage{pgfplots}
\usepackage{grffile}
\pgfplotsset{compat=newest}
\usetikzlibrary{plotmarks}
\usepackage{amsmath}

\begin{document}
\definecolor{mycolor1}{rgb}{0.00000,0.44700,0.74100}%
\definecolor{mycolor2}{rgb}{0.85000,0.32500,0.09800}%
%
\begin{tikzpicture}

\begin{axis}[%
width=1.5in,
height=1.5in,
scale only axis,
xmin=-3,
xmax=3,
ymin=-3.2,
ymax=3.2,
title={fourth component}
]
\addplot [color=mycolor1,only marks,mark=o,mark options={solid},forget plot]
  table[row sep=crcr]{%
0.0259388858243524	1.99651291936462\\
0.0479365430139229	2.00136242712261\\
0.0808011428276144	2.00015957097738\\
0.0960018408277835	2.00397357604688\\
0.123924398192739	1.99393565592043\\
0.144716297235007	1.99609680237626\\
0.172432773795358	2.00782480916528\\
0.211096968463268	2.00289657674787\\
0.241323554557747	2.00054011148902\\
0.248450748340069	1.98323290473631\\
0.278037158020472	1.99775137204772\\
0.295123845636812	2.024317082564\\
0.278089869880903	1.98953669925669\\
0.349224330459094	2.01774494803247\\
0.389769425315202	1.96915513448386\\
0.443104169526806	2.00181328676435\\
0.445366816453369	2.00702160442149\\
0.447621894481189	1.9639008226548\\
0.459268204218018	1.96874477518371\\
0.525152271376063	1.93928073190811\\
0.479912634035923	1.9531725626356\\
0.541892480429807	1.98767075962513\\
0.596086730914084	1.95159718447859\\
0.642409661124227	1.88862872560156\\
0.59672705695044	2.06269182867749\\
0.577160609928358	1.93769298667604\\
0.61270638646484	1.9863001413635\\
0.738131551064678	1.99453476973758\\
0.731223036941714	1.9720505976831\\
0.728694499517447	2.00387465318476\\
0.792620470002546	2.01820212338234\\
0.76404643370489	1.92712620890764\\
0.783621400051233	1.97170029129066\\
0.844923525499164	2.0002383443526\\
0.933088804435648	1.86929978545337\\
0.826522508566799	1.96489496451136\\
0.857943301337058	1.93574991230876\\
0.862415931324452	1.91824418132958\\
0.85957751145223	1.94915043906307\\
1.10549276410824	1.89209624896601\\
0.992380983980335	1.9389778055659\\
1.05163144833557	1.9077966876626\\
0.971362952394099	1.94526987131734\\
1.10648177744783	1.98917932950755\\
1.12898410068108	1.88302810812235\\
1.14935856334692	1.80741423289017\\
1.2475966096538	1.80328803923413\\
1.02094574055669	1.89016584759305\\
1.18313200649952	1.80836196844854\\
1.11713521982497	1.97114350530034\\
1.21095637410431	1.6606506602979\\
1.09752975594461	1.89186114518647\\
1.1286092446374	1.8521928055118\\
1.17641657796315	1.85179775653125\\
1.2027986521108	1.79291729003239\\
1.40862693119303	1.68188184454732\\
1.29643649567147	1.6835431657974\\
1.46182858553516	1.76685142278392\\
1.44694035368565	1.75403257372017\\
1.49092139521946	1.61038316326389\\
1.3379059086772	1.66077993928176\\
1.46168112951844	1.88866965377674\\
1.41023026436323	1.73928409103319\\
1.4714192456311	1.78600061226619\\
1.35577541010529	1.79014999256957\\
1.4051660334094	1.67552928758717\\
1.61761103881626	1.92386710385478\\
1.24305770227106	1.88248740497196\\
1.49729193411573	1.81459528264337\\
1.63186681467879	1.63329561110855\\
1.53796361955124	1.77930541551158\\
1.46647340378635	1.89266145452761\\
1.59115626629009	1.92971274617182\\
1.55598375769603	1.68322562636936\\
1.60162857304111	1.74522783244834\\
1.55820638402395	1.58511515395533\\
1.6445086680066	1.69602995640505\\
1.43117976594691	1.39522816594688\\
1.66046656580406	1.5020485139169\\
1.89457571586182	1.8060245580165\\
1.7690851756095	1.803354010806\\
1.51808575625609	1.62902388762982\\
1.71974982125714	1.56603639813406\\
1.63851255324909	1.37209586707968\\
1.73596507613967	1.53438459796941\\
1.8856967379105	1.45170562832068\\
1.83646220685143	1.51078810063935\\
1.6807272288616	1.41341108157641\\
1.6794592820606	1.42428516993435\\
1.84697126074185	1.69896546443275\\
1.65822816966463	1.41270723759625\\
1.78371507953502	1.42947586399192\\
1.87899798012913	1.57760927136629\\
1.89142792776872	1.58934082453833\\
1.89301329213783	1.04572276609604\\
1.75219375143258	1.4487568304138\\
2.1047354401674	1.41876136406707\\
2.01123547254846	1.3110508611222\\
1.96781276802938	1.52540722041352\\
1.51166361010094	1.32992221483295\\
2.22158609914819	1.4805071317003\\
1.97788676892182	1.64185849242464\\
1.94851336550087	1.45074289017143\\
2.05150009067797	1.21139868557201\\
1.93322927547617	1.22986741742318\\
1.99640422827958	0.927539227571993\\
2.02784448795865	1.33792744894102\\
2.05296048996672	1.40894341042615\\
1.79681210188576	1.18409186751766\\
2.04186829813144	1.41141478615\\
2.06457591322178	1.27562881170451\\
1.9610792821858	1.32599888091024\\
2.02517402706913	1.29054836978655\\
1.90898477009428	1.23168003520599\\
1.95382687452845	1.07890142563229\\
1.98007488912549	0.865945916419211\\
1.91942744299385	1.23088568855929\\
1.90203680990463	1.46675740172822\\
1.79605423362615	0.99553318086307\\
1.93675059768613	1.26622062159566\\
1.80589580300649	1.38156657036724\\
1.78116659707435	1.22679608853667\\
1.4859041706582	1.42125093730957\\
1.80527660882125	1.44977387260663\\
2.02922661129011	1.0207490772509\\
1.81798786252392	0.647952952360472\\
2.32931027046338	1.2400230228602\\
1.62944341809323	0.943292490703794\\
1.98834525677686	1.34211488379452\\
1.91927966663161	1.25902905273399\\
2.0814706955723	1.07488996429983\\
1.99353342912541	0.988261144028966\\
2.10800463348708	0.766165616265449\\
1.8550225815185	1.13356072105016\\
1.89272673258568	1.36537195907968\\
1.8902074337978	0.757056328403824\\
1.53885687334523	1.07694569130808\\
1.76112331184547	0.998690472291168\\
2.09154477395659	0.694075837194668\\
1.90858542651557	0.757727713019561\\
1.9330049256032	0.976874558375833\\
1.93983302339883	0.816051509797898\\
2.24506862568672	0.672826581876916\\
1.82718437556353	1.10936506743491\\
1.96057264525256	0.893925986104898\\
1.93646566414943	0.64731870645308\\
2.39122724922164	1.24345442192263\\
2.24889413709987	0.793537657333463\\
2.39607565749392	0.91178625334131\\
1.80986080122559	0.707169946665853\\
1.97704296558096	0.591120152065339\\
2.00293771191178	1.36005407696541\\
2.24796362886873	0.686533510972829\\
2.11424994022843	0.877619102581024\\
2.34406728575871	0.795199528805402\\
1.56821878862338	0.60163344885657\\
1.82330236671711	0.291695714299127\\
1.45039697294321	0.590849030637304\\
1.88899283418446	1.04552268182906\\
1.52058053004542	0.435160238141619\\
1.66251832326001	0.804648110802819\\
1.64664565454418	0.505949448785172\\
1.91508532580206	0.847486502505432\\
1.34263224779645	0.88256267651136\\
2.26055865989652	0.538404924773607\\
1.69524767103065	0.741992169305531\\
1.94016209417356	0.741514759245998\\
1.82019093577253	0.328368525179733\\
1.4839520831286	0.681796346611881\\
1.13805529399511	0.728796623608036\\
1.73651102992287	0.402875684463721\\
2.40970926685092	0.766519696154169\\
2.12006326132181	0.724614576326541\\
1.33511311753437	0.0606578626326615\\
1.21694480442495	0.939960475214773\\
1.49919597789097	0.779844441507984\\
1.63203786050577	0.615522830684201\\
1.28009907725904	0.603133599618405\\
1.83363385147587	0.447426311107601\\
1.25294254329276	0.891238649047264\\
1.38589448744568	0.110193693911327\\
2.12557258718458	0.480767020439752\\
1.76853338890148	0.345961715962911\\
1.1838175823629	-0.0335760456264687\\
1.07748933668513	0.396363550377259\\
1.91951574719719	0.710847307761236\\
0.890323101344239	0.539672205647595\\
1.65348820931753	0.512238368108479\\
1.65111543031058	-0.265446989923681\\
1.6070900231769	0.488398313760616\\
0.901758997337281	0.183058802837383\\
1.71870592048393	-0.251816153684363\\
1.55442462019974	0.00466747014038243\\
0.795021559669278	0.0361205015350433\\
0.958568445751217	0.212363974452783\\
1.67442948239855	0.4882871131843\\
1.2206685044667	-0.324863168774303\\
1.50767341273933	-0.389108334817502\\
1.37836608274919	-0.0373397947956927\\
1.11156784444486	0.701945681672377\\
1.4984756552433	0.00584512978944459\\
1.24690289872505	0.226154641123381\\
1.09194532469786	-0.130683470031559\\
0.678183222535072	-0.0365335083510088\\
1.73612089885204	0.397031801733272\\
0.933320654619196	0.746473890326733\\
1.15525443706948	-0.179109971394189\\
0.458916732169228	-0.213954536366935\\
1.58359507914698	-0.231297472151952\\
0.931434909226943	0.114256234330456\\
1.1322927346265	0.17900469841877\\
0.780331897364257	0.102401717565955\\
0.501871266865262	-0.0628649708451471\\
0.503048632418745	-0.459092444286199\\
0.604765068083075	-0.011679852416595\\
0.666178137609782	0.219390527881344\\
0.413245356250045	-0.0695882331429949\\
0.467795007897315	-0.0999511872763669\\
1.15246888796731	-0.0589372144022903\\
0.792490565783635	-0.386299556418842\\
0.747652249238335	0.192053680178325\\
0.6730227064451	-0.457495772861639\\
0.657608820873446	-0.397844794344219\\
0.80283938775403	0.00586599616798039\\
1.29819955341968	-0.204261874156702\\
0.700969575790672	-0.360362213508646\\
0.577484279146232	0.667638314415951\\
1.53670668302431	-0.432957390684138\\
-0.157947037327338	-0.206989394214058\\
-0.250033960160869	0.369996686022994\\
0.35679132363435	-0.672812036115748\\
0.307418930825796	-0.491736063949304\\
0.612754451313721	-0.591725661644702\\
0.513152577927538	-0.242776993818146\\
0.457076244801449	-0.714876480294682\\
0.374036335051672	-0.347506625594372\\
0.579283772090358	0.220733907465001\\
0.187849487939868	0.169639639117464\\
-0.0361733332976721	0.281051835212802\\
0.772027729747197	-0.505718105015689\\
0.269625343575084	-0.756939350867156\\
0.60570524403307	-0.119581878977858\\
-0.192532712828272	-0.0344061540276999\\
-0.179793749480468	-0.900222299631903\\
0.0699543465880473	-0.840398840157916\\
0.168779646202848	-0.973477052749994\\
-0.00586739214808801	-0.369383158623758\\
0.465908963186411	-0.973925709442261\\
0.160490995593304	-0.295058807475094\\
0.168397029916368	-0.980178812503749\\
};
\addplot [color=mycolor2,only marks,mark=o,mark options={solid},forget plot]
  table[row sep=crcr]{%
-0.026947097397927	-1.54967503001056\\
-0.0499096052447985	-1.548838725865\\
-0.0878921441942618	-1.54358858745209\\
-0.102243240432917	-1.54118951004161\\
-0.115588964597783	-1.54384949729\\
-0.166126369580897	-1.54954111761082\\
-0.171775059288128	-1.53255169798828\\
-0.202468902552123	-1.53693920317449\\
-0.212030786452003	-1.57194441918315\\
-0.242543396618313	-1.55982750802416\\
-0.268638030522741	-1.55184551810696\\
-0.307999262514236	-1.54141240307698\\
-0.34184706998786	-1.55670987888773\\
-0.321654511497463	-1.58426559222047\\
-0.392925170560618	-1.51563730344866\\
-0.369402481562153	-1.52198515289873\\
-0.416941632627096	-1.51482183374776\\
-0.47214126557288	-1.57339572749443\\
-0.493265130467738	-1.49632259648208\\
-0.512861932358619	-1.50290927620725\\
-0.557340464169725	-1.50842598279773\\
-0.515456646472125	-1.56774163478792\\
-0.608332652640224	-1.42764267683647\\
-0.571584245650045	-1.49817383565417\\
-0.662130333502388	-1.56697697704636\\
-0.669954622356556	-1.57493033524355\\
-0.694374248931051	-1.46017650026605\\
-0.763136406470192	-1.44654664398095\\
-0.64832005525922	-1.5311300933111\\
-0.721079189047473	-1.57176135978772\\
-0.762850186528795	-1.48791858695594\\
-0.771668694696658	-1.4609916619696\\
-0.82794824307844	-1.46985243432682\\
-0.757762090173658	-1.51439340279633\\
-0.73200316350585	-1.54643662607006\\
-0.849650868675982	-1.4739825416715\\
-0.869633988303967	-1.36513677682954\\
-0.929617676304213	-1.40255715057468\\
-0.976307028854808	-1.46241069529678\\
-0.929398084649915	-1.49557897888553\\
-1.08884466383786	-1.60008247625742\\
-1.1061052512272	-1.4667516416831\\
-0.989287983622614	-1.44461809994125\\
-0.985959410143955	-1.5033836797293\\
-1.18771458192431	-1.33744464545877\\
-1.00331746689445	-1.40505396838939\\
-0.892432339761976	-1.52461527768079\\
-1.06850301949737	-1.42382215016215\\
-1.21578154069187	-1.35143372749743\\
-1.1054763672977	-1.37105316655968\\
-1.05324198188984	-1.36449836483918\\
-1.36583986729047	-1.47144251941379\\
-1.21498723375497	-1.58153140750668\\
-1.15864586873963	-1.35211138149604\\
-1.29274883019543	-1.24706936498748\\
-1.21165267830111	-1.36424729300889\\
-1.13339068136518	-1.38791313506574\\
-1.38647009921518	-1.36164505679436\\
-1.38447268580765	-1.24424678906453\\
-1.30846082709955	-1.31326104146817\\
-1.26981825327827	-1.22686897127221\\
-1.49956364455672	-1.20155729619685\\
-1.44709290956236	-1.3499491526295\\
-1.42148110929907	-1.2096773418177\\
-1.48960505179888	-1.41647829527412\\
-1.3264878697695	-1.47045835775913\\
-1.68833293650043	-1.35607977232846\\
-1.58968210104544	-1.26255149104674\\
-1.55268500555184	-1.2291738166282\\
-1.68292402606647	-1.11026634760032\\
-1.72154852078698	-1.24202688963492\\
-1.43590169300641	-1.37871676926254\\
-1.69132652673162	-1.07877471235978\\
-1.70551483678654	-1.42804960005592\\
-1.53520709567707	-1.17299419798963\\
-1.63554863873912	-1.28753234363309\\
-1.65650253897248	-1.09178635476079\\
-1.62198948544192	-1.19413197420841\\
-1.47519706775498	-1.28565950980904\\
-1.43433635492227	-1.32753123102665\\
-1.73478223846456	-1.20660071287845\\
-1.90432859972523	-1.29120302815414\\
-1.9071129230002	-0.972886141766062\\
-1.69183439158024	-1.10780628569212\\
-1.68284658996465	-1.31833231939511\\
-1.85992398322545	-1.01141698924493\\
-1.72772929472913	-0.98157402042344\\
-1.74885476713543	-1.4351574402123\\
-1.84150517090143	-0.95425928858046\\
-1.69315517542604	-0.958656958688715\\
-1.7286495891314	-0.954601665072524\\
-1.81811333022962	-1.19579885376329\\
-1.91064332117842	-1.29835844701107\\
-2.115454982706	-1.0182126578052\\
-1.81014864687136	-1.11941452772307\\
-1.88306383492019	-1.01716868637323\\
-1.86797031445803	-1.04760399233855\\
-1.99290332758122	-1.38532172129099\\
-1.57013566090976	-0.954717363457782\\
-1.6305853956127	-0.926490388852283\\
-1.82503267963498	-1.2054049019875\\
-2.04745234990293	-0.70712356859999\\
-1.91302909485597	-0.765494691747267\\
-1.79315031506749	-1.07048160263372\\
-1.67520961679342	-0.869000607592547\\
-1.98886073309524	-1.32715925014926\\
-2.01375717222897	-0.9315973741601\\
-1.93408159383817	-0.878274086050914\\
-2.26246434296342	-0.886063271151665\\
-1.72639050831228	-0.560229703702508\\
-2.06045827186263	-0.977280310897143\\
-2.21522054773345	-0.832044314048742\\
-2.12802823080291	-0.765383980210819\\
-2.47819250264965	-0.688993880398068\\
-1.68985107972107	-0.732789288101983\\
-1.81469301090601	-0.558603652168838\\
-2.02534666967788	-0.825878531557616\\
-1.90668680745126	-0.687396711733751\\
-2.06031052575321	-0.615192926362136\\
-1.99585448031789	-0.990208999863819\\
-2.14213953096814	-0.765898153601535\\
-1.99073350792875	-0.38534769649198\\
-1.96145912243946	-0.756790477559718\\
-1.88353079013264	-0.493700231213568\\
-2.3752794639133	-0.852864483514567\\
-1.71756604008283	-0.839522058331343\\
-1.96581549537573	-0.658952618483849\\
-2.46496189427916	-0.82131795587749\\
-2.10629520423653	-0.845571433601728\\
-2.11100602865062	-0.528455628514166\\
-1.70854667582737	-0.522179557510307\\
-2.00450257438808	-1.16055808599304\\
-2.21156096039358	-0.482354346628644\\
-2.04160631785436	-0.533352916239695\\
-1.84559801482221	-0.658966536067479\\
-1.92091051009354	-0.644336627288141\\
-1.92864622951692	-0.408174908184284\\
-2.04092091269914	-0.53236208450286\\
-1.86431188937102	-0.409957901472978\\
-1.97521350506084	-0.613741462440286\\
-2.12216417761872	-0.431925911153432\\
-1.80339734046248	-0.583776426535711\\
-1.72974769480169	-0.586333850372364\\
-1.97641639419026	-0.40354436882594\\
-1.66790009602428	-0.535102761781156\\
-1.60550901999747	-0.517641138116707\\
-1.98093745067308	-0.248643233716586\\
-1.77048803449917	-0.0215214325064952\\
-1.82669905771791	-0.344780626915585\\
-2.19905576091282	-0.408420695054419\\
-1.83101161586013	-0.16573497405526\\
-1.99022362231772	-0.240850680829\\
-1.37200304413117	-0.218020520951735\\
-1.83273900042908	0.0375840492151414\\
-1.56236387312051	-0.0622109695320063\\
-1.83847741136532	-0.629170879373212\\
-1.62842916741517	-0.261531442505625\\
-1.760757004596	-0.122954408755528\\
-1.92693522151942	0.0317672214706444\\
-0.795922426677282	-0.305147429881106\\
-1.66083624793154	-0.460640523838259\\
-1.79682003601176	0.166845603602293\\
-1.64022836211743	-0.361636009773446\\
-2.53886522482741	-0.236580562503111\\
-2.03581414704918	0.013541805129293\\
-1.84630209912121	0.230376960175914\\
-1.91209527248309	-0.360580206613225\\
-1.81233556853063	0.0860452228596581\\
-1.85601948484678	-0.446016224299677\\
-1.44425457774794	0.133932704188332\\
-1.82082024175442	-0.524871278676388\\
-1.90283579546103	-0.024980003152043\\
-0.918616001910668	0.598439720318925\\
-1.37397981159442	-0.228688766885865\\
-2.13124001952824	0.0394435337110551\\
-1.51184628119662	-0.339909320146771\\
-1.42078378674312	0.259637468967449\\
-1.46934828152567	-0.48658770179399\\
-1.45036714720006	0.561650634597791\\
-1.6235939585632	0.100495743328196\\
-1.56361953447043	0.429563917963797\\
-1.72011242640493	0.125548136165587\\
-1.70161637144955	-0.00552791997817297\\
-1.3830196422212	0.484897744780771\\
-1.94541153808096	-0.142579452253231\\
-1.17461760721602	0.106879495508337\\
-1.09674397776978	0.918979238595924\\
-1.37782133309489	0.510300838744834\\
-1.88845389859122	0.238194094828098\\
-1.05273228631062	0.534963752730167\\
-1.29045492870113	0.541672326993093\\
-1.51108660201011	0.115095697290448\\
-1.60478470830124	0.761938729066085\\
-1.16297289088485	0.712515972958128\\
-1.36626122649495	0.0459699369006284\\
-1.00569677090725	0.566053726817701\\
-1.15605146422804	0.32048685029173\\
-0.822878786956236	0.299885052175656\\
-1.40380642068769	0.703790875671732\\
-1.59757716568005	0.713604210308872\\
-1.32350395473092	-0.0181115754076439\\
-1.15090617990717	0.0297575564492658\\
-1.21924769269614	0.636116178851027\\
-1.66330401562556	0.274767468591772\\
-1.34530843504664	0.53762093307587\\
-1.39326431751636	1.12504793697234\\
-0.833618774635974	0.512669380246674\\
-0.573367167351866	0.652510525923379\\
-0.276320238657219	0.18763449775689\\
-1.17194492857265	0.572769304168273\\
-0.784840550207506	0.714006761323611\\
-0.924013272173041	0.0557708685451278\\
-1.15282586642296	0.334082023567727\\
-0.935348154153094	0.774875910909737\\
-1.09740398474138	0.634480504642842\\
-1.11946267734234	0.610532582269006\\
-0.838968288924051	1.38322547926388\\
-1.22448365542895	1.32213641543783\\
-0.476481839809044	1.40105426500117\\
-0.769114345508466	0.66169599251629\\
-1.40027835036432	1.28884494337088\\
-1.50132634028759	0.613610138665922\\
-0.949826176623956	0.594374431492637\\
-0.88164099127801	0.552395160007791\\
-0.994300613393627	0.256827952699087\\
-0.632450277038536	0.881951176938878\\
-0.331331900379547	0.85471608258404\\
-1.13074002827094	0.521881807762682\\
-0.86984371951327	0.417083028748866\\
-0.420618985877099	0.726297500339449\\
-0.110157996853952	0.742495519066295\\
0.644069697204106	0.64901805109848\\
-0.262085393832434	0.700643157487521\\
-0.276442795861882	0.467852427149516\\
-0.14443352788167	0.705856371233983\\
-0.727695229224663	1.06740698127406\\
-0.626210638028053	0.986934718277524\\
-0.658293890852093	1.48386542068786\\
-0.71066942064998	0.843047675153383\\
0.185985214616693	0.552601466396477\\
-0.615839609456226	0.931267504635312\\
-0.533903453015464	1.49417117664361\\
-0.278043832459567	1.40015122194036\\
-0.374673778674507	0.760845850670364\\
0.0347222292805265	1.27907372368006\\
-0.41631147842867	0.89614152469315\\
-0.148017321318051	0.860943748299246\\
-0.318167564411286	1.66905910965134\\
0.565869201509072	1.63328081306408\\
0.456182094678272	1.2472468033001\\
};
\addplot[contour prepared, contour prepared format=matlab, contour/labels=false] table[row sep=crcr] {%
%
-0.5	17\\
1.70676048020263	0.3\\
1.7	0.299156640217436\\
1.69698633646441	0.3\\
1.6	0.389493532238523\\
1.59638194734851	0.4\\
1.6	0.411152426507645\\
1.63585194286411	0.5\\
1.7	0.566403274315718\\
1.74625213091292	0.6\\
1.8	0.624623151993581\\
1.9	0.633765970630217\\
1.96145194547385	0.6\\
1.97998044629886	0.5\\
1.92365566327624	0.4\\
1.9	0.377276992074144\\
1.8	0.316914227256172\\
1.70676048020263	0.3\\
-0.4	45\\
1.91301163913524	0\\
1.9	-0.00685756092659516\\
1.8	-0.0465179434311474\\
1.7	-0.0733112431020327\\
1.6	-0.0866948287306187\\
1.5	-0.0850085162081831\\
1.4	-0.065176565672214\\
1.3	-0.0222326018275292\\
1.26705634714023	0\\
1.2	0.0704996609482627\\
1.17996456349127	0.1\\
1.15025329129528	0.2\\
1.1558813769336	0.3\\
1.18976719457974	0.4\\
1.2	0.416551834391423\\
1.25025913351717	0.5\\
1.3	0.559636633732991\\
1.33342086141519	0.6\\
1.4	0.662826741419161\\
1.43996934533986	0.7\\
1.5	0.746272505463811\\
1.57247006161222	0.8\\
1.6	0.817771983151435\\
1.7	0.875973012361793\\
1.7482828255701	0.9\\
1.8	0.92333617837181\\
1.9	0.957193011023933\\
2	0.977228977422514\\
2.1	0.977254661280084\\
2.2	0.94880130502746\\
2.27441243925304	0.9\\
2.3	0.872030447608686\\
2.34427890545115	0.8\\
2.37178933984785	0.7\\
2.37344971203469	0.6\\
2.35455974372928	0.5\\
2.31764988660429	0.4\\
2.3	0.367116703595041\\
2.26009593027706	0.3\\
2.2	0.221464609424123\\
2.18111651019335	0.2\\
2.1	0.122728785111105\\
2.07149839679296	0.1\\
2	0.0494307354022326\\
1.91301163913524	0\\
-0.3	69\\
1.82594757246463	-0.3\\
1.8	-0.309828341035827\\
1.7	-0.338133111847403\\
1.6	-0.356764514574877\\
1.5	-0.365657842198449\\
1.4	-0.364253474058824\\
1.3	-0.351513273299618\\
1.2	-0.325959609843375\\
1.13395340138763	-0.3\\
1.1	-0.284581899182819\\
1	-0.221587771436074\\
0.973143156678153	-0.2\\
0.9	-0.121667829835032\\
0.883075108719884	-0.1\\
0.832905498372569	0\\
0.812938606942794	0.1\\
0.819914041091168	0.2\\
0.851584813580628	0.3\\
0.9	0.386932517324133\\
0.906767149011435	0.4\\
0.982685463046685	0.5\\
1	0.517254825770436\\
1.07755204135739	0.6\\
1.1	0.61926409396111\\
1.1888087206838	0.7\\
1.2	0.708567735174092\\
1.3	0.788007218100353\\
1.31484662174559	0.8\\
1.4	0.860328085998329\\
1.45571427008226	0.9\\
1.5	0.928772644926749\\
1.6	0.992292386302203\\
1.61316393081376	1\\
1.7	1.04824152344324\\
1.8	1.09986360829682\\
1.80032513731271	1.1\\
1.9	1.14129362324811\\
2	1.17439650541756\\
2.1	1.19623999939257\\
2.15754241294752	1.2\\
2.2	1.20286546389907\\
2.22695656653654	1.2\\
2.3	1.19034197981325\\
2.4	1.15180253101597\\
2.47474075822216	1.1\\
2.5	1.07549196434996\\
2.55779177185055	1\\
2.6	0.911281404964829\\
2.60463950425218	0.9\\
2.62853835190629	0.8\\
2.63609495993998	0.7\\
2.62958045487565	0.6\\
2.6100396561014	0.5\\
2.6	0.46853370288944\\
2.57786812278747	0.4\\
2.53296250623714	0.3\\
2.5	0.242652317089366\\
2.47411643727188	0.2\\
2.40028577047532	0.1\\
2.4	0.0996605356441543\\
2.3080454480416	0\\
2.3	-0.00782691045071994\\
2.2	-0.0944394824840883\\
2.1926145474411	-0.1\\
2.1	-0.165648369676012\\
2.04223895540162	-0.2\\
2	-0.224421188069128\\
1.9	-0.272112596194195\\
1.82594757246463	-0.3\\
-0.2	59\\
-1.87842982377014	-0.6\\
-1.9	-0.606242887706069\\
-2	-0.620759059326759\\
-2.1	-0.618719600890914\\
-2.18034331410244	-0.6\\
-2.2	-0.593871100528265\\
-2.3	-0.530574514860424\\
-2.33082767925282	-0.5\\
-2.3951812042537	-0.4\\
-2.4	-0.386486933392658\\
-2.42573866658973	-0.3\\
-2.43620517993048	-0.2\\
-2.43075662218399	-0.1\\
-2.41137616257504	0\\
-2.4	0.0352137329743494\\
-2.37830316689868	0.1\\
-2.33208284301965	0.2\\
-2.3	0.254750832687177\\
-2.27140214313772	0.3\\
-2.2	0.393880771203247\\
-2.19480407420723	0.4\\
-2.1	0.49719058554905\\
-2.09682018943365	0.5\\
-2	0.577327065346908\\
-1.96547563436854	0.6\\
-1.9	0.640108299167143\\
-1.8	0.687546185927552\\
-1.76302458960777	0.7\\
-1.7	0.720392128719044\\
-1.6	0.738855513127656\\
-1.5	0.74289477882222\\
-1.4	0.73172050685988\\
-1.3	0.704410851866378\\
-1.28975493909349	0.7\\
-1.2	0.655850615351032\\
-1.11836076299355	0.6\\
-1.1	0.584225200023684\\
-1.02074534123103	0.5\\
-1	0.467829104671897\\
-0.961223944817459	0.4\\
-0.932149983491462	0.3\\
-0.931700269384024	0.2\\
-0.958948754443592	0.1\\
-1	0.0244152856204402\\
-1.01275557712833	0\\
-1.09021533871094	-0.1\\
-1.1	-0.109563168640199\\
-1.1893649702689	-0.2\\
-1.2	-0.208805030536931\\
-1.3	-0.292728539762976\\
-1.30882256566477	-0.3\\
-1.4	-0.364867381992757\\
-1.45099357372866	-0.4\\
-1.5	-0.430446918750461\\
-1.6	-0.48857946661598\\
-1.62256534154267	-0.5\\
-1.7	-0.536826132269454\\
-1.8	-0.576797105036942\\
-1.87842982377014	-0.6\\
-0.2	91\\
1.74531431622508	-0.6\\
1.7	-0.613143774809271\\
1.6	-0.633922697850445\\
1.5	-0.646305371373297\\
1.4	-0.650228140544074\\
1.3	-0.645303214958673\\
1.2	-0.63088391601162\\
1.1	-0.606167785728217\\
1.08232473234419	-0.6\\
1	-0.570745403754258\\
0.9	-0.523121454918251\\
0.860637114530744	-0.5\\
0.8	-0.461236751310789\\
0.719350036894079	-0.4\\
0.7	-0.382943289824579\\
0.615876804685202	-0.3\\
0.6	-0.280227735883931\\
0.538919221895476	-0.2\\
0.5	-0.126572198913528\\
0.485802460772175	-0.1\\
0.456026496139982	0\\
0.45213142316403	0.1\\
0.475827892805926	0.2\\
0.5	0.244614138458492\\
0.526849512696186	0.3\\
0.6	0.393207365148121\\
0.604802538483982	0.4\\
0.7	0.49499919544113\\
0.704541028581657	0.5\\
0.8	0.580030809055788\\
0.821702160479816	0.6\\
0.9	0.657876782806891\\
0.952227352235523	0.7\\
1	0.732337702731076\\
1.09226494223848	0.8\\
1.1	0.80495085792257\\
1.2	0.872923421170043\\
1.2378975894715	0.9\\
1.3	0.940338556668489\\
1.38846139008092	1\\
1.4	1.00733856786416\\
1.5	1.07094960048975\\
1.54617745673556	1.1\\
1.6	1.1332219010704\\
1.7	1.19291118057161\\
1.71305108224008	1.2\\
1.8	1.24821205227068\\
1.9	1.29917454358219\\
1.90198305825888	1.3\\
2	1.34336544430238\\
2.1	1.37953265075757\\
2.17993586376411	1.4\\
2.2	1.40569268904077\\
2.3	1.41930777376568\\
2.4	1.41755428096706\\
2.48705440017711	1.4\\
2.5	1.39695437065842\\
2.6	1.35189203320419\\
2.67176894964866	1.3\\
2.7	1.27387790161468\\
2.76279067096818	1.2\\
2.8	1.1390316175459\\
2.82129928591493	1.1\\
2.85988421562682	1\\
2.88338152031588	0.9\\
2.89470073559203	0.8\\
2.89554134227424	0.7\\
2.88675665282048	0.6\\
2.86855139246106	0.5\\
2.84057692475554	0.4\\
2.80195579710681	0.3\\
2.8	0.295876280859717\\
2.75500384392929	0.2\\
2.7	0.106782543068185\\
2.69591695282733	0.1\\
2.62640141221094	0\\
2.6	-0.0330926393675115\\
2.54346927216932	-0.1\\
2.5	-0.146393600681493\\
2.445377074975	-0.2\\
2.4	-0.241595043839532\\
2.3289182193168	-0.3\\
2.3	-0.322894822389185\\
2.2	-0.392716321995335\\
2.18811526927519	-0.4\\
2.1	-0.453989048513868\\
2.00974918574597	-0.5\\
2	-0.50508637310647\\
1.9	-0.549642301161126\\
1.8	-0.58500378658926\\
1.74531431622508	-0.6\\
-0.1	193\\
3	-0.00596111023547163\\
2.94158912174466	-0.1\\
2.9	-0.157248955646932\\
2.86856608437167	-0.2\\
2.8	-0.282800997864906\\
2.78520296642686	-0.3\\
2.7	-0.390994503576037\\
2.69101897137028	-0.4\\
2.6	-0.486492014142465\\
2.58449469303065	-0.5\\
2.5	-0.571731880466381\\
2.46281774263682	-0.6\\
2.4	-0.647722326356982\\
2.32129281848265	-0.7\\
2.3	-0.714456720113324\\
2.2	-0.77506044335122\\
2.15183976760829	-0.8\\
2.1	-0.828261386127548\\
2	-0.874434570651772\\
1.93250803290643	-0.9\\
1.9	-0.913311829090063\\
1.8	-0.946897680682254\\
1.7	-0.972071503293689\\
1.6	-0.989455224777384\\
1.5	-0.999264432388078\\
1.46525237970224	-1\\
1.4	-1.00155442626899\\
1.37781374494836	-1\\
1.3	-0.995281400701309\\
1.2	-0.980334470496817\\
1.1	-0.95566074489624\\
1	-0.920401460556915\\
0.955874750708821	-0.9\\
0.9	-0.876002185209495\\
0.8	-0.822316922572794\\
0.764974235890853	-0.8\\
0.7	-0.759859230159937\\
0.615294867738363	-0.7\\
0.6	-0.689108898747476\\
0.5	-0.61135225060067\\
0.486070966541095	-0.6\\
0.4	-0.526736387618946\\
0.36858708767651	-0.5\\
0.3	-0.436632764959057\\
0.258042853871453	-0.4\\
0.2	-0.342657337960739\\
0.150987356826449	-0.3\\
0.0999999999999996	-0.247337759110467\\
0.0426678324127566	-0.2\\
0	-0.155570146486791\\
-0.0783972836072332	-0.1\\
-0.0999999999999996	-0.0788363932397645\\
-0.2	-0.0296006810592794\\
-0.3	-0.0234530443730789\\
-0.4	-0.0480294385246191\\
-0.5	-0.0951225599744217\\
-0.507105417140766	-0.1\\
-0.6	-0.145119676447812\\
-0.68981464180306	-0.2\\
-0.7	-0.204863112799439\\
-0.8	-0.263176350311018\\
-0.854966027788635	-0.3\\
-0.9	-0.325347538619873\\
-1	-0.388488404104267\\
-1.01675362111639	-0.4\\
-1.1	-0.450678101945343\\
-1.17578241145584	-0.5\\
-1.2	-0.514523066697992\\
-1.3	-0.577027452426408\\
-1.33613003823939	-0.6\\
-1.4	-0.63903379760591\\
-1.49932134346142	-0.7\\
-1.5	-0.700414888861016\\
-1.6	-0.758810348898644\\
-1.67382201120766	-0.8\\
-1.7	-0.815136437165596\\
-1.8	-0.867279759333198\\
-1.87077214092561	-0.9\\
-1.9	-0.914565844975813\\
-2	-0.955186316949849\\
-2.1	-0.987296177278468\\
-2.16068495173671	-1\\
-2.2	-1.00928091795727\\
-2.3	-1.01863269083806\\
-2.4	-1.01246749830614\\
-2.45310805038338	-1\\
-2.5	-0.987331105525097\\
-2.6	-0.93806132810112\\
-2.65164564544063	-0.9\\
-2.7	-0.855058831945152\\
-2.7471951685746	-0.8\\
-2.8	-0.715898993006595\\
-2.80898038956564	-0.7\\
-2.85122159549014	-0.6\\
-2.87838252840203	-0.5\\
-2.89369230544577	-0.4\\
-2.89915874119732	-0.3\\
-2.89596024726024	-0.2\\
-2.88466895223323	-0.1\\
-2.86537241122025	0\\
-2.83772861676101	0.1\\
-2.80097150516884	0.2\\
-2.8	0.202204008879627\\
-2.75834263017896	0.3\\
-2.70536630977364	0.4\\
-2.7	0.408915763445836\\
-2.64502716399718	0.5\\
-2.6	0.563935199273169\\
-2.5738266892195	0.6\\
-2.5	0.690200250176089\\
-2.49150321164082	0.7\\
-2.4	0.796478039423406\\
-2.39635748504336	0.8\\
-2.3	0.887554577382244\\
-2.2845783085046	0.9\\
-2.2	0.965775542192082\\
-2.14852324416571	1\\
-2.1	1.03181882543473\\
-2	1.08628085293103\\
-1.96883291085965	1.1\\
-1.9	1.13077047670028\\
-1.8	1.16392916007625\\
-1.7	1.18535414401157\\
-1.6	1.19531177257891\\
-1.5	1.19360948332382\\
-1.4	1.17976353579083\\
-1.3	1.15323864072929\\
-1.2	1.113788953176\\
-1.1737850454213	1.1\\
-1.1	1.06269210359525\\
-1	1.00052149014389\\
-0.999281239603567	1\\
-0.9	0.927956550610602\\
-0.86483827728525	0.9\\
-0.8	0.846454087042503\\
-0.745650196823618	0.8\\
-0.7	0.757855560999599\\
-0.635449501220982	0.7\\
-0.6	0.664198100210526\\
-0.52924612657668	0.6\\
-0.5	0.568576886636753\\
-0.42103767506643	0.5\\
-0.4	0.4768809217027\\
-0.3	0.402088917025482\\
-0.294872500807442	0.4\\
-0.2	0.348351160127949\\
-0.0999999999999996	0.341124896581094\\
0	0.364832726316493\\
0.0769371146818393	0.4\\
0.0999999999999996	0.406639667415808\\
0.2	0.449876087965733\\
0.293899555026563	0.5\\
0.3	0.502360592099427\\
0.4	0.551129606041439\\
0.486852331321412	0.6\\
0.5	0.605781083853169\\
0.6	0.657670909526519\\
0.671996809008349	0.7\\
0.7	0.713579200870393\\
0.8	0.768784508658861\\
0.850570594530105	0.8\\
0.9	0.826351777676347\\
1	0.885480030693058\\
1.0225022279795	0.9\\
1.1	0.945077431318031\\
1.18722884007258	1\\
1.2	1.00751680399793\\
1.3	1.06973232923471\\
1.34666232582186	1.1\\
1.4	1.13369810148797\\
1.5	1.19855130856892\\
1.50227810010583	1.2\\
1.6	1.26302023971754\\
1.65852205832967	1.3\\
1.7	1.32756910264391\\
1.8	1.39062361046315\\
1.81643947781053	1.4\\
1.9	1.45217585002308\\
1.98473755068302	1.5\\
2	1.5097817380056\\
2.1	1.56346522064023\\
2.18142733626089	1.6\\
2.2	1.6098494269636\\
2.3	1.64902314364015\\
2.4	1.67660170490482\\
2.5	1.69127855744528\\
2.6	1.6912204529583\\
2.7	1.67380915121196\\
2.8	1.63528441087566\\
2.85871650896367	1.6\\
2.9	1.57120583949436\\
2.97736997248811	1.5\\
3	1.47426899814258\\
0	100\\
2.48309974098119	-3.2\\
2.4	-3.101892091109\\
2.3993254207251	-3.1\\
2.32606078062849	-3\\
2.3	-2.9818532279949\\
2.24822704901038	-2.9\\
2.2	-2.85981934611273\\
2.16673451574941	-2.8\\
2.1	-2.73490714299064\\
2.08260108999893	-2.7\\
2	-2.60768190672884\\
1.99650017093598	-2.6\\
1.91500194464931	-2.5\\
1.9	-2.48938488170236\\
1.83254962052771	-2.4\\
1.8	-2.37432302977175\\
1.74733263193811	-2.3\\
1.7	-2.25916516433804\\
1.65993098788478	-2.2\\
1.6	-2.14446383386136\\
1.57073852734138	-2.1\\
1.5	-2.0307569875122\\
1.48002300101497	-2\\
1.4	-1.91851173927636\\
1.38796295424408	-1.9\\
1.3	-1.80808262483277\\
1.294670890492	-1.8\\
1.20027967857024	-1.7\\
1.2	-1.69978607274487\\
1.10598708371939	-1.6\\
1.1	-1.59526250649402\\
1.00985354945768	-1.5\\
1	-1.49199619003126\\
0.911951750904131	-1.4\\
0.9	-1.39009811180018\\
0.812272998252105	-1.3\\
0.8	-1.28968491507241\\
0.71070415247423	-1.2\\
0.7	-1.19091710380497\\
0.606975583065087	-1.1\\
0.6	-1.09405261942474\\
0.500560836659486	-1\\
0.5	-0.999522207001213\\
0.4	-0.90853227949736\\
0.390904393339451	-0.9\\
0.3	-0.82121087848646\\
0.27515587577865	-0.8\\
0.2	-0.738859286788938\\
0.148860730774125	-0.7\\
0.0999999999999996	-0.663540254123429\\
0.00252401537964789	-0.6\\
0	-0.598335633564104\\
-0.0999999999999996	-0.543728463282421\\
-0.2	-0.504450655627536\\
-0.217733668105962	-0.5\\
-0.3	-0.47868223944785\\
-0.4	-0.468202813864797\\
-0.5	-0.471852861784038\\
-0.6	-0.488000816785928\\
-0.643250808305685	-0.5\\
-0.7	-0.513979064701795\\
-0.8	-0.548006817722244\\
-0.9	-0.590068281747334\\
-0.919975296137471	-0.6\\
-1	-0.63730667569578\\
-1.1	-0.690033810173539\\
-1.11725834276586	-0.7\\
-1.2	-0.746815082677424\\
-1.28775680438793	-0.8\\
-1.3	-0.807544721443137\\
-1.4	-0.871596514337564\\
-1.44362733976761	-0.9\\
-1.5	-0.938909879787983\\
-1.58897955786109	-1\\
-1.6	-1.00832583327403\\
-1.7	-1.08088042327295\\
-1.72804837437799	-1.1\\
-1.8	-1.15615882813773\\
-1.86120720181768	-1.2\\
-1.9	-1.23301832156248\\
-1.98830909153603	-1.3\\
-2	-1.3109401424331\\
-2.1	-1.39137371645185\\
-2.1128332141439	-1.4\\
-2.2	-1.47497980707393\\
-2.23567516806846	-1.5\\
-2.3	-1.55987582561678\\
-2.35463325897909	-1.6\\
-2.4	-1.64586504556096\\
-2.47020107914636	-1.7\\
-2.5	-1.73280572725861\\
-2.58284167831601	-1.8\\
-2.6	-1.82060322549496\\
-2.69297732837591	-1.9\\
-2.7	-1.90920404574433\\
-2.8	-1.99906510056772\\
-2.80156684072832	-2\\
-2.9	-2.09285805768962\\
-2.91175899448733	-2.1\\
-3	-2.1876836823818\\
0	99\\
-2.51820401093374	3.2\\
-2.5	3.18693959574329\\
-2.44705678447005	3.1\\
-2.4	3.05835700400737\\
-2.36944678811301	3\\
-2.3	2.92676386303328\\
-2.28759564090203	2.9\\
-2.20521482005977	2.8\\
-2.2	2.79641905028026\\
-2.12804896602522	2.7\\
-2.1	2.67800011050703\\
-2.04637010094565	2.6\\
-2	2.55948477079174\\
-1.96148231710185	2.5\\
-1.9	2.44144781267903\\
-1.87422744772617	2.4\\
-1.8	2.32440604249236\\
-1.78517241650123	2.3\\
-1.7	2.20877398739836\\
-1.69471166225824	2.2\\
-1.60465750905858	2.1\\
-1.6	2.0964214325297\\
-1.51524631857483	2\\
-1.5	1.98762328661505\\
-1.42401594109244	1.9\\
-1.4	1.87960971095443\\
-1.33126145290379	1.8\\
-1.3	1.77248704694302\\
-1.23715816871884	1.7\\
-1.2	1.66637635611809\\
-1.14176560034802	1.6\\
-1.1	1.56144683745917\\
-1.04500794713913	1.5\\
-1	1.45796065821536\\
-0.946623517719971	1.4\\
-0.9	1.35633247888922\\
-0.846063327940832	1.3\\
-0.8	1.25720809804576\\
-0.742293676268367	1.2\\
-0.7	1.16156787263181\\
-0.633395239647983	1.1\\
-0.6	1.07086316424802\\
-0.515677632510534	1\\
-0.5	0.987202396452819\\
-0.4	0.912516979233111\\
-0.380609814352603	0.9\\
-0.3	0.848351136236277\\
-0.200633082006185	0.8\\
-0.2	0.799684795447187\\
-0.0999999999999996	0.762643071576338\\
0	0.741414743813181\\
0.0999999999999996	0.733654639739243\\
0.2	0.737228235247928\\
0.3	0.750438667324241\\
0.4	0.772089096301248\\
0.494756159790527	0.8\\
0.5	0.801292909253371\\
0.6	0.833734320087034\\
0.7	0.872496116269976\\
0.760255689730473	0.9\\
0.8	0.916032519210885\\
0.9	0.962796529935062\\
0.970434019460543	1\\
1	1.01442324200874\\
1.1	1.06881481721332\\
1.15250293039934	1.1\\
1.2	1.12720442211618\\
1.3	1.18873185793673\\
1.31754417146707	1.2\\
1.4	1.25327019827769\\
1.4700633766387	1.3\\
1.5	1.32085080241566\\
1.6	1.39080578450326\\
1.61352139363042	1.4\\
1.7	1.46398001716107\\
1.75083246370284	1.5\\
1.8	1.5393650669277\\
1.88134112013441	1.6\\
1.9	1.61632619677487\\
2	1.69511265248224\\
2.00713973603707	1.7\\
2.1	1.77754142571688\\
2.13158854055533	1.8\\
2.2	1.86158829312137\\
2.25179198212445	1.9\\
2.3	1.94699327446007\\
2.36829659427076	2\\
2.4	2.03356752372226\\
2.48161886466599	2.1\\
2.5	2.12118115127917\\
2.59222960183193	2.2\\
2.6	2.20975456739695\\
2.7	2.2994783039543\\
2.70084302629173	2.3\\
2.8	2.39308594057159\\
2.81100720855876	2.4\\
2.9	2.48783441513558\\
2.91913104055373	2.5\\
3	2.58366789036083\\
0.1	92\\
0.0111976833689177	3.2\\
0	3.19578147506999\\
-0.0999999999999996	3.15125029401082\\
-0.190720666993898	3.1\\
-0.2	3.09481633289347\\
-0.3	3.03077228304762\\
-0.340963635954971	3\\
-0.4	2.95342755723795\\
-0.459642787960958	2.9\\
-0.5	2.86020615532051\\
-0.556140944454813	2.8\\
-0.6	2.74549636396345\\
-0.63509139617161	2.7\\
-0.698106570568241	2.6\\
-0.7	2.59638154551655\\
-0.751288765365337	2.5\\
-0.78981509258919	2.4\\
-0.8	2.36305392081108\\
-0.818711675717513	2.3\\
-0.836782933278862	2.2\\
-0.843316272270422	2.1\\
-0.838585917110965	2\\
-0.822219693253426	1.9\\
-0.8	1.82320932176741\\
-0.794208527034009	1.8\\
-0.757122833738058	1.7\\
-0.70608485624096	1.6\\
-0.7	1.59054464657685\\
-0.645874080179132	1.5\\
-0.6	1.43841252638721\\
-0.572214343362955	1.4\\
-0.5	1.3167729971454\\
-0.485157623067871	1.3\\
-0.4	1.21680613535142\\
-0.381429412364956	1.2\\
-0.3	1.13403693175699\\
-0.251281920282763	1.1\\
-0.2	1.06683131417294\\
-0.0999999999999996	1.0148646939684\\
-0.0623054075989165	1\\
0	0.97641880722017\\
0.0999999999999996	0.951031160142205\\
0.2	0.937793479418705\\
0.3	0.935130632299026\\
0.4	0.941774719762769\\
0.5	0.956791346050216\\
0.6	0.979557400971454\\
0.66694253738719	1\\
0.7	1.00911170022113\\
0.8	1.04407018561222\\
0.9	1.08557678234296\\
0.929834212250817	1.1\\
1	1.13229668559116\\
1.1	1.1844391985699\\
1.12700129769665	1.2\\
1.2	1.24188452476141\\
1.29325572020378	1.3\\
1.3	1.3043516453698\\
1.4	1.37247448441187\\
1.43905904366209	1.4\\
1.5	1.44645366780371\\
1.56957111620903	1.5\\
1.6	1.52637571499763\\
1.68654182135657	1.6\\
1.7	1.61344148730941\\
1.7910584527652	1.7\\
1.8	1.71041634424315\\
1.88347780856096	1.8\\
1.9	1.82271902135247\\
1.96312723236412	1.9\\
2	1.96059554937957\\
2.02789978628457	2\\
2.07773743003145	2.1\\
2.1	2.16990934162853\\
2.11171543650609	2.2\\
2.13110978538117	2.3\\
2.133286341743	2.4\\
2.11821816271393	2.5\\
2.1	2.55694104186359\\
2.08642637533054	2.6\\
2.03777759343156	2.7\\
2	2.75627096351608\\
1.96888374984436	2.8\\
1.9	2.87772815002336\\
1.87811078330488	2.9\\
1.8	2.97005314666061\\
1.76132726197364	3\\
1.7	3.04484995736795\\
1.60885062097104	3.1\\
1.6	3.10534949134645\\
1.5	3.15872579699166\\
1.40089421456815	3.2\\
0.1	83\\
-1.19441391848373	-2.5\\
-1.1	-2.53090022423372\\
-1	-2.55479277671902\\
-0.9	-2.57092666323108\\
-0.8	-2.58005692141297\\
-0.7	-2.58257150585514\\
-0.6	-2.57853163488335\\
-0.5	-2.56768224323932\\
-0.4	-2.54943436177789\\
-0.3	-2.52281746894878\\
-0.235692276059169	-2.5\\
-0.2	-2.48763608270935\\
-0.0999999999999996	-2.44371255862938\\
-0.0220112400447092	-2.4\\
0	-2.38711481204998\\
0.0999999999999996	-2.31678158090921\\
0.120465065270185	-2.3\\
0.2	-2.22638637073938\\
0.225076445239173	-2.2\\
0.3	-2.10425233799104\\
0.303065131738387	-2.1\\
0.361323175937793	-2\\
0.4	-1.90086972778336\\
0.400332634673751	-1.9\\
0.424650993231814	-1.8\\
0.432997774032881	-1.7\\
0.426191917032264	-1.6\\
0.404319874894257	-1.5\\
0.4	-1.48861533604914\\
0.36924395159794	-1.4\\
0.319176711045852	-1.3\\
0.3	-1.27089479346524\\
0.254783665810609	-1.2\\
0.2	-1.13140008160579\\
0.174577031335918	-1.1\\
0.0999999999999996	-1.0231586651579\\
0.0759686715940534	-1\\
0	-0.936427783616657\\
-0.0497097157180136	-0.9\\
-0.0999999999999996	-0.866783838787154\\
-0.2	-0.813282549158007\\
-0.233230566202947	-0.8\\
-0.3	-0.775115621157507\\
-0.4	-0.751656515268275\\
-0.5	-0.741921307341556\\
-0.6	-0.744865455028609\\
-0.7	-0.759346938089424\\
-0.8	-0.784246435627154\\
-0.845384466903404	-0.8\\
-0.9	-0.818837800144719\\
-1	-0.862152179800085\\
-1.07436748298141	-0.9\\
-1.1	-0.913588235568292\\
-1.2	-0.973575350131604\\
-1.24047138638277	-1\\
-1.3	-1.04240528824142\\
-1.37656656509458	-1.1\\
-1.4	-1.12009208899757\\
-1.4910597548025	-1.2\\
-1.5	-1.20936919217268\\
-1.58732655705594	-1.3\\
-1.6	-1.31651956604029\\
-1.66692601983745	-1.4\\
-1.7	-1.45478289332484\\
-1.72955383559178	-1.5\\
-1.77501120754536	-1.6\\
-1.8	-1.69136867018211\\
-1.80268307692702	-1.7\\
-1.81353609549499	-1.8\\
-1.80528028724488	-1.9\\
-1.8	-1.92000282056343\\
-1.7781855640184	-2\\
-1.72972696450102	-2.1\\
-1.7	-2.14336552601492\\
-1.65699229692349	-2.2\\
-1.6	-2.25933119331813\\
-1.55464015193189	-2.3\\
-1.5	-2.34290360880817\\
-1.41146004761139	-2.4\\
-1.4	-2.40696451960832\\
-1.3	-2.45869220205438\\
-1.2	-2.498233294217\\
-1.19441391848373	-2.5\\
0.2	53\\
-1.1110107810032	-2.1\\
-1.1	-2.10586377438713\\
-1	-2.14623486344838\\
-0.9	-2.17284571392114\\
-0.8	-2.18734537985269\\
-0.7	-2.19058372591816\\
-0.6	-2.18274026624267\\
-0.5	-2.16337162921036\\
-0.4	-2.13138885885827\\
-0.331077280087101	-2.1\\
-0.3	-2.0840451339635\\
-0.2	-2.01634593367577\\
-0.180513226465674	-2\\
-0.0999999999999996	-1.91535412434921\\
-0.0876566932933196	-1.9\\
-0.029008226138482	-1.8\\
0	-1.7159478950801\\
0.00500917197271893	-1.7\\
0.0168323164140557	-1.6\\
0.00862292198235633	-1.5\\
0	-1.46946333930091\\
-0.0202599564096325	-1.4\\
-0.0711490921231717	-1.3\\
-0.0999999999999996	-1.26057919440402\\
-0.149892189822977	-1.2\\
-0.2	-1.15382500850274\\
-0.271899142028284	-1.1\\
-0.3	-1.08291042770874\\
-0.4	-1.03822037257334\\
-0.5	-1.01154719171238\\
-0.6	-1.00199796105032\\
-0.7	-1.00833835573888\\
-0.8	-1.02914829898895\\
-0.9	-1.06292434677611\\
-0.981652183301716	-1.1\\
-1	-1.10946787056388\\
-1.1	-1.17319652418028\\
-1.13650061634516	-1.2\\
-1.2	-1.25628620138988\\
-1.24491408543305	-1.3\\
-1.3	-1.36946769629913\\
-1.32303086763793	-1.4\\
-1.37529539453001	-1.5\\
-1.4	-1.58435059325662\\
-1.40459238292914	-1.6\\
-1.40990601447372	-1.7\\
-1.4	-1.75353690786374\\
-1.39024204535643	-1.8\\
-1.34146505751322	-1.9\\
-1.3	-1.95347024205576\\
-1.25581516563001	-2\\
-1.2	-2.04509730651388\\
-1.1110107810032	-2.1\\
0.2	89\\
0.274751860717394	1.1\\
0.3	1.0963427307329\\
0.4	1.09202360985427\\
0.5	1.09761206995613\\
0.515622177157291	1.1\\
0.6	1.11189479557843\\
0.7	1.13427262770186\\
0.8	1.16438772170873\\
0.894694564337819	1.2\\
0.9	1.20194747691912\\
1	1.24653245053817\\
1.1	1.29785890334108\\
1.10372011712272	1.3\\
1.2	1.35708744607483\\
1.26617426445665	1.4\\
1.3	1.42356737891682\\
1.4	1.49785763872992\\
1.40279204211875	1.5\\
1.5	1.58388617702001\\
1.51852588368217	1.6\\
1.6	1.68328830780011\\
1.616731998554	1.7\\
1.69807275217778	1.8\\
1.7	1.80309806274334\\
1.76436123433942	1.9\\
1.8	1.97482780179351\\
1.81327744563824	2\\
1.8465517564932	2.1\\
1.86221927858181	2.2\\
1.86054989801762	2.3\\
1.840759273288	2.4\\
1.80086621516834	2.5\\
1.8	2.50155365290898\\
1.741193243225	2.6\\
1.7	2.65116621347043\\
1.65587849858821	2.7\\
1.6	2.75181870936844\\
1.53897453681262	2.8\\
1.5	2.82798519689248\\
1.4	2.88780908055148\\
1.37534920589074	2.9\\
1.3	2.93678649368536\\
1.2	2.9752090613526\\
1.11540597642977	3\\
1.1	3.0047628138521\\
1	3.0285468808015\\
0.9	3.04466308851194\\
0.8	3.053840107774\\
0.7	3.0564569772967\\
0.6	3.05258181333995\\
0.5	3.04198342228955\\
0.4	3.02411785721369\\
0.306981908091964	3\\
0.3	2.99829349422546\\
0.2	2.96656757145591\\
0.0999999999999996	2.9246829376031\\
0.052107498913799	2.9\\
0	2.87268712031644\\
-0.0999999999999996	2.80826357833536\\
-0.111212039296437	2.8\\
-0.2	2.72905091892268\\
-0.23167711715216	2.7\\
-0.3	2.6283704667781\\
-0.324693488834874	2.6\\
-0.396777924399311	2.5\\
-0.4	2.49448569257583\\
-0.453658838843335	2.4\\
-0.494391804657475	2.3\\
-0.5	2.28021965668903\\
-0.523459469237749	2.2\\
-0.539738923505826	2.1\\
-0.543597713058122	2\\
-0.535207831584275	1.9\\
-0.514070869764187	1.8\\
-0.5	1.7592073523372\\
-0.480743967751949	1.7\\
-0.433938731496893	1.6\\
-0.4	1.54540081106016\\
-0.371546019897955	1.5\\
-0.3	1.41051725410057\\
-0.291051249281245	1.4\\
-0.2	1.31175541484434\\
-0.186216912706773	1.3\\
-0.0999999999999996	1.23656072751925\\
-0.0383069866040012	1.2\\
0	1.17958723004407\\
0.0999999999999996	1.13856086692821\\
0.2	1.11140465696997\\
0.274751860717394	1.1\\
0.3	19\\
-0.843126623940703	-1.7\\
-0.8	-1.72291105273864\\
-0.7	-1.73634672489264\\
-0.6	-1.71243299700914\\
-0.579430785600774	-1.7\\
-0.5	-1.60377388441653\\
-0.49805248910124	-1.6\\
-0.5	-1.53741505495576\\
-0.50187368933501	-1.5\\
-0.6	-1.40454889870366\\
-0.617854543179808	-1.4\\
-0.7	-1.38965562779331\\
-0.748090793691788	-1.4\\
-0.8	-1.4195391713314\\
-0.891179189724286	-1.5\\
-0.9	-1.53288548164201\\
-0.911895174196413	-1.6\\
-0.9	-1.62777732723053\\
-0.843126623940703	-1.7\\
0.3	73\\
0.145694144746521	1.3\\
0.2	1.27781562225614\\
0.3	1.25001321589568\\
0.4	1.23525671085728\\
0.5	1.23214098409858\\
0.6	1.23956035508768\\
0.7	1.25666885672882\\
0.8	1.28282617312748\\
0.848861647609635	1.3\\
0.9	1.31817619038905\\
1	1.36244080551148\\
1.07200863149204	1.4\\
1.1	1.41555097075452\\
1.2	1.47894784930215\\
1.23015941716758	1.5\\
1.3	1.55464997478256\\
1.3542738179061	1.6\\
1.4	1.64504401029638\\
1.45398828447295	1.7\\
1.5	1.75838277240941\\
1.53287324348268	1.8\\
1.59219460814085	1.9\\
1.6	1.91976616260476\\
1.63329201495804	2\\
1.65551111053064	2.1\\
1.65866736338282	2.2\\
1.64188633744008	2.3\\
1.60323225213639	2.4\\
1.6	2.40564484318514\\
1.53991519266854	2.5\\
1.5	2.54605020726729\\
1.44540551407404	2.6\\
1.4	2.63744379235577\\
1.30684549115897	2.7\\
1.3	2.70417422644546\\
1.2	2.75497900699802\\
1.1	2.79298970245893\\
1.07538141159186	2.8\\
1	2.82152527251575\\
0.9	2.84090795243646\\
0.8	2.85171902605871\\
0.7	2.85444704453686\\
0.6	2.84918119401376\\
0.5	2.83563154054751\\
0.4	2.81311590202476\\
0.35843561707518	2.8\\
0.3	2.78147702013287\\
0.2	2.73930573294264\\
0.127541060400504	2.7\\
0.0999999999999996	2.68395907043906\\
0	2.61289986268359\\
-0.0155592870598249	2.6\\
-0.0999999999999996	2.51915456399411\\
-0.117643670592767	2.5\\
-0.192950406582537	2.4\\
-0.2	2.38790587923601\\
-0.248363520779329	2.3\\
-0.286146390116099	2.2\\
-0.3	2.13899308884509\\
-0.308899938910995	2.1\\
-0.317481323657045	2\\
-0.311655995170156	1.9\\
-0.3	1.84320965856124\\
-0.291378553904867	1.8\\
-0.256010716533409	1.7\\
-0.202697722581435	1.6\\
-0.2	1.59613505644914\\
-0.127995662446093	1.5\\
-0.0999999999999996	1.47074073455455\\
-0.0222534684010465	1.4\\
0	1.38321244313828\\
0.0999999999999996	1.32127811853214\\
0.145694144746521	1.3\\
0.4	59\\
0.331322415647426	1.4\\
0.4	1.38221485192624\\
0.5	1.36988877716586\\
0.6	1.37031630674661\\
0.7	1.38237087940014\\
0.776736202406938	1.4\\
0.8	1.40559172221491\\
0.9	1.44104339163475\\
1	1.48640380436519\\
1.02471661850994	1.5\\
1.1	1.54618620648149\\
1.17628877792826	1.6\\
1.2	1.61976688968628\\
1.28696578992231	1.7\\
1.3	1.71521499337439\\
1.36830773393995	1.8\\
1.4	1.85423808820971\\
1.42619536690478	1.9\\
1.46161528958403	2\\
1.47546215318948	2.1\\
1.46670798130024	2.2\\
1.43345945414298	2.3\\
1.4	2.35742200750056\\
1.37103343923237	2.4\\
1.3	2.4743320817395\\
1.26967980802453	2.5\\
1.2	2.54848389857004\\
1.10225270802037	2.6\\
1.1	2.60106901015158\\
1	2.63737195904724\\
0.9	2.66146626252349\\
0.8	2.67469040769362\\
0.7	2.67774046345198\\
0.6	2.67076943395766\\
0.5	2.65342901000084\\
0.4	2.62486280374595\\
0.338335576014468	2.6\\
0.3	2.58320877821784\\
0.2	2.52572580519236\\
0.164058097396633	2.5\\
0.0999999999999996	2.446029425574\\
0.0540929557395149	2.4\\
0	2.33080152904543\\
-0.0214270464733655	2.3\\
-0.0719326698172458	2.2\\
-0.0999999999999996	2.10865488864196\\
-0.102539505708154	2.1\\
-0.114998151866222	2\\
-0.110060717700357	1.9\\
-0.0999999999999996	1.85552770139992\\
-0.0870560042317715	1.8\\
-0.0438412702922933	1.7\\
0	1.63310212799718\\
0.0244469120566312	1.6\\
0.0999999999999996	1.52503618321718\\
0.131029761722136	1.5\\
0.2	1.45577393369994\\
0.3	1.40950002644591\\
0.331322415647426	1.4\\
0.5	45\\
0.296513302179633	1.6\\
0.3	1.59773402490539\\
0.4	1.55038184118522\\
0.5	1.52336726483859\\
0.6	1.51453088909944\\
0.7	1.52207527739298\\
0.8	1.54447309226641\\
0.9	1.58037340571928\\
0.940171684769469	1.6\\
1	1.63453646808017\\
1.09211729642188	1.7\\
1.1	1.70718832260496\\
1.18749812335384	1.8\\
1.2	1.8190975248171\\
1.2480065761736	1.9\\
1.28103455849397	2\\
1.28630118986504	2.1\\
1.26145735636177	2.2\\
1.2031946392843	2.3\\
1.2	2.30364324431682\\
1.1	2.39334802150047\\
1.08985264672641	2.4\\
1	2.44687499441054\\
0.9	2.48116508184204\\
0.8	2.49970526334771\\
0.792830409182889	2.5\\
0.7	2.50350589440376\\
0.661697726707831	2.5\\
0.6	2.493825778876\\
0.5	2.46940679328918\\
0.4	2.42938001370316\\
0.347602770416208	2.4\\
0.3	2.36748793845671\\
0.223686314855409	2.3\\
0.2	2.27159867098164\\
0.149999896452268	2.2\\
0.106530173685931	2.1\\
0.0999999999999996	2.06500451698463\\
0.0889425823929084	2\\
0.0943842555885331	1.9\\
0.0999999999999996	1.88020493214764\\
0.125338523092789	1.8\\
0.185932387194911	1.7\\
0.2	1.68443209621098\\
0.296513302179633	1.6\\
0.6	27\\
0.599032097756521	1.7\\
0.6	1.69978907219004\\
0.7	1.69913388189385\\
0.704280679644863	1.7\\
0.8	1.72516834386136\\
0.9	1.77504661398277\\
0.934827235398727	1.8\\
1	1.87027989128214\\
1.02184779020345	1.9\\
1.05512142525222	2\\
1.04545257281535	2.1\\
1	2.179560662024\\
0.982623900215369	2.2\\
0.9	2.25503492135313\\
0.8	2.29062219050482\\
0.7	2.29888626746151\\
0.6	2.2816699765207\\
0.5	2.23888871819641\\
0.44359729629551	2.2\\
0.4	2.15349136142643\\
0.363049971734843	2.1\\
0.333499091291324	2\\
0.344197336613251	1.9\\
0.398990431290686	1.8\\
0.4	1.79899806620934\\
0.5	1.73172365544712\\
0.599032097756521	1.7\\
};
\end{axis}
\end{tikzpicture}%
\end{document}
% This file was created by matlab2tikz.
% Minimal pgfplots version: 1.3
%
%The latest updates can be retrieved from
%  http://www.mathworks.com/matlabcentral/fileexchange/22022-matlab2tikz
%where you can also make suggestions and rate matlab2tikz.
%
\documentclass[tikz]{standalone}
\usepackage{pgfplots}
\usepackage{grffile}
\pgfplotsset{compat=newest}
\usetikzlibrary{plotmarks}
\usepackage{amsmath}

\begin{document}
\definecolor{mycolor1}{rgb}{0.00000,0.44700,0.74100}%
\definecolor{mycolor2}{rgb}{0.85000,0.32500,0.09800}%
%
\begin{tikzpicture}

\begin{axis}[%
width=1.5in,
height=1.5in,
scale only axis,
xmin=-3,
xmax=3,
ymin=-3.2,
ymax=3.2,
title={fifth component}
]
\addplot [color=mycolor1,only marks,mark=o,mark options={solid},forget plot]
  table[row sep=crcr]{%
0.0259388858243524	1.99651291936462\\
0.0479365430139229	2.00136242712261\\
0.0808011428276144	2.00015957097738\\
0.0960018408277835	2.00397357604688\\
0.123924398192739	1.99393565592043\\
0.144716297235007	1.99609680237626\\
0.172432773795358	2.00782480916528\\
0.211096968463268	2.00289657674787\\
0.241323554557747	2.00054011148902\\
0.248450748340069	1.98323290473631\\
0.278037158020472	1.99775137204772\\
0.295123845636812	2.024317082564\\
0.278089869880903	1.98953669925669\\
0.349224330459094	2.01774494803247\\
0.389769425315202	1.96915513448386\\
0.443104169526806	2.00181328676435\\
0.445366816453369	2.00702160442149\\
0.447621894481189	1.9639008226548\\
0.459268204218018	1.96874477518371\\
0.525152271376063	1.93928073190811\\
0.479912634035923	1.9531725626356\\
0.541892480429807	1.98767075962513\\
0.596086730914084	1.95159718447859\\
0.642409661124227	1.88862872560156\\
0.59672705695044	2.06269182867749\\
0.577160609928358	1.93769298667604\\
0.61270638646484	1.9863001413635\\
0.738131551064678	1.99453476973758\\
0.731223036941714	1.9720505976831\\
0.728694499517447	2.00387465318476\\
0.792620470002546	2.01820212338234\\
0.76404643370489	1.92712620890764\\
0.783621400051233	1.97170029129066\\
0.844923525499164	2.0002383443526\\
0.933088804435648	1.86929978545337\\
0.826522508566799	1.96489496451136\\
0.857943301337058	1.93574991230876\\
0.862415931324452	1.91824418132958\\
0.85957751145223	1.94915043906307\\
1.10549276410824	1.89209624896601\\
0.992380983980335	1.9389778055659\\
1.05163144833557	1.9077966876626\\
0.971362952394099	1.94526987131734\\
1.10648177744783	1.98917932950755\\
1.12898410068108	1.88302810812235\\
1.14935856334692	1.80741423289017\\
1.2475966096538	1.80328803923413\\
1.02094574055669	1.89016584759305\\
1.18313200649952	1.80836196844854\\
1.11713521982497	1.97114350530034\\
1.21095637410431	1.6606506602979\\
1.09752975594461	1.89186114518647\\
1.1286092446374	1.8521928055118\\
1.17641657796315	1.85179775653125\\
1.2027986521108	1.79291729003239\\
1.40862693119303	1.68188184454732\\
1.29643649567147	1.6835431657974\\
1.46182858553516	1.76685142278392\\
1.44694035368565	1.75403257372017\\
1.49092139521946	1.61038316326389\\
1.3379059086772	1.66077993928176\\
1.46168112951844	1.88866965377674\\
1.41023026436323	1.73928409103319\\
1.4714192456311	1.78600061226619\\
1.35577541010529	1.79014999256957\\
1.4051660334094	1.67552928758717\\
1.61761103881626	1.92386710385478\\
1.24305770227106	1.88248740497196\\
1.49729193411573	1.81459528264337\\
1.63186681467879	1.63329561110855\\
1.53796361955124	1.77930541551158\\
1.46647340378635	1.89266145452761\\
1.59115626629009	1.92971274617182\\
1.55598375769603	1.68322562636936\\
1.60162857304111	1.74522783244834\\
1.55820638402395	1.58511515395533\\
1.6445086680066	1.69602995640505\\
1.43117976594691	1.39522816594688\\
1.66046656580406	1.5020485139169\\
1.89457571586182	1.8060245580165\\
1.7690851756095	1.803354010806\\
1.51808575625609	1.62902388762982\\
1.71974982125714	1.56603639813406\\
1.63851255324909	1.37209586707968\\
1.73596507613967	1.53438459796941\\
1.8856967379105	1.45170562832068\\
1.83646220685143	1.51078810063935\\
1.6807272288616	1.41341108157641\\
1.6794592820606	1.42428516993435\\
1.84697126074185	1.69896546443275\\
1.65822816966463	1.41270723759625\\
1.78371507953502	1.42947586399192\\
1.87899798012913	1.57760927136629\\
1.89142792776872	1.58934082453833\\
1.89301329213783	1.04572276609604\\
1.75219375143258	1.4487568304138\\
2.1047354401674	1.41876136406707\\
2.01123547254846	1.3110508611222\\
1.96781276802938	1.52540722041352\\
1.51166361010094	1.32992221483295\\
2.22158609914819	1.4805071317003\\
1.97788676892182	1.64185849242464\\
1.94851336550087	1.45074289017143\\
2.05150009067797	1.21139868557201\\
1.93322927547617	1.22986741742318\\
1.99640422827958	0.927539227571993\\
2.02784448795865	1.33792744894102\\
2.05296048996672	1.40894341042615\\
1.79681210188576	1.18409186751766\\
2.04186829813144	1.41141478615\\
2.06457591322178	1.27562881170451\\
1.9610792821858	1.32599888091024\\
2.02517402706913	1.29054836978655\\
1.90898477009428	1.23168003520599\\
1.95382687452845	1.07890142563229\\
1.98007488912549	0.865945916419211\\
1.91942744299385	1.23088568855929\\
1.90203680990463	1.46675740172822\\
1.79605423362615	0.99553318086307\\
1.93675059768613	1.26622062159566\\
1.80589580300649	1.38156657036724\\
1.78116659707435	1.22679608853667\\
1.4859041706582	1.42125093730957\\
1.80527660882125	1.44977387260663\\
2.02922661129011	1.0207490772509\\
1.81798786252392	0.647952952360472\\
2.32931027046338	1.2400230228602\\
1.62944341809323	0.943292490703794\\
1.98834525677686	1.34211488379452\\
1.91927966663161	1.25902905273399\\
2.0814706955723	1.07488996429983\\
1.99353342912541	0.988261144028966\\
2.10800463348708	0.766165616265449\\
1.8550225815185	1.13356072105016\\
1.89272673258568	1.36537195907968\\
1.8902074337978	0.757056328403824\\
1.53885687334523	1.07694569130808\\
1.76112331184547	0.998690472291168\\
2.09154477395659	0.694075837194668\\
1.90858542651557	0.757727713019561\\
1.9330049256032	0.976874558375833\\
1.93983302339883	0.816051509797898\\
2.24506862568672	0.672826581876916\\
1.82718437556353	1.10936506743491\\
1.96057264525256	0.893925986104898\\
1.93646566414943	0.64731870645308\\
2.39122724922164	1.24345442192263\\
2.24889413709987	0.793537657333463\\
2.39607565749392	0.91178625334131\\
1.80986080122559	0.707169946665853\\
1.97704296558096	0.591120152065339\\
2.00293771191178	1.36005407696541\\
2.24796362886873	0.686533510972829\\
2.11424994022843	0.877619102581024\\
2.34406728575871	0.795199528805402\\
1.56821878862338	0.60163344885657\\
1.82330236671711	0.291695714299127\\
1.45039697294321	0.590849030637304\\
1.88899283418446	1.04552268182906\\
1.52058053004542	0.435160238141619\\
1.66251832326001	0.804648110802819\\
1.64664565454418	0.505949448785172\\
1.91508532580206	0.847486502505432\\
1.34263224779645	0.88256267651136\\
2.26055865989652	0.538404924773607\\
1.69524767103065	0.741992169305531\\
1.94016209417356	0.741514759245998\\
1.82019093577253	0.328368525179733\\
1.4839520831286	0.681796346611881\\
1.13805529399511	0.728796623608036\\
1.73651102992287	0.402875684463721\\
2.40970926685092	0.766519696154169\\
2.12006326132181	0.724614576326541\\
1.33511311753437	0.0606578626326615\\
1.21694480442495	0.939960475214773\\
1.49919597789097	0.779844441507984\\
1.63203786050577	0.615522830684201\\
1.28009907725904	0.603133599618405\\
1.83363385147587	0.447426311107601\\
1.25294254329276	0.891238649047264\\
1.38589448744568	0.110193693911327\\
2.12557258718458	0.480767020439752\\
1.76853338890148	0.345961715962911\\
1.1838175823629	-0.0335760456264687\\
1.07748933668513	0.396363550377259\\
1.91951574719719	0.710847307761236\\
0.890323101344239	0.539672205647595\\
1.65348820931753	0.512238368108479\\
1.65111543031058	-0.265446989923681\\
1.6070900231769	0.488398313760616\\
0.901758997337281	0.183058802837383\\
1.71870592048393	-0.251816153684363\\
1.55442462019974	0.00466747014038243\\
0.795021559669278	0.0361205015350433\\
0.958568445751217	0.212363974452783\\
1.67442948239855	0.4882871131843\\
1.2206685044667	-0.324863168774303\\
1.50767341273933	-0.389108334817502\\
1.37836608274919	-0.0373397947956927\\
1.11156784444486	0.701945681672377\\
1.4984756552433	0.00584512978944459\\
1.24690289872505	0.226154641123381\\
1.09194532469786	-0.130683470031559\\
0.678183222535072	-0.0365335083510088\\
1.73612089885204	0.397031801733272\\
0.933320654619196	0.746473890326733\\
1.15525443706948	-0.179109971394189\\
0.458916732169228	-0.213954536366935\\
1.58359507914698	-0.231297472151952\\
0.931434909226943	0.114256234330456\\
1.1322927346265	0.17900469841877\\
0.780331897364257	0.102401717565955\\
0.501871266865262	-0.0628649708451471\\
0.503048632418745	-0.459092444286199\\
0.604765068083075	-0.011679852416595\\
0.666178137609782	0.219390527881344\\
0.413245356250045	-0.0695882331429949\\
0.467795007897315	-0.0999511872763669\\
1.15246888796731	-0.0589372144022903\\
0.792490565783635	-0.386299556418842\\
0.747652249238335	0.192053680178325\\
0.6730227064451	-0.457495772861639\\
0.657608820873446	-0.397844794344219\\
0.80283938775403	0.00586599616798039\\
1.29819955341968	-0.204261874156702\\
0.700969575790672	-0.360362213508646\\
0.577484279146232	0.667638314415951\\
1.53670668302431	-0.432957390684138\\
-0.157947037327338	-0.206989394214058\\
-0.250033960160869	0.369996686022994\\
0.35679132363435	-0.672812036115748\\
0.307418930825796	-0.491736063949304\\
0.612754451313721	-0.591725661644702\\
0.513152577927538	-0.242776993818146\\
0.457076244801449	-0.714876480294682\\
0.374036335051672	-0.347506625594372\\
0.579283772090358	0.220733907465001\\
0.187849487939868	0.169639639117464\\
-0.0361733332976721	0.281051835212802\\
0.772027729747197	-0.505718105015689\\
0.269625343575084	-0.756939350867156\\
0.60570524403307	-0.119581878977858\\
-0.192532712828272	-0.0344061540276999\\
-0.179793749480468	-0.900222299631903\\
0.0699543465880473	-0.840398840157916\\
0.168779646202848	-0.973477052749994\\
-0.00586739214808801	-0.369383158623758\\
0.465908963186411	-0.973925709442261\\
0.160490995593304	-0.295058807475094\\
0.168397029916368	-0.980178812503749\\
};
\addplot [color=mycolor2,only marks,mark=o,mark options={solid},forget plot]
  table[row sep=crcr]{%
-0.026947097397927	-1.54967503001056\\
-0.0499096052447985	-1.548838725865\\
-0.0878921441942618	-1.54358858745209\\
-0.102243240432917	-1.54118951004161\\
-0.115588964597783	-1.54384949729\\
-0.166126369580897	-1.54954111761082\\
-0.171775059288128	-1.53255169798828\\
-0.202468902552123	-1.53693920317449\\
-0.212030786452003	-1.57194441918315\\
-0.242543396618313	-1.55982750802416\\
-0.268638030522741	-1.55184551810696\\
-0.307999262514236	-1.54141240307698\\
-0.34184706998786	-1.55670987888773\\
-0.321654511497463	-1.58426559222047\\
-0.392925170560618	-1.51563730344866\\
-0.369402481562153	-1.52198515289873\\
-0.416941632627096	-1.51482183374776\\
-0.47214126557288	-1.57339572749443\\
-0.493265130467738	-1.49632259648208\\
-0.512861932358619	-1.50290927620725\\
-0.557340464169725	-1.50842598279773\\
-0.515456646472125	-1.56774163478792\\
-0.608332652640224	-1.42764267683647\\
-0.571584245650045	-1.49817383565417\\
-0.662130333502388	-1.56697697704636\\
-0.669954622356556	-1.57493033524355\\
-0.694374248931051	-1.46017650026605\\
-0.763136406470192	-1.44654664398095\\
-0.64832005525922	-1.5311300933111\\
-0.721079189047473	-1.57176135978772\\
-0.762850186528795	-1.48791858695594\\
-0.771668694696658	-1.4609916619696\\
-0.82794824307844	-1.46985243432682\\
-0.757762090173658	-1.51439340279633\\
-0.73200316350585	-1.54643662607006\\
-0.849650868675982	-1.4739825416715\\
-0.869633988303967	-1.36513677682954\\
-0.929617676304213	-1.40255715057468\\
-0.976307028854808	-1.46241069529678\\
-0.929398084649915	-1.49557897888553\\
-1.08884466383786	-1.60008247625742\\
-1.1061052512272	-1.4667516416831\\
-0.989287983622614	-1.44461809994125\\
-0.985959410143955	-1.5033836797293\\
-1.18771458192431	-1.33744464545877\\
-1.00331746689445	-1.40505396838939\\
-0.892432339761976	-1.52461527768079\\
-1.06850301949737	-1.42382215016215\\
-1.21578154069187	-1.35143372749743\\
-1.1054763672977	-1.37105316655968\\
-1.05324198188984	-1.36449836483918\\
-1.36583986729047	-1.47144251941379\\
-1.21498723375497	-1.58153140750668\\
-1.15864586873963	-1.35211138149604\\
-1.29274883019543	-1.24706936498748\\
-1.21165267830111	-1.36424729300889\\
-1.13339068136518	-1.38791313506574\\
-1.38647009921518	-1.36164505679436\\
-1.38447268580765	-1.24424678906453\\
-1.30846082709955	-1.31326104146817\\
-1.26981825327827	-1.22686897127221\\
-1.49956364455672	-1.20155729619685\\
-1.44709290956236	-1.3499491526295\\
-1.42148110929907	-1.2096773418177\\
-1.48960505179888	-1.41647829527412\\
-1.3264878697695	-1.47045835775913\\
-1.68833293650043	-1.35607977232846\\
-1.58968210104544	-1.26255149104674\\
-1.55268500555184	-1.2291738166282\\
-1.68292402606647	-1.11026634760032\\
-1.72154852078698	-1.24202688963492\\
-1.43590169300641	-1.37871676926254\\
-1.69132652673162	-1.07877471235978\\
-1.70551483678654	-1.42804960005592\\
-1.53520709567707	-1.17299419798963\\
-1.63554863873912	-1.28753234363309\\
-1.65650253897248	-1.09178635476079\\
-1.62198948544192	-1.19413197420841\\
-1.47519706775498	-1.28565950980904\\
-1.43433635492227	-1.32753123102665\\
-1.73478223846456	-1.20660071287845\\
-1.90432859972523	-1.29120302815414\\
-1.9071129230002	-0.972886141766062\\
-1.69183439158024	-1.10780628569212\\
-1.68284658996465	-1.31833231939511\\
-1.85992398322545	-1.01141698924493\\
-1.72772929472913	-0.98157402042344\\
-1.74885476713543	-1.4351574402123\\
-1.84150517090143	-0.95425928858046\\
-1.69315517542604	-0.958656958688715\\
-1.7286495891314	-0.954601665072524\\
-1.81811333022962	-1.19579885376329\\
-1.91064332117842	-1.29835844701107\\
-2.115454982706	-1.0182126578052\\
-1.81014864687136	-1.11941452772307\\
-1.88306383492019	-1.01716868637323\\
-1.86797031445803	-1.04760399233855\\
-1.99290332758122	-1.38532172129099\\
-1.57013566090976	-0.954717363457782\\
-1.6305853956127	-0.926490388852283\\
-1.82503267963498	-1.2054049019875\\
-2.04745234990293	-0.70712356859999\\
-1.91302909485597	-0.765494691747267\\
-1.79315031506749	-1.07048160263372\\
-1.67520961679342	-0.869000607592547\\
-1.98886073309524	-1.32715925014926\\
-2.01375717222897	-0.9315973741601\\
-1.93408159383817	-0.878274086050914\\
-2.26246434296342	-0.886063271151665\\
-1.72639050831228	-0.560229703702508\\
-2.06045827186263	-0.977280310897143\\
-2.21522054773345	-0.832044314048742\\
-2.12802823080291	-0.765383980210819\\
-2.47819250264965	-0.688993880398068\\
-1.68985107972107	-0.732789288101983\\
-1.81469301090601	-0.558603652168838\\
-2.02534666967788	-0.825878531557616\\
-1.90668680745126	-0.687396711733751\\
-2.06031052575321	-0.615192926362136\\
-1.99585448031789	-0.990208999863819\\
-2.14213953096814	-0.765898153601535\\
-1.99073350792875	-0.38534769649198\\
-1.96145912243946	-0.756790477559718\\
-1.88353079013264	-0.493700231213568\\
-2.3752794639133	-0.852864483514567\\
-1.71756604008283	-0.839522058331343\\
-1.96581549537573	-0.658952618483849\\
-2.46496189427916	-0.82131795587749\\
-2.10629520423653	-0.845571433601728\\
-2.11100602865062	-0.528455628514166\\
-1.70854667582737	-0.522179557510307\\
-2.00450257438808	-1.16055808599304\\
-2.21156096039358	-0.482354346628644\\
-2.04160631785436	-0.533352916239695\\
-1.84559801482221	-0.658966536067479\\
-1.92091051009354	-0.644336627288141\\
-1.92864622951692	-0.408174908184284\\
-2.04092091269914	-0.53236208450286\\
-1.86431188937102	-0.409957901472978\\
-1.97521350506084	-0.613741462440286\\
-2.12216417761872	-0.431925911153432\\
-1.80339734046248	-0.583776426535711\\
-1.72974769480169	-0.586333850372364\\
-1.97641639419026	-0.40354436882594\\
-1.66790009602428	-0.535102761781156\\
-1.60550901999747	-0.517641138116707\\
-1.98093745067308	-0.248643233716586\\
-1.77048803449917	-0.0215214325064952\\
-1.82669905771791	-0.344780626915585\\
-2.19905576091282	-0.408420695054419\\
-1.83101161586013	-0.16573497405526\\
-1.99022362231772	-0.240850680829\\
-1.37200304413117	-0.218020520951735\\
-1.83273900042908	0.0375840492151414\\
-1.56236387312051	-0.0622109695320063\\
-1.83847741136532	-0.629170879373212\\
-1.62842916741517	-0.261531442505625\\
-1.760757004596	-0.122954408755528\\
-1.92693522151942	0.0317672214706444\\
-0.795922426677282	-0.305147429881106\\
-1.66083624793154	-0.460640523838259\\
-1.79682003601176	0.166845603602293\\
-1.64022836211743	-0.361636009773446\\
-2.53886522482741	-0.236580562503111\\
-2.03581414704918	0.013541805129293\\
-1.84630209912121	0.230376960175914\\
-1.91209527248309	-0.360580206613225\\
-1.81233556853063	0.0860452228596581\\
-1.85601948484678	-0.446016224299677\\
-1.44425457774794	0.133932704188332\\
-1.82082024175442	-0.524871278676388\\
-1.90283579546103	-0.024980003152043\\
-0.918616001910668	0.598439720318925\\
-1.37397981159442	-0.228688766885865\\
-2.13124001952824	0.0394435337110551\\
-1.51184628119662	-0.339909320146771\\
-1.42078378674312	0.259637468967449\\
-1.46934828152567	-0.48658770179399\\
-1.45036714720006	0.561650634597791\\
-1.6235939585632	0.100495743328196\\
-1.56361953447043	0.429563917963797\\
-1.72011242640493	0.125548136165587\\
-1.70161637144955	-0.00552791997817297\\
-1.3830196422212	0.484897744780771\\
-1.94541153808096	-0.142579452253231\\
-1.17461760721602	0.106879495508337\\
-1.09674397776978	0.918979238595924\\
-1.37782133309489	0.510300838744834\\
-1.88845389859122	0.238194094828098\\
-1.05273228631062	0.534963752730167\\
-1.29045492870113	0.541672326993093\\
-1.51108660201011	0.115095697290448\\
-1.60478470830124	0.761938729066085\\
-1.16297289088485	0.712515972958128\\
-1.36626122649495	0.0459699369006284\\
-1.00569677090725	0.566053726817701\\
-1.15605146422804	0.32048685029173\\
-0.822878786956236	0.299885052175656\\
-1.40380642068769	0.703790875671732\\
-1.59757716568005	0.713604210308872\\
-1.32350395473092	-0.0181115754076439\\
-1.15090617990717	0.0297575564492658\\
-1.21924769269614	0.636116178851027\\
-1.66330401562556	0.274767468591772\\
-1.34530843504664	0.53762093307587\\
-1.39326431751636	1.12504793697234\\
-0.833618774635974	0.512669380246674\\
-0.573367167351866	0.652510525923379\\
-0.276320238657219	0.18763449775689\\
-1.17194492857265	0.572769304168273\\
-0.784840550207506	0.714006761323611\\
-0.924013272173041	0.0557708685451278\\
-1.15282586642296	0.334082023567727\\
-0.935348154153094	0.774875910909737\\
-1.09740398474138	0.634480504642842\\
-1.11946267734234	0.610532582269006\\
-0.838968288924051	1.38322547926388\\
-1.22448365542895	1.32213641543783\\
-0.476481839809044	1.40105426500117\\
-0.769114345508466	0.66169599251629\\
-1.40027835036432	1.28884494337088\\
-1.50132634028759	0.613610138665922\\
-0.949826176623956	0.594374431492637\\
-0.88164099127801	0.552395160007791\\
-0.994300613393627	0.256827952699087\\
-0.632450277038536	0.881951176938878\\
-0.331331900379547	0.85471608258404\\
-1.13074002827094	0.521881807762682\\
-0.86984371951327	0.417083028748866\\
-0.420618985877099	0.726297500339449\\
-0.110157996853952	0.742495519066295\\
0.644069697204106	0.64901805109848\\
-0.262085393832434	0.700643157487521\\
-0.276442795861882	0.467852427149516\\
-0.14443352788167	0.705856371233983\\
-0.727695229224663	1.06740698127406\\
-0.626210638028053	0.986934718277524\\
-0.658293890852093	1.48386542068786\\
-0.71066942064998	0.843047675153383\\
0.185985214616693	0.552601466396477\\
-0.615839609456226	0.931267504635312\\
-0.533903453015464	1.49417117664361\\
-0.278043832459567	1.40015122194036\\
-0.374673778674507	0.760845850670364\\
0.0347222292805265	1.27907372368006\\
-0.41631147842867	0.89614152469315\\
-0.148017321318051	0.860943748299246\\
-0.318167564411286	1.66905910965134\\
0.565869201509072	1.63328081306408\\
0.456182094678272	1.2472468033001\\
};
\addplot[contour prepared, contour prepared format=matlab, contour/labels=false] table[row sep=crcr] {%
%
-0.4	35\\
-0.993042891731576	0.2\\
-1	0.197025943534734\\
-1.1	0.186872537300184\\
-1.17293489552501	0.2\\
-1.2	0.205432871662981\\
-1.3	0.252121118388022\\
-1.37102113917939	0.3\\
-1.4	0.324509805384991\\
-1.47014468364092	0.4\\
-1.5	0.446400917413425\\
-1.52945760780433	0.5\\
-1.55779532916907	0.6\\
-1.55864577143698	0.7\\
-1.53053158784621	0.8\\
-1.5	0.853132324716212\\
-1.46447654739919	0.9\\
-1.4	0.957233984861332\\
-1.32528630734457	1\\
-1.3	1.01143841443397\\
-1.2	1.03319486318732\\
-1.1	1.03089136984885\\
-1	1.00417718398177\\
-0.991978942629969	1\\
-0.9	0.941579991218085\\
-0.857305892379541	0.9\\
-0.8	0.822510372767615\\
-0.787707691441127	0.8\\
-0.755274352536136	0.7\\
-0.745148667151422	0.6\\
-0.755948671990232	0.5\\
-0.786745949786038	0.4\\
-0.8	0.373218405686102\\
-0.851370415740462	0.3\\
-0.9	0.250459311110424\\
-0.993042891731576	0.2\\
-0.3	39\\
-0.293983246129865	-2\\
-0.3	-2.00349935834037\\
-0.4	-2.04127617510327\\
-0.5	-2.05782258933294\\
-0.6	-2.05395285068039\\
-0.7	-2.02824019863089\\
-0.756035769922788	-2\\
-0.8	-1.97230725706178\\
-0.871948992322501	-1.9\\
-0.9	-1.85781874000034\\
-0.927637386512861	-1.8\\
-0.952198279878012	-1.7\\
-0.956241664960575	-1.6\\
-0.943406867599171	-1.5\\
-0.916488594279111	-1.4\\
-0.9	-1.35768531557369\\
-0.871128968467651	-1.3\\
-0.806553672853479	-1.2\\
-0.8	-1.19128991615736\\
-0.7	-1.1104017648182\\
-0.665557259801489	-1.1\\
-0.6	-1.08277102601943\\
-0.5	-1.09506925831427\\
-0.488870578181193	-1.1\\
-0.4	-1.14011990048621\\
-0.314311090345958	-1.2\\
-0.3	-1.21123270085901\\
-0.213379301362769	-1.3\\
-0.2	-1.31703501683532\\
-0.144694101251948	-1.4\\
-0.0999999999999996	-1.49428885927246\\
-0.0975230904500991	-1.5\\
-0.0759237491176777	-1.6\\
-0.0771088877699307	-1.7\\
-0.0999999999999996	-1.78604085169407\\
-0.104617965091088	-1.8\\
-0.169619157290788	-1.9\\
-0.2	-1.93116038173212\\
-0.293983246129865	-2\\
-0.3	53\\
-0.790039420222189	0\\
-0.8	-0.00981641170209084\\
-0.9	-0.0647627707461437\\
-1	-0.0801957416543369\\
-1.1	-0.0681117241829888\\
-1.2	-0.0365234712496697\\
-1.27895508964311	0\\
-1.3	0.00957704156795994\\
-1.4	0.0686423112694663\\
-1.4455067859955	0.1\\
-1.5	0.140674550425339\\
-1.5707925509813	0.2\\
-1.6	0.22862136224611\\
-1.66710457729089	0.3\\
-1.7	0.344526975397657\\
-1.73913637370458	0.4\\
-1.78897501971667	0.5\\
-1.8	0.537895122519826\\
-1.81802559898388	0.6\\
-1.82717333151563	0.7\\
-1.81688738781151	0.8\\
-1.8	0.857367227711793\\
-1.78627923811407	0.9\\
-1.73303222108605	1\\
-1.7	1.04447324019578\\
-1.65110832694481	1.1\\
-1.6	1.14644379901651\\
-1.52424248938976	1.2\\
-1.5	1.2149110838449\\
-1.4	1.25991789702733\\
-1.3	1.28696153591894\\
-1.2	1.29733078237901\\
-1.1	1.29135797195743\\
-1	1.26853683822195\\
-0.9	1.22758627993591\\
-0.854251298953189	1.2\\
-0.8	1.16421550615909\\
-0.728062259473908	1.1\\
-0.7	1.0704335618638\\
-0.647144348700618	1\\
-0.6	0.917714165441136\\
-0.59153639777351	0.9\\
-0.558439196749166	0.8\\
-0.540769362339296	0.7\\
-0.5371384786493	0.6\\
-0.546078671773692	0.5\\
-0.566204797889548	0.4\\
-0.596301717296579	0.3\\
-0.6	0.290259295333154\\
-0.641906052919133	0.2\\
-0.699133145475411	0.1\\
-0.7	0.098627583493299\\
-0.790039420222189	0\\
-0.2	63\\
-0.212365537947582	-2.3\\
-0.3	-2.33065918794943\\
-0.4	-2.35259757164922\\
-0.5	-2.3625007185488\\
-0.6	-2.36071868578582\\
-0.7	-2.34665498445734\\
-0.8	-2.31855906641732\\
-0.842650209454301	-2.3\\
-0.9	-2.27318213571328\\
-1	-2.20378942541653\\
-1.0041448785534	-2.2\\
-1.09276945114395	-2.1\\
-1.1	-2.08891945110072\\
-1.14595999865437	-2\\
-1.17690384625025	-1.9\\
-1.19085498674497	-1.8\\
-1.1912560391244	-1.7\\
-1.18078582854852	-1.6\\
-1.16162365022861	-1.5\\
-1.13564761139166	-1.4\\
-1.10459244867406	-1.3\\
-1.1	-1.28662927076472\\
-1.06646253938913	-1.2\\
-1.02552134452042	-1.1\\
-1	-1.03687834373861\\
-0.980807093742776	-1\\
-0.931696724305082	-0.9\\
-0.9	-0.826991009598419\\
-0.88014994727145	-0.8\\
-0.819316667136219	-0.7\\
-0.8	-0.650978972666514\\
-0.7	-0.628918001151237\\
-0.6	-0.688402469409924\\
-0.588929277619977	-0.7\\
-0.5	-0.756742587069813\\
-0.450146646775093	-0.8\\
-0.4	-0.83394029107127\\
-0.322533389209593	-0.9\\
-0.3	-0.917107663898872\\
-0.208144124331795	-1\\
-0.2	-1.00717715643713\\
-0.106852895016133	-1.1\\
-0.0999999999999996	-1.10719037917358\\
-0.0181367184259383	-1.2\\
0	-1.22316955210477\\
0.0581009154821018	-1.3\\
0.0999999999999996	-1.366695350204\\
0.121064662215295	-1.4\\
0.168937120873013	-1.5\\
0.2	-1.59648334533652\\
0.201194852666301	-1.6\\
0.216701945711365	-1.7\\
0.214083097782383	-1.8\\
0.2	-1.86504528495885\\
0.191773932786	-1.9\\
0.147310158514781	-2\\
0.0999999999999996	-2.06968136191248\\
0.0759054696198708	-2.1\\
0	-2.17406190496213\\
-0.0332032127361372	-2.2\\
-0.0999999999999996	-2.24510967241223\\
-0.2	-2.29547790932933\\
-0.212365537947582	-2.3\\
-0.2	75\\
-0.725452902757602	-0.4\\
-0.8	-0.483763106584956\\
-0.9	-0.453738066771558\\
-0.977654896838484	-0.4\\
-1	-0.391615965028255\\
-1.1	-0.331321724518668\\
-1.14242613974189	-0.3\\
-1.2	-0.267416889228658\\
-1.3	-0.201683839059989\\
-1.30226490952232	-0.2\\
-1.4	-0.135138557820933\\
-1.44919778864815	-0.1\\
-1.5	-0.0648357267176757\\
-1.58788223925232	0\\
-1.6	0.00930768867182243\\
-1.7	0.0901267227850691\\
-1.71199309814075	0.1\\
-1.8	0.181791298100383\\
-1.81953692964245	0.2\\
-1.9	0.29035798947269\\
-1.90874463717247	0.3\\
-1.97935788806517	0.4\\
-2	0.44086468045152\\
-2.03140829897128	0.5\\
-2.06510543145071	0.6\\
-2.08132428228136	0.7\\
-2.08137926779711	0.8\\
-2.06569900954519	0.9\\
-2.03380492147862	1\\
-2	1.07007073110946\\
-1.98479320626674	1.1\\
-1.91733166630453	1.2\\
-1.9	1.2209784415949\\
-1.82630704924524	1.3\\
-1.8	1.32427371097771\\
-1.70171735421749	1.4\\
-1.7	1.40120879404177\\
-1.6	1.45974421790593\\
-1.50429820343182	1.5\\
-1.5	1.50176425624123\\
-1.4	1.53087611428902\\
-1.3	1.54616791434429\\
-1.2	1.54820935643968\\
-1.1	1.53687947166467\\
-1	1.51143841091268\\
-0.971876179709424	1.5\\
-0.9	1.47205006216963\\
-0.8	1.4167693786974\\
-0.776788861352685	1.4\\
-0.7	1.34351460338209\\
-0.652877162928971	1.3\\
-0.6	1.24734479764834\\
-0.560443455201133	1.2\\
-0.5	1.11626400437126\\
-0.489851429405353	1.1\\
-0.438898596063364	1\\
-0.402303227257542	0.9\\
-0.4	0.890303728178876\\
-0.380698779386865	0.8\\
-0.371033148580965	0.7\\
-0.371750945397413	0.6\\
-0.381346302274005	0.5\\
-0.398320067944506	0.4\\
-0.4	0.39262455370606\\
-0.423005905839859	0.3\\
-0.45299299625474	0.2\\
-0.486703121828282	0.1\\
-0.5	0.0616580818495225\\
-0.526074394355027	0\\
-0.568876832646543	-0.1\\
-0.6	-0.176041337168007\\
-0.614130905836458	-0.2\\
-0.666405784872846	-0.3\\
-0.7	-0.376475861415593\\
-0.725452902757602	-0.4\\
-0.2	43\\
2.54416707273514	1\\
2.5	0.971503709788058\\
2.4	0.93598493179465\\
2.3	0.927894245712046\\
2.2	0.941816862771926\\
2.1	0.973971734549051\\
2.04735947539281	1\\
2	1.02568125522959\\
1.9	1.09805700448741\\
1.89786082697216	1.1\\
1.80679953638158	1.2\\
1.8	1.20998550856449\\
1.74978184495561	1.3\\
1.71545978353699	1.4\\
1.70343617864813	1.5\\
1.71341524950258	1.6\\
1.74421074568938	1.7\\
1.79420318089378	1.8\\
1.8	1.80897137244159\\
1.88235784437653	1.9\\
1.9	1.91660394891851\\
2	1.97597545981692\\
2.08508325463174	2\\
2.1	2.00402436681707\\
2.2	2.00555094029448\\
2.22591454428283	2\\
2.3	1.9828107198003\\
2.4	1.93534761189208\\
2.44992613707913	1.9\\
2.5	1.85808948556499\\
2.55396691100923	1.8\\
2.6	1.73671127489107\\
2.62304613095433	1.7\\
2.66907707072122	1.6\\
2.69803218744717	1.5\\
2.7	1.4846881642154\\
2.71054440618346	1.4\\
2.70693673340583	1.3\\
2.7	1.26675930132525\\
2.68450011283997	1.2\\
2.63710629589892	1.1\\
2.6	1.05198316357996\\
2.54416707273514	1\\
-0.1	58\\
3	0.874309143709164\\
2.94914897442179	0.8\\
2.9	0.747594772834515\\
2.84254589740622	0.7\\
2.8	0.672276292416392\\
2.7	0.629266138776406\\
2.6	0.609353954733345\\
2.5	0.6076690412896\\
2.4	0.620816516965405\\
2.3	0.646437344605633\\
2.2	0.682904925809447\\
2.16480134519031	0.7\\
2.1	0.729577244323057\\
2	0.785406773014946\\
1.97812212104459	0.8\\
1.9	0.852292899142085\\
1.83800941725798	0.9\\
1.8	0.931135856739726\\
1.72607953056172	1\\
1.7	1.02767910336231\\
1.63906857705272	1.1\\
1.6	1.15742422059151\\
1.57351420372027	1.2\\
1.52856821734048	1.3\\
1.50089018733573	1.4\\
1.5	1.40785560471123\\
1.49023988537485	1.5\\
1.49333686665579	1.6\\
1.5	1.64509238441356\\
1.50903359516318	1.7\\
1.53651726450155	1.8\\
1.57366415061172	1.9\\
1.6	1.95921697015397\\
1.62218390684458	2\\
1.68349749931957	2.1\\
1.7	2.12528294810077\\
1.76474424511916	2.2\\
1.8	2.23949487087617\\
1.87768221992238	2.3\\
1.9	2.31783437869244\\
2	2.36945269738178\\
2.1	2.39725806289694\\
2.13014461325817	2.4\\
2.2	2.40721126877784\\
2.2932580889062	2.4\\
2.3	2.39951567684385\\
2.4	2.37718449226143\\
2.5	2.33869190439797\\
2.57038609432613	2.3\\
2.6	2.2832289523682\\
2.7	2.2102773265357\\
2.71189557820626	2.2\\
2.8	2.11586579596001\\
2.8146134090429	2.1\\
2.89459244089022	2\\
2.9	1.99202722129494\\
2.95943053333738	1.9\\
3	1.81966936437009\\
-0.1	161\\
-0.392726181082359	-2.7\\
-0.4	-2.701129345731\\
-0.5	-2.70853431794294\\
-0.6	-2.70757331707392\\
-0.679092525751708	-2.7\\
-0.7	-2.69823025200976\\
-0.8	-2.68159159388763\\
-0.9	-2.65501686009363\\
-1	-2.61577569384348\\
-1.03039235570425	-2.6\\
-1.1	-2.56329505365882\\
-1.18706394448694	-2.5\\
-1.2	-2.48960093287481\\
-1.28613094932539	-2.4\\
-1.3	-2.38196992122544\\
-1.35161205767799	-2.3\\
-1.39318592960051	-2.2\\
-1.4	-2.1740629350445\\
-1.41713512938596	-2.1\\
-1.42692464081728	-2\\
-1.42540207869504	-1.9\\
-1.41483972202366	-1.8\\
-1.4	-1.71784528842484\\
-1.39710991374913	-1.7\\
-1.37417367818036	-1.6\\
-1.34699868771539	-1.5\\
-1.31691637302517	-1.4\\
-1.3	-1.34690538873181\\
-1.28506892517687	-1.3\\
-1.25289054051122	-1.2\\
-1.22254207516038	-1.1\\
-1.2	-1.01371596412557\\
-1.19600331583615	-1\\
-1.17487910635904	-0.9\\
-1.16489367931753	-0.8\\
-1.17150303054585	-0.7\\
-1.2	-0.606698036683766\\
-1.20197342183764	-0.6\\
-1.25691391618456	-0.5\\
-1.3	-0.449885085968192\\
-1.34185155559331	-0.4\\
-1.4	-0.346701810300885\\
-1.45034362242865	-0.3\\
-1.5	-0.260278592604699\\
-1.5748855054253	-0.2\\
-1.6	-0.181020319457552\\
-1.7	-0.105174532310996\\
-1.70698048030131	-0.1\\
-1.8	-0.0292430929201501\\
-1.83948011984941	0\\
-1.9	0.0489204115692177\\
-1.96531775023273	0.1\\
-2	0.131406118698321\\
-2.0792671439901	0.2\\
-2.1	0.222049162748658\\
-2.17812479490525	0.3\\
-2.2	0.328547469540691\\
-2.25976908943258	0.4\\
-2.3	0.467349956808298\\
-2.32190271082801	0.5\\
-2.3668883854918	0.6\\
-2.3932383357535	0.7\\
-2.4	0.761548404328865\\
-2.40505184597553	0.8\\
-2.40228226423892	0.9\\
-2.4	0.913696136433558\\
-2.38678789496585	1\\
-2.35844684527017	1.1\\
-2.31561544276932	1.2\\
-2.3	1.22854117670838\\
-2.26077817636631	1.3\\
-2.2	1.38667501066317\\
-2.19020345100283	1.4\\
-2.10364431248958	1.5\\
-2.1	1.50376829865007\\
-2	1.59687076313429\\
-1.99614712428649	1.6\\
-1.9	1.67341802633216\\
-1.85741730569147	1.7\\
-1.8	1.73520709824471\\
-1.7	1.7835199263269\\
-1.65482963565365	1.8\\
-1.6	1.82080609128975\\
-1.5	1.8470164688032\\
-1.4	1.86099687528402\\
-1.3	1.86318020468044\\
-1.2	1.85327995602757\\
-1.1	1.83030594054407\\
-1.01980649281519	1.8\\
-1	1.79352859696121\\
-0.9	1.74600121012694\\
-0.828621232378266	1.7\\
-0.8	1.68323644718971\\
-0.7	1.60760832559023\\
-0.691802682063836	1.6\\
-0.6	1.51844625665617\\
-0.582572295178548	1.5\\
-0.5	1.41225956833905\\
-0.489981399804626	1.4\\
-0.412358349327291	1.3\\
-0.4	1.28233920998731\\
-0.348459966739976	1.2\\
-0.3	1.10582546164272\\
-0.297239112927024	1.1\\
-0.259792743516047	1\\
-0.23463485231601	0.9\\
-0.220890168091489	0.8\\
-0.217250057270806	0.7\\
-0.22218934745899	0.6\\
-0.234117203701238	0.5\\
-0.251454040246613	0.4\\
-0.272642882945504	0.3\\
-0.296109270730447	0.2\\
-0.3	0.183514222121471\\
-0.321510471985883	0.1\\
-0.346172310294243	0\\
-0.367592649798508	-0.1\\
-0.382960640447469	-0.2\\
-0.388771164006203	-0.3\\
-0.380726487057252	-0.4\\
-0.353779286564914	-0.5\\
-0.302439435633071	-0.6\\
-0.3	-0.603334580136014\\
-0.234784502344237	-0.7\\
-0.2	-0.740187413940002\\
-0.151239904699242	-0.8\\
-0.0999999999999996	-0.854999860819264\\
-0.0596072314909505	-0.9\\
0	-0.963016996487469\\
0.0345547039879943	-1\\
0.0999999999999996	-1.07089843379857\\
0.127263183794387	-1.1\\
0.2	-1.1830786087936\\
0.215476996224113	-1.2\\
0.29666826096798	-1.3\\
0.3	-1.30488818093311\\
0.369906038522704	-1.4\\
0.4	-1.45108518015267\\
0.432159995352153	-1.5\\
0.481500353208209	-1.6\\
0.5	-1.65544075569295\\
0.517335982102998	-1.7\\
0.538559431478699	-1.8\\
0.543317346637111	-1.9\\
0.53113897834692	-2\\
0.500613028136905	-2.1\\
0.5	-2.10135347956118\\
0.453458734537804	-2.2\\
0.4	-2.27852349655373\\
0.384001859224072	-2.3\\
0.3	-2.39069513625943\\
0.290081805362461	-2.4\\
0.2	-2.47459332325889\\
0.162755779582483	-2.5\\
0.0999999999999996	-2.54051176618196\\
0	-2.59167191224578\\
-0.0202441102478049	-2.6\\
-0.0999999999999996	-2.63358870099458\\
-0.2	-2.66508922507322\\
-0.3	-2.68712492885131\\
-0.392726181082359	-2.7\\
0	63\\
-2.20361805108572	-3.2\\
-2.2	-3.19384824975229\\
-2.17130125796679	-3.1\\
-2.13566816348384	-3\\
-2.1	-2.91585889784621\\
-2.0962173038633	-2.9\\
-2.061917591178	-2.8\\
-2.02314977385873	-2.7\\
-2	-2.65285204871158\\
-1.98405956641362	-2.6\\
-1.94626480640784	-2.5\\
-1.90402878384247	-2.4\\
-1.9	-2.39271104674377\\
-1.86568253148036	-2.3\\
-1.8243373677052	-2.2\\
-1.8	-2.14993085134864\\
-1.78252550848993	-2.1\\
-1.74162842480805	-2\\
-1.7	-1.90608505592427\\
-1.6979154600814	-1.9\\
-1.65741441980774	-1.8\\
-1.61485673485096	-1.7\\
-1.6	-1.66777278932499\\
-1.57373090461888	-1.6\\
-1.53348021073977	-1.5\\
-1.5	-1.4167094435333\\
-1.4939355164544	-1.4\\
-1.4579258513676	-1.3\\
-1.4247994387861	-1.2\\
-1.4	-1.11246310929735\\
-1.39650726892726	-1.1\\
-1.3755257340722	-1\\
-1.36424020341727	-0.9\\
-1.36616394728713	-0.8\\
-1.38546945839828	-0.7\\
-1.4	-0.663357257382221\\
-1.42607636972628	-0.6\\
-1.49089718324801	-0.5\\
-1.5	-0.489520568260032\\
-1.58107554035338	-0.4\\
-1.6	-0.382659063066075\\
-1.69472878173277	-0.3\\
-1.7	-0.29581677724254\\
-1.8	-0.220510639957976\\
-1.82942923957039	-0.2\\
-1.9	-0.150950499722655\\
-1.97891035421655	-0.1\\
-2	-0.0855578132996997\\
-2.1	-0.0230876825469388\\
-2.14106389003195	0\\
-2.2	0.0377402978954404\\
-2.3	0.0958824806172895\\
-2.30835121777713	0.1\\
-2.4	0.154996294048017\\
-2.48352840269051	0.2\\
-2.5	0.211377878972725\\
-2.6	0.268923249210266\\
-2.66269165133422	0.3\\
-2.7	0.325237300057044\\
-2.8	0.381140939093987\\
-2.84164255524613	0.4\\
-2.9	0.438324887122086\\
-3	0.492704622347961\\
0	55\\
3	-0.0778725327654489\\
2.9	-0.0148024030177974\\
2.88218039286623	0\\
2.8	0.0464133586309133\\
2.72208807265426	0.1\\
2.7	0.110817672739748\\
2.6	0.173010968315814\\
2.56431677666437	0.2\\
2.5	0.236565159951996\\
2.40482351941894	0.3\\
2.4	0.302530215216676\\
2.3	0.366778694457912\\
2.25501734703504	0.4\\
2.2	0.43382474889667\\
2.10523088938902	0.5\\
2.1	0.50318748417573\\
2	0.573645314940044\\
1.96623072443055	0.6\\
1.9	0.6478891833091\\
1.83429351074968	0.7\\
1.8	0.726524426402436\\
1.71273380499747	0.8\\
1.7	0.811047423624571\\
1.60490630062777	0.9\\
1.6	0.905027165204581\\
1.51291016582161	1\\
1.5	1.01656212195689\\
1.43781933227212	1.1\\
1.4	1.16527900472678\\
1.38033432257594	1.2\\
1.34051486277545	1.3\\
1.316756555615	1.4\\
1.30683467314891	1.5\\
1.30844103562537	1.6\\
1.31935866363379	1.7\\
1.33756376969044	1.8\\
1.36127424255347	1.9\\
1.38896262009634	2\\
1.4	2.03797502538063\\
1.42036218475957	2.1\\
1.45410797620936	2.2\\
1.4886996313536	2.3\\
1.5	2.3351923666047\\
1.52567052877885	2.4\\
1.56316731806154	2.5\\
1.59958951984116	2.6\\
1.6	2.60138056366018\\
1.63944634446715	2.7\\
1.67729198697021	2.8\\
1.7	2.86816664367544\\
1.71533636394567	2.9\\
1.7547261565909	3\\
1.79115694124615	3.1\\
1.8	3.13141661154319\\
1.83046030893352	3.2\\
0	109\\
-1.64565984234554	3.2\\
-1.6	3.14902537997995\\
-1.57654739103031	3.1\\
-1.50403013807493	3\\
-1.5	2.9965004239448\\
-1.43727100840405	2.9\\
-1.4	2.86136542096052\\
-1.36483250244201	2.8\\
-1.3	2.7221676272313\\
-1.28849220178917	2.7\\
-1.21297201721304	2.6\\
-1.2	2.5879760732521\\
-1.13770961335025	2.5\\
-1.1	2.46125209515216\\
-1.05908623732566	2.4\\
-1	2.33411529160396\\
-0.978081716685479	2.3\\
-0.9	2.20724042670742\\
-0.895449771306727	2.2\\
-0.814254410699121	2.1\\
-0.8	2.08586110216393\\
-0.73243284216561	2\\
-0.7	1.96562775030415\\
-0.649783875122564	1.9\\
-0.6	1.84386971210838\\
-0.567271606393254	1.8\\
-0.5	1.71926150851263\\
-0.48602008312775	1.7\\
-0.407862301873663	1.6\\
-0.4	1.59029829866584\\
-0.334160568732281	1.5\\
-0.3	1.45192145455656\\
-0.265796148377005	1.4\\
-0.205429939200432	1.3\\
-0.2	1.28981285804086\\
-0.154524728896468	1.2\\
-0.115039454748006	1.1\\
-0.0999999999999996	1.047000254227\\
-0.0870471377339646	1\\
-0.0704336269847516	0.9\\
-0.0643683676276016	0.8\\
-0.0671872340457536	0.7\\
-0.0771036969326048	0.6\\
-0.0923420289370845	0.5\\
-0.0999999999999996	0.459800152651436\\
-0.111165636938949	0.4\\
-0.132050578292956	0.3\\
-0.153414565413615	0.2\\
-0.173536648271939	0.1\\
-0.190515718103016	0\\
-0.2	-0.0818399105795949\\
-0.202298820880453	-0.1\\
-0.20643572401655	-0.2\\
-0.2	-0.2914004235529\\
-0.199396226054852	-0.3\\
-0.179208388552872	-0.4\\
-0.142688718929025	-0.5\\
-0.0999999999999996	-0.578636685718052\\
-0.0883668157667252	-0.6\\
-0.0191056606655776	-0.7\\
0	-0.723365767468619\\
0.0633433450129204	-0.8\\
0.0999999999999996	-0.840492046703649\\
0.155622714828962	-0.9\\
0.2	-0.946258525177659\\
0.254485344768092	-1\\
0.3	-1.04618948411647\\
0.357468584703711	-1.1\\
0.4	-1.14300096372381\\
0.462770889507953	-1.2\\
0.5	-1.23816025555854\\
0.569112677330736	-1.3\\
0.6	-1.33252978209481\\
0.675608238292061	-1.4\\
0.7	-1.42666595357563\\
0.781660818164736	-1.5\\
0.8	-1.52097065261\\
0.886882043061607	-1.6\\
0.9	-1.61577292952388\\
0.991031740751515	-1.7\\
1	-1.71137394587521\\
1.09397308783069	-1.8\\
1.1	-1.80806962227457\\
1.19563871402087	-1.9\\
1.2	-1.90615693707355\\
1.29600454513681	-2\\
1.3	-2.00592588386822\\
1.3950691345805	-2.1\\
1.4	-2.10763789016079\\
1.49283684725501	-2.2\\
1.5	-2.21149252939908\\
1.58930354883509	-2.3\\
1.6	-2.31758732983329\\
1.68444349022465	-2.4\\
1.7	-2.42587948346281\\
1.77819589850748	-2.5\\
1.8	-2.53616131018222\\
1.8704493527118	-2.6\\
1.9	-2.64806083741124\\
1.96102118160918	-2.7\\
2	-2.76107317503631\\
2.04962752035256	-2.8\\
2.1	-2.87461867409445\\
2.1358365486865	-2.9\\
2.2	-2.98811439675096\\
2.21899112916745	-3\\
2.29904267967173	-3.1\\
2.3	-3.10247745054336\\
2.3866147224323	-3.2\\
0.1	53\\
-3	-1.40178458877748\\
-2.93941926011479	-1.5\\
-2.9	-1.55189490601546\\
-2.85988362802792	-1.6\\
-2.8	-1.66171813376247\\
-2.75611079731629	-1.7\\
-2.7	-1.74440247673619\\
-2.60832349841002	-1.8\\
-2.6	-1.80483128970232\\
-2.5	-1.84803899915321\\
-2.4	-1.87235371764905\\
-2.3	-1.87871573527721\\
-2.2	-1.86635477625346\\
-2.1	-1.83239642632805\\
-2.04647519758315	-1.8\\
-2	-1.77396938924115\\
-1.91486475868895	-1.7\\
-1.9	-1.68753114361128\\
-1.82579441389939	-1.6\\
-1.8	-1.56937555221183\\
-1.75507397144825	-1.5\\
-1.7	-1.41084594626654\\
-1.69448804206571	-1.4\\
-1.64682302445509	-1.3\\
-1.60553046470664	-1.2\\
-1.6	-1.1831071497333\\
-1.57554926787768	-1.1\\
-1.5561056653941	-1\\
-1.54950748656186	-0.9\\
-1.5589798087319	-0.8\\
-1.58780765668103	-0.7\\
-1.6	-0.675390554543997\\
-1.64154795686328	-0.6\\
-1.7	-0.524955317163106\\
-1.72207671944295	-0.5\\
-1.8	-0.42817485053711\\
-1.83512611555176	-0.4\\
-1.9	-0.353280174208278\\
-1.9851758518185	-0.3\\
-2	-0.291023222700163\\
-2.1	-0.240291596306273\\
-2.19503398057253	-0.2\\
-2.2	-0.197791284186655\\
-2.3	-0.164979465003338\\
-2.4	-0.14193665874688\\
-2.5	-0.130196148255327\\
-2.6	-0.131911221007972\\
-2.7	-0.150103459173443\\
-2.8	-0.189024344546119\\
-2.81946119507519	-0.2\\
-2.9	-0.256675868403934\\
-2.94628598395668	-0.3\\
-3	-0.36714598767033\\
0.1	151\\
1.13983408879736	-1.3\\
1.2	-1.30864865736468\\
1.3	-1.31077325398298\\
1.4	-1.30116739391639\\
1.40584809187254	-1.3\\
1.5	-1.28237532419216\\
1.6	-1.25262207534443\\
1.7	-1.21032443786123\\
1.71997897930136	-1.2\\
1.8	-1.15853074916925\\
1.88901500156864	-1.1\\
1.9	-1.09243661388779\\
2	-1.01252980555824\\
2.01384772115534	-1\\
2.1	-0.912640229081086\\
2.11136226849974	-0.9\\
2.18855483395771	-0.8\\
2.2	-0.781714493124963\\
2.24968210143168	-0.7\\
2.29360541049061	-0.6\\
2.3	-0.579185614490936\\
2.32524508226368	-0.5\\
2.34242836289648	-0.4\\
2.34508903261672	-0.3\\
2.33291666310935	-0.2\\
2.30434129442909	-0.1\\
2.3	-0.0908703398925549\\
2.26305146845161	0\\
2.20280982964344	0.1\\
2.2	0.103443134677944\\
2.1302678805404	0.2\\
2.1	0.232280697770359\\
2.04259346923653	0.3\\
2	0.340834776755074\\
1.94317388551898	0.4\\
1.9	0.438464927070657\\
1.83541347028626	0.5\\
1.8	0.530379251867075\\
1.72311306744169	0.6\\
1.7	0.619842055085291\\
1.61050957611886	0.7\\
1.6	0.70941886739216\\
1.50201339171312	0.8\\
1.5	0.801973910852748\\
1.40180699134878	0.9\\
1.4	0.902043158173626\\
1.31355777702218	1\\
1.3	1.01883053161291\\
1.24048375079521	1.1\\
1.2	1.17506968142919\\
1.18593894988708	1.2\\
1.14817081970462	1.3\\
1.12744447038652	1.4\\
1.11998831490654	1.5\\
1.12249196547044	1.6\\
1.13208658598054	1.7\\
1.14628235029231	1.8\\
1.16288615158534	1.9\\
1.17991052851011	2\\
1.19547650648616	2.1\\
1.2	2.13894020578367\\
1.20749412311268	2.2\\
1.21421678198366	2.3\\
1.2135428157731	2.4\\
1.20278894449388	2.5\\
1.2	2.51312504093999\\
1.17800478759302	2.6\\
1.13473586050441	2.7\\
1.1	2.75569108759663\\
1.06446726608941	2.8\\
1	2.86418281871525\\
0.950555433674011	2.9\\
0.9	2.93321025221187\\
0.8	2.97951228710491\\
0.732548666787248	3\\
0.7	3.01012694640688\\
0.6	3.02919152844431\\
0.5	3.03706194985705\\
0.4	3.03487053979708\\
0.3	3.02285571604706\\
0.2	3.00048176488547\\
0.198492312320191	3\\
0.0999999999999996	2.96960980890238\\
0	2.92601162231424\\
-0.0462271307287413	2.9\\
-0.0999999999999996	2.86826316425291\\
-0.19014878700746	2.8\\
-0.2	2.79157566482056\\
-0.288672238681737	2.7\\
-0.3	2.68536989851579\\
-0.357758160068983	2.6\\
-0.4	2.50710911172817\\
-0.402968348092729	2.5\\
-0.429385696849543	2.4\\
-0.438005814672755	2.3\\
-0.430930326756238	2.2\\
-0.409725863796963	2.1\\
-0.4	2.07273685277833\\
-0.376967209562334	2\\
-0.334072131701188	1.9\\
-0.3	1.83528164204637\\
-0.28275887330274	1.8\\
-0.225950432053288	1.7\\
-0.2	1.65794873185777\\
-0.165563593954577	1.6\\
-0.105350553478692	1.5\\
-0.0999999999999996	1.49093100882661\\
-0.0468868780461414	1.4\\
0	1.30782595286864\\
0.00405584444405975	1.3\\
0.0473413120277635	1.2\\
0.0777121420971699	1.1\\
0.095488304100964	1\\
0.0999999999999996	0.930930309177375\\
0.102064134718947	0.9\\
0.0999999999999996	0.842278704046863\\
0.0986912351826779	0.8\\
0.0878615552617251	0.7\\
0.0713983211807815	0.6\\
0.0511971841101419	0.5\\
0.0290333413045542	0.4\\
0.00659563758174735	0.3\\
0	0.268674064324852\\
-0.0141874696251054	0.2\\
-0.0319167808553999	0.1\\
-0.0448599374968444	0\\
-0.0509269364416876	-0.1\\
-0.0478032346695197	-0.2\\
-0.0330178143331756	-0.3\\
-0.00412923110312725	-0.4\\
0	-0.409630715936514\\
0.0402078318395572	-0.5\\
0.0999999999999996	-0.598591509974629\\
0.10090504262052	-0.6\\
0.177817592928502	-0.7\\
0.2	-0.725374165821862\\
0.270115521176256	-0.8\\
0.3	-0.829933935719431\\
0.376944727437407	-0.9\\
0.4	-0.920890338054849\\
0.498445761710143	-1\\
0.5	-1.00130475687879\\
0.6	-1.07396171086954\\
0.643045396247633	-1.1\\
0.7	-1.13787885917238\\
0.8	-1.19158207773129\\
0.820470153196133	-1.2\\
0.9	-1.23771872227684\\
1	-1.27233292128309\\
1.1	-1.29530523488569\\
1.13983408879736	-1.3\\
0.2	41\\
-2.467215650175	-1.4\\
-2.4	-1.43759067659454\\
-2.3	-1.4671117478951\\
-2.2	-1.46980542826982\\
-2.1	-1.44500677531088\\
-2.01785312194441	-1.4\\
-2	-1.38947147471973\\
-1.90613892152423	-1.3\\
-1.9	-1.29315048916108\\
-1.84022037319831	-1.2\\
-1.8	-1.11735678476896\\
-1.79331240174734	-1.1\\
-1.77023941698515	-1\\
-1.76593213420848	-0.9\\
-1.7835362179218	-0.8\\
-1.8	-0.760280820147302\\
-1.83101678172359	-0.7\\
-1.9	-0.613346843567346\\
-1.91381075924775	-0.6\\
-2	-0.535001557836737\\
-2.06287395218503	-0.5\\
-2.1	-0.481917275922665\\
-2.2	-0.450587331737071\\
-2.3	-0.436770699944816\\
-2.4	-0.443335495250298\\
-2.5	-0.474522081001267\\
-2.54466957096103	-0.5\\
-2.6	-0.544457797080852\\
-2.64877125278457	-0.6\\
-2.7	-0.694008758972206\\
-2.7027591545976	-0.7\\
-2.72866541458968	-0.8\\
-2.73553570460243	-0.9\\
-2.72555387524636	-1\\
-2.7	-1.09660723875932\\
-2.6990294258737	-1.1\\
-2.6525961176114	-1.2\\
-2.6	-1.27867420216426\\
-2.58256621779899	-1.3\\
-2.5	-1.37828200037395\\
-2.467215650175	-1.4\\
0.2	125\\
0.921307071509807	-1\\
1	-1.01701158182748\\
1.1	-1.02653983914135\\
1.2	-1.02429746944378\\
1.3	-1.01014744237547\\
1.33978471799597	-1\\
1.4	-0.984561575570199\\
1.5	-0.946834371120903\\
1.59030311973574	-0.9\\
1.6	-0.894646612964775\\
1.7	-0.827057066493108\\
1.73321580855293	-0.8\\
1.8	-0.737767866940238\\
1.83533899215374	-0.7\\
1.9	-0.614890232689074\\
1.91036546281098	-0.6\\
1.96484316837705	-0.5\\
2	-0.400921076737985\\
2.00031959378206	-0.4\\
2.02030272542284	-0.3\\
2.02378355540193	-0.2\\
2.01100647131711	-0.1\\
2	-0.0632700378864157\\
1.98281124982058	0\\
1.93907982533943	0.1\\
1.9	0.163699197580413\\
1.87908009781965	0.2\\
1.80447833630023	0.3\\
1.8	0.304942731110903\\
1.71717432718692	0.4\\
1.7	0.417132096861219\\
1.61858968268537	0.5\\
1.6	0.51737584135138\\
1.51190358144445	0.6\\
1.5	0.610835176963322\\
1.40084238486479	0.7\\
1.4	0.700779082430408\\
1.3	0.790382802792554\\
1.28855666145708	0.8\\
1.2	0.883129188938889\\
1.18041051371918	0.9\\
1.1	0.983826262541873\\
1.08258740739653	1\\
1.00236664699455	1.1\\
1	1.10429681699688\\
0.939428409901939	1.2\\
0.905766381575703	1.3\\
0.9	1.34135807767433\\
0.890103498981445	1.4\\
0.890374935608362	1.5\\
0.9	1.58966137090672\\
0.900851074609119	1.6\\
0.913813741703986	1.7\\
0.928080456257381	1.8\\
0.940806709407595	1.9\\
0.949268993019101	2\\
0.950648232607334	2.1\\
0.941810034408809	2.2\\
0.919050799831334	2.3\\
0.9	2.34899818264055\\
0.873437014903275	2.4\\
0.8	2.49428828768932\\
0.7933995150303	2.5\\
0.7	2.5647086625576\\
0.61091886547268	2.6\\
0.6	2.60386919546233\\
0.5	2.61992769236331\\
0.4	2.61780908978233\\
0.308547476760461	2.6\\
0.3	2.59821137709476\\
0.2	2.55824965588107\\
0.107209711810481	2.5\\
0.0999999999999996	2.49448523470414\\
0.00999252336624896	2.4\\
0	2.38509032061453\\
-0.0455036153984973	2.3\\
-0.074412889706341	2.2\\
-0.0818106459452839	2.1\\
-0.0709999498653482	2\\
-0.0451331710876134	1.9\\
-0.00756677571406228	1.8\\
0	1.78417070885179\\
0.0422738749146913	1.7\\
0.0966817865721632	1.6\\
0.0999999999999996	1.59413308072252\\
0.158161676995118	1.5\\
0.2	1.4240769581529\\
0.215220626560776	1.4\\
0.266513273275099	1.3\\
0.299658348179188	1.2\\
0.3	1.19807093268007\\
0.319763259493251	1.1\\
0.320853364588965	1\\
0.307907014789556	0.9\\
0.3	0.867276662355275\\
0.287454767995924	0.8\\
0.262356881773937	0.7\\
0.233743837445557	0.6\\
0.203711078463146	0.5\\
0.2	0.487831876735838\\
0.176117166210966	0.4\\
0.150500717774688	0.3\\
0.12818086367332	0.2\\
0.110846973301818	0.1\\
0.100317729877993	0\\
0.0999999999999996	-0.0185765922871412\\
0.0986173634697685	-0.1\\
0.0999999999999996	-0.116030692070531\\
0.107720500962808	-0.2\\
0.129873537313458	-0.3\\
0.166905555678935	-0.4\\
0.2	-0.463247112707671\\
0.221055701098086	-0.5\\
0.293618903704237	-0.6\\
0.3	-0.607519096919415\\
0.387739096272731	-0.7\\
0.4	-0.71175339466624\\
0.5	-0.795337080209901\\
0.506645876455551	-0.8\\
0.6	-0.863924582942308\\
0.665079616891039	-0.9\\
0.7	-0.919844260709159\\
0.8	-0.963924586935212\\
0.9	-0.995790559759905\\
0.921307071509807	-1\\
0.3	65\\
0.992129865039172	-0.8\\
1	-0.801140930218515\\
1.1	-0.802006429140245\\
1.11657639365077	-0.8\\
1.2	-0.789446715848194\\
1.3	-0.762876949623858\\
1.4	-0.721026301488623\\
1.43743766098095	-0.7\\
1.5	-0.660235027644644\\
1.57406793743042	-0.6\\
1.6	-0.574322923440448\\
1.66320874048845	-0.5\\
1.7	-0.441212328094659\\
1.72301110471276	-0.4\\
1.75955585944328	-0.3\\
1.77619843416722	-0.2\\
1.77437484128269	-0.1\\
1.75480684419625	0\\
1.71776222589972	0.1\\
1.7	0.132383447613634\\
1.66335971373211	0.2\\
1.6	0.289532939441153\\
1.59245798273515	0.3\\
1.50539691415421	0.4\\
1.5	0.405456441540683\\
1.40285814599461	0.5\\
1.4	0.502604636367769\\
1.3	0.587958873652353\\
1.28431441831123	0.6\\
1.2	0.665277553902348\\
1.14857691517017	0.7\\
1.1	0.735425553654386\\
1	0.796332610117574\\
0.991515873017854	0.8\\
0.9	0.847540678982481\\
0.8	0.878752334783272\\
0.7	0.878955697421754\\
0.6	0.829888760667447\\
0.573396036673292	0.8\\
0.5	0.717201480485405\\
0.490206634784299	0.7\\
0.43218515649893	0.6\\
0.4	0.540821512703327\\
0.382545229118269	0.5\\
0.342367611232751	0.4\\
0.30754932765282	0.3\\
0.3	0.273393985846962\\
0.281506292594194	0.2\\
0.263530763461669	0.1\\
0.254571470043272	0\\
0.256522312308286	-0.1\\
0.271376977516337	-0.2\\
0.3	-0.296541478830452\\
0.301158591026338	-0.3\\
0.351306063769407	-0.4\\
0.4	-0.471811041587056\\
0.422381506589668	-0.5\\
0.5	-0.580557032705826\\
0.522812975922458	-0.6\\
0.6	-0.658583698958432\\
0.669498486204925	-0.7\\
0.7	-0.717220095970077\\
0.8	-0.759142749294999\\
0.9	-0.78694625051763\\
0.992129865039172	-0.8\\
0.4	45\\
0.705926524896027	-0.5\\
0.8	-0.544958403893467\\
0.9	-0.573919964565954\\
1	-0.584792875388956\\
1.1	-0.577774930221805\\
1.2	-0.552346567259021\\
1.3	-0.507051683831235\\
1.31111443645076	-0.5\\
1.4	-0.429674950387802\\
1.42870357948205	-0.4\\
1.49691545052493	-0.3\\
1.5	-0.292367398001295\\
1.53159512437146	-0.2\\
1.54180971696818	-0.1\\
1.5298058827441	0\\
1.5	0.0908669254340281\\
1.49683473340295	0.1\\
1.4406180879367	0.2\\
1.4	0.254315095973711\\
1.36232659508997	0.3\\
1.3	0.362721187186107\\
1.25721763158659	0.4\\
1.2	0.444807883880696\\
1.11403835163028	0.5\\
1.1	0.508713438750432\\
1	0.55311137504316\\
0.9	0.577695610016149\\
0.8	0.575984619919979\\
0.7	0.53954194833661\\
0.64927147179472	0.5\\
0.6	0.455384388546853\\
0.559628881879822	0.4\\
0.500273629499752	0.3\\
0.5	0.29939738609724\\
0.464104902451225	0.2\\
0.440527324436747	0.1\\
0.430947825970863	0\\
0.437161157541018	-0.1\\
0.461123296214172	-0.2\\
0.5	-0.289593660542579\\
0.505543376040414	-0.3\\
0.581960924199731	-0.4\\
0.6	-0.418379418140601\\
0.7	-0.496733328455294\\
0.705926524896027	-0.5\\
0.5	25\\
0.898848867079939	-0.3\\
0.9	-0.300449252072593\\
1	-0.30938206540798\\
1.04515569314408	-0.3\\
1.1	-0.283567843358008\\
1.2	-0.207766236028396\\
1.20639505893815	-0.2\\
1.24685929115203	-0.1\\
1.24595896029766	0\\
1.20946395894469	0.1\\
1.2	0.113742891098164\\
1.12135615265069	0.2\\
1.1	0.216998952190141\\
1	0.265130061437848\\
0.9	0.276044529830501\\
0.8	0.239210127241471\\
0.75727214170555	0.2\\
0.7	0.113051703630108\\
0.694495162399311	0.1\\
0.679031304406474	0\\
0.692022241277805	-0.1\\
0.7	-0.118791095374334\\
0.750255526691335	-0.2\\
0.8	-0.247538809426755\\
0.898848867079939	-0.3\\
};
\end{axis}
\end{tikzpicture}%
\end{document}
% This file was created by matlab2tikz.
% Minimal pgfplots version: 1.3
%
%The latest updates can be retrieved from
%  http://www.mathworks.com/matlabcentral/fileexchange/22022-matlab2tikz
%where you can also make suggestions and rate matlab2tikz.
%
\documentclass[tikz]{standalone}
\usepackage{pgfplots}
\usepackage{grffile}
\pgfplotsset{compat=newest}
\usetikzlibrary{plotmarks}
\usepackage{amsmath}

\begin{document}
\definecolor{mycolor1}{rgb}{0.00000,0.44700,0.74100}%
\definecolor{mycolor2}{rgb}{0.85000,0.32500,0.09800}%
%
\begin{tikzpicture}

\begin{axis}[%
width=1.5in,
height=1.5in,
scale only axis,
xmin=-3,
xmax=3,
ymin=-3.2,
ymax=3.2,
title={sixth component}
]
\addplot [color=mycolor1,only marks,mark=o,mark options={solid},forget plot]
  table[row sep=crcr]{%
0.0259388858243524	1.99651291936462\\
0.0479365430139229	2.00136242712261\\
0.0808011428276144	2.00015957097738\\
0.0960018408277835	2.00397357604688\\
0.123924398192739	1.99393565592043\\
0.144716297235007	1.99609680237626\\
0.172432773795358	2.00782480916528\\
0.211096968463268	2.00289657674787\\
0.241323554557747	2.00054011148902\\
0.248450748340069	1.98323290473631\\
0.278037158020472	1.99775137204772\\
0.295123845636812	2.024317082564\\
0.278089869880903	1.98953669925669\\
0.349224330459094	2.01774494803247\\
0.389769425315202	1.96915513448386\\
0.443104169526806	2.00181328676435\\
0.445366816453369	2.00702160442149\\
0.447621894481189	1.9639008226548\\
0.459268204218018	1.96874477518371\\
0.525152271376063	1.93928073190811\\
0.479912634035923	1.9531725626356\\
0.541892480429807	1.98767075962513\\
0.596086730914084	1.95159718447859\\
0.642409661124227	1.88862872560156\\
0.59672705695044	2.06269182867749\\
0.577160609928358	1.93769298667604\\
0.61270638646484	1.9863001413635\\
0.738131551064678	1.99453476973758\\
0.731223036941714	1.9720505976831\\
0.728694499517447	2.00387465318476\\
0.792620470002546	2.01820212338234\\
0.76404643370489	1.92712620890764\\
0.783621400051233	1.97170029129066\\
0.844923525499164	2.0002383443526\\
0.933088804435648	1.86929978545337\\
0.826522508566799	1.96489496451136\\
0.857943301337058	1.93574991230876\\
0.862415931324452	1.91824418132958\\
0.85957751145223	1.94915043906307\\
1.10549276410824	1.89209624896601\\
0.992380983980335	1.9389778055659\\
1.05163144833557	1.9077966876626\\
0.971362952394099	1.94526987131734\\
1.10648177744783	1.98917932950755\\
1.12898410068108	1.88302810812235\\
1.14935856334692	1.80741423289017\\
1.2475966096538	1.80328803923413\\
1.02094574055669	1.89016584759305\\
1.18313200649952	1.80836196844854\\
1.11713521982497	1.97114350530034\\
1.21095637410431	1.6606506602979\\
1.09752975594461	1.89186114518647\\
1.1286092446374	1.8521928055118\\
1.17641657796315	1.85179775653125\\
1.2027986521108	1.79291729003239\\
1.40862693119303	1.68188184454732\\
1.29643649567147	1.6835431657974\\
1.46182858553516	1.76685142278392\\
1.44694035368565	1.75403257372017\\
1.49092139521946	1.61038316326389\\
1.3379059086772	1.66077993928176\\
1.46168112951844	1.88866965377674\\
1.41023026436323	1.73928409103319\\
1.4714192456311	1.78600061226619\\
1.35577541010529	1.79014999256957\\
1.4051660334094	1.67552928758717\\
1.61761103881626	1.92386710385478\\
1.24305770227106	1.88248740497196\\
1.49729193411573	1.81459528264337\\
1.63186681467879	1.63329561110855\\
1.53796361955124	1.77930541551158\\
1.46647340378635	1.89266145452761\\
1.59115626629009	1.92971274617182\\
1.55598375769603	1.68322562636936\\
1.60162857304111	1.74522783244834\\
1.55820638402395	1.58511515395533\\
1.6445086680066	1.69602995640505\\
1.43117976594691	1.39522816594688\\
1.66046656580406	1.5020485139169\\
1.89457571586182	1.8060245580165\\
1.7690851756095	1.803354010806\\
1.51808575625609	1.62902388762982\\
1.71974982125714	1.56603639813406\\
1.63851255324909	1.37209586707968\\
1.73596507613967	1.53438459796941\\
1.8856967379105	1.45170562832068\\
1.83646220685143	1.51078810063935\\
1.6807272288616	1.41341108157641\\
1.6794592820606	1.42428516993435\\
1.84697126074185	1.69896546443275\\
1.65822816966463	1.41270723759625\\
1.78371507953502	1.42947586399192\\
1.87899798012913	1.57760927136629\\
1.89142792776872	1.58934082453833\\
1.89301329213783	1.04572276609604\\
1.75219375143258	1.4487568304138\\
2.1047354401674	1.41876136406707\\
2.01123547254846	1.3110508611222\\
1.96781276802938	1.52540722041352\\
1.51166361010094	1.32992221483295\\
2.22158609914819	1.4805071317003\\
1.97788676892182	1.64185849242464\\
1.94851336550087	1.45074289017143\\
2.05150009067797	1.21139868557201\\
1.93322927547617	1.22986741742318\\
1.99640422827958	0.927539227571993\\
2.02784448795865	1.33792744894102\\
2.05296048996672	1.40894341042615\\
1.79681210188576	1.18409186751766\\
2.04186829813144	1.41141478615\\
2.06457591322178	1.27562881170451\\
1.9610792821858	1.32599888091024\\
2.02517402706913	1.29054836978655\\
1.90898477009428	1.23168003520599\\
1.95382687452845	1.07890142563229\\
1.98007488912549	0.865945916419211\\
1.91942744299385	1.23088568855929\\
1.90203680990463	1.46675740172822\\
1.79605423362615	0.99553318086307\\
1.93675059768613	1.26622062159566\\
1.80589580300649	1.38156657036724\\
1.78116659707435	1.22679608853667\\
1.4859041706582	1.42125093730957\\
1.80527660882125	1.44977387260663\\
2.02922661129011	1.0207490772509\\
1.81798786252392	0.647952952360472\\
2.32931027046338	1.2400230228602\\
1.62944341809323	0.943292490703794\\
1.98834525677686	1.34211488379452\\
1.91927966663161	1.25902905273399\\
2.0814706955723	1.07488996429983\\
1.99353342912541	0.988261144028966\\
2.10800463348708	0.766165616265449\\
1.8550225815185	1.13356072105016\\
1.89272673258568	1.36537195907968\\
1.8902074337978	0.757056328403824\\
1.53885687334523	1.07694569130808\\
1.76112331184547	0.998690472291168\\
2.09154477395659	0.694075837194668\\
1.90858542651557	0.757727713019561\\
1.9330049256032	0.976874558375833\\
1.93983302339883	0.816051509797898\\
2.24506862568672	0.672826581876916\\
1.82718437556353	1.10936506743491\\
1.96057264525256	0.893925986104898\\
1.93646566414943	0.64731870645308\\
2.39122724922164	1.24345442192263\\
2.24889413709987	0.793537657333463\\
2.39607565749392	0.91178625334131\\
1.80986080122559	0.707169946665853\\
1.97704296558096	0.591120152065339\\
2.00293771191178	1.36005407696541\\
2.24796362886873	0.686533510972829\\
2.11424994022843	0.877619102581024\\
2.34406728575871	0.795199528805402\\
1.56821878862338	0.60163344885657\\
1.82330236671711	0.291695714299127\\
1.45039697294321	0.590849030637304\\
1.88899283418446	1.04552268182906\\
1.52058053004542	0.435160238141619\\
1.66251832326001	0.804648110802819\\
1.64664565454418	0.505949448785172\\
1.91508532580206	0.847486502505432\\
1.34263224779645	0.88256267651136\\
2.26055865989652	0.538404924773607\\
1.69524767103065	0.741992169305531\\
1.94016209417356	0.741514759245998\\
1.82019093577253	0.328368525179733\\
1.4839520831286	0.681796346611881\\
1.13805529399511	0.728796623608036\\
1.73651102992287	0.402875684463721\\
2.40970926685092	0.766519696154169\\
2.12006326132181	0.724614576326541\\
1.33511311753437	0.0606578626326615\\
1.21694480442495	0.939960475214773\\
1.49919597789097	0.779844441507984\\
1.63203786050577	0.615522830684201\\
1.28009907725904	0.603133599618405\\
1.83363385147587	0.447426311107601\\
1.25294254329276	0.891238649047264\\
1.38589448744568	0.110193693911327\\
2.12557258718458	0.480767020439752\\
1.76853338890148	0.345961715962911\\
1.1838175823629	-0.0335760456264687\\
1.07748933668513	0.396363550377259\\
1.91951574719719	0.710847307761236\\
0.890323101344239	0.539672205647595\\
1.65348820931753	0.512238368108479\\
1.65111543031058	-0.265446989923681\\
1.6070900231769	0.488398313760616\\
0.901758997337281	0.183058802837383\\
1.71870592048393	-0.251816153684363\\
1.55442462019974	0.00466747014038243\\
0.795021559669278	0.0361205015350433\\
0.958568445751217	0.212363974452783\\
1.67442948239855	0.4882871131843\\
1.2206685044667	-0.324863168774303\\
1.50767341273933	-0.389108334817502\\
1.37836608274919	-0.0373397947956927\\
1.11156784444486	0.701945681672377\\
1.4984756552433	0.00584512978944459\\
1.24690289872505	0.226154641123381\\
1.09194532469786	-0.130683470031559\\
0.678183222535072	-0.0365335083510088\\
1.73612089885204	0.397031801733272\\
0.933320654619196	0.746473890326733\\
1.15525443706948	-0.179109971394189\\
0.458916732169228	-0.213954536366935\\
1.58359507914698	-0.231297472151952\\
0.931434909226943	0.114256234330456\\
1.1322927346265	0.17900469841877\\
0.780331897364257	0.102401717565955\\
0.501871266865262	-0.0628649708451471\\
0.503048632418745	-0.459092444286199\\
0.604765068083075	-0.011679852416595\\
0.666178137609782	0.219390527881344\\
0.413245356250045	-0.0695882331429949\\
0.467795007897315	-0.0999511872763669\\
1.15246888796731	-0.0589372144022903\\
0.792490565783635	-0.386299556418842\\
0.747652249238335	0.192053680178325\\
0.6730227064451	-0.457495772861639\\
0.657608820873446	-0.397844794344219\\
0.80283938775403	0.00586599616798039\\
1.29819955341968	-0.204261874156702\\
0.700969575790672	-0.360362213508646\\
0.577484279146232	0.667638314415951\\
1.53670668302431	-0.432957390684138\\
-0.157947037327338	-0.206989394214058\\
-0.250033960160869	0.369996686022994\\
0.35679132363435	-0.672812036115748\\
0.307418930825796	-0.491736063949304\\
0.612754451313721	-0.591725661644702\\
0.513152577927538	-0.242776993818146\\
0.457076244801449	-0.714876480294682\\
0.374036335051672	-0.347506625594372\\
0.579283772090358	0.220733907465001\\
0.187849487939868	0.169639639117464\\
-0.0361733332976721	0.281051835212802\\
0.772027729747197	-0.505718105015689\\
0.269625343575084	-0.756939350867156\\
0.60570524403307	-0.119581878977858\\
-0.192532712828272	-0.0344061540276999\\
-0.179793749480468	-0.900222299631903\\
0.0699543465880473	-0.840398840157916\\
0.168779646202848	-0.973477052749994\\
-0.00586739214808801	-0.369383158623758\\
0.465908963186411	-0.973925709442261\\
0.160490995593304	-0.295058807475094\\
0.168397029916368	-0.980178812503749\\
};
\addplot [color=mycolor2,only marks,mark=o,mark options={solid},forget plot]
  table[row sep=crcr]{%
-0.026947097397927	-1.54967503001056\\
-0.0499096052447985	-1.548838725865\\
-0.0878921441942618	-1.54358858745209\\
-0.102243240432917	-1.54118951004161\\
-0.115588964597783	-1.54384949729\\
-0.166126369580897	-1.54954111761082\\
-0.171775059288128	-1.53255169798828\\
-0.202468902552123	-1.53693920317449\\
-0.212030786452003	-1.57194441918315\\
-0.242543396618313	-1.55982750802416\\
-0.268638030522741	-1.55184551810696\\
-0.307999262514236	-1.54141240307698\\
-0.34184706998786	-1.55670987888773\\
-0.321654511497463	-1.58426559222047\\
-0.392925170560618	-1.51563730344866\\
-0.369402481562153	-1.52198515289873\\
-0.416941632627096	-1.51482183374776\\
-0.47214126557288	-1.57339572749443\\
-0.493265130467738	-1.49632259648208\\
-0.512861932358619	-1.50290927620725\\
-0.557340464169725	-1.50842598279773\\
-0.515456646472125	-1.56774163478792\\
-0.608332652640224	-1.42764267683647\\
-0.571584245650045	-1.49817383565417\\
-0.662130333502388	-1.56697697704636\\
-0.669954622356556	-1.57493033524355\\
-0.694374248931051	-1.46017650026605\\
-0.763136406470192	-1.44654664398095\\
-0.64832005525922	-1.5311300933111\\
-0.721079189047473	-1.57176135978772\\
-0.762850186528795	-1.48791858695594\\
-0.771668694696658	-1.4609916619696\\
-0.82794824307844	-1.46985243432682\\
-0.757762090173658	-1.51439340279633\\
-0.73200316350585	-1.54643662607006\\
-0.849650868675982	-1.4739825416715\\
-0.869633988303967	-1.36513677682954\\
-0.929617676304213	-1.40255715057468\\
-0.976307028854808	-1.46241069529678\\
-0.929398084649915	-1.49557897888553\\
-1.08884466383786	-1.60008247625742\\
-1.1061052512272	-1.4667516416831\\
-0.989287983622614	-1.44461809994125\\
-0.985959410143955	-1.5033836797293\\
-1.18771458192431	-1.33744464545877\\
-1.00331746689445	-1.40505396838939\\
-0.892432339761976	-1.52461527768079\\
-1.06850301949737	-1.42382215016215\\
-1.21578154069187	-1.35143372749743\\
-1.1054763672977	-1.37105316655968\\
-1.05324198188984	-1.36449836483918\\
-1.36583986729047	-1.47144251941379\\
-1.21498723375497	-1.58153140750668\\
-1.15864586873963	-1.35211138149604\\
-1.29274883019543	-1.24706936498748\\
-1.21165267830111	-1.36424729300889\\
-1.13339068136518	-1.38791313506574\\
-1.38647009921518	-1.36164505679436\\
-1.38447268580765	-1.24424678906453\\
-1.30846082709955	-1.31326104146817\\
-1.26981825327827	-1.22686897127221\\
-1.49956364455672	-1.20155729619685\\
-1.44709290956236	-1.3499491526295\\
-1.42148110929907	-1.2096773418177\\
-1.48960505179888	-1.41647829527412\\
-1.3264878697695	-1.47045835775913\\
-1.68833293650043	-1.35607977232846\\
-1.58968210104544	-1.26255149104674\\
-1.55268500555184	-1.2291738166282\\
-1.68292402606647	-1.11026634760032\\
-1.72154852078698	-1.24202688963492\\
-1.43590169300641	-1.37871676926254\\
-1.69132652673162	-1.07877471235978\\
-1.70551483678654	-1.42804960005592\\
-1.53520709567707	-1.17299419798963\\
-1.63554863873912	-1.28753234363309\\
-1.65650253897248	-1.09178635476079\\
-1.62198948544192	-1.19413197420841\\
-1.47519706775498	-1.28565950980904\\
-1.43433635492227	-1.32753123102665\\
-1.73478223846456	-1.20660071287845\\
-1.90432859972523	-1.29120302815414\\
-1.9071129230002	-0.972886141766062\\
-1.69183439158024	-1.10780628569212\\
-1.68284658996465	-1.31833231939511\\
-1.85992398322545	-1.01141698924493\\
-1.72772929472913	-0.98157402042344\\
-1.74885476713543	-1.4351574402123\\
-1.84150517090143	-0.95425928858046\\
-1.69315517542604	-0.958656958688715\\
-1.7286495891314	-0.954601665072524\\
-1.81811333022962	-1.19579885376329\\
-1.91064332117842	-1.29835844701107\\
-2.115454982706	-1.0182126578052\\
-1.81014864687136	-1.11941452772307\\
-1.88306383492019	-1.01716868637323\\
-1.86797031445803	-1.04760399233855\\
-1.99290332758122	-1.38532172129099\\
-1.57013566090976	-0.954717363457782\\
-1.6305853956127	-0.926490388852283\\
-1.82503267963498	-1.2054049019875\\
-2.04745234990293	-0.70712356859999\\
-1.91302909485597	-0.765494691747267\\
-1.79315031506749	-1.07048160263372\\
-1.67520961679342	-0.869000607592547\\
-1.98886073309524	-1.32715925014926\\
-2.01375717222897	-0.9315973741601\\
-1.93408159383817	-0.878274086050914\\
-2.26246434296342	-0.886063271151665\\
-1.72639050831228	-0.560229703702508\\
-2.06045827186263	-0.977280310897143\\
-2.21522054773345	-0.832044314048742\\
-2.12802823080291	-0.765383980210819\\
-2.47819250264965	-0.688993880398068\\
-1.68985107972107	-0.732789288101983\\
-1.81469301090601	-0.558603652168838\\
-2.02534666967788	-0.825878531557616\\
-1.90668680745126	-0.687396711733751\\
-2.06031052575321	-0.615192926362136\\
-1.99585448031789	-0.990208999863819\\
-2.14213953096814	-0.765898153601535\\
-1.99073350792875	-0.38534769649198\\
-1.96145912243946	-0.756790477559718\\
-1.88353079013264	-0.493700231213568\\
-2.3752794639133	-0.852864483514567\\
-1.71756604008283	-0.839522058331343\\
-1.96581549537573	-0.658952618483849\\
-2.46496189427916	-0.82131795587749\\
-2.10629520423653	-0.845571433601728\\
-2.11100602865062	-0.528455628514166\\
-1.70854667582737	-0.522179557510307\\
-2.00450257438808	-1.16055808599304\\
-2.21156096039358	-0.482354346628644\\
-2.04160631785436	-0.533352916239695\\
-1.84559801482221	-0.658966536067479\\
-1.92091051009354	-0.644336627288141\\
-1.92864622951692	-0.408174908184284\\
-2.04092091269914	-0.53236208450286\\
-1.86431188937102	-0.409957901472978\\
-1.97521350506084	-0.613741462440286\\
-2.12216417761872	-0.431925911153432\\
-1.80339734046248	-0.583776426535711\\
-1.72974769480169	-0.586333850372364\\
-1.97641639419026	-0.40354436882594\\
-1.66790009602428	-0.535102761781156\\
-1.60550901999747	-0.517641138116707\\
-1.98093745067308	-0.248643233716586\\
-1.77048803449917	-0.0215214325064952\\
-1.82669905771791	-0.344780626915585\\
-2.19905576091282	-0.408420695054419\\
-1.83101161586013	-0.16573497405526\\
-1.99022362231772	-0.240850680829\\
-1.37200304413117	-0.218020520951735\\
-1.83273900042908	0.0375840492151414\\
-1.56236387312051	-0.0622109695320063\\
-1.83847741136532	-0.629170879373212\\
-1.62842916741517	-0.261531442505625\\
-1.760757004596	-0.122954408755528\\
-1.92693522151942	0.0317672214706444\\
-0.795922426677282	-0.305147429881106\\
-1.66083624793154	-0.460640523838259\\
-1.79682003601176	0.166845603602293\\
-1.64022836211743	-0.361636009773446\\
-2.53886522482741	-0.236580562503111\\
-2.03581414704918	0.013541805129293\\
-1.84630209912121	0.230376960175914\\
-1.91209527248309	-0.360580206613225\\
-1.81233556853063	0.0860452228596581\\
-1.85601948484678	-0.446016224299677\\
-1.44425457774794	0.133932704188332\\
-1.82082024175442	-0.524871278676388\\
-1.90283579546103	-0.024980003152043\\
-0.918616001910668	0.598439720318925\\
-1.37397981159442	-0.228688766885865\\
-2.13124001952824	0.0394435337110551\\
-1.51184628119662	-0.339909320146771\\
-1.42078378674312	0.259637468967449\\
-1.46934828152567	-0.48658770179399\\
-1.45036714720006	0.561650634597791\\
-1.6235939585632	0.100495743328196\\
-1.56361953447043	0.429563917963797\\
-1.72011242640493	0.125548136165587\\
-1.70161637144955	-0.00552791997817297\\
-1.3830196422212	0.484897744780771\\
-1.94541153808096	-0.142579452253231\\
-1.17461760721602	0.106879495508337\\
-1.09674397776978	0.918979238595924\\
-1.37782133309489	0.510300838744834\\
-1.88845389859122	0.238194094828098\\
-1.05273228631062	0.534963752730167\\
-1.29045492870113	0.541672326993093\\
-1.51108660201011	0.115095697290448\\
-1.60478470830124	0.761938729066085\\
-1.16297289088485	0.712515972958128\\
-1.36626122649495	0.0459699369006284\\
-1.00569677090725	0.566053726817701\\
-1.15605146422804	0.32048685029173\\
-0.822878786956236	0.299885052175656\\
-1.40380642068769	0.703790875671732\\
-1.59757716568005	0.713604210308872\\
-1.32350395473092	-0.0181115754076439\\
-1.15090617990717	0.0297575564492658\\
-1.21924769269614	0.636116178851027\\
-1.66330401562556	0.274767468591772\\
-1.34530843504664	0.53762093307587\\
-1.39326431751636	1.12504793697234\\
-0.833618774635974	0.512669380246674\\
-0.573367167351866	0.652510525923379\\
-0.276320238657219	0.18763449775689\\
-1.17194492857265	0.572769304168273\\
-0.784840550207506	0.714006761323611\\
-0.924013272173041	0.0557708685451278\\
-1.15282586642296	0.334082023567727\\
-0.935348154153094	0.774875910909737\\
-1.09740398474138	0.634480504642842\\
-1.11946267734234	0.610532582269006\\
-0.838968288924051	1.38322547926388\\
-1.22448365542895	1.32213641543783\\
-0.476481839809044	1.40105426500117\\
-0.769114345508466	0.66169599251629\\
-1.40027835036432	1.28884494337088\\
-1.50132634028759	0.613610138665922\\
-0.949826176623956	0.594374431492637\\
-0.88164099127801	0.552395160007791\\
-0.994300613393627	0.256827952699087\\
-0.632450277038536	0.881951176938878\\
-0.331331900379547	0.85471608258404\\
-1.13074002827094	0.521881807762682\\
-0.86984371951327	0.417083028748866\\
-0.420618985877099	0.726297500339449\\
-0.110157996853952	0.742495519066295\\
0.644069697204106	0.64901805109848\\
-0.262085393832434	0.700643157487521\\
-0.276442795861882	0.467852427149516\\
-0.14443352788167	0.705856371233983\\
-0.727695229224663	1.06740698127406\\
-0.626210638028053	0.986934718277524\\
-0.658293890852093	1.48386542068786\\
-0.71066942064998	0.843047675153383\\
0.185985214616693	0.552601466396477\\
-0.615839609456226	0.931267504635312\\
-0.533903453015464	1.49417117664361\\
-0.278043832459567	1.40015122194036\\
-0.374673778674507	0.760845850670364\\
0.0347222292805265	1.27907372368006\\
-0.41631147842867	0.89614152469315\\
-0.148017321318051	0.860943748299246\\
-0.318167564411286	1.66905910965134\\
0.565869201509072	1.63328081306408\\
0.456182094678272	1.2472468033001\\
};
\addplot[contour prepared, contour prepared format=matlab, contour/labels=false] table[row sep=crcr] {%
%
-0.4	35\\
-0.213547332855987	1.1\\
-0.3	1.04952812221184\\
-0.4	1.01077206464046\\
-0.5	1.000750056633\\
-0.6	1.03344391461483\\
-0.675769986470734	1.1\\
-0.7	1.14525424651179\\
-0.719179232212168	1.2\\
-0.723111439421643	1.3\\
-0.702081958349177	1.4\\
-0.7	1.40499193934206\\
-0.654790279449274	1.5\\
-0.6	1.58826851041679\\
-0.591620994294669	1.6\\
-0.50725098824226	1.7\\
-0.5	1.70765799287112\\
-0.4	1.7988705018467\\
-0.398416269343342	1.8\\
-0.3	1.86404753838139\\
-0.21795526707994	1.9\\
-0.2	1.90723064169707\\
-0.0999999999999996	1.91530419529788\\
-0.0568264528253919	1.9\\
0	1.86847929816662\\
0.0456898108118818	1.8\\
0.0756976180268957	1.7\\
0.0782979057903426	1.6\\
0.0609449833466974	1.5\\
0.0277020260921345	1.4\\
0	1.34288646724704\\
-0.0253194060467919	1.3\\
-0.0995285230559748	1.2\\
-0.0999999999999996	1.19948774743763\\
-0.2	1.10943081264301\\
-0.213547332855987	1.1\\
-0.3	19\\
2.27465613882454	0.3\\
2.2	0.24892405897543\\
2.1	0.222703618680043\\
2	0.239334256179706\\
1.9	0.297891890647882\\
1.89789772811627	0.3\\
1.84369367842398	0.4\\
1.84572849814915	0.5\\
1.9	0.578732858390318\\
1.92475391626406	0.6\\
2	0.633518386178228\\
2.1	0.650737853332358\\
2.2	0.638083695211185\\
2.27911831015394	0.6\\
2.3	0.57641096275596\\
2.33969630014846	0.5\\
2.33505313738197	0.4\\
2.3	0.332587561064288\\
2.27465613882454	0.3\\
-0.3	63\\
-0.484080192695999	0.7\\
-0.5	0.696999080052854\\
-0.6	0.696207502423565\\
-0.620068546883927	0.7\\
-0.7	0.718151169434936\\
-0.8	0.766146621019509\\
-0.847070157725096	0.8\\
-0.9	0.848180176951882\\
-0.945013832911172	0.9\\
-1	0.98586050796387\\
-1.00796428882021	1\\
-1.04496235292533	1.1\\
-1.06318665985891	1.2\\
-1.06324500613573	1.3\\
-1.04609956314439	1.4\\
-1.01289207562836	1.5\\
-1	1.52743428217573\\
-0.963797724030163	1.6\\
-0.90149419120431	1.7\\
-0.9	1.70207881609318\\
-0.826135720576083	1.8\\
-0.8	1.83133062419776\\
-0.739709272183289	1.9\\
-0.7	1.94218327881539\\
-0.642032666745452	2\\
-0.6	2.03970638025142\\
-0.530300757395578	2.1\\
-0.5	2.12499909702021\\
-0.4	2.19746921984105\\
-0.395783039645325	2.2\\
-0.3	2.25503153695489\\
-0.2	2.29784223838613\\
-0.191405048614206	2.3\\
-0.0999999999999996	2.3223025006228\\
0	2.32634978924187\\
0.0999999999999996	2.30467034163665\\
0.10890547927936	2.3\\
0.2	2.24179915054795\\
0.236508172466626	2.2\\
0.299626259239729	2.1\\
0.3	2.09915092791157\\
0.330832142090264	2\\
0.349463456724022	1.9\\
0.358147915720152	1.8\\
0.358478852282136	1.7\\
0.35120435524309	1.6\\
0.336246433983283	1.5\\
0.312649188159837	1.4\\
0.3	1.3621655019565\\
0.276265260887391	1.3\\
0.225647183506172	1.2\\
0.2	1.1605749302136\\
0.156187644576887	1.1\\
0.0999999999999996	1.03615494759625\\
0.065050619655916	1\\
0	0.942242519602179\\
-0.052164025689399	0.9\\
-0.0999999999999996	0.865725676278684\\
-0.2	0.801563275645553\\
-0.202953722569141	0.8\\
-0.3	0.75351374325933\\
-0.4	0.717278364510764\\
-0.484080192695999	0.7\\
-0.2	37\\
-1.56529411545794	-2\\
-1.6	-2.01245971388063\\
-1.7	-2.02461890419678\\
-1.8	-2.01541243371065\\
-1.8526889962753	-2\\
-1.9	-1.98432388066796\\
-2	-1.9277318885777\\
-2.03480896513991	-1.9\\
-2.1	-1.83079810384426\\
-2.12310650767031	-1.8\\
-2.16978614404824	-1.7\\
-2.18582933596125	-1.6\\
-2.17158440986655	-1.5\\
-2.1243890810738	-1.4\\
-2.1	-1.37121134268407\\
-2.03264117524997	-1.3\\
-2	-1.27681271202683\\
-1.9	-1.21746143192003\\
-1.86127846012603	-1.2\\
-1.8	-1.17882911629796\\
-1.7	-1.15704141461781\\
-1.6	-1.1506477453101\\
-1.5	-1.16393935044052\\
-1.40860323752345	-1.2\\
-1.4	-1.20493746883343\\
-1.30701477955087	-1.3\\
-1.3	-1.3126571405073\\
-1.26549592755933	-1.4\\
-1.24995786500678	-1.5\\
-1.25448421791155	-1.6\\
-1.27718036756503	-1.7\\
-1.3	-1.75760836777183\\
-1.32259149602269	-1.8\\
-1.39787820549193	-1.9\\
-1.4	-1.90228843625212\\
-1.5	-1.97429388548785\\
-1.56529411545794	-2\\
-0.2	47\\
2.31747380247264	-0.1\\
2.3	-0.10663431673306\\
2.2	-0.126686045289765\\
2.1	-0.127094714285621\\
2	-0.107736146220979\\
1.98093484138189	-0.1\\
1.9	-0.0671904975292269\\
1.8	-0.00474471294271601\\
1.79442156100678	0\\
1.7	0.0855934852583633\\
1.68722884878894	0.1\\
1.6094252957692	0.2\\
1.6	0.215410801446114\\
1.55448947160412	0.3\\
1.51832524584596	0.4\\
1.5056075211203	0.5\\
1.5244552941215	0.6\\
1.58646742822202	0.7\\
1.6	0.712348989762723\\
1.7	0.788456812688648\\
1.71877766508911	0.8\\
1.8	0.837278804622304\\
1.9	0.874326670162927\\
1.98931748051139	0.9\\
2	0.902667552318772\\
2.1	0.919953335397786\\
2.2	0.929256452745863\\
2.3	0.929005650248858\\
2.4	0.916804271143253\\
2.46387379683718	0.9\\
2.5	0.886997296227336\\
2.6	0.828928256640016\\
2.63527318814256	0.8\\
2.7	0.714771988207349\\
2.70932330545105	0.7\\
2.74456972694494	0.6\\
2.75533055103326	0.5\\
2.7462893131718	0.4\\
2.718949222523	0.3\\
2.7	0.25782453727528\\
2.67227541848102	0.2\\
2.60349172532672	0.1\\
2.6	0.0959642401464631\\
2.50141650671931	0\\
2.5	-0.00114895768880383\\
2.4	-0.0653636390055184\\
2.31747380247264	-0.1\\
-0.2	83\\
-0.445798458826147	0.5\\
-0.5	0.487621435641501\\
-0.6	0.479789470267651\\
-0.7	0.48921978950566\\
-0.73854532248265	0.5\\
-0.8	0.518862539475812\\
-0.9	0.569885998677907\\
-0.943352970186259	0.6\\
-1	0.644182094189188\\
-1.05960591917642	0.7\\
-1.1	0.743892317288715\\
-1.14695703020258	0.8\\
-1.2	0.876259670197099\\
-1.21604020357569	0.9\\
-1.26928068113888	1\\
-1.3	1.07921085095033\\
-1.30835621122342	1.1\\
-1.33308189052688	1.2\\
-1.3430884761114	1.3\\
-1.33820760867208	1.4\\
-1.31830499303129	1.5\\
-1.3	1.55383395110644\\
-1.28389288190686	1.6\\
-1.23594721047366	1.7\\
-1.2	1.75950442218712\\
-1.17485214936973	1.8\\
-1.10225976496543	1.9\\
-1.1	1.90283229349396\\
-1.020583036285	2\\
-1	2.02326057406985\\
-0.930229241259774	2.1\\
-0.9	2.13144867927069\\
-0.831706147761201	2.2\\
-0.8	2.23050130525037\\
-0.723842075394616	2.3\\
-0.7	2.3210095493367\\
-0.60294015419461	2.4\\
-0.6	2.40232564153913\\
-0.5	2.47432381323837\\
-0.458439938394525	2.5\\
-0.4	2.53552698404636\\
-0.3	2.58498253285614\\
-0.260134058327492	2.6\\
-0.2	2.62287251292282\\
-0.0999999999999996	2.64792078595615\\
0	2.65879949388709\\
0.0999999999999996	2.65363993793469\\
0.2	2.62857765532253\\
0.256804999644752	2.6\\
0.3	2.57588462935389\\
0.380444967450065	2.5\\
0.4	2.47721032245443\\
0.445469291997438	2.4\\
0.487831961540187	2.3\\
0.5	2.26009590540102\\
0.514269301944478	2.2\\
0.53058401362351	2.1\\
0.540643516354619	2\\
0.545956644710642	1.9\\
0.547536955581723	1.8\\
0.545939327227325	1.7\\
0.541233376489842	1.6\\
0.532902559407292	1.5\\
0.519641126181167	1.4\\
0.5	1.30450887662994\\
0.49896275839633	1.3\\
0.466652181590988	1.2\\
0.418379014697131	1.1\\
0.4	1.07240173346856\\
0.347869527972215	1\\
0.3	0.948986702023241\\
0.251567475058593	0.9\\
0.2	0.856742298594573\\
0.130214072639797	0.8\\
0.0999999999999996	0.77852188389956\\
0	0.709139473132571\\
-0.0138941800469413	0.7\\
-0.0999999999999996	0.647953632521598\\
-0.185763646948509	0.6\\
-0.2	0.59248531121629\\
-0.3	0.547001259706505\\
-0.4	0.510743327393388\\
-0.445798458826147	0.5\\
-0.1	59\\
3	0.0235975790885923\\
2.98645390865299	0\\
2.91571899069444	-0.1\\
2.9	-0.118899812262981\\
2.82646471162656	-0.2\\
2.8	-0.225536997109168\\
2.71017081831388	-0.3\\
2.7	-0.30776341369691\\
2.6	-0.371855718983394\\
2.54036367221086	-0.4\\
2.5	-0.418769569489577\\
2.4	-0.450778616745397\\
2.3	-0.466665440215264\\
2.2	-0.466925185927978\\
2.1	-0.450918631252037\\
2	-0.416831252944364\\
1.96956084473383	-0.4\\
1.9	-0.365941224641034\\
1.80674775104811	-0.3\\
1.8	-0.295592384735454\\
1.7	-0.208118808767509\\
1.6926213633682	-0.2\\
1.6	-0.101890519024066\\
1.59849659600584	-0.1\\
1.51864394352001	0\\
1.5	0.0244298582036683\\
1.44780780637843	0.1\\
1.4	0.175156101769571\\
1.38471528493107	0.2\\
1.33107141037296	0.3\\
1.3	0.37177671594147\\
1.28687326669668	0.4\\
1.25525909899868	0.5\\
1.24203671009463	0.6\\
1.25841043681596	0.7\\
1.3	0.764059527695949\\
1.32676544292682	0.8\\
1.4	0.852863303526888\\
1.47679812603014	0.9\\
1.5	0.910159270708295\\
1.6	0.952101745287328\\
1.7	0.989763225261812\\
1.72923377890138	1\\
1.8	1.02186094650533\\
1.9	1.0505546444789\\
2	1.07686696542192\\
2.09770343855384	1.1\\
2.1	1.10054317213329\\
2.2	1.12015371894082\\
2.3	1.1360489211834\\
2.4	1.14722059888088\\
2.5	1.15226880309944\\
2.6	1.14926226874629\\
2.7	1.1355322703219\\
2.8	1.10736805760487\\
2.8180889634877	1.1\\
2.9	1.05644721390213\\
2.97390968814505	1\\
3	0.971613705072144\\
-0.1	63\\
-1.56825276196537	-2.4\\
-1.6	-2.40856512450377\\
-1.7	-2.41951911215279\\
-1.8	-2.41643853817922\\
-1.9	-2.40034670681308\\
-1.90125167827611	-2.4\\
-2	-2.37395888026396\\
-2.1	-2.33424992467672\\
-2.16410045487639	-2.3\\
-2.2	-2.28011562151062\\
-2.3	-2.20967976417123\\
-2.31168276992021	-2.2\\
-2.4	-2.11676050968998\\
-2.41554442928338	-2.1\\
-2.49159892424539	-2\\
-2.5	-1.98546121976052\\
-2.54712204646036	-1.9\\
-2.58269430916939	-1.8\\
-2.6	-1.70047877133001\\
-2.60008717152797	-1.7\\
-2.6	-1.67716008109049\\
-2.59974260764521	-1.6\\
-2.5800517095203	-1.5\\
-2.53635720614389	-1.4\\
-2.5	-1.35111800343044\\
-2.46287865276718	-1.3\\
-2.4	-1.24301706227705\\
-2.35103238710063	-1.2\\
-2.3	-1.16755529990354\\
-2.2	-1.10845117438884\\
-2.18471257893288	-1.1\\
-2.1	-1.06284600386648\\
-2	-1.02400746430287\\
-1.92877952395404	-1\\
-1.9	-0.991592070140975\\
-1.8	-0.967026666553032\\
-1.7	-0.948508066057132\\
-1.6	-0.937449507726477\\
-1.5	-0.936189101119119\\
-1.4	-0.948487308932506\\
-1.3	-0.98041119305714\\
-1.26515821601016	-1\\
-1.2	-1.04719544396485\\
-1.15216510424945	-1.1\\
-1.1	-1.18153495036333\\
-1.09088527999887	-1.2\\
-1.0562239629279	-1.3\\
-1.03542874431667	-1.4\\
-1.02582161195922	-1.5\\
-1.02578585828881	-1.6\\
-1.03444724630231	-1.7\\
-1.05152621923392	-1.8\\
-1.07729474321553	-1.9\\
-1.1	-1.96670243789898\\
-1.11404923188019	-2\\
-1.16561362126682	-2.1\\
-1.2	-2.15536969630254\\
-1.23743988871219	-2.2\\
-1.3	-2.26661854012363\\
-1.34593455823767	-2.3\\
-1.4	-2.3381675816026\\
-1.5	-2.38289453996984\\
-1.56825276196537	-2.4\\
-0.1	105\\
-0.508316616295135	0.3\\
-0.6	0.286682393862679\\
-0.7	0.288414801313911\\
-0.757314452503386	0.3\\
-0.8	0.308859972178965\\
-0.9	0.349119119099532\\
-0.98454645316912	0.4\\
-1	0.409673906648392\\
-1.1	0.48827549983275\\
-1.11291748503853	0.5\\
-1.2	0.583103248309244\\
-1.21655774790011	0.6\\
-1.3	0.691971672969596\\
-1.30724656440665	0.7\\
-1.38827321796998	0.8\\
-1.4	0.816841027962312\\
-1.46115652885801	0.9\\
-1.5	0.963543194710875\\
-1.52459250986713	1\\
-1.5778984365648	1.1\\
-1.6	1.15610965872805\\
-1.62000464010244	1.2\\
-1.64999019263801	1.3\\
-1.66571387889108	1.4\\
-1.66729007558504	1.5\\
-1.65456762418209	1.6\\
-1.62704122147052	1.7\\
-1.6	1.76491800131139\\
-1.58553917738004	1.8\\
-1.53203715670237	1.9\\
-1.5	1.94877431704197\\
-1.46619209702523	2\\
-1.4	2.08615817569819\\
-1.38925623626253	2.1\\
-1.30397267539152	2.2\\
-1.3	2.20438218517896\\
-1.21211994263449	2.3\\
-1.2	2.31251907392305\\
-1.11365200540909	2.4\\
-1.1	2.41334042438366\\
-1.00863582923434	2.5\\
-1	2.50798045786422\\
-0.9	2.59662112053069\\
-0.895992554605289	2.6\\
-0.8	2.67957905618784\\
-0.773072024334727	2.7\\
-0.7	2.75504111972184\\
-0.632326154192889	2.8\\
-0.6	2.82164239838003\\
-0.5	2.88017694025045\\
-0.459695595091869	2.9\\
-0.4	2.93045125799839\\
-0.3	2.97141354059078\\
-0.206398543119905	3\\
-0.2	3.00211214864885\\
-0.0999999999999996	3.02571686181231\\
0	3.03850976729552\\
0.0999999999999996	3.04013479103016\\
0.2	3.02911751707504\\
0.3	3.002215207338\\
0.304975093830312	3\\
0.4	2.95743835165335\\
0.476684216258855	2.9\\
0.5	2.88058305921257\\
0.565037424264184	2.8\\
0.6	2.74416829025021\\
0.620285147328424	2.7\\
0.655073633938289	2.6\\
0.678185249372815	2.5\\
0.692944335910684	2.4\\
0.7	2.32204598740994\\
0.701682929076206	2.3\\
0.706179958733249	2.2\\
0.708165107411989	2.1\\
0.708626766505356	2\\
0.708332179514905	1.9\\
0.707861491921941	1.8\\
0.707610614428576	1.7\\
0.707756882976919	1.6\\
0.708169600222263	1.5\\
0.708223704497352	1.4\\
0.706419099675154	1.3\\
0.7	1.20549751592988\\
0.699588754109182	1.2\\
0.682840440099257	1.1\\
0.6450499605476	1\\
0.6	0.938819740547069\\
0.569476443245	0.9\\
0.5	0.841341437395216\\
0.450180709872453	0.8\\
0.4	0.767790395724954\\
0.3	0.701592839030683\\
0.297705953206223	0.7\\
0.2	0.640781550131429\\
0.134618059429415	0.6\\
0.0999999999999996	0.580243260818937\\
0	0.521810590976203\\
-0.0379383897168491	0.5\\
-0.0999999999999996	0.465785788770407\\
-0.2	0.413489119924388\\
-0.229020068188368	0.4\\
-0.3	0.367188558149802\\
-0.4	0.328922961105305\\
-0.5	0.301155189542831\\
-0.508316616295135	0.3\\
0	116\\
3	-2.3785986207232\\
2.95933726083798	-2.3\\
2.9	-2.23713016361543\\
2.88494452512225	-2.2\\
2.81243236256826	-2.1\\
2.8	-2.09069803227607\\
2.74560789671572	-2\\
2.7	-1.95577375465672\\
2.67266378425195	-1.9\\
2.6	-1.81026178488554\\
2.59583532109849	-1.8\\
2.52743346197044	-1.7\\
2.5	-1.67522565487427\\
2.45625428910096	-1.6\\
2.4	-1.53758481828022\\
2.38126478129313	-1.5\\
2.30594566572988	-1.4\\
2.3	-1.3948324960226\\
2.23672323805788	-1.3\\
2.2	-1.26177746149736\\
2.16378735922775	-1.2\\
2.1	-1.12193742461438\\
2.08851775002893	-1.1\\
2.01627577899295	-1\\
2	-0.983261475626774\\
1.94547410353004	-0.9\\
1.9	-0.846076127254861\\
1.87252893660173	-0.8\\
1.8	-0.702734247488454\\
1.79848817876122	-0.7\\
1.72935372033826	-0.6\\
1.7	-0.564174741945269\\
1.65883584245853	-0.5\\
1.6	-0.42034399036577\\
1.58764205727868	-0.4\\
1.51875057977748	-0.3\\
1.5	-0.275063623900232\\
1.45068717072519	-0.2\\
1.4	-0.127145725484327\\
1.38257668850623	-0.1\\
1.31594668698136	0\\
1.3	0.0240323713642851\\
1.24981212444137	0.1\\
1.2	0.176839477186571\\
1.18393860045879	0.2\\
1.11739809619954	0.3\\
1.1	0.32845751045362\\
1.04653207428897	0.4\\
1	0.46589890958019\\
0.965284638136418	0.5\\
0.9	0.566235238970843\\
0.83017608809123	0.6\\
0.8	0.614641972912198\\
0.7	0.625481802465304\\
0.6	0.611676395392857\\
0.562019671778806	0.6\\
0.5	0.584694169374131\\
0.4	0.548800675029267\\
0.3	0.505084066089615\\
0.290105739489408	0.5\\
0.2	0.457328431109715\\
0.0999999999999996	0.40523129720711\\
0.0905275038741672	0.4\\
0	0.351377842364948\\
-0.0941203704876395	0.3\\
-0.0999999999999996	0.296772229517582\\
-0.2	0.243472681707679\\
-0.287978502634384	0.2\\
-0.3	0.193760800805842\\
-0.4	0.149158167412619\\
-0.5	0.113494613205423\\
-0.557737163627815	0.1\\
-0.6	0.0887738125535445\\
-0.7	0.0787647958438295\\
-0.8	0.0887451717666898\\
-0.834019157048116	0.1\\
-0.9	0.1208935985407\\
-1	0.176690049848546\\
-1.02964567242111	0.2\\
-1.1	0.252642489298928\\
-1.15159552679506	0.3\\
-1.2	0.343143939803581\\
-1.25757128343983	0.4\\
-1.3	0.441724797544357\\
-1.35756020904568	0.5\\
-1.4	0.544045056297113\\
-1.45554294091695	0.6\\
-1.5	0.647401783197346\\
-1.55343226336025	0.7\\
-1.6	0.750204028943684\\
-1.65230221839649	0.8\\
-1.7	0.851551109903897\\
-1.75284663863285	0.9\\
-1.8	0.950936617800124\\
-1.8555614344184	1\\
-1.9	1.04805332883867\\
-1.96080616626072	1.1\\
-2	1.14266493686871\\
-2.06880295276424	1.2\\
-2.1	1.23451488992615\\
-2.17960141696691	1.3\\
-2.2	1.32324828938568\\
-2.29303258182544	1.4\\
-2.3	1.4083246589951\\
-2.4	1.49247544075459\\
-2.41303062230156	1.5\\
-2.5	1.57644938540077\\
-2.53967933562475	1.6\\
-2.6	1.65713842328356\\
-2.66823271102031	1.7\\
-2.7	1.7334773121569\\
-2.79693432044433	1.8\\
-2.8	1.80372356932501\\
-2.9	1.87974998764743\\
-2.94187716626769	1.9\\
-3	1.95249525305584\\
0	53\\
-3	-1.11164245492119\\
-2.96933095073349	-1.1\\
-2.9	-1.0811831952391\\
-2.8	-1.05186765617105\\
-2.7	-1.02231676437857\\
-2.62803739884055	-1\\
-2.6	-0.993221727014945\\
-2.5	-0.966092542031668\\
-2.4	-0.939060810615031\\
-2.3	-0.912204374663418\\
-2.25620492189876	-0.9\\
-2.2	-0.886419759112897\\
-2.1	-0.861816004183261\\
-2	-0.837963013431033\\
-1.9	-0.815162462502472\\
-1.82905715920473	-0.8\\
-1.8	-0.793929422629531\\
-1.7	-0.774997818831811\\
-1.6	-0.759138535455374\\
-1.5	-0.747824810639899\\
-1.4	-0.743429276143615\\
-1.3	-0.749918029705091\\
-1.2	-0.77424659788089\\
-1.14852142186405	-0.8\\
-1.1	-0.828723467979198\\
-1.02843277746757	-0.9\\
-1	-0.936315782358085\\
-0.963312145386739	-1\\
-0.919556164388996	-1.1\\
-0.9	-1.16023846498211\\
-0.888735052561335	-1.2\\
-0.867423447486194	-1.3\\
-0.852607215121547	-1.4\\
-0.843118948055914	-1.5\\
-0.838036221695341	-1.6\\
-0.836615162615277	-1.7\\
-0.838246631120221	-1.8\\
-0.842425936664749	-1.9\\
-0.848731073899555	-2\\
-0.856806576572134	-2.1\\
-0.866351229943938	-2.2\\
-0.877108528291877	-2.3\\
-0.888859143646409	-2.4\\
-0.9	-2.49064863589151\\
-0.901432908706691	-2.5\\
-0.914781466091156	-2.6\\
-0.928608916248326	-2.7\\
-0.942795907879191	-2.8\\
-0.957239441846545	-2.9\\
-0.971850568627482	-3\\
-0.98655247509704	-3.1\\
-1	-3.19366376481915\\
-1.00134156452181	-3.2\\
0	51\\
1.02540422510567	3.2\\
1.00791157666192	3.1\\
1	3.06025310181756\\
0.990933359090041	3\\
0.974520614614124	2.9\\
0.958435696726269	2.8\\
0.942809405787257	2.7\\
0.9277929809904	2.6\\
0.913561105406296	2.5\\
0.900315516522378	2.4\\
0.9	2.3976165177597\\
0.888492564856532	2.3\\
0.878203404197746	2.2\\
0.869754087099036	2.1\\
0.863504937456453	2\\
0.85987646444579	1.9\\
0.859364582514036	1.8\\
0.862564010275182	1.7\\
0.870207053428433	1.6\\
0.88323326062642	1.5\\
0.9	1.41348306977167\\
0.903205114314873	1.4\\
0.935158973387972	1.3\\
0.983999976726837	1.2\\
1	1.17430324007175\\
1.09327565568897	1.1\\
1.1	1.09590010776146\\
1.2	1.07200548387213\\
1.3	1.07323271446264\\
1.4	1.08610508844309\\
1.46999759294651	1.1\\
1.5	1.10504070442276\\
1.6	1.12705751669358\\
1.7	1.15167530249626\\
1.8	1.1781636899724\\
1.87828135168336	1.2\\
1.9	1.20611067886615\\
2	1.2352675910732\\
2.1	1.26525291009877\\
2.2	1.29588125255853\\
2.21420804373197	1.3\\
2.3	1.32827760501499\\
2.4	1.36105992252241\\
2.5	1.39388433185897\\
2.52078040574793	1.4\\
2.6	1.4289947575817\\
2.7	1.46417566089881\\
2.8	1.49874815659497\\
2.80439528656047	1.5\\
2.9	1.53685690091567\\
3	1.57385727232263\\
0.1	167\\
-3	-0.324517706014149\\
-2.96862839412348	-0.4\\
-2.9	-0.498542823685414\\
-2.89885844979073	-0.5\\
-2.8	-0.585490725742235\\
-2.77785757615102	-0.6\\
-2.7	-0.637944242395569\\
-2.6	-0.669746693969371\\
-2.5	-0.687492429870182\\
-2.4	-0.694784610213082\\
-2.3	-0.694202841739423\\
-2.2	-0.687613112483786\\
-2.1	-0.676381309951972\\
-2	-0.66152347224677\\
-1.9	-0.643815387357359\\
-1.8	-0.623878111010703\\
-1.7	-0.602253461940859\\
-1.68977966849094	-0.6\\
-1.6	-0.577277411723512\\
-1.5	-0.551187865213336\\
-1.4	-0.524997945060768\\
-1.3	-0.500050298374338\\
-1.29971155991791	-0.5\\
-1.2	-0.472391589868432\\
-1.1	-0.454396212568487\\
-1	-0.472468942253381\\
-0.976301369773636	-0.5\\
-0.9	-0.590585585273937\\
-0.896331417158866	-0.6\\
-0.850073992176443	-0.7\\
-0.810358782710588	-0.8\\
-0.8	-0.827970005647358\\
-0.778682332581431	-0.9\\
-0.752015834612101	-1\\
-0.72929171396992	-1.1\\
-0.710141831598909	-1.2\\
-0.7	-1.26229405289173\\
-0.693890484490556	-1.3\\
-0.679630040777964	-1.4\\
-0.666735157640361	-1.5\\
-0.654354766470619	-1.6\\
-0.64146944978506	-1.7\\
-0.62684892854121	-1.8\\
-0.608976684507189	-1.9\\
-0.6	-1.94181415055294\\
-0.584421535735481	-2\\
-0.550198927059576	-2.1\\
-0.503435861181971	-2.2\\
-0.5	-2.20629927800938\\
-0.427704026464517	-2.3\\
-0.4	-2.32998279699167\\
-0.3	-2.399164069839\\
-0.297982419529688	-2.4\\
-0.2	-2.44023330430376\\
-0.0999999999999996	-2.45981526995552\\
0	-2.46351952235219\\
0.0999999999999996	-2.45403419582796\\
0.2	-2.43233633584765\\
0.295081694787418	-2.4\\
0.3	-2.39843053431014\\
0.4	-2.3562594862182\\
0.5	-2.30081401058381\\
0.501267994924424	-2.3\\
0.6	-2.23741637790068\\
0.649965612702604	-2.2\\
0.7	-2.1623593220715\\
0.773286743634559	-2.1\\
0.8	-2.07686923486674\\
0.881262725514557	-2\\
0.9	-1.9817776057082\\
0.979078740846939	-1.9\\
1	-1.87752132134141\\
1.06923215638059	-1.8\\
1.1	-1.76368033026976\\
1.15252110711294	-1.7\\
1.2	-1.63789848619997\\
1.22851290777316	-1.6\\
1.29620300288988	-1.5\\
1.3	-1.4938477938665\\
1.35742459606095	-1.4\\
1.4	-1.31337207221542\\
1.40659245429509	-1.3\\
1.44712122922727	-1.2\\
1.4745347661309	-1.1\\
1.48973134715184	-1\\
1.49337633226572	-0.9\\
1.48599726046461	-0.8\\
1.46806539612645	-0.7\\
1.44006831489529	-0.6\\
1.40257481744783	-0.5\\
1.4	-0.494662744276471\\
1.359387328569	-0.4\\
1.3088440205598	-0.3\\
1.3	-0.284797421158125\\
1.25317165524114	-0.2\\
1.2	-0.113006958310831\\
1.19199720048866	-0.1\\
1.12499112939669	0\\
1.1	0.0354987047483643\\
1.04996752456392	0.1\\
1	0.161752742449681\\
0.962528406092491	0.2\\
0.9	0.261570837184631\\
0.845518610193831	0.3\\
0.8	0.331148299471702\\
0.7	0.370840800404848\\
0.6	0.385153597353276\\
0.5	0.379916693189436\\
0.4	0.359501282760385\\
0.3	0.327256845265781\\
0.234761216795369	0.3\\
0.2	0.285838441665791\\
0.0999999999999996	0.237494790626392\\
0.0293137432929089	0.2\\
0	0.184095660240526\\
-0.0999999999999996	0.127236360360362\\
-0.148138875637213	0.1\\
-0.2	0.0686016452182757\\
-0.3	0.0105553909734579\\
-0.320189815073772	0\\
-0.4	-0.0474596782096074\\
-0.499366957990664	-0.1\\
-0.5	-0.100408470195845\\
-0.6	-0.152914097944047\\
-0.7	-0.195241138322672\\
-0.719603099232088	-0.2\\
-0.8	-0.232625524896809\\
-0.9	-0.24529508212703\\
-0.977637977745832	-0.2\\
-1	-0.189567841531118\\
-1.07528037724092	-0.1\\
-1.1	-0.07666096677492\\
-1.15453935613558	0\\
-1.2	0.0516619670511138\\
-1.23472009795199	0.1\\
-1.3	0.17609264202016\\
-1.31876687931103	0.2\\
-1.4	0.29044387275941\\
-1.40852753271207	0.3\\
-1.5	0.393803973478852\\
-1.50641707553369	0.4\\
-1.6	0.486626212109605\\
-1.61626050736193	0.5\\
-1.7	0.569136648241735\\
-1.74434561333757	0.6\\
-1.8	0.640677852716457\\
-1.9	0.699326331856795\\
-1.9015692136492	0.7\\
-2	0.74670158875368\\
-2.1	0.779192220381915\\
-2.2	0.797054908143851\\
-2.29490827391824	0.8\\
-2.3	0.800187127639657\\
-2.30153199007336	0.8\\
-2.4	0.787818236158561\\
-2.5	0.758459912940685\\
-2.6	0.709489436009221\\
-2.61486701770148	0.7\\
-2.7	0.639652855645051\\
-2.7446627969237	0.6\\
-2.8	0.542347911489425\\
-2.83591641514709	0.5\\
-2.9	0.405965747389212\\
-2.90386769625754	0.4\\
-2.95707936778001	0.3\\
-2.99503183539544	0.2\\
-3	0.181215603330518\\
0.1	65\\
1.30770712388817	1.3\\
1.4	1.28287519179522\\
1.5	1.28127035406867\\
1.6	1.29084386590629\\
1.65129441455344	1.3\\
1.7	1.30882015037581\\
1.8	1.33370065073618\\
1.9	1.36373122563843\\
2	1.39827501852929\\
2.00462458597552	1.4\\
2.1	1.44116588835292\\
2.2	1.48873276048371\\
2.22281210222277	1.5\\
2.3	1.54797225033717\\
2.37921983450095	1.6\\
2.4	1.61850555804118\\
2.49058667960949	1.7\\
2.5	1.7125392945989\\
2.56826931956635	1.8\\
2.6	1.86755858941805\\
2.61665917565003	1.9\\
2.64296603805585	2\\
2.64859125007536	2.1\\
2.63603611679904	2.2\\
2.60546319156413	2.3\\
2.6	2.31192465069363\\
2.55943929432783	2.4\\
2.5	2.49007183427393\\
2.49300058679814	2.5\\
2.40501018698385	2.6\\
2.4	2.60489648858776\\
2.3	2.69039031770258\\
2.2863999991563	2.7\\
2.2	2.75712122495305\\
2.11410667621515	2.8\\
2.1	2.80701816223915\\
2	2.84548964155664\\
1.9	2.87027985386636\\
1.8	2.88257618332305\\
1.7	2.88238514989349\\
1.6	2.86829600570642\\
1.5	2.83680055278538\\
1.43233831184865	2.8\\
1.4	2.78297197784022\\
1.30054292532441	2.7\\
1.3	2.69953024554595\\
1.21982902791555	2.6\\
1.2	2.57235053154881\\
1.16088811427098	2.5\\
1.11537424225516	2.4\\
1.1	2.36033874567849\\
1.0808003943098	2.3\\
1.05480335029024	2.2\\
1.03533351131059	2.1\\
1.02224934131298	2\\
1.01581821883764	1.9\\
1.01676526936444	1.8\\
1.02641883260542	1.7\\
1.0470327105971	1.6\\
1.08248745824254	1.5\\
1.1	1.46612979420466\\
1.14883689238384	1.4\\
1.2	1.35202520693996\\
1.3	1.30174680124775\\
1.30770712388817	1.3\\
0.2	81\\
-0.164247663084913	-2\\
-0.0999999999999996	-2.02709024234528\\
0	-2.04324521052056\\
0.0999999999999996	-2.03869046565796\\
0.2	-2.01673708290355\\
0.244703056713898	-2\\
0.3	-1.97967465457781\\
0.4	-1.92872466334043\\
0.444863750756376	-1.9\\
0.5	-1.86453838657145\\
0.584204537759656	-1.8\\
0.6	-1.7877259170363\\
0.699266659215694	-1.7\\
0.7	-1.69933713458597\\
0.799810016016699	-1.6\\
0.8	-1.59980416484918\\
0.889868388743102	-1.5\\
0.9	-1.4881013010325\\
0.970635838176589	-1.4\\
1	-1.35992110575033\\
1.04176027131566	-1.3\\
1.1	-1.20337442081618\\
1.10195262426207	-1.2\\
1.15170004936778	-1.1\\
1.18871360829923	-1\\
1.2	-0.955777185705956\\
1.2137554604519	-0.9\\
1.22632947143857	-0.8\\
1.22648463573599	-0.7\\
1.21477656187571	-0.6\\
1.2	-0.5366607337588\\
1.19188262988168	-0.5\\
1.15841366306811	-0.4\\
1.11490062420968	-0.3\\
1.1	-0.272024916870731\\
1.06017199621403	-0.2\\
1	-0.107473868558624\\
0.994569906458959	-0.1\\
0.911141357650959	0\\
0.9	0.0120916121257432\\
0.8	0.0976501595258424\\
0.79610746760583	0.1\\
0.7	0.152675368913035\\
0.6	0.181775126786089\\
0.5	0.188728064608559\\
0.4	0.177271084359127\\
0.3	0.150904180156201\\
0.2	0.112790478500056\\
0.173117903870687	0.1\\
0.0999999999999996	0.0637402655365346\\
0	0.00734046150505274\\
-0.0121482240230062	0\\
-0.0999999999999996	-0.0578653384305858\\
-0.162232105751485	-0.1\\
-0.2	-0.129094121341928\\
-0.291866820009151	-0.2\\
-0.3	-0.207556281477613\\
-0.4	-0.29944268899756\\
-0.400635418122631	-0.3\\
-0.484402282007773	-0.4\\
-0.5	-0.428368563324993\\
-0.539235830683178	-0.5\\
-0.567582987239308	-0.6\\
-0.577777725773664	-0.7\\
-0.576828256527202	-0.8\\
-0.569300173228861	-0.9\\
-0.558040871223632	-1\\
-0.544686524772258	-1.1\\
-0.530008607251526	-1.2\\
-0.514142043239017	-1.3\\
-0.5	-1.38145619691543\\
-0.496365601106754	-1.4\\
-0.474179205217046	-1.5\\
-0.447579974969624	-1.6\\
-0.414630864053781	-1.7\\
-0.4	-1.73708857598746\\
-0.366765075701313	-1.8\\
-0.3	-1.89936785964472\\
-0.299358322769849	-1.9\\
-0.2	-1.98367151003625\\
-0.164247663084913	-2\\
0.2	43\\
-2.4252368222829	-0.4\\
-2.4	-0.412794979326071\\
-2.3	-0.447155296232001\\
-2.2	-0.464752394231418\\
-2.1	-0.469166087717431\\
-2	-0.462856994488697\\
-1.9	-0.447424559226473\\
-1.8	-0.423752601594832\\
-1.72470981721869	-0.4\\
-1.7	-0.389715721754599\\
-1.6	-0.336703727407683\\
-1.54430115355975	-0.3\\
-1.5	-0.247193076850094\\
-1.46734301123486	-0.2\\
-1.45075245014838	-0.1\\
-1.47303183612442	0\\
-1.5	0.052932585956805\\
-1.52441770721651	0.1\\
-1.59932919387064	0.2\\
-1.6	0.20072138462529\\
-1.7	0.29669524685107\\
-1.70415679558269	0.3\\
-1.8	0.367019691377595\\
-1.86200928073258	0.4\\
-1.9	0.419171288945754\\
-2	0.451531951612215\\
-2.1	0.465586304408194\\
-2.2	0.459808930220239\\
-2.3	0.432262577697003\\
-2.36435166978247	0.4\\
-2.4	0.378002435558997\\
-2.48855449953287	0.3\\
-2.5	0.28633023159451\\
-2.55967336118122	0.2\\
-2.6	0.108761874153566\\
-2.60354164435165	0.1\\
-2.62453189444827	0\\
-2.62461793851196	-0.1\\
-2.60120673463239	-0.2\\
-2.6	-0.202440699844528\\
-2.54411810540118	-0.3\\
-2.5	-0.344320308637903\\
-2.4252368222829	-0.4\\
0.2	45\\
1.40704193087413	1.5\\
1.5	1.47664369846818\\
1.6	1.47301335555676\\
1.7	1.48481191907472\\
1.76379637855244	1.5\\
1.8	1.51006854565802\\
1.9	1.54970827928298\\
2	1.59919059554414\\
2.00142246847569	1.6\\
2.1	1.67376261172076\\
2.13161541992833	1.7\\
2.2	1.78300463531326\\
2.21333923151978	1.8\\
2.25955306791673	1.9\\
2.27735158299312	2\\
2.26958352531762	2.1\\
2.23653588459407	2.2\\
2.2	2.26282613845452\\
2.17473133556639	2.3\\
2.1	2.37822880963594\\
2.07304021016232	2.4\\
2	2.44869054546951\\
1.9	2.49346276203656\\
1.87381248359957	2.5\\
1.8	2.51732246914782\\
1.7	2.52184244958676\\
1.6	2.50595501546583\\
1.5846322534902	2.5\\
1.5	2.46593155270099\\
1.40727801411918	2.4\\
1.4	2.39418688416799\\
1.32108621439675	2.3\\
1.3	2.2689224398126\\
1.26513460825363	2.2\\
1.22766299219242	2.1\\
1.20378138123618	2\\
1.2	1.96145182381274\\
1.19493599932547	1.9\\
1.19994452074361	1.8\\
1.2	1.79972026471787\\
1.22553760680641	1.7\\
1.27809740966103	1.6\\
1.3	1.5738673989704\\
1.4	1.50240344703005\\
1.40704193087413	1.5\\
0.3	61\\
-0.125167309926428	-1.6\\
-0.0999999999999996	-1.62188672329912\\
0	-1.66684666332338\\
0.0999999999999996	-1.6770214101183\\
0.2	-1.66138343674667\\
0.3	-1.6252293808796\\
0.346880986994398	-1.6\\
0.4	-1.57122058899555\\
0.5	-1.50159886715521\\
0.50193243040015	-1.5\\
0.6	-1.41692381104194\\
0.617480773700314	-1.4\\
0.7	-1.31665240521453\\
0.714798821905651	-1.3\\
0.797407170703461	-1.2\\
0.8	-1.19654505688559\\
0.865449492384307	-1.1\\
0.9	-1.03940935551558\\
0.920562224819395	-1\\
0.961129712643076	-0.9\\
0.987974107009624	-0.8\\
1	-0.705022712634276\\
1.00058309426331	-0.7\\
1	-0.669047662502528\\
0.998668900183856	-0.6\\
0.981943698978069	-0.5\\
0.950460170792981	-0.4\\
0.904413821540618	-0.3\\
0.9	-0.292678681827316\\
0.834598806848481	-0.2\\
0.8	-0.160566772615142\\
0.729908826235589	-0.1\\
0.7	-0.0783540913177252\\
0.6	-0.0324954798265929\\
0.5	-0.0133278104603304\\
0.4	-0.0174321681000138\\
0.3	-0.0408400159590406\\
0.2	-0.0796296525450901\\
0.160269364286805	-0.1\\
0.0999999999999996	-0.13420114259667\\
0.00156524992759764	-0.2\\
0	-0.201203470306729\\
-0.0999999999999996	-0.287083570553066\\
-0.114157447640315	-0.3\\
-0.2	-0.395972491132083\\
-0.203454317919749	-0.4\\
-0.268939663377937	-0.5\\
-0.3	-0.567515725880741\\
-0.314103940567917	-0.6\\
-0.341228143040908	-0.7\\
-0.354680064490255	-0.8\\
-0.357717732836187	-0.9\\
-0.352686877920104	-1\\
-0.341059587265298	-1.1\\
-0.323491785221707	-1.2\\
-0.3	-1.29946693928278\\
-0.299843354905908	-1.3\\
-0.262212609819667	-1.4\\
-0.212498726596144	-1.5\\
-0.2	-1.52050620655886\\
-0.125167309926428	-1.6\\
0.3	17\\
1.53036673575927	1.8\\
1.6	1.7626425886704\\
1.7	1.76307343303413\\
1.79361773569245	1.8\\
1.8	1.80475527105754\\
1.8718195697818	1.9\\
1.88242162114072	2\\
1.82007367399999	2.1\\
1.8	2.1149930990435\\
1.7	2.14153136571199\\
1.6	2.11876131672246\\
1.57464509928375	2.1\\
1.5	2.01054869084412\\
1.49491034669418	2\\
1.4827867546655	1.9\\
1.5	1.84512613316987\\
1.53036673575927	1.8\\
0.4	37\\
0.015557338689864	-1.2\\
0.0999999999999996	-1.25510295938646\\
0.2	-1.26763805194458\\
0.3	-1.24395020299477\\
0.388100817451847	-1.2\\
0.4	-1.1937862361086\\
0.5	-1.11693673583322\\
0.517316822277742	-1.1\\
0.6	-1.01056950619627\\
0.608074318455662	-1\\
0.671628343009568	-0.9\\
0.7	-0.84057636099594\\
0.716168469697753	-0.8\\
0.738779876805841	-0.7\\
0.741739957977157	-0.6\\
0.72288040499321	-0.5\\
0.7	-0.446081672981832\\
0.673189232921145	-0.4\\
0.6	-0.322888275155892\\
0.560474534978212	-0.3\\
0.5	-0.274987407873277\\
0.4	-0.266639914150435\\
0.3	-0.28649844032988\\
0.268914000605434	-0.3\\
0.2	-0.335808936531094\\
0.112163038753925	-0.4\\
0.0999999999999996	-0.411125780632996\\
0.0221574137228243	-0.5\\
0	-0.533585891521295\\
-0.0372876532937009	-0.6\\
-0.0748917515767719	-0.7\\
-0.0952265632216867	-0.8\\
-0.0985061309281697	-0.9\\
-0.0839868754531443	-1\\
-0.0495001849177397	-1.1\\
0	-1.18565805115837\\
0.015557338689864	-1.2\\
};
\end{axis}
\end{tikzpicture}%
\end{document}
\caption{Plots of the kernel matrix projections using each of the six computed principal components. The quality of the plots decreases together with the relative importance of the associated eigenvalue.  }
\label{fig:toyProj}
\end{figure}
The actual de-noising is done using:
\begin{equation}
\bar{\mathbf{x}} = h(z)
\end{equation}
Where the function $h$ must minimize:
\begin{equation}
\min \sum\limits_{k=1}^{N} \| \mathbf{x}_k - h(z_k)\|^2.
\end{equation} 
The function $h$ is generally an MLP trained using Bayesian learning. \footnote{Support Vector Machines: Methods and Applications, Suykens et al., page 213} But solving an the unconstrained optimization problem using a specified kernel function is also possible.  